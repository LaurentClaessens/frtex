% This is part of Mes notes de mathématique
% Copyright (c) 2011-2012,2015
%   Laurent Claessens
% See the file fdl-1.3.txt for copying conditions.

%+++++++++++++++++++++++++++++++++++++++++++++++++++++++++++++++++++++++++++++++++++++++++++++++++++++++++++++++++++++++++++
\section{La différentielle revisitée}
%+++++++++++++++++++++++++++++++++++++++++++++++++++++++++++++++++++++++++++++++++++++++++++++++++++++++++++++++++++++++++++

%---------------------------------------------------------------------------------------------------------------------------
\subsection{Les formes différentielles de base}
%---------------------------------------------------------------------------------------------------------------------------

Si la fonction $f\colon \eR^n\to \eR$ est différentiable alors la différentielle en $a\in\eR^n$ est l'application
\begin{equation}        \label{EqFormDiffdfahOM}
    \begin{aligned}
        df_a\colon \eR^n&\to \eR \\
        u&\mapsto \frac{ \partial f }{ \partial x_1 }(a)u_1+\ldots+\frac{ \partial f }{ \partial x_n }(a)u_n.
    \end{aligned}
\end{equation}
Considérons en particulier la fonction qui à $x\in\eR^n$ fait correspondre $x_i\in\eR$. Par abus de notations,  nous la noterons $x_i$. Nous avons 
\begin{equation}
    \frac{ \partial x_i }{ \partial x_j }=\delta_{ij}.
\end{equation}
Par exemple $\partial_yx=0$ et $\partial_xx=1$. Toutes les dérivées partielles de $x_i$ s'annulent sauf la $i$ème qui vaut $1$. Par conséquent
\begin{equation}
    \begin{aligned}
        dx_i\colon \eR^n&\to \eR \\
        u&\mapsto u_i. 
    \end{aligned}
\end{equation}

\begin{remark}
    En toute rigueur nous devrions écrire $(dx_i)_a$. Mais étant donné que
    \begin{equation}
        (dx_i)_a(u)=(dx_i)_b(u)
    \end{equation}
    pour tout points $a$, $b$ et pour tout vecteurs $u$, nous nous permettons de simplifier la notation en ne précisant pas en quel point nous calculons la différentielle de $x_i$.
\end{remark}

Étant donné que $dx_i(u)=u_i$, nous pouvons récrire la formule \eqref{EqFormDiffdfahOM} en remplaçant $u_i$ par $dx_i(u)$ :
\begin{equation}
    df_a(u)=\frac{ \partial f }{ \partial x_1 }(a)dx_1(u)+\ldots+\frac{ \partial f }{ \partial x_n }(a)dx_n(u).
\end{equation}
En tant que application linéaire, $df_a$ est une combinaison linéaire des $dx_i$. En notations compacte :
\begin{equation}
    df_a=\sum_{i=1}^n\frac{ \partial f }{ \partial x_i }(a)dx_i.
\end{equation}

%---------------------------------------------------------------------------------------------------------------------------
\subsection{Différentielles de fonctions composées}
%---------------------------------------------------------------------------------------------------------------------------

Cette façon de voir la différentielle nous permet de jeter un nouveau regard sur la formule de différentiation des fonctions composées. Soient
\begin{equation}
    \begin{aligned}[]
        f\colon \eR^p&\to \eR^n\\
        g\colon \eR^n&\to \eR,
    \end{aligned}
\end{equation}
et $h\colon \eR^p\to \eR$ définie par 
\begin{equation}
    h(u)=h\big( f(u) \big)=(g\circ f)(u).
\end{equation}
Nous allons noter $x$ les coordonnées de $\eR^p$, $a$ un point de $\eR^p$ et $u$, un vecteur de $\eR^p$ accroché au point $a$. Pour $\eR^n$, les notations seront que les coordonnées sont $y$, $b$ est un point de $\eR^n$ et $v$ est un vecteur «accroché» au point $b$.

Nous avons
\begin{equation}
    dg_b(v)=\sum_{i=1}^n\frac{ \partial g }{ \partial y_i }(b)dy_i(v).
\end{equation}
Ici $dy_i(v)$ signifie la $i$ème composante de $v$. C'est simplement $v_i$. Cette formule étant valable pour tout point $b\in\eR^n$ et pour tout vecteur $v$, nous pouvons l'écrire en particulier pour
\begin{subequations}
    \begin{numcases}{}
        b=f(a)\\
        v=df_a(u).
    \end{numcases}
\end{subequations}
Cela donne
\begin{equation}        \label{EqdgfadfauOM}
    dg_{f(a)}\big( df_a(u) \big)=\sum_{i=1}^n\frac{ \partial g }{ \partial y_i }\big( f(a) \big)dy_i\big( df_a(u) \big).
\end{equation}
Mais 
\begin{equation}
    df_a(u)=\sum_{j=1}^p\frac{ \partial f }{ \partial x_j }(a)dx_j(u),
\end{equation}
donc la $i$ème composante de ce vecteur est
\begin{equation}
     \big( df_a(u)\big)_i=\sum_{j=1}^p\frac{ \partial f_i }{ \partial x_j }(a)dx_j(u).
\end{equation}
En remplaçant $dy_i\big( df_a(u) \big)$ par cela dans l'expression \eqref{EqdgfadfauOM}, nous trouvons
\begin{equation}
    dg_{f(a)}\big( df_a(u) \big)=\sum_{i=1}^n\frac{ \partial g }{ \partial y_i }\big( f(a) \big)\sum_{j=1}^p\frac{ \partial f_i }{ \partial x_j }(a)dx_j(u).
\end{equation}
Nous pouvons vérifier que cela est la différentielle de $g\circ f$ au point $a$ appliquée au vecteur $u$. En effet
\begin{equation}
    d(g\circ f)_a(u)=\sum_{j=1}^p\frac{ \partial (g\circ f) }{ \partial x_j }(a)dx_j(u),
\end{equation}
tandis que, par la dérivation de fonctions composées, 
\begin{equation}        \label{EqDerCompofgOM}
    \frac{ \partial (g\circ f) }{ \partial x_j }(a)=\sum_{i=1}^n\frac{ \partial g }{ \partial y_i }\big( f(a) \big)\frac{ \partial f_i }{ \partial x_j }(a).
\end{equation}
Au final, ce que nous avons prouvé est que
\begin{equation}
    d(g\circ f)_a(u)=dg_{f(a)}\big( df_a(u) \big).
\end{equation}

%---------------------------------------------------------------------------------------------------------------------------
\subsection{Passage aux coordonnées polaires}
%---------------------------------------------------------------------------------------------------------------------------

Le changement de coordonnées pour les polaires est la fonction
\begin{equation}
    f\begin{pmatrix}
        r    \\ 
        \theta    
    \end{pmatrix}=\begin{pmatrix}
        x    \\ 
        y    
    \end{pmatrix}=\begin{pmatrix}
        r\cos\theta    \\ 
        r\sin\theta    
    \end{pmatrix}.
\end{equation}
Considérons une fonction $g$ sur $\eR^2$, et définissons la fonction $\tilde g$ par
\begin{equation}
    \tilde g(r,\theta)=g(r\cos\theta,r\sin\theta).
\end{equation}
La formule \eqref{EqDerCompofgOM} permet de trouver les dérivées partielles de $g$ par rapport à $r$ et $\theta$ en termes de celles par rapport à $x$ et $y$ de $g$.

Pour faire le lien avec les notations du point précédent, nous avons
\begin{equation}
    \begin{aligned}[]
        f_1(r,\theta)&=r\cos(\theta)\\
        f_2(r,\theta)&=r\sin(\theta)\\
        (x_1,x_2)&\to(r,\theta)\\
        (y_1,y_2)&\to(x,y).
    \end{aligned}
\end{equation}
Nous avons donc 
\begin{equation}
    \begin{aligned}[]
        \frac{ \partial \tilde g }{ \partial r }(r,\theta)&=\sum_{i=1}^2\frac{ \partial g }{ \partial x_i }\big( f(r,\theta) \big)\frac{ \partial f_i }{ \partial r }(r,\theta)\\
        &=\frac{ \partial g }{ \partial x }(r\cos\theta,r\sin\theta)\frac{ \partial \big( r\cos\theta \big) }{ \partial r }(r,\theta)\\
        &\quad+\frac{ \partial g }{ \partial y }(r\cos\theta,r\sin\theta)\frac{ \partial \big( r\sin\theta\big) }{ \partial r }(r,\theta)\\
        &=\cos\theta\frac{ \partial g }{ \partial x }(r\cos\theta,r\sin\theta)+\sin\theta\frac{ \partial g }{ \partial y }(r\cos\theta,r\sin\theta).
    \end{aligned}
\end{equation}

Prenons par exemple $g(x,y)=\frac{1}{ x^2+y^2 }$. Étant donné que
\begin{equation}
    \frac{ \partial g }{ \partial x }=\frac{ -2x }{ (x^2+y^2)^2 },
\end{equation}
nous avons
\begin{equation}
    \frac{ \partial g }{ \partial x }(r\cos\theta,r\sin\theta)=\frac{ -2\cos\theta }{ r^3 }.
\end{equation}
En utilisant la formule,
\begin{equation}
    \frac{ \partial \tilde g }{ \partial r }(r,\theta)=\cos(\theta)\left( \frac{ -2\cos\theta }{ r^3 } \right)+\sin(\theta)\left( \frac{ -2\sin\theta }{ r^3 } \right)=-\frac{ 2 }{ r^3 }.
\end{equation}
Nous pouvons vérifier directement que cela est correct. En effet
\begin{equation}
    \tilde g(r,\theta)=g(r\cos\theta,r\sin\theta)=\frac{1}{ r^2 },
\end{equation}
dont la dérivée par rapport à $r$ vaut $-2/r^3$.

En ce qui concerne la dérivée par rapport à $\theta$, nous avons
\begin{equation}
    \begin{aligned}[]
    \frac{ \partial \tilde g }{ \partial \theta }&=\frac{ \partial g }{ \partial x }(r\cos\theta,r\sin\theta)\frac{ \partial \big( r\cos(\theta) \big) }{ \partial \theta }+\frac{ \partial g }{ \partial y }(r\cos\theta,r\sin\theta)\frac{ \partial \big( r\sin(\theta) \big) }{ \partial \theta }\\
    &=\left( \frac{ -2\cos\theta }{ r^3 } \right)(-r\sin\theta)+\left( \frac{ -2\sin\theta }{ r^3 } \right)(r\cos\theta)\\
    &=0.
    \end{aligned}
\end{equation}

En résumé et avec quelques abus de notations :
\begin{equation}
    \begin{aligned}[]
        \frac{ \partial \tilde g }{ \partial r }&=\cos(\theta)\frac{ \partial g }{ \partial x }+\sin(\theta)\frac{ \partial g }{ \partial y }\\
        \frac{ \partial \tilde g }{ \partial \theta }&=-r\sin(\theta)\frac{ \partial g }{ \partial x }+r\cos(\theta)\frac{ \partial g }{ \partial y }\\
    \end{aligned}
\end{equation}

%+++++++++++++++++++++++++++++++++++++++++++++++++++++++++++++++++++++++++++++++++++++++++++++++++++++++++++++++++++++++++++
\section{Théorie générale}
%+++++++++++++++++++++++++++++++++++++++++++++++++++++++++++++++++++++++++++++++++++++++++++++++++++++++++++++++++++++++++++

%---------------------------------------------------------------------------------------------------------------------------
\subsection{Base locale}
%---------------------------------------------------------------------------------------------------------------------------

Les coordonnées sphériques et cylindriques sont deux systèmes de coordonnées «un peu courbe» qui existent sur $\eR^3$. Il en existe de nombreux autres, que nous appelons \defe{coordonnées curvilignes}{coordonnées!curvilignes}. Des coordonnées curvilignes sur $\eR^3$ est n'importe quel\footnote{Nous n'entrons pas dans les détails de régularité.} système qui permet de repérer un point de $\eR^3$ à partir de trois nombres.

Il s'agit donc d'un ensemble de trois applications 
\begin{equation}
    x_i\colon \eR^3\to \eR.
\end{equation}
Les coordonnées cylindriques sont
\begin{subequations}
    \begin{numcases}{}
        x_1(r,\theta,z)=r\cos\theta\\
        x_2(r,\theta,z)=r\sin\theta\\
        x_3(r,\theta,z)=z
    \end{numcases}
\end{subequations}

Soit donc un système général $q=(q_1,q_2,q_3)$ et 
\begin{equation}
    M(q)=\begin{pmatrix}
        x_1(q)    \\ 
        x_2(q)    \\ 
        x_3(q)    
    \end{pmatrix}.
\end{equation}
Si nous fixons $q_2$ et $q_3$ et que nous laissons varier $q_1$, nous obtenons une courbe\footnote{Dans le cas des sphériques, c'est une demi-droite horizontale d'angle $q_2$ et de hauteur $q_3$.} dont nous pouvons considérer le vecteur vitesse, c'est à dire le vecteur tangent. En chaque point nous avons ainsi trois vecteurs
\begin{equation}
    \frac{ \partial M }{ \partial q_i }(q).
\end{equation}
Nous disons que le système de coordonnées curviligne est \defe{orthogonal}{orthogonal!coordonnées curviligne} si ces trois vecteurs sont orthogonaux. Dans la suite nous supposerons que c'est toujours le cas.

Nous posons
\begin{equation}
    h_i=\left\| \frac{ \partial M }{ \partial q_i } \right\|
\end{equation}
et nous considérons les trois vecteurs normés
\begin{equation}        \label{EqDefeihMqOM}
    e_i=h_i^{-1}\frac{ \partial M }{ \partial q_i }.
\end{equation}
Les trois vecteurs $\{ e_1,e_2,e_3 \}$ forment une base orthonormée dite \defe{base locale}{base!locale}. Ce sont des vecteurs liés\footnote{En géométrie différentielle on dira que ce sont des élément de l'espace tangent, mais c'est une toute autre histoire.} au point $M$.

%---------------------------------------------------------------------------------------------------------------------------
\subsection{Importance de l'orthogonalité}
%---------------------------------------------------------------------------------------------------------------------------

Nous avons dit que nous nous restreignons au cas où les vecteurs $e_i$ sont orthogonaux. En termes de produits scalaires, cela signifie
\begin{equation}
    e_i\cdot e_j=\delta_{ij}.
\end{equation}
Nous en étudions maintenant quelque conséquence. L'équation \eqref{EqDefeihMqOM} peut s'écrire plus explicitement sous la forme
\begin{equation}
    e_i=\sum_k h_i^{-1}\frac{ \partial x_k }{ \partial q_i }1_k.
\end{equation}
Notez que pour chaque $k$ et $i$, la quantité $h_i^{-1}\frac{ \partial x_k }{ \partial q_i }$ est un simple nombre. Nous allons les mettre dans une matrice :
\begin{equation}
    A_{ki}=h_i^{-1}\frac{ \partial x_k }{ \partial q_i }.
\end{equation}
Cela nous donne le changement de base
\begin{equation}        \label{EqChmBaseeisAkiAkOM}
    e_i=\sum_kA_{ki}1_k.
\end{equation}
Le produit $e_i\cdot e_j$ s'écrit alors
\begin{equation}
    \begin{aligned}[]
        e_i\cdot e_j&=\sum_{kl}A_{ki}A_{lj}\underbrace{1_k\cdot 1_l}_{=\delta_{kl}}\\
        &=\sum_{kl}A_{ki}A_{lj}\delta_{kl}\\
        &=\sum_kA_{ki}A_{kj}\\
        &=\sum_k(A^T)_{ik}A_{kj}.
    \end{aligned}
\end{equation}
Or cela doit valoir $\delta_{ij}$. Par conséquent 
\begin{equation}
    A^T=A^{-1}.
\end{equation}
Le fait que les coordonnées curvilignes considérées soient orthogonales s'exprime donc par la fait que la matrice de changement de base est une matrice orthogonale.

Cette circonstance nous permet d'inverser le changement de base \eqref{EqChmBaseeisAkiAkOM} en multipliant cette équation par $(A^{-1})_{il}$ des deux côtés et en faisant la somme sur $i$ :
\begin{equation}
    \sum_i (A^{-1})_{il}e_i=\sum_{kl}\underbrace{A_{ki}(A^{-1})_{il}}_{=\delta_{kl}}1_k,
\end{equation}
par conséquent
\begin{equation}
    \sum_i(A^T)_{il}e_i=1_l,    
\end{equation}
et
\begin{equation}        \label{EqChamvarunlAeiOM}
    1_l=\sum_iA_{li}e_i=\sum_ih_i^{-1}\frac{ \partial x_l }{ \partial q_i }e_i.
\end{equation}
Armés de cette importante formule, nous pouvons exprimer les quantités que nous connaissons dans la base canonique en termes de la base locale.

Une autre conséquence du fait que $e_1$, $e_2$ et $e_3$ est une base orthonormée est que, éventuellement en réordonnant les vecteurs, on a
\begin{equation}
    \begin{aligned}[]
        e_1\times e_2&=e_3\\
        e_2\times e_3&=e_1\\
        e_3\times e_1&=e_2
    \end{aligned}
\end{equation}
Ces trois relations s'écrivent en une seule avec
\begin{equation}
    e_i\times e_j=\sum_{k}\epsilon_{ijk}e_k
\end{equation}
où 
\begin{equation}
    \epsilon_{ijk}=\begin{cases}
        0    &   \text{si $i$, $j$ et $k$ ne sont pas tous différents}\\
        1    &    \text{si $ijk$ se ramène à $123$ par un nombre pair de permutations}\\
        -1    &    \text{si $ijk$ se ramène à $123$ par un nombre impair de permutations}
    \end{cases}
\end{equation}
est le \defe{symbole de Levi-Civita}{Levi-Civita}. La formule du produit vectoriel peut également être utilisée à l'envers sous la forme
\begin{equation}        \label{EqekeitimesejOM}
    e_k=\frac{ 1 }{2}\sum_{ij}\epsilon_{ijk}\,e_i\times e_j.
\end{equation}

Le symbole de Levi-Civita possède de nombreuses formules. En voici certaines, facilement démontrables en considérant tous les cas :
\begin{equation}
    \epsilon_{ijk}\epsilon_{ijl}=\delta_{kl}| \epsilon_{ijk} |.
\end{equation}
Grâce au symboles de Levi-Civita, le produit mixte des vecteurs de base a une belle forme :
\begin{equation}        \label{EqProdMixteepsilonCicivrOM}
    e_l\cdot(e_i\times e_j)=\sum_k\epsilon_{ijk}e_l\times e_k=\sum_k\epsilon_{ijk}\delta_{lk}=\epsilon_{ijl}.
\end{equation}

%+++++++++++++++++++++++++++++++++++++++++++++++++++++++++++++++++++++++++++++++++++++++++++++++++++++++++++++++++++++++++++
\section{Gradient en coordonnées curvilignes}
%+++++++++++++++++++++++++++++++++++++++++++++++++++++++++++++++++++++++++++++++++++++++++++++++++++++++++++++++++++++++++++

Soit $(x,y,z)\mapsto f(x,y,z)$ une fonction sur $\eR^3$. Nous pouvons la composer avec les coordonnées curvilignes $q$ pour obtenir la fonction
\begin{equation}
    \tilde f(q_1,q_2,q_3)=f\big( x_1(q),x_2(x),x_3(q) \big).
\end{equation}
Nous disons que $\tilde f$ est l'expression de $f$ dans les coordonnées $q$. Nous savons déjà comment calculer le gradient de $f$ en coordonnées cartésiennes :
\begin{equation}
    F(x,y, z)=\nabla f(x,y,z)=\begin{pmatrix}
        \partial_xf(x,y,z)    \\ 
        \partial_yf(x,y,z)    \\ 
        \partial_zf(x,y,z)    \
    \end{pmatrix}.
\end{equation}
Cela est un vecteur lié au point $(x,y,z)$. Nous voudrions exprimer ce vecteur dans la base $\{ e_1,e_2,e_3 \}$. En d'autres termes, nous voudrions trouver les nombres $\tilde F_1$, $\tilde F_2$ et $\tilde F_3$ tels que
\begin{equation}
    F(x,y,z)=F\big( x(q),y(q),z(q) \big)=\tilde F_1e_1+\tilde F_2e_2+\tilde F_3e_3.
\end{equation}
Ces nombres seront des fonctions de $(q_1,q_2,q_3)$.

Par définition,
\begin{equation}
    \nabla f=\sum_l\frac{ \partial f }{ \partial x_l }1_l.
\end{equation}
En remplaçant $1_l$ par sa valeur en termes des $e_i$ par la formule \eqref{EqChamvarunlAeiOM},
\begin{equation}
    \begin{aligned}[]
        \nabla f&=\sum_l\frac{ \partial f }{ \partial x_l }1_l\\
        &=\sum_l\frac{ \partial f }{ \partial x_l }\sum_ih_i^{-1}\frac{ \partial x_l }{ \partial q_i }e_i\\
        &=\sum_{il}\frac{1}{ h_i }\frac{ \partial f }{ \partial x_l }\frac{ \partial x_l }{ \partial q_i }e_i\\
        &=\sum_i\frac{1}{ h_i }\frac{ \partial \tilde f }{ \partial q_i }e_i.
    \end{aligned}
\end{equation}

Plus explicitement,
\begin{equation}        \label{EqGradientenCurviligneOM}
    \nabla f\big( x(q),y(q),z(q) \big)=\sum_i \frac{1}{ h_i(q) }\frac{ \partial \tilde f }{ \partial q_i }(q)e_i
\end{equation}
où
\begin{equation}
    h_i(q)=\left\| \frac{ \partial M }{ \partial q_i }(q) \right\|.
\end{equation}
Le plus souvent nous n'allons pas noter explicitement la dépendance de $h_i$ en $q$.

Quelques formulaires intéressants sont en appendice de \cite{Schomblond_em}.

%---------------------------------------------------------------------------------------------------------------------------
\subsection{Coordonnées polaires}
%---------------------------------------------------------------------------------------------------------------------------

Les coordonnées curvilignes polaires sont données par
\begin{equation}
    M(r,\theta)=\begin{pmatrix}
        r\cos(\theta)    \\ 
        r\sin(\theta)    
    \end{pmatrix},
\end{equation}
et par conséquent
\begin{equation}
    \begin{aligned}[]
        \frac{ \partial M }{ \partial r }&=\begin{pmatrix}
            \cos(\theta)    \\ 
            \sin(\theta)    
        \end{pmatrix},&\frac{ \partial M }{ \partial \theta }=\begin{pmatrix}
            -r\sin(\theta)    \\ 
            r\cos(\theta)    
        \end{pmatrix}.
    \end{aligned}
\end{equation}
Nous avons les normes $h_r=1$ et $h_{\theta}=r$, et donc les vecteurs de la base locale en $(r,\theta)$ sont
\begin{equation}
    e_r=\begin{pmatrix}
        \cos(\theta)    \\ 
        \sin(\theta)    
    \end{pmatrix}=\cos(\theta)e_x+\sin(\theta)e_y
\end{equation}
ainsi que
\begin{equation}
    e_{\theta}=\begin{pmatrix}
        -\sin(\theta)    \\ 
        \cos(\theta)    
    \end{pmatrix}=-\sin(\theta)e_x+\cos(\theta)e_y.
\end{equation}


Ces vecteurs sont représentés à la figure \ref{LabelFigCurvilignesPolaires}. Notez qu'il y en a une paire différente en chaque point.
\newcommand{\CaptionFigCurvilignesPolaires}{En brun, les lignes que le point suivrait si on ne variait qu'une coordonnées polaire à la fois. Les vecteurs rouges sont les vecteurs $e_{r}$ et $e_{\theta}$.}
\input{Fig_CurvilignesPolaires.pstricks}


%---------------------------------------------------------------------------------------------------------------------------
\subsection{Coordonnées cylindriques}
%---------------------------------------------------------------------------------------------------------------------------

Les coordonnées cylindriques sont les mêmes que les coordonnées polaires à part qu'il faut écrire
\begin{equation}
    M(r,\theta,z)=\begin{pmatrix}
        r\cos(\theta)    \\ 
        r\sin(\theta)    \\ 
        z    
    \end{pmatrix},
\end{equation}
et nous avons le vecteur de base supplémentaire
\begin{equation}
    e_z=\frac{ \partial M }{ \partial z }=\begin{pmatrix}
        0    \\ 
        0    \\ 
        1    
    \end{pmatrix}
\end{equation}
parce que $h_z=1$.

%---------------------------------------------------------------------------------------------------------------------------
\subsection{Coordonnées sphériques}
%---------------------------------------------------------------------------------------------------------------------------

Les coordonnées curvilignes sphériques sont données par
\begin{equation}
    M(\rho,\theta,\varphi)=
    \begin{pmatrix}
        \rho\sin(\theta)\cos(\varphi)    \\ 
        \rho\sin(\theta)\sin(\varphi)    \\ 
        \rho\cos(\theta)
    \end{pmatrix},
\end{equation}
dont les dérivées sont données par
\begin{equation}
    \begin{aligned}[]
        \frac{ \partial M }{ \partial r }&=\begin{pmatrix}
        \sin(\theta)\cos(\varphi)    \\ 
        \sin(\theta)\sin(\varphi)    \\ 
        \cos(\theta)
    \end{pmatrix},
    &\frac{ \partial M }{ \partial \theta }&=
    \begin{pmatrix}
        \rho\cos(\theta)\cos(\varphi)    \\ 
        \rho\cos(\theta)\sin(\varphi)    \\ 
        -\rho\sin(\theta)
    \end{pmatrix},\\
    \frac{ \partial M }{ \partial \varphi }&=
    \begin{pmatrix}
        -\rho\sin(\theta)\sin(\varphi)    \\ 
        \rho\sin(\theta)\cos(\varphi)    \\ 
        0
    \end{pmatrix}
    \end{aligned}
\end{equation}
Les normes de ces vecteurs sont $h_{\rho}=1$, $h_{\theta}=\rho$ et $h_{\varphi}=\rho\sin(\theta)$. Les vecteurs de la base locale en $(\rho,\theta,\varphi)$ sont donc
\begin{equation}
    \begin{aligned}[]
        e_r&=\begin{pmatrix}
        \sin(\theta)\cos(\varphi)    \\ 
        \sin(\theta)\sin(\varphi)    \\ 
        \cos(\theta)
    \end{pmatrix},
    &e_{\theta}&=
    \begin{pmatrix}
        \cos(\theta)\cos(\varphi)    \\ 
        \cos(\theta)\sin(\varphi)    \\ 
        -\sin(\theta)
    \end{pmatrix},\\
    e_{\varphi}&=
    \begin{pmatrix}
        -\sin(\varphi)    \\ 
        \cos(\varphi)    \\ 
        0
    \end{pmatrix}
    \end{aligned}
\end{equation}

Nous pouvons exprimer le gradient d'une fonction en coordonnées sphériques en utilisant la formule \eqref{EqGradientenCurviligneOM} :
\begin{equation}        \label{EqGradientSpheriqueOM}
    \nabla\tilde f(\rho,\theta,\varphi)=\frac{ \partial \tilde f }{ \partial \rho }e_{\rho}+\frac{1}{ \rho }\frac{ \partial \tilde f }{ \partial \theta }e_{\theta}+\frac{1}{ \rho\sin(\theta) }\frac{ \partial \tilde f }{ \partial \varphi }r_{\varphi}.
\end{equation}
Cette expression peut paraître peu pratique parce que les vecteurs $e_{\rho}$, $e_{\theta}$ et $e_{\varphi}$ eux-mêmes changent en chaque point. Elle est effectivement peu adaptée au dessin, mais elle est très pratique pour des fonctions ayant des symétries.

\begin{example}
    Le potentiel de la gravitation est la fonction
    \begin{equation}
        V(x,y,z)=\frac{1}{ \sqrt{x^2+y^2+z^2} }.
    \end{equation}
    En coordonnées sphériques elle s'écrit
    \begin{equation}
        \tilde V(\rho,\theta,\varphi)=\frac{1}{ \rho }.
    \end{equation}
    En voila une fonction qu'elle est facile à dériver, contrairement à $V$ ! En suivant la formule \eqref{EqGradientSpheriqueOM}, nous avons immédiatement
    \begin{equation}
        \nabla\tilde V=-\frac{1}{ \rho^2 }e_{\rho}.
    \end{equation}
    Nous voyons immédiatement que cela est un champ de vecteurs dont la norme diminue comme le carré de la distance à l'origine et qui est en permanence dirigé vers l'origine.
\end{example}


%+++++++++++++++++++++++++++++++++++++++++++++++++++++++++++++++++++++++++++++++++++++++++++++++++++++++++++++++++++++++++++
\section{Divergence en coordonnées curvilignes}
%+++++++++++++++++++++++++++++++++++++++++++++++++++++++++++++++++++++++++++++++++++++++++++++++++++++++++++++++++++++++++++

Nous savons que 
\begin{equation}
    \nabla\tilde f=\sum_j\frac{1}{ h_j }\frac{ \partial \tilde f }{ \partial q_j }e_j.
\end{equation}
Nous pouvons en particulier considérer la fonction $f(q)=q_i$. De la même manière que nous avions noté $x_i$ la fonction $x\mapsto x_i$, nous notons $q_i$ la fonction $q\mapsto q_i$. Le gradient de cette fonction est donné par
\begin{equation}
    \nabla q_i=\sum_j\frac{1}{ h_j }\frac{ \partial q_i }{ \partial q_j }e_j,
\end{equation}
mais $\frac{ \partial q_i }{ \partial q_j }=\delta_{ij}$, donc
\begin{equation}
    \nabla q_i=\frac{ e_i }{ h_i },
\end{equation}
ou encore
\begin{equation}
    e_i=h_i\nabla q_i.
\end{equation}
Cela n'est pas étonnant : la direction dans laquelle la coordonnées $q_i$ varie le plus est le vecteur $e_i$ qui donne la tangente à la courbe obtenue lorsque \emph{seul} $q_i$ varie.

Commençons par calculer la divergence de $e_i$. En utilisant la formule \eqref{EqekeitimesejOM},
\begin{equation}
    \nabla\cdot e_k=\frac{ 1 }{2}\sum_{ij}\epsilon_{ijk}\,\nabla\cdot (e_i\times e_j).
\end{equation}
Nous avons, en utilisant les règles de Leibnitz \eqref{EqLeinDivNablRotOM}, 
\begin{equation}
    \begin{aligned}[]
        \nabla\cdot(e_i\times e_j)&=\nabla\cdot(h_i\nabla q_i\times h_j\nabla q_j)\\
        &=\nabla(h_ih_j)\cdot\big( \nabla q_i\times\nabla q_j \big)+h_ih_j\nabla\cdot\big( \nabla q_i\times\nabla q_j \big)\\
        &=\nabla(h_ih_j)\cdot\big( \nabla q_i\times\nabla q_j \big)\\
        &\quad+h_ih_j\nabla q_j\cdot\big( \underbrace{\nabla\times\nabla q_i}_{=0} \big)\\
        &\quad+h_ih_j\nabla q_i\cdot\big( \underbrace{\nabla\times\nabla q_j}_{=0} \big)
    \end{aligned}
\end{equation}
Cela nous fait
\begin{equation}
    \nabla\cdot e_k=\sum_{ij}\epsilon_{ijk}\frac{ \nabla(h_ih_j) }{ h_ih_j }\cdot (e_i\times e_j).
\end{equation}
parce que $\nabla q_i=h_i^{-1}e_i$. Nous pouvons développer le gradient qui intervient :
\begin{equation}
    \nabla(h_ih_j)=\sum_l\frac{1}{ h_l }\frac{ \partial  }{ \partial q_l }(h_ih_j)e_l.
\end{equation}
Nous voyons donc arriver le produit mixte $e_l\cdot (e_i\times e_j)$. En utilisant la formule \eqref{EqProdMixteepsilonCicivrOM}, cela s'exprime directement sous la forme $\epsilon_{ijl}$.

Nous avons alors
\begin{equation}        \label{EqFragradekdviOM}
    \begin{aligned}[]
        \nabla\cdot e_k&=\frac{ 1 }{2}\sum_{ijl}\frac{1}{ h_ih_jh_l }\frac{ \partial  }{ \partial q_l }(h_ih_j)\epsilon_{ijk}\epsilon_{ijl}\\
        &=\frac{ 1 }{2}\sum_{ijl}\delta_{kl}| \epsilon_{ijk} |\frac{ \partial  }{ \partial q_l }(h_ih_j)\\
        &=\frac{ 1 }{2}\sum_{ij}\frac{| \epsilon_{ijk} |}{ h_ih_jh_k }\frac{ \partial  }{ \partial q_k }(h_ih_j).
    \end{aligned}
\end{equation}
Par exemple,
\begin{equation}
    \nabla\cdot e_1=\frac{1}{ h_1h_2h_3 }\frac{ \partial  }{ \partial q_1 }(h_2h_3).
\end{equation}

Nous devons maintenant chercher le gradient d'un champ général
\begin{equation}
    F(q)=\sum_kF_k(q)e_k.
\end{equation}
La première chose à faire est d'utiliser la formule de Leibnitz :
\begin{equation}        \label{EqLeibnbablaFekOM}
    \nabla\cdot F=\sum_k\nabla F_k(q)\cdot e_k+\sum_kF_k(q)\nabla\cdot e_k.
\end{equation}
Afin d'alléger les notations, nous allons nous concentrer sur le terme numéro $k$ et ne pas écrire la somme. Si $i$ et $j$ sont les nombres tels que $\epsilon_{ijk}=1$, alors ce que la formule \eqref{EqFragradekdviOM} signifie, c'est que
\begin{equation}
    \nabla\cdot e_k=\frac{1}{ h_1h_2h_3 }\frac{ \partial  }{ \partial q_k }(h_ih_j).
\end{equation}
Nous savons déjà par la formule \eqref{EqGradientenCurviligneOM} que
\begin{equation}
    \nabla F_k=\sum_l\frac{1}{ h_l }\frac{ \partial F_k }{ \partial q_l }e_l,
\end{equation}
par conséquent
\begin{equation}
    \nabla F_k\cdot e_k=\sum_l\frac{1}{ h_l }\frac{ \partial F_k }{ \partial q_l }\delta_{kl}=\frac{1}{ h_k }\frac{ \partial F_k }{ \partial q_k }.
\end{equation}
Pour obtenir cela nous avons utilisé le fait que $e_l\cdot e_k=\delta_{lk}$. Le terme numéro $k$ de la somme \eqref{EqLeibnbablaFekOM} est donc
\begin{equation}
    \frac{1}{ h_k }\frac{ \partial F_k }{ \partial q_k }+\frac{ F_k }{ h_kh_ih_j }\frac{ \partial (h_ih_j) }{ \partial q_k }=\frac{1}{ h_ih_jh_k }\frac{ \partial (F_kh_ih_j) }{ \partial q_k }
\end{equation}
où il est entendu que $i$ et $j$ représentent les nombres tels que $\epsilon_{ijk}=1$.

Au final, nous avons
\begin{equation}
    \nabla\cdot F=\frac{1}{ h_1h_2h_3 }\sum_{ijk}| \epsilon_{ijk} |\frac{ \partial (F_kh_ih_j) }{ \partial q_k }.
\end{equation}
Ici, la somme sur $i$ et $j$ consiste seulement à sélectionner les termes tels que $i$ et $j$ ne sont pas $k$. En écrivant la somme explicitement,
\begin{equation}
    \begin{aligned}[]
        \nabla\cdot F=\frac{1}{ h_1h_2h_3 }\left[ \frac{ \partial  }{ \partial q_1 }(F_1h_2h_3)+\frac{ \partial  }{ \partial q_2 }(F_2h_1h_3)+\frac{ \partial  }{ \partial q_3 }(F_3h_1h_2) \right].
    \end{aligned}
\end{equation}

%---------------------------------------------------------------------------------------------------------------------------
\subsection{Coordonnées cylindriques}
%---------------------------------------------------------------------------------------------------------------------------

En coordonnées cylindriques, nous avons déjà vu que $h_r=1$, $h_{\theta}=r$ et $h_z=1$. La divergence est donc donnée par
\begin{equation}        \label{EqDivEnCylonfOM}
    \nabla\cdot F=\frac{1}{ r }\left[ \frac{ \partial  }{ \partial r }(rF_r)+\frac{ \partial  }{ \partial \theta }(F_{\theta})+\frac{ \partial  }{ \partial z }(rF_z) \right].
\end{equation}
Par exemple si
\begin{equation}
    F(r,\theta,z)=re_{\theta}+e_z,
\end{equation}
nous avons
\begin{equation}
    (\nabla\cdot F)(r,\theta,z)=\frac{1}{ r }\left[ \frac{ \partial  }{ \partial \theta }(r)+\frac{ \partial  }{ \partial z }(r) \right]=0.
\end{equation}
Cela est logique parce que $re_{\theta}$ est à peu près le champ dont nous avons parlé dans l'exemple \eqref{ExamDivFrotOM}, qui était à divergence nulle. En réalité, le champ dont on parlait dans cet exemple était exactement $-e_{\theta}$. Le champ $e_z$ est également à divergence nulle parce qu'il est constant.

%---------------------------------------------------------------------------------------------------------------------------
\subsection{Coordonnées sphériques}
%---------------------------------------------------------------------------------------------------------------------------

En coordonnées sphériques, nous avons $h_{\rho}=1$, $h_{\theta}=r$ et $h_{\varphi}=r\sin\theta$, donc
\begin{equation}
    \nabla\cdot F=\frac{1}{ r^2\sin\theta }\left[ \frac{ \partial  }{ \partial \rho }(\rho^2\sin\theta F_{\rho})+\frac{ \partial  }{ \partial \theta }(\rho\sin\theta F_{\theta})+\frac{ \partial  }{ \partial \varphi }(\rho F_{\varphi}) \right].
\end{equation}
si $F(\rho,\theta,\varphi)=F_{\rho}e_{\rho}+F_{\theta}e_{\theta}+F_{\varphi}e_{\varphi}$.

%+++++++++++++++++++++++++++++++++++++++++++++++++++++++++++++++++++++++++++++++++++++++++++++++++++++++++++++++++++++++++++
\section{Laplacien en coordonnées curvilignes orthogonales}
%+++++++++++++++++++++++++++++++++++++++++++++++++++++++++++++++++++++++++++++++++++++++++++++++++++++++++++++++++++++++++++

Soit une fonction $f\colon \eR^3\to \eR$. Le \defe{Laplacien}{Laplacien} de $f$ est donné par
\begin{equation}
    \Delta f=\nabla\cdot(\nabla f).
\end{equation}
En utilisant les formules données, nous avons
\begin{equation}
    \Delta f=\frac{1}{ h_1h_2h_3 }\left[ \frac{ \partial  }{ \partial q_1 }\left( \frac{ h_2h_3 }{ h_1 }\frac{ \partial f }{ \partial q_1 } \right)  +\frac{ \partial  }{ \partial q_2 }\left( \frac{ h_1h_3 }{ h_2 }\frac{ \partial f }{ \partial q_2 } \right)  +\frac{ \partial  }{ \partial q_3 }\left( \frac{ h_1h_2 }{ h_3 }\frac{ \partial f }{ \partial q_3 } \right)     \right].
\end{equation}
Dans cette expression, la fonction $f$ est donnée comme fonction de $q_1$, $q_2$ et $q_3$.

En coordonnées cylindriques, cela s'écrit
\begin{equation}
    \begin{aligned}[]
        \Delta f&=\frac{1}{ r }\left[ \frac{ \partial  }{ \partial r }\left( r\frac{ \partial f }{ \partial r } \right)+\frac{ \partial  }{ \partial \theta }\left( \frac{1}{ r }\frac{ \partial f }{ \partial \theta } \right)+\frac{ \partial  }{ \partial z }\left( r\frac{ \partial f }{ \partial z } \right) \right]\\
        &=\frac{ \partial^2f  }{ \partial r^2 }+\frac{1}{ r }\frac{ \partial f }{ \partial r }+\frac{1}{ r^2 }\frac{ \partial^2f }{ \partial \theta^2 }+\frac{ \partial^2f }{ \partial z^2 }.
    \end{aligned}
\end{equation}
Dans cette expression, $f$ est fonction de $r$, $\theta$ et $z$.

En coordonnées sphériques, cela devient
\begin{equation}        \label{EqLaplaceSpheOM}
    \Delta f=\frac{1}{ \rho^2\sin\theta }\left[ \frac{ \partial  }{ \partial \rho }\left( \rho^2\sin\theta\frac{ \partial f }{ \partial \rho } \right)+\frac{ \partial  }{ \partial \theta }\left( \sin\theta\frac{ \partial f }{ \partial \theta } \right)+\frac{ \partial  }{ \partial \varphi }\left( \frac{1}{ \sin\theta }\frac{ \partial f }{ \partial \varphi } \right) \right].
\end{equation}
Dans cette expression, $f$ est fonction de $\rho$, $\theta$ et $\varphi$.

%+++++++++++++++++++++++++++++++++++++++++++++++++++++++++++++++++++++++++++++++++++++++++++++++++++++++++++++++++++++++++++
\section{Rotationnel en coordonnées curvilignes orthogonales}
%+++++++++++++++++++++++++++++++++++++++++++++++++++++++++++++++++++++++++++++++++++++++++++++++++++++++++++++++++++++++++++

Nous voulons calculer le rotationnel de $F(q)=\sum_kF_k(q)e_k$. Pour cela nous commençons par écrire $e_k=h_k\nabla q_k$ et nous utilisons la formule \eqref{EqLeinRotfFFOM} avec $F_kh_k$ en guise de $f$ :
\begin{equation}
    \begin{aligned}[]
        \nabla\times F_ke_k&=\nabla\times(F_kh_k\nabla q_k)\\
        &=F_kh_k\underbrace{\nabla\times(\nabla q_k)}_{=0}+\nabla(F_kh_k)\times\nabla q_k\\
        &=\frac{1}{ h_k }\nabla(F_kh_k)\times e_k.
    \end{aligned}
\end{equation}
Nous utilisons à présent la formule \eqref{EqGradientenCurviligneOM} du gradient et le formule $e_j\times e_k=\sum_l\epsilon_{jkl}e_l$ :
\begin{equation}
    \begin{aligned}[]
        \nabla\times(F_ke_k)&=\sum_{j}\frac{1}{ h_jh_k }\frac{ \partial  }{ \partial q_j }(F_kh_k)e_j\times e_k\\
        &=\sum_{jl}\frac{1}{ h_jh_k }\epsilon_{jkl}\frac{ \partial  }{ \partial q_j }(F_kh_k)e_l.
    \end{aligned}
\end{equation}
Le rotationnel s'écrit donc
\begin{equation}
    \nabla\times F=\sum_{jkl}\frac{1}{ h_jh_k }\epsilon_{jkl}\frac{ \partial  }{ \partial q_j }(F_kh_k)e_l.
\end{equation}
Devant $e_1$ par exemple nous avons seulement les termes $j=2$, $k=3$ et $j=3$, $k=2$. Étant donné que $\epsilon_{231}=1$ et $\epsilon_{321}=-1$, le coefficient de $e_1$ sera simplement
\begin{equation}
    \frac{1}{ h_2h_3 }\left( \frac{ \partial  }{ \partial q_2 }(F_3h_3)-\frac{ \partial  }{ \partial q_3 }(F_2h_2) \right).
\end{equation}
La formule complète devient
\begin{equation}
    \begin{aligned}[]
        \nabla\times\sum_k F_ke_k&=\frac{1}{ h_2h_3 }\left( \frac{ \partial  }{ \partial q_2 }(F_3h_3)-\frac{ \partial  }{ \partial q_3 }(F_2h_2) \right)\\
            &\quad+\frac{1}{ h_1h_3 }\left( \frac{ \partial  }{ \partial q_3 }(F_1h_1)-\frac{ \partial  }{ \partial q_1 }(F_3h_3) \right)\\
            &\quad+\frac{1}{ h_2h_1 }\left( \frac{ \partial  }{ \partial q_1 }(F_2h_2)-\frac{ \partial  }{ \partial q_2 }(F_1h_1) \right).  
    \end{aligned} 
\end{equation} 

%---------------------------------------------------------------------------------------------------------------------------
\subsection{Coordonnées cylindriques}
%---------------------------------------------------------------------------------------------------------------------------

En utilisant $h_r=1$, $h_{\theta}=r$ et $h_z=1$, nous trouvons
\begin{equation}        \label{EqRotationnelCylinOM}
    \begin{aligned}[]
        \nabla\times(F_re_r+F_{\theta}e_{\theta}+F_ze_z)&=\frac{1}{ r }\left( \frac{ \partial F_z }{ \partial \theta }-\frac{ \partial (F_{\theta}r) }{ \partial z } \right)e_r\\
        &\quad+\left( \frac{ \partial F_r }{ \partial z }-\frac{ \partial F_z }{ \partial r } \right)e_{\theta}\\
        &\quad+\left( \frac{ \partial (F_{\theta} r) }{ \partial r }-\frac{ \partial F_r }{ \partial \theta } \right)e_z.
    \end{aligned}
\end{equation}

%---------------------------------------------------------------------------------------------------------------------------
\subsection{Coordonnées sphériques}
%---------------------------------------------------------------------------------------------------------------------------

En utilisant $h_{\rho}=1$, $h_{\theta}=\rho$ et $h_{\varphi}=\rho\sin\theta$, nous trouvons
\begin{equation}
    \begin{aligned}[]
        \nabla\times(F_{\rho}e_{\rho}+F_{\theta}e_{\theta}+F_{\varphi}e_{\varphi})&=\frac{1}{ \rho\sin\theta }\left( \frac{ \partial (F_{\varphi})\sin\theta }{ \partial \theta }-\frac{ \partial F_{\theta} }{ \partial \varphi } \right)e_{\rho}\\
        &\quad+\frac{1}{ \rho\sin\theta }\left( \frac{ \partial F_{\rho} }{ \partial \varphi }-\frac{ \partial (F_{\varphi}\rho\sin\theta) }{ \partial \rho } \right)e_{\theta}\\
        &\quad+\frac{1}{ \rho }\left( \frac{ \partial F_{\theta}\rho }{ \partial \rho }-\frac{ \partial F_r }{ \partial \theta } \right)e_{\varphi}.
    \end{aligned}
\end{equation}
Note : dans le premier terme, il y a une simplification par $\rho$.

%+++++++++++++++++++++++++++++++++++++++++++++++++++++++++++++++++++++++++++++++++++++++++++++++++++++++++++++++++++++++++++
\section{Les formules}
%+++++++++++++++++++++++++++++++++++++++++++++++++++++++++++++++++++++++++++++++++++++++++++++++++++++++++++++++++++++++++++

%---------------------------------------------------------------------------------------------------------------------------
\subsection{Coordonnées polaires}
%---------------------------------------------------------------------------------------------------------------------------

Les vecteurs de base :
\begin{subequations}
    \begin{align}
    e_r=\begin{pmatrix}
        \cos(\theta)    \\ 
        \sin(\theta)    
    \end{pmatrix}=\cos(\theta)e_x+\sin(\theta)e_y\\
    e_{\theta}=\begin{pmatrix}
        -\sin(\theta)    \\ 
        \cos(\theta)    
    \end{pmatrix}=-\sin(\theta)e_x+\cos(\theta)e_y.
    \end{align}
\end{subequations}

Le gradient :
\begin{equation}
    \nabla\tilde f(r,\theta)=\frac{ \partial \tilde f }{ \partial r }(r,\theta)e_r+\frac{1}{ r }\frac{ \partial \tilde f }{ \partial \theta }(r,\theta)e_{\theta}.
\end{equation}

La divergence :
\begin{equation}    \label{EqgRxJKdOM}
    \nabla\cdot F=\frac{1}{ r }\left[ \frac{ \partial  }{ \partial r }(rF_r)+\frac{ \partial  }{ \partial \theta }(F_{\theta}) \right].
\end{equation}

Le rotationnel :
\begin{equation}    \label{EqtBnoCwOM}
    \nabla\times(F_re_r+F_{\theta}e_{\theta})=\left( \frac{ \partial (F_{\theta} r) }{ \partial r }-\frac{ \partial F_r }{ \partial \theta } \right)e_z.
\end{equation}
Notons que le rotationnel n'existe pas vraiment en deux dimensions. Ici nous avons vu le champ \( F(r,\theta)\) comme un champs dans \( \eR^3\) ne dépendant pas de \( z\) et n'ayant pas de composante \( z\). Le résultat est un rotationnel qui est dirigé selon l'axe \( z\).


%---------------------------------------------------------------------------------------------------------------------------
\subsection{Coordonnées cylindriques}
%---------------------------------------------------------------------------------------------------------------------------

Les vecteurs de base : idem qu'en coordonnées polaires, et on ajoute \( e_z\) sans modifications.

Le gradient :
\begin{equation}
    \nabla\tilde f(r,\theta,z)=\frac{ \partial \tilde f }{ \partial r }(r,\theta,z)e_r+\frac{1}{ r }\frac{ \partial \tilde f }{ \partial \theta }(r,\theta,z)e_{\theta}+\frac{ \partial \tilde f }{ \partial z }(r,\theta,z)e_z.
\end{equation}

La divergence :
\begin{equation} 
    \nabla\cdot F=\frac{1}{ r }\left[ \frac{ \partial  }{ \partial r }(rF_r)+\frac{ \partial  }{ \partial \theta }(F_{\theta})+\frac{ \partial  }{ \partial z }(rF_z) \right].
\end{equation}

Le rotationnel :
\begin{equation}    
    \begin{aligned}[]
        \nabla\times(F_re_r+F_{\theta}e_{\theta}+F_ze_z)&=\frac{1}{ r }\left( \frac{ \partial F_z }{ \partial \theta }-\frac{ \partial (F_{\theta}r) }{ \partial z } \right)e_r\\
        &\quad+\left( \frac{ \partial F_r }{ \partial z }-\frac{ \partial F_z }{ \partial r } \right)e_{\theta}\\
        &\quad+\left( \frac{ \partial (F_{\theta} r) }{ \partial r }-\frac{ \partial F_r }{ \partial \theta } \right)e_z.
    \end{aligned}
\end{equation}

Note : les formules concernant les coordonnées polaires se réduisent de celles-ci en enlevant toutes les références à \( z\).

%---------------------------------------------------------------------------------------------------------------------------
\subsection{Coordonnées sphériques}
%---------------------------------------------------------------------------------------------------------------------------

Les vecteurs de base :
\begin{equation}
    \begin{aligned}[]
        e_r&=\begin{pmatrix}
        \sin(\theta)\cos(\varphi)    \\ 
        \sin(\theta)\sin(\varphi)    \\ 
        \cos(\theta)
    \end{pmatrix},
    &e_{\theta}&=
    \begin{pmatrix}
        \cos(\theta)\cos(\varphi)    \\ 
        \cos(\theta)\sin(\varphi)    \\ 
        -\sin(\theta)
    \end{pmatrix},\\
    e_{\varphi}&=
    \begin{pmatrix}
        -\sin(\varphi)    \\ 
        \cos(\varphi)    \\ 
        0
    \end{pmatrix}
    \end{aligned}
\end{equation}


Le gradient :
\begin{equation}
    \nabla\tilde f(\rho,\theta,\varphi)=\frac{ \partial \tilde f }{ \partial \rho }e_{\rho}+\frac{1}{ \rho }\frac{ \partial \tilde f }{ \partial \theta }e_{\theta}+\frac{1}{ \rho\sin(\theta) }\frac{ \partial \tilde f }{ \partial \varphi }r_{\varphi}.
\end{equation}


La divergence :
\begin{equation}
    \nabla\cdot F=\frac{1}{ \rho^2\sin\theta }\left[ \frac{ \partial  }{ \partial \rho }(\rho^2\sin\theta F_{\rho})+\frac{ \partial  }{ \partial \theta }(\rho\sin\theta F_{\theta})+\frac{ \partial  }{ \partial \varphi }(\rho F_{\varphi}) \right].
\end{equation}

Le rotationnel :
\begin{equation}
    \begin{aligned}[]
        \nabla\times(F_{\rho}e_{\rho}+F_{\theta}e_{\theta}+F_{\varphi}e_{\varphi})&=\frac{1}{ \rho\sin\theta }\left( \frac{ \partial (F_{\varphi})\sin\theta }{ \partial \theta }-\frac{ \partial F_{\theta} }{ \partial \varphi } \right)e_{\rho}\\
        &\quad+\frac{1}{ \rho\sin\theta }\left( \frac{ \partial F_{\rho} }{ \partial \varphi }-\frac{ \partial (F_{\varphi}\rho\sin\theta) }{ \partial \rho } \right)e_{\theta}\\
        &\quad+\frac{1}{ \rho }\left( \frac{ \partial F_{\theta}\rho }{ \partial \rho }-\frac{ \partial F_r }{ \partial \theta } \right)e_{\varphi}.
    \end{aligned}
\end{equation}
