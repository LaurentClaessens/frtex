% This is part of the Exercices et corrigés de mathématique générale.
% Copyright (C) 2009
%   Laurent Claessens
% See the file fdl-1.3.txt for copying conditions.
\begin{exercice}\label{exoLineraire0027}

	Dans l'espace vectoriel $F$ des fonction de $\eR$ dans $\eR$, on considère le sous-ensemble $L$ des fonctions $f$ de la forme
	\begin{equation}
		f(x)=A(x)\cos(x)+B(x)\sin(x)
	\end{equation}
	où $A(x)$ et $B(x)$ sont deux polynômes en $x$ de degré $\leq 1$.
	\begin{enumerate}

		\item
			Montrer que $L$ est un sous-espace de $F$,
		\item
			Montrer que les fonctions $f_1(x)=\cos(x)$, $f_2(x)=\sin(x)$, $f_3(x)=x\cos(x)$, $f_4(x)=x\sin(x)$ est une base de $L$.

	\end{enumerate}

\corrref{Lineraire0027}
\end{exercice}
