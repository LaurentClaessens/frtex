\begin{corrige}{devoir1-0006}
  \begin{enumerate}
  \item La limite double $\displaystyle \lim_{y\to 0}\lim_{x\to 0} f(x,y)$ est nulle parce que  
    \begin{equation}
      \lim_{x\to 0} f(x,y)=\lim_{x\to 0}\frac{x^3y}{x^6+3y^2}=0,
    \end{equation}
    et $\lim_{y\to 0} 0=0$. De même $\displaystyle \lim_{x\to 0}\lim_{y\to 0} f(x,y)$ est zéro parce que 
    \begin{equation}
      \lim_{y\to 0} f(x,y)=\lim_{y\to 0}\frac{x^3y}{x^6+3y^2}=0,
    \end{equation}
    et $\lim_{x\to 0} 0=0$.
  \item 
    \begin{equation}
       \lim_{t\to 0}f(t,at)=\lim_{t\to 0}\frac{at^4}{t^6+3a^2t^2},
    \end{equation}
    le terme dominant  au dénominateur lorsque $t \to 0$ est $3a^2t^2$, qui tend vers zéro moins vite que $t^4$. La valeur de la limite est donc zéro.

  \item
      Pour montrer cela on calcule la limite le long de la courbe $t\mapsto (t, t^3)$:
      \begin{equation}
        \lim_{t\to 0} f(t,t^3)= 1/4\neq 0.
      \end{equation}
     L'existence de la limite serait en contradiction avec le théorème d'unicité de la limite. Par conséquent la limite n'existe pas. 

  \end{enumerate}
\end{corrige}
