% This is part of Exercices et corrigés de CdI-1
% Copyright (c) 2011
%   Laurent Claessens
% See the file fdl-1.3.txt for copying conditions.

\begin{corrige}{OutilsMath-0088}

    Les dérivées partielles de $f$ se calculent :
    \begin{equation}
        \begin{aligned}[]
            \frac{ \partial f }{ \partial x }&=y\cos\big( y\ln(2x) \big)+\sin\big( y\ln(2x) \big)\\
            \frac{ \partial f }{ \partial y }&=x \ln(2x)\cos\big( y\ln(2x) \big).
        \end{aligned}
    \end{equation}
    Au point $(e,0)$ nous avons
    \begin{equation}
        \begin{aligned}[]
            \frac{ \partial f }{ \partial x }(e,0)&=0\\
            \frac{ \partial f }{ \partial y }(e,0)&=e\ln(2e).
        \end{aligned}
    \end{equation}
    Le plan a donc pour équation
    \begin{equation}
        T_{(e,0)}(x,y)=ye\ln(2e)
    \end{equation}
    parce que $f(e,0)=0$.

    Pour trouver trois points dans ce plan, il suffit d'écrire $\big( x,y,T_{(e,0)}(x,y)\big)$ pour n'importe quel choix de $x$ et $y$. Par exemple $(0,0,0)$, $(0,1,e\ln(2e))$ et $(1,0,0)$.

\end{corrige}
