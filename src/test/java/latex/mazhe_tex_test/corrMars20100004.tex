% This is part of the Exercices et corrigés de mathématique générale.
% Copyright (C) 2009-2010
%   Laurent Claessens
% See the file fdl-1.3.txt for copying conditions.


\begin{corrige}{Mars20100004}

	Vu qu'on est dans $\eR^3$, l'espace $E$ est au maximum de dimension $3$. Il est directement évident que les deux premiers sont linéairement indépendants. Donc $\dim(E)\geq 2$. La question est de savoir si il est de dimension $2$ ou $3$.

	Nommons $v_1$, $v_2$, $v_3$ et $v_4$ les quatre vecteurs proposés.

	Il y a de nombreuses façons de procéder. Une façon est de considérer la matrice formée par les $4$ vecteurs proposés et de voir, en l'échelonnant, quelle est la matrice carré de taille maximum qu'on peut trouver dedans. En partant de
	\begin{equation}
		\begin{pmatrix}
			 1	&	0	&	3	&	2	\\
			 3	&	2	&	7	&	2	\\ 
			 2	&	-1	&	7	&	6
		 \end{pmatrix},
	\end{equation}
	on peut arriver à
	\begin{equation}
		\begin{pmatrix}
			 1	&	0	&	3	&	2	\\
			 1	&	0	&	3	&	2	\\ 
			 2	&	-1	&	7	&	6
		 \end{pmatrix}.
	\end{equation}
	Dans cette matrice, c'est tout de suite visible qu'il n'y a aucune matrice de taille $3$ de déterminant non nul parce que toute la première ligne est égale à la seconde.

	Une autre façon de procéder (plus piétonne) est de vérifier si les vecteurs $v_3$ et/ou $v_4$ sont des combinaisons de $v_1$ et $v_2$. Si ils sont de telles combinaisons, alors les deux premiers génèrent tout $E$.  Si en revanche $v_3$ n'est pas une combinaison de $v_1$ et $v_2$, alors ce sont ces trois vecteurs $v_1$, $v_2$ et $v_3$ qui généreront tout $E$.

	Vérifier si trois vecteurs dans $\eR^3$ sont linéairement indépendants revient à calculer le déterminant de la matrice formée par ces trois vecteurs. Dans notre cas, on doit vérifier les deux déterminants
	\begin{equation}
		\det\begin{pmatrix}
			1	&	0	&	3	\\
			3	&	2	&	7	\\
			2	&	-1	&	7
		\end{pmatrix}
	\end{equation}
	et
	\begin{equation}
		\det\begin{pmatrix}
			1	&	0	&	2	\\
			3	&	2	&	2	\\
			2	&	-1	&	6
		\end{pmatrix}.
	\end{equation}
	Le calcul montre que ces deux déterminants sont nuls, donc chacun des deux vecteurs $v_3$ et $v_4$ n'est combinaison linéaire de $v_1$ et $v_2$. 

	Une base de $E$ est donc donnée par $\{ v_1,v_2 \}$. La dimension est donc $2$.
	
	Pour trouver une base orthogonale, on utilise Gram-Schmidt. Nous posons $f_1=v_1$, et puis
	\begin{equation}
		f_2=v_2-\frac{ \langle v_2, f_1\rangle  }{ \langle f_1, f_1\rangle  }f_1=\begin{pmatrix}
			0	\\ 
			2	\\ 
			-1	
		\end{pmatrix}-\frac{ 4 }{ 14 }\begin{pmatrix}
			1	\\ 
			3	\\ 
			2	
		\end{pmatrix}=\frac{1}{ 7 }\begin{pmatrix}
			-2	\\ 
			8	\\ 
			-11	
		\end{pmatrix}.
	\end{equation}
	Une base orthogonale de $E$ est donc donnée par
	\begin{equation}
		\begin{aligned}[]
			f_1&=\begin{pmatrix}
				1	\\ 
				3	\\ 
				2	
			\end{pmatrix},&\text{et}&
			&f_2&=\begin{pmatrix}
				-2	\\ 
				8	\\ 
				-11	
			\end{pmatrix}.
		\end{aligned}
	\end{equation}
	Vérification : $\langle f_1, f_2\rangle =-2+24-22=0$. Notez que j'ai «oublié» le $\frac{1}{ 7 }$ dans $f_2$ parce que de toutes façons, il n'est pas encore normé.

	Une base orthonormale se trouve en divisant par la norme :
	\begin{equation}
		\begin{aligned}[]
			g_1&=\frac{1}{ \sqrt{14} }\begin{pmatrix}
				1	\\ 
				3	\\ 
				2	
			\end{pmatrix},&\text{et}&
			&g_2&=\frac{1}{ 3\sqrt{21} }\begin{pmatrix}
				-2	\\ 
				8	\\ 
				-11	
			\end{pmatrix}.
		\end{aligned}
	\end{equation}


	

\end{corrige}
