% This is part of Exercices et corrigés de CdI-1
% Copyright (c) 2011
%   Laurent Claessens
% See the file fdl-1.3.txt for copying conditions.

\begin{corrige}{OutilsMath-0081}

    Le passage en cylindriques se fait en remplaçant $x^2+y^2$ par $r$ :
    \begin{equation}
        \tilde f(r,\theta,z)=\frac{ z }{ r }.
    \end{equation}
    Le gradient est alors donné par
    \begin{equation}
        \nabla \tilde f=-\frac{ z }{ r }e_r+\frac{1}{ r }e_z.
    \end{equation}
    Le rotationnel de ce gradient est nul parce que le rotationnel d'un gradient est toujours nul.

\end{corrige}
