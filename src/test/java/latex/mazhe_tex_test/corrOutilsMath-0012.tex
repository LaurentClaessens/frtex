% This is part of Exercices et corrigés de CdI-1
% Copyright (c) 2011
%   Laurent Claessens
% See the file fdl-1.3.txt for copying conditions.

\begin{corrige}{OutilsMath-0012}

	Nous avons représenté la situation sur la figure \ref{LabelFigExoPolaire}. Tout d'abord nous trouvons la longueur du segment qui joint l'origine au point $P=(\sqrt{3},1)$. Nous avons $l=\sqrt{ (\sqrt{3})^2+1^2 }=2$.
	\newcommand{\CaptionFigExoPolaire}{Il s'agit de trouver $l$ et $\theta$ en sachant que $x=\sqrt{3}$ et $y=1$.}
	\input{Fig_ExoPolaire.pstricks}
	Nous savons donc déjà que la coordonnée radiale sera $r=2$. Afin de trouver l'angle, nous utilisons par exemple le fait que le sinus de $\theta$ est la hauteur divisée par l'hypoténuse :
	\begin{equation}
		\sin(\theta)=\frac{ 1 }{ 2 }.
	\end{equation}
	Cela nous permet de savoir que $\frac{ \pi }{ 6 }$ ou bien $\frac{ \pi }{ 6 }+\pi$. Nous choisissons $\frac{ \pi }{ 6 }$ parce que l'angle $\frac{ \pi }{ 6 }+\pi$ donnerait des coordonnée négatives.

\end{corrige}
