\begin{exercice}\label{exoCourbesSurfaces0016}

	Des surfaces en coordonnées polaires.
	\begin{enumerate}
		\item	\label{ItemeXS016i}

On considère la sphère dans $ \eR^3$ paramétrée par 
\begin{equation}
\left\{
\begin{array}{l}
x( \theta, \varphi) = \cos \theta \cos \varphi, \\
y( \theta, \varphi) = \sin \theta \cos \varphi, \\
y( \theta, \varphi) = \sin \varphi, 
\end{array}
\right.
\qquad \mbox{ où } \theta \in [0, 2 \pi] \mbox{ et } \varphi \in \left[ - \frac{ \pi}{2}, \frac{ \pi}{2} \right].
\end{equation}
En chaque point de la sphère déterminer deux vecteurs qui engendrent le plan tangent, 
puis déterminer un vecteur normal unitaire à la sphère. 

\item
	Même question pour \defe{l'ellipsoïde}{ellipsoïde} d'équation paramétrique
	\begin{equation}
\left\{
\begin{array}{l}
x( \theta, \varphi) = a \cos \theta \cos \varphi, \\
y( \theta, \varphi) = b \sin \theta \cos \varphi, \\
y( \theta, \varphi) = \sin \varphi, 
\end{array}
\right.
\qquad a, \; b > 0, 
\qquad \mbox{ où } \theta \in [0, 2 \pi] \mbox{ et } \varphi \in \left[ - \frac{ \pi}{2}, \frac{ \pi}{2} \right]
\end{equation}

\item
Même question pour l'hyperboloïde 
\begin{equation}
\left\{
\begin{array}{l}
x( \theta, \varphi) = \cos \theta \cosh \varphi, \\
y( \theta, \varphi) = \sin \theta \cosh \varphi, \\
y( \theta, \varphi) = \sinh \varphi, 
\end{array}
\right.
\qquad \mbox{ où } \theta \in [0, 2 \pi] \mbox{ et } \varphi \in \eR. % \left[ - \frac{ \pi}{2}, \frac{ \pi}{2} \right].
\end{equation}

\item
	On considère «l'hémisphère nord»  paramétrée par $(x, y) \in B_{\eR^2}(0, 1)$, $ z = \sqrt{ 1 - x^2 - y^2}$.  En chaque point de l'hémisphère déterminer deux vecteurs qui engendrent le plan tangent, ainsi qu'un vecteur normal unitaire. Comparer avec le résultat trouvé à la question \ref{ItemeXS016i}. 


	\end{enumerate}

\corrref{CourbesSurfaces0016}
\end{exercice}
			
