\begin{exercice}[\coolexo]\label{exoGeomAnal-0041}

On va montrer que pour $p=\infty$, la norme matricielle $p$ vérifie pour tout $A=(a_{i,j})\in \eM_n(\eR)$ :
\begin{equation}    \label{zzquGAAP}
    \|A\|_{\infty}=\max_{1\le i\le n}\sum_{j=1}^n|a_{ij}|
\end{equation}

\begin{enumerate}
\item

Soit $x\in\eR^n$ tel que $||x||_{\infty}=1$. On pose $b=Ax$ (on a donc $||Ax||_{\infty}=\max_{1\le i\le n}|b_i|$).
\begin{enumerate}
\item Montrer que $|b_i|\le\sum_{j=1}^n|a_{ij}|$ pour tout $i\in\{1,..,n\}$.
\item En déduire que $$||Ax||_{\infty}\le\max_{1\le i\le n}\sum_{j=1}^n|a_{ij}|$$ puis que $$||A||_{\infty}\le\max_{1\le i\le n}\sum_{j=1}^n|a_{ij}|$$
\end{enumerate}

\item

On va montrer l'inégalité inverse. Pour cela soit $i^*$, l'indice correspondant à l'égalité suivante :
\begin{equation}
    \sum_{j=1}^n|a_{i^*j}|=\max_{i\in\{ 1,\ldots,n \}}\sum_{j=1}^n|a_{ij}|.
\end{equation}
L'indice \( i^*\) est donc le numéro de la ligne qui réalise le maximum dont on parle dans le membre de droite. Soit par ailleurs $x^*$ le vecteur de norme $||x||_{\infty}=1$ défini par : 
$$
x^*_j=\left\{
\begin{array}{ll}
	1 & \mbox{si}\ a_{i^*j}\ge 0 \\
	-1 & \mbox{si}\ a_{i^*j}< 0
\end{array}
\right.
$$
\begin{enumerate}
    \item
        Monter que \( a_{i^*j}x^*_j=| a_{i^*j} |\), sans sommation sur \( j\) (ni sur \( i\) ni sur \( i^*\)).
    \item Montrer que $$||Ax^*||_{\infty}\ge\max_{1\le i\le n}\sum_{j=1}|a_{ij}|.$$
\item En déduire le résultat \eqref{zzquGAAP}.
\end{enumerate} 

        
\end{enumerate}

\corrref{GeomAnal-0041}
\end{exercice}
