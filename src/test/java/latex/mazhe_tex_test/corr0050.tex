% This is part of Exercices et corrigés de CdI-1
% Copyright (c) 2011
%   Laurent Claessens
% See the file fdl-1.3.txt for copying conditions.

\begin{corrige}{0050}

La différentielle $df_{(1,1)}$ existe parce que dans un voisinage de ce point, la fonction est parfaitement $ C^{\infty}$. Pour la calculer, nous calculons les dérivées partielles. D'une part
\begin{equation}
	\frac{ \partial f }{ \partial x }(1,1)=\left[ \frac{1}{ y } e^{x/y}\ln(\frac{ x }{ y })+ e^{x/y}\frac{1}{ x/y }\frac{1}{ y } \right]_{(1,1)}=e\ln(1)+e=e.
\end{equation}
De la même manière,
\begin{equation}
	\frac{ \partial f }{ \partial y }(1,1)=-e,
\end{equation}
ce qui nous donne
\begin{equation}
	df_{(1,1)}=e(dx-dy),
\end{equation}
c'est à dire
\begin{equation}
	df_{(1,1)}(u)=e(u_1-u_2).
\end{equation}
Juste pour le plaisir, nous allons calculer les dérivées partielles de $g$ en utilisant la définition, mais vous devriez remarquer qu'il est tout à fait suffisant d'appliquer les règles de calcul composante par composantes, sans vous laisser effrayer par le fait que les variables s'appellent $r$ et $\theta$ au lieu des habituelles $x$ et $y$.
\begin{equation}
	\begin{aligned}[]
		\frac{ \partial g }{ \partial r }(r,\theta)
			&=	\lim_{t\to 0}\frac{ g(r+t,\theta)-g(r,\theta) }{ t }\\
			&=	\lim_{t\to 0}\frac{ \big( (r+t)\cos(\theta),(r+t)\sin(\theta) \big)-\big( r\cos(\theta),r\sin(\theta) \big) }{ t }\\
			&=	\lim_{t\to 0}\frac{ \big( t\cos(\theta),0 \big) }{ t }\\
			&=	\big( \cos(\theta),\sin(\theta) \big).
	\end{aligned}
\end{equation}
En utilisant la même technique, ou les règles usuelles de calcul,
\begin{equation}
	\frac{ \partial g }{ \partial \theta }=\big( -r\sin(\theta),r\cos(\theta) \big).
\end{equation}
Magie ! Remarque que si tu fais le produit scalaire entre $\partial_{\theta}g$ et $g(r,\theta)$, tu obtiens zéro. En effet, $g$ est la fonction qui transforme des coordonnées polaires en cartésiennes. Mais un déplacement infinitésimal de $\theta$ (c'est à dire $\partial_{\theta}g$) correspond à un vecteur tangent au cercle de rayon $r$. Cette tangente est perpendiculaire au rayon.

Nous avons donc, au point demandé :
\begin{equation}		\label{EqDiffPArtg0050}
	\begin{aligned}[]
		\frac{ \partial g }{ \partial r }(\sqrt{2},\frac{ \pi }{ 4 })	&=\left( \frac{1}{ \sqrt{2} },\frac{1}{ \sqrt{2} } \right),\\
		\frac{ \partial g }{ \partial \theta }(\sqrt{2},\frac{ \pi }{ 4 }) 	&=(-1,1).
	\end{aligned}
\end{equation}
En remplaçant $g(r,\theta)$ par sa valeur dans $f$, nous trouvons
\begin{equation}
	\tilde f(r,\theta)= e^{1/\tan(\theta)}\ln\left( \frac{1}{ \tan(\theta) } \right).
\end{equation}
Remarquons que cette fonction ne dépend que de l'angle. Afin de calculer maintenant la différentielle de la composée, analysons la formule donnée dans le cours :
\begin{equation}
	d(f\circ g)(a)=df\big( g(a) \big)\circ dg(a).
\end{equation}
En notant cela avec nos notations plus habituelles,
\begin{equation}
	d(f\circ g)_a(u)=df_{g(a)}\big( dg_a(u) \big).
\end{equation}
Dans notre cas, nous avons que $g(\sqrt{2},\frac{ \pi }{ 4 })=(1,1)$, c'est à dire précisément le point où nous avons déjà calculé la différentielle de $f$. Cool, nous n'avons pas perdu notre temps. Écrite dans le langage des formes, les équations \eqref{EqDiffPArtg0050} signifient ceci :
\begin{equation}
	dg_{(\sqrt{2},\frac{ \pi }{ 4 })}=\begin{pmatrix}
	1/\sqrt{2}	\\ 
	1/\sqrt{2}	
\end{pmatrix}dr+
\begin{pmatrix}
	-1	\\ 
	1	
\end{pmatrix}d\theta.
\end{equation}
En déballant un peu tout,
\begin{equation}
	\begin{aligned}[]
		d(f\circ g)_{(\sqrt{2},\frac{ \pi }{ 4 })}(u_r,u_{\theta})
					&=df_{g (\sqrt{2},\frac{ \pi }{ 4 }) }\left( 
		u_r\begin{pmatrix}
	1/\sqrt{2}	\\ 
	1/\sqrt{2}	
\end{pmatrix} +
u_{\theta}\begin{pmatrix}
	-1	\\ 
	1	
\end{pmatrix}\right)	\\
					&=(edx-edy)\begin{pmatrix}
	\frac{ u_r }{ \sqrt{2} }-u_{\theta}	\\ 
	\frac{ u_r }{ \sqrt{2} }+u_{\theta}	
\end{pmatrix}					\\
					&=e\left( \frac{ u_r }{ \sqrt{2} }-u_{\theta} \right)-e\left( \frac{ u_r }{ \sqrt{2} }+u_{\theta} \right)\\
					&=-2eu_{\theta}.
	\end{aligned}
\end{equation}

\end{corrige}
