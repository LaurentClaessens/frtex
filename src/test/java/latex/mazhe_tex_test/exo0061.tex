% This is part of Exercices et corrigés de CdI-1
% Copyright (c) 2011,2014
%   Laurent Claessens
% See the file fdl-1.3.txt for copying conditions.

\begin{exercice}\label{exo0061}

Soit $f:\eR^2 \rightarrow \eR: (x, y) \rightarrow f(x,y)$ une
fonction de classe $C^2$. On définit le {\bf Laplacien} de $f$ par la
formule suivante:
\[
\Delta f \stackrel{def}= \frac{\partial ^2}{\partial x^2 } f +
\frac{\partial^2}{\partial y^2} f
\]
\begin{enumerate}
\item
Calculez le Laplacien en coordonnées polaires, c'est-\`a-dire, posant
$\tilde{f} (r, \theta) \stackrel{def}= f(r\cos(\theta),r\sin(\theta))$
prouvez que:
\[
(\Delta f)(r\cos(\theta), r\sin(\theta))
= (\frac{\partial^2}{\partial r^2}\tilde{f})(r, \theta)
+ \frac{1}{r} (\frac{\partial}{\partial r}\tilde{f})(r,\theta)
+ \frac{1}{r^2} (\frac{\partial^2}{\partial \theta^2}\tilde{f})(r,\theta)
\]
\item
Vérifiez que la fonction suivante $F:( x,y) \rightarrow \ln(\sqrt{x^2+y^2})
$ est une solution de l'équation aux dérivée partielles $\Delta f =0$
\end{enumerate}
 
\corrref{0061}
\end{exercice}
