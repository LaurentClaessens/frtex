% This is part of Analyse Starter CTU
% Copyright (c) 2014
%   Laurent Claessens,Carlotta Donadello
% See the file fdl-1.3.txt for copying conditions.

\begin{corrige}{session1-0008}

 \begin{enumerate}
  \item  La fonction $f(x) = \cos(2x)$  est paire, donc son développement limité au voisinage de zéro ne contient que les termes d'ordre paire
    \begin{equation*}
      f(x) = f(0) + \frac{x^2}{2} f''(0) + \frac{x^4}{4!} f^{(4)}(0) + \frac{x^6}{6!} f^{(6)}(0) + x^7\varepsilon(x).  
    \end{equation*}
Les dérivées de $f$ sont faciles \`a calculer  : $f^{(n)}(x) = 2^n \cos^{(n)}(2x) $. Par conséquent 
\begin{equation*}
      f(x) = 1 -2x^2  + \frac{2x^4}{3}  - \frac{4x^6}{45} + x^7\varepsilon(x).  
    \end{equation*}
  \item La fonction $g(x) = \cos^2(x)$ est égale \`a $\frac{1 + \cos(2x)}{2}$, donc nous avons 
    \begin{equation*}
      g(x) = \frac{1 + \cos(2x)}{2} = 1 -x^2  + \frac{x^4}{3}  - \frac{2x^6}{45} + x^7\varepsilon(x).
    \end{equation*}
Analogue ment,  $h(x) = \sin^2(x)$ est égale \`a $1-\cos(x)$, donc 
\begin{equation*}
      h(x) = 1-\cos(x) = x^2  - \frac{x^4}{3}  + \frac{2x^6}{45} + x^7\varepsilon(x).
    \end{equation*}
  \item 
    \begin{equation*}
      \lim{x\to 0}\frac{e^x-1-x}{\sin^2(x)} = \lim{x\to 0}\frac{\frac{x^2}{2}}{ x^2} = \frac{1}{2} .
    \end{equation*}
  \end{enumerate}

\end{corrige}
