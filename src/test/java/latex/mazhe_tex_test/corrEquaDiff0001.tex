% This is part of the Exercices et corrigés de mathématique générale.
% Copyright (C) 2009
%   Laurent Claessens
% See the file fdl-1.3.txt for copying conditions.
\begin{corrige}{EquaDiff0001}

\begin{enumerate}

\item
En mettant tous les $y$ à gauche et tous les $x$ à droite, et en utilisant le fait que $y'=dy/dx$, nous trouvons
\begin{equation}
	\begin{aligned}[]
		\frac{ y' }{ y }&=-\frac{ x }{ \sqrt{1-x^2} }\\
		\frac{ \frac{ dy }{ dx } }{ y }&=-\frac{ x }{ \sqrt{1-x^2} }\\
		\frac{ dy }{ y }&=-\frac{ x }{ \sqrt{1-x^2} }dx
	\end{aligned}
\end{equation}
En prenant l'intégrale des deux côtés, nous avons
\begin{equation}		\label{EqDiff001presol}
	\ln(y)=\sqrt{1-x^2}+C,
\end{equation}
donc la solution s'écrit
\begin{equation}
	y(x)= e^{\sqrt{1-x^2}+C}=K e^{\sqrt{1-x^2}},
\end{equation}
si $K=e^C$.

Remarque : étant donné que $K=e^C$, on pourrait croire que la solution n'est valable que pour $K$ positif. En réalité il n'en est rien parce que l'équation \eqref{EqDiff001presol} devrait contenir une valeur absolue de $y$.

\item
L'équation $(1-y^2)dy-ydx$ se récrit sous la forme 
\begin{equation}
	\frac{ 1-y^2 }{ y }dy=dx,
\end{equation}
qui peut être intégrée des deux côtés :
\begin{equation}
	x=\int \frac{ 1-y^2 }{ y }dy=\ln(y)-\frac{ y^2 }{2},
\end{equation}
donc la solution est $2\ln(y)-y^2=2x+C$. 

Note : nous n'avons pas la solution sous la forme $y=y(x)$, mais sous la forme $x=x(y)$\ldots la vie ne peut pas être parfaite !

\end{enumerate}

\end{corrige}
