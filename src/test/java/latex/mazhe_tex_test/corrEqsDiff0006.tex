% This is part of Exercices et corrigés de CdI-1
% Copyright (c) 2011
%   Laurent Claessens
% See the file fdl-1.3.txt for copying conditions.

\begin{corrige}{EqsDiff0006}

\begin{enumerate}

\item
Le polynôme caractéristique est $x^2-2x=0$, donc $x=0$ et $x=2$. La solution générale est donc
\begin{equation}
	y(t)=A+B e^{2t}.
\end{equation}
Les dérivées se calculent aisément :
\begin{equation}
	\begin{aligned}[]
		y'(t)&=2B e^{2t}\\
		y''(t)&=4B e^{2t}.
	\end{aligned}
\end{equation}
Les données de Cauchy demandent de résoudre le système
\begin{subequations}
\begin{numcases}{}
	y(0)=A+B=0\\
	y'(0)=2B=1,
\end{numcases}
\end{subequations}
donc $A=-\frac{ 1 }{2}$ et $B=\frac{ 1 }{2}$.

\item
Polynôme caractéristique : $3x^2+4x+1=0$, solutions : $x=-\frac{1}{ 3 }$ et $x=-1$, donc l'équation différentielle est résolue par
\begin{equation}
	y(t)=A e^{-x/3}+B e^{-x}.
\end{equation}

\item
Cette fois, le polynôme caractéristique a une racine double $x=3$. Le système fondamental de solutions est alors donné par
\begin{equation}
	\begin{aligned}[]
		y_1(t)&= e^{-3t}\\
		y_2(t)&= te^{-3t},
	\end{aligned}
\end{equation}
par la proposition 2 de la page $331$. La donnée de Cauchy est résolue avec
\begin{equation}
	y(t)=Ay_1(t)+By_2(t)
\end{equation}
et $A=\frac{ e^3 }{ 6 }$ et $B=-\frac{ e^3 }{ 6 }$.
	
\item
Le polynôme caractéristique est $x^2-4x+4=0$, qui accepte une racine double en $x=2$, donc la solution générale de l'équation est
\begin{equation}
	y(t)=A e^{2t}+Bt e^{2t}.
\end{equation}

\item
Le polynôme caractéristique est $x^2+x=0$, et ses solutions sont $x=0$ et $x=-1$. La solution générale est donc
\begin{equation}
	y(t)=A+B e^{-t}.
\end{equation}
La donnée de Cauchy se résous par le système
\begin{subequations}
\begin{numcases}{}
	y(3)=A+B e^{-3}=0\\
	y'(3)=-B e^{-3}=1,
\end{numcases}
\end{subequations}
qui a pour solution $A=1$ et $B=- e^{3}$.

\item
Le polynôme caractéristique est donné par $x^2-2x+3=0$, dont les deux solutions complexes conjuguées sont
\begin{equation}
	\begin{aligned}[]
		x_1&=1+\sqrt{2}i\\
		x_2&=1-\sqrt{2}i.
	\end{aligned}
\end{equation}
Un ensemble fondamental de solutions complexes est donné par
\begin{equation}
	\begin{aligned}[]
		y_1(t)=e^{(1+\sqrt{2}i)t}\\
		y_2(t)=e^{(1-\sqrt{2}i)t}
	\end{aligned}
\end{equation}
Les combinaisons linéaires $\frac{ 1 }{2}(y_1+y_2)$ et $\frac{1}{ 2i }(y_1-y_2)$ donnent les solutions réelles :
\begin{equation}
	y(t)=A e^{t}\cos(\sqrt{2}t)+B e^{t}\sin(\sqrt{t}).
\end{equation}
En effet, $ e^{(a+bi)t}= e^{at} e^{it}= e^{at}\big( \cos(bt)+i\sin(bt) \big)$.

\item
Le polynôme caractéristique se factorise en
\begin{equation}
	(x-2)(x-1)^2(x+2),
\end{equation}
donc la solution générale est
\begin{equation}
	y(t)=A e^{2t}+B e^{-t}+Ct e^{-t}+D e^{-2t}.
\end{equation}


\end{enumerate}


\end{corrige}
