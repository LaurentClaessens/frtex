\begin{corrige}{CourbesSurfaces0003}
	\newcommand{\CaptionFigSubfiguresCDUTraceCycloide}{La cycloïde de l'exercice \ref{exoCourbesSurfaces0003}.}
\input{Fig_SubfiguresCDUTraceCycloide.pstricks}

	Si nous supposons une vitesse angulaire $\omega$, la roue avance horizontalement d'une longueur $\omega t$ en un temps $t$. Si le centre du cercle se trouve en $(0,1)$ au départ, cela signifie que le centre de la roue décrit le mouvement $C(t)=(\omega t,1)$. Un point du cercle décrit un cercle autour de ce centre. Si nous suivons le point $P$ situé en $(0,1)$ au départ, nous avons l'équation $P(t)=C(t)+\big( \cos(\omega t),\sin(\omega t) \big)$. En posant $\omega=1$, nous avons
	\begin{subequations}
		\begin{numcases}{}
			x(t)=t+\cos(t)\\
			y(t)=1+\sin(t).
		\end{numcases}
	\end{subequations}
	Il est facile de trouver quelque points. Les dérivées premières et secondes donnent des indices :
	\begin{subequations}
		\begin{numcases}{}
			x'(t)=1-\sin(t)\\
			y'(t)=\cos(t).
		\end{numcases}
	\end{subequations}
	et
	\begin{subequations}
		\begin{numcases}{}
			x''(t)=-\cos(t)\\
			y''(t)=-\sin(t).
		\end{numcases}
	\end{subequations}
	Nous voyons que la fonction $x(t)$ est toujours croissante. La courbe que nous cherchons n'aura donc pas d'auto-intersections. La dérivée seconde donne la concavité; ses valeurs remarquables sont évidement tous les multiples de $\pi/2$.

Nous avons mis sur la figure \ref{LabelFigSubfiguresCDUTraceCycloidessSS1TraceCycloide} la tangente, la dérivée seconde et la normale intérieure en divers points. La courbe est tracée à la figure \ref{LabelFigSubfiguresCDUTraceCycloidessSS2TraceCycloide}.

\end{corrige}
