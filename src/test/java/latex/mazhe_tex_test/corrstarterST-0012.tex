% This is part of Analyse Starter CTU
% Copyright (c) 2014
%   Laurent Claessens,Carlotta Donadello
% See the file fdl-1.3.txt for copying conditions.

\begin{corrige}{starterST-0012}

 \begin{enumerate}
  \item La fonction $\arcsin$ est impaire. Pour le voir il suffit d'utiliser la relation $\arcsin(\sin(x)) =x$, qui est satisfaite pour tout $x$ dans l'intervalle $\displaystyle \left[\frac{-\pi}{2}, \frac{\pi}{2}\right]$. On a donc que pour tout $y$ dans le domaine de $\arcsin$, il existe un $x$ dans  $\displaystyle \left[\frac{-\pi}{2}, \frac{\pi}{2}\right]$ tel que $y=\sin(x)$. On a alors   $\arcsin(-y) = \arcsin(-\sin(x)) = \arcsin(\sin(-x))= -x = -\arcsin(y)$, ce qui revient à dire que $\arcsin$ est impaire.  
  \item La fonction $\arccos$ prend se valeurs sur un intervalle qui n'est pas symétrique par rapport à $0$, par conséquent, $\cos$ n'est ni paire ni impaire. On peut tout de m\^eme essayer d'exprimer $\arccos(-y)$ en fonction de $\arccos(y)$, en se disant que si $y$ est dans le domaine de $\arccos$ alors il existe un $x$ entre $0$ et $\pi$ tel que $y = \cos(x)$. Alors  $\arccos(-y) = \arccos(-\cos(x)) = \arccos (\cos(\pi-x)) = \pi- \arccos(y)$. Dans cette dernière suite d'égalités on a utilisé la formule pour le cosinus d'une différence d'angles. 
  \item Pour tous les $x$ dans l'intervalle $[-1,1]$ on a que par définition $\cos(\arccos(x)) =x$ et que 
 \[
\cos\left(-\arcsin (x) +\dfrac{ \pi}{2}\right) = \cos(\arcsin(x))\cos\left(\dfrac{ \pi}{2}\right)+\sin(\arcsin(x))\sin\left(\dfrac{ \pi}{2}\right) = \sin(\arcsin(x)) = x. 
\]
 Donc $\arccos(x) =  -\arcsin (x) + \dfrac{ \pi}{2}$.
    \end{enumerate}

\end{corrige}
