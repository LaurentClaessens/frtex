\begin{exercice}\label{exoLineraire0016}

	Exercice 6, page 80. Les parties suivantes de $\eR^3$ sont-elles libres ?
	\begin{enumerate}

		\item
			$\{ (1,2,1), (1,1,1) \}$,
		\item
			$\{ (1,0,0),(0,1, 0),(0,0,1) \}$,
		\item
			$\{ (0,0,0),(3,2,1) \}$,
		\item
			$\{ (2,1,-3), (1,4,0), (4,9,-3) \}$,
		\item
			$\{ (1,0,0), (0,1,0),(0,0,1),(0,1,1),(1,0,1), (1,1,0) \}$,
		\item
			$\{ (1,0,1),(0,-8,0),(1,7,-2) \}$,
		\item
			$\{ (-1,0,1) \}$
	\end{enumerate}

\corrref{Lineraire0016}
\end{exercice}
% This is part of the Exercices et corrigés de mathématique générale.
% Copyright (C) 2009
%   Laurent Claessens
% See the file fdl-1.3.txt for copying conditions.
