% This is part of Exercices et corrigés de CdI-1
% Copyright (c) 2011
%   Laurent Claessens
% See the file fdl-1.3.txt for copying conditions.

\begin{corrige}{OutilsMath-0057}

    Les dérivées partielles de $f$ au point $(2,1)$ sont calculables comme d'habitude :
    \begin{verbatim}
----------------------------------------------------------------------
| Sage Version 4.6.1, Release Date: 2011-01-11                       |
| Type notebook() for the GUI, and license() for information.        |
----------------------------------------------------------------------
sage: f(x,y)=x**2-y**2
sage: f(2,1)
3
sage: f.diff(x)(2,1)
4
sage: f.diff(y)(2,1)
-2

    \end{verbatim}
    Nous utilisons la formule du plan tangent
    \begin{equation}
        \begin{aligned}[]
            T_{(2,1)}(x,y)&=f(2,1)+\frac{ \partial f }{ \partial x }(2,1)(x-2)+\frac{ \partial f }{ \partial y }(2,1)(y-1)\\
                &=3+4(x-2)+3(y-1)\\
                &=4x+3y-12.
        \end{aligned}
    \end{equation}
    Pour trouver une vecteur perpendiculaire à ce plan, il suffit de trouver deux vecteurs parallèles à ce plan et puis de trouver une vecteur qui est simultanément perpendiculaire à ces deux vecteurs.

    Le plan passant par l'origine et étant parallèle à notre plan tangent est le plan d'équation
    \begin{equation}
        z=4x+3y.
    \end{equation}
    Il possède par exemple les deux vecteurs suivants:
    \begin{equation}
        \begin{aligned}[]
            v_1&=\begin{pmatrix}
                1    \\ 
                0    \\ 
                4    
            \end{pmatrix}&v_2&=\begin{pmatrix}
                0    \\ 
                1    \\ 
                3    
            \end{pmatrix}.
        \end{aligned}
    \end{equation}
    Un vecteur $(x,y,z)$ simultanément perpendiculaire à $v_1$ et à $v_2$ doit vérifier
    \begin{subequations}
        \begin{numcases}{}
            x+4z=0\\
            y+3z=0.
        \end{numcases}
    \end{subequations}
    Vu qu'il y a moins d'équations que d'inconnues, on peut par exemple choisir $x=1$ et puis les équations fixent $z=-1/4$ et $y=3/4$. Par conséquent un vecteur perpendiculaire au plan est
    \begin{equation}
        \begin{pmatrix}
            1    \\ 
            -1/4    \\ 
            3/4    
        \end{pmatrix}.
    \end{equation}

\end{corrige}
