% This is part of Un soupçon de physique, sans être agressif pour autant
% Copyright (C) 2006-2009
%   Laurent Claessens
% See the file fdl-1.3.txt for copying conditions.


\begin{corrige}{INGE11140027}

	Pour rappel, l'inverse de la fonction $f$ est donnée par
	\begin{equation}
		f^{-1}(y)=\{ x\tq f(x)=y \}.
	\end{equation}
	Cela est une fonction pour les valeurs de $y$ où l'ensemble se réduit à un singleton.

	\begin{enumerate}

		\item
			L'équation à résoudre est
			\begin{equation}
				\sqrt[5]{x+1}=y.
			\end{equation}
			Donc nous avons $f^{-1}(y)=y^5-1$. Notez que cette fonction est définie partout, mais la fonction de départ $f$ ne prend que des valeurs positives. Par conséquent, la fonction $f^{-1}$ ne doit être vue que entre $0$ et $\infty$.

		\item
			Cette fonction a une asymptote verticale en $x=-4$ et prend toutes les valeurs réelles sauf $3$. Son inverse va donc \emph{a priori} être définie sur $\eR\setminus\{ 3 \}$. Résolvons l'équation qui donne $g^{-1}(y)$~:
			\begin{equation}
				y=\frac{ 3x-1 }{ x+4 }
			\end{equation}
			par rapport à $x$. La réponse est que
			\begin{equation}
				g^{-1}(y)=\frac{ 1+4y }{ 3-y }.
			\end{equation}
			Notez la condition d'existence $y\neq 3$. La fonction de départ ne prenant jamais la valeur $3$, cette condition d'existence n'en est pas vraiment une parce que $f^{-1}$ n'était déjà définie que sur $\eR\setminus\{ 3 \}$.	

	\end{enumerate}
	

\end{corrige}
