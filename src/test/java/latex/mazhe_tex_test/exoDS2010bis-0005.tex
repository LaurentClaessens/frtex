% This is part of Exercices de mathématique pour SVT
% Copyright (c) 2010
%   Laurent Claessens et Carlotta Donadello
% See the file fdl-1.3.txt for copying conditions.

\begin{exercice}\label{exoDS2010bis-0005}

Considérons la suite définie par 
\begin{equation}
  u_n= \frac{(-1)^n n}{n+1}, 
\end{equation}
pour $n\geq 1$, $n\in\mathbb{N}$.
 
\begin{enumerate}
\item Calculer la valeur des termes $u_1, u_2, u_3$ et $u_4$.
  \item Démontrer que la suite $(u_n)_{n\in\mathbb{N}^*}$ est bornée.
    \item Est-ce que la suite $(u_n)_{n\in\mathbb{N}^*}$ est monotone ? 
      \item On admet que la suite $(|u_n|)_{n\in\mathbb{N}^*}$ est convergente. Calculer sa limite.
        \item Calculer la valeur des premiers 3 termes des sous-suites $(u_{2n})_{n\in\mathbb{N}^*}$ (sous-suite des indices pairs) et $(u_{2n-1})_{n\in\mathbb{N}^*}$ (sous-suite des indices impairs), pour $n\geq 1$. 
          \item Est-ce que la suite $(u_n)_{n\in\mathbb{N}^*}$ est convergente ? Et les suites $(u_{2n})_{n\in\mathbb{N}^*}$  et $(u_{2n-1})_{n\in\mathbb{N}^*}$ ?
\end{enumerate}

\corrref{DS2010bis-0005}
\end{exercice}
