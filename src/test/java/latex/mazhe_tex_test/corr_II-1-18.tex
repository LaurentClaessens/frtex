% This is part of the Exercices et corrigés de CdI-2.
% Copyright (C) 2008, 2009
%   Laurent Claessens
% See the file fdl-1.3.txt for copying conditions.


\begin{corrige}{_II-1-18}

La solution de ce type d'équation est donnée sous forme paramétrique aux pages II.82 et II.84 du cours.

\begin{enumerate}

	\item
		Nous posons $z(t)=y'(t)$, ce qui implique
		\begin{equation}
			t(z)=\frac{ a+bz^2 }{ z }=f(z).
		\end{equation}
		Cette équation doit être vue comme $t$ donné en terme du paramètre $z$. Le jeu est maintenant de trouver $y$ en fonction du même paramètre $z$. Ainsi nous aurons la courbe $y(t)$ sous forme paramétrique. Afin d'utiliser la formule du cours, nous calculons
		\begin{equation}
			f'(z)=\frac{ bz^2-a }{ z^2 },
		\end{equation}
		et nous posons
		\begin{equation}
			\begin{aligned}[]
				y	&=y_0+\int_{z_0}^z \xi f'(\xi)d\xi +C\\
				&=y_0+\int_{z_0}^z(b\xi+\frac{ a }{ \xi })d\xi	+C\\
				&=y_0+b\left( \frac{ z^2 }{ 2 }-\frac{ z_0^2 }{2} \right)-a\ln| z/z_0 |+C.
			\end{aligned}
		\end{equation}
		Dans cette expression pour $y(z)$, nous pouvons rassembler toutes les constantes arbitraires ($y_0$, $C$ et $z_0$) en une seule~:
		\begin{equation}
			y(z)=\frac{ b }{2}z^2-a\ln(z)+C.
		\end{equation}
		
		Pour le problème de Cauchy $y(t_0)=y_0$, il faut d'abord voir pour quelle valeur du paramètre $z$ nous avons $t(z)=t_0$. Cela se fait en résolvant l'équation
		\begin{equation}
			t_0=\frac{ a+bz_0^2 }{ z_0 }
		\end{equation}
		par rapport à $z_0$. Les solutions sont données par
		\begin{equation}
			z_0=\frac{ t_0\pm\sqrt{t_0^2-4ab} }{ 2b }.
		\end{equation}
		Noter que si $b=0$, alors l'équation différentielle de départ est évidente\footnote{Prouvez-vous qu'elle est évidente en écrianve toutes les solutions !!}, nous supposons donc que $b\neq 0$. 

		Si $a=b=1$, nous avons les possibilités
		\begin{enumerate}

			\item
				Si $| t_0 |>2$, deux solutions,
			\item
				Si $| t_0 |<2$, pas de solutions,
			\item
				Si $| t_0 |=2$, une seule solutions.

		\end{enumerate}
		
		Si par contre $a=-1=b$, alors nous avons
		\begin{equation}
			z_0=\frac{ t_0\pm\sqrt{t_0^2+4} }{ -2 },
		\end{equation}
		et donc toujours deux solutions.

		Dans tous les cas, la solution se trouve sous forme paramétrique
		\begin{subequations}
			\begin{numcases}{}
				t(z)=\frac{ a+bz^2 }{ z },\\
				y(z)=\frac{ b }{2}z^2-a\ln(z)+C.
			\end{numcases}
		\end{subequations}
		Une fois que $z_0$ est trouvé en fonction de $t_0$, il convient de résoudre $y(z_0)=y_0$ pour fixer la constante $C$.

	\item
		En posant $z=y'$, nous écrivons
		\begin{equation}
			y=f(z)=\frac{ a+bz^2 }{ z },
		\end{equation}
		qui doit, encore une fois, être vue comme une équation paramétrique. Pour appliquer la formule du cours, nous calculons
		\begin{equation}
			f'(z)=\frac{ bz^2-a }{ z^2 },
		\end{equation}
		et nous posons
		\begin{equation}
			\begin{aligned}[]
				t	&=t_0+\int_{z_0}^z\frac{ b\xi^2-a }{ \xi^3 }d\xi+C\\
				&=t_0+b\ln(| z/z_0 |)+2a\left( \frac{1}{ z^2 }-\frac{1}{ z_0^2 } \right)+C
			\end{aligned}
		\end{equation}
		Encore une fois, il est important de regrouper toutes les constantes arbitraires ($z_0$, $t_0$ et $C$) en une seule  :
		\begin{equation}
			t(z)=b\ln(z)+\frac{ 2a }{ z^2 }+C.
		\end{equation}
		
		Pour résoudre le problème de Cauchy $y(t_0)=y_0$, nous commençons par trouver pour quelle valeur $z_0$ du paramètre $z$ nous avons $t(z_0)=t_0$. L'équation à résoudre est
		\begin{equation}
			y_0=\frac{ a+bz_0^2 }{ z_0 },
		\end{equation}
		et les solutions sont
		\begin{equation}
			z_0=\frac{ y_0\pm\sqrt{y_0^2-4ab} }{ 2b }.
		\end{equation}
		Si $a=b=1$, nous avons les possibilités
		\begin{enumerate}

			\item
				Si $| y_0 |<2$, alors il y a deux solutions,
			\item
				Si $| y_0 |>2$, alors il n'y a pas de solutions,
			\item
				Si $| y_0 |=2$, alors il y a une seule solution.

		\end{enumerate}
		Si $a=-b=1$, par contre, il y a toujours deux solutions.
		
\end{enumerate}

\begin{alternative}

\begin{enumerate}

\item 
Nous commençons par dilater les données $y=az$ et $y=\sqrt{| ab |}u$, de telle façon à mettre l'équation sous la forme
\begin{equation}		\label{EqII118ufraczz}
	u=\frac{ 1+\epsilon z'^2 }{ z' }
\end{equation}
où $\epsilon=\pm 1$, selon les valeurs de $a$ et $b$.  La solution de l'équation \eqref{EqII118ufraczz} est donné sous forme paramétrique à la page II.82 du cours :
\begin{subequations}
\begin{numcases}{}
	u(\lambda)=\frac{ 1+\epsilon\lambda^2 }{ \lambda }\\
	z(\lambda)=C+\int_{\lambda_0}^{\lambda}\xi f'(\xi)d\xi
\end{numcases}
\end{subequations}
où $f(\xi)=\frac{ 1+\epsilon\xi^2 }{ \xi }$. L'intégrale n'est pas très compliquée à effectuer :
\begin{equation}
	z(\lambda)=C+\left[ \frac{ \epsilon\xi^2 }{ 2 }-\ln(\xi) \right]_{\lambda_0}^{\lambda}=C+\frac{ \epsilon }{ 2 }(\lambda^2-\lambda_0^2)+\ln\left( \frac{ \lambda_0 }{ \lambda } \right).
\end{equation}
Le $C$ peut être redéfini pour englober toutes les termes contenant $\lambda_0$ (qui est une constante arbitraire). Nous avons donc
\begin{subequations}
\begin{numcases}{}
	u(\lambda)=\frac{ 1+\epsilon\lambda^2 }{ \lambda }\\
	z(\lambda)=C+\frac{ \epsilon\lambda^2 }{ 2 }-\ln(\lambda).		\label{EqParmPpurzII118}
\end{numcases}
\end{subequations}
Résolvons le problème de Cauchy $z(u_0)=z_0$. Pour cela, cherchons la valeur $\lambda_0$ du paramètre pour lequel $u(\lambda)=u_0$. Il vient l'équation 
\begin{equation}
	\epsilon\lambda_0^2-u_0\lambda+1=0,
\end{equation}
dont la solution est 
\begin{equation}
	\lambda_0=\frac{ u_0\pm\sqrt{u_0^2-4\epsilon} }{ 2\epsilon }.
\end{equation}
Il suffit maintenant de remplacer cette valeur dans l'équation \eqref{EqParmPpurzII118} de $z(\lambda)$. Il faut distinguer les cas $\epsilon=\pm 1$.

Étudions d'abord le cas $\epsilon=-1$. Dans ce cas, $\sqrt{u_0^2-4\epsilon}$ existe toujours (c'est à dire pour tout $u_0$). Nous trouvons donc deux valeurs pour la constante $C$ dans \eqref{EqParmPpurzII118}, données par
\begin{equation}
	C^{(i)}=z_0-\frac{ \epsilon(\lambda_0^{(i)})^2 }{ 2 }-\ln(\lambda_0^{(i)})
\end{equation}
où
\begin{equation}
	\begin{aligned}[]
		\lambda_0^{(1)}&=\frac{ -u_0-\sqrt{u_0^2+4} }{ 2 },&\quad
		\lambda_0^{(2)}&=\frac{ -u_0+\sqrt{u_0^2+4} }{ 2 }
	\end{aligned}
\end{equation}

Dans le cas où $\epsilon=1$, la racine n'existe pas toujours.
\begin{enumerate}
\item Si $-2<u_0<2$, il n'y a aucune solutions.
\item Si $u_0=\pm 2$, il y a une seule solution.
\item Si $u_0>2$ ou $u_0<-2$, il y a deux solutions.
\end{enumerate}
Toutes ces solutions s'obtiennent par la même méthode que plus haut.

Cette multiplicité de solutions peut être vue en récrivant l'équation de départ \eqref{EqII118ufraczz} sous la forme
\begin{equation}
	z'=\frac{ u\pm\sqrt{u^2+4\epsilon} }{ 2\epsilon },
\end{equation}
qui sont réellement deux équations différentielles différentes qui peuvent être résolues moyennant le calcul de l'intégrale $\int\sqrt{u^2+4\epsilon}du$.

\item
$y=(a+by'^2 )/ y'$.
Comme précédemment, nous écrivons l'équation sous la forme $y=(1+\epsilon y'^2)/y'$ avec $\epsilon=\pm 1$. Cette fois, la forme paramétrique de la solution est donné à la page II.84 :
\begin{subequations}
\begin{numcases}{}
	y(\lambda)=\frac{ 1+\epsilon\lambda^2 }{ \lambda }\\
	t(\lambda)=t_0+\int_{\lambda_0}^{\lambda} \frac{ f'(\xi) }{ \xi }d\xi
\end{numcases}
\end{subequations}
où $f(\xi)=(1+\epsilon\xi^2)/\xi$. L'intégrale n'est pas très difficile à effectuer, et nous avons
\begin{subequations}
\begin{numcases}{}
	y(\lambda)=\frac{ 1+\epsilon\lambda^2 }{ \lambda }\\
	t(\lambda)=C+\epsilon\ln(\lambda)+\frac{ 1 }{ 2\lambda^2 }.
\end{numcases}
\end{subequations}
\end{enumerate}

Encore une fois, il y avait moyen de résoudre cette équation par une autre voie : $yy'=1+\epsilon y'^2$ peut être résolue algébriquement en fonction de $y'$ pour donner
\begin{equation}
	y'=\frac{ y\pm \sqrt{y^2+4\epsilon} }{ 2\epsilon },
\end{equation}
qui sont deux équations à variables séparées.


\end{alternative}
\end{corrige}
