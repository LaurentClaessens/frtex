% This is part of Outils mathématiques
% Copyright (c) 2012
%   Laurent Claessens
% See the file fdl-1.3.txt for copying conditions.

\begin{exercice}\label{exoDerive-0008} Épreuve complémentaire mai 2012\\

    On considère la fonction
    \begin{equation}
        f\colon (x,y,z)\in \eR^3\to x^2yz+2y^2\sin(xyz).
    \end{equation}
    \begin{enumerate}
        \item
            Calculer les dérivées partielles premières de \( f\).
        \item
            Soit \( F(x,y,z)=\nabla f(x,y,z)\) (le gradient de \( f\) au point \( x,y,z\)). Que vaut le rotationnel de \( F\) ?
        \item
            Que vaut la circulation du champ de vecteurs \( F\) le long d'une courbe reliant les points \( (1,2,\pi)\) et \( (1,1,\frac{ \pi }{2})\).
    \end{enumerate}

\corrref{Derive-0008}
\end{exercice}
