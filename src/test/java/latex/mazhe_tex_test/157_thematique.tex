% This is part of le Frido
% Copyright (c) 2016
%   Laurent Claessens
% See the file fdl-1.3.txt for copying conditions.

Ceci est un type d'index thématique. Plus bas, il y a une liste de différentes versions du même théorème.

%--------------------------------------------------------------------------------------------------------------------------- 
\subsection*{thèmes}
%---------------------------------------------------------------------------------------------------------------------------

\InternalLinks{Fonctions Lipschitz}
    \begin{enumerate}
    \item
        Définition : \ref{DEFooQHVEooDbYKmz}.
    \item
        La notion de Lipschitz est utilisée pour définir la stabilité d'un problème, définition \ref{DEFooYIFAooSJbMkC}.
    \end{enumerate}

\InternalLinks{Polynôme de Taylor}
    \begin{enumerate}
    \item
        Énoncé : théorème \ref{ThoTaylor}.
        \item
            Le polynôme de Taylor généralise à l'utilisation de toutes les dérivées disponibles le résultat de développement limité donné par la proposition \ref{PropUTenzfQ}.
        \item
            Il est utilisé pour justifier la méthode de Newton autour de l'équation \eqref{EQooOPUBooYaznay}.
        \end{enumerate}

\InternalLinks{Points fixes}
    \begin{enumerate}
\item 
    Il y a plusieurs théorèmes de points fixes.
    \begin{description}
        \item[Théorème de Picard] \ref{ThoEPVkCL} donne un point fixe comme limite d'itéré d'une fonction Lipschitz. Il aura pour conséquence le théorème de Cauchy-Lipschitz \ref{ThokUUlgU}, l'équation de Fredholm, théorème \ref{ThoagJPZJ} et le théorème d'inversion locale dans le cas des espaces de Banach \ref{ThoXWpzqCn}.
    \item[Théorème de Brouwer] qui donne un point fixe pour une application d'une boule vers elle-même. Nous allons donner plusieurs versions et preuves.
            \begin{enumerate}
                \item
                    Dans \( \eR^n\) en version \( C^{\infty}\) via le théorème de Stokes, proposition \ref{PropDRpYwv}.
                \item
                    Dans \( \eR^n\) en version continue, en s'appuyant sur le cas \( C^{\infty}\) et en faisant un passage à la limite, théorème \ref{ThoRGjGdO}.
                \item
                    Dans \( \eR^2\) via l'homotopie, théorème \ref{ThoLVViheK}. Oui, c'est très loin. Et c'est normal parce que ça va utiliser la formule de l'indice qui est de l'analyse complexe\footnote{On aime bien parce que ça ne demande pas Stokes, mais quand même hein, c'est pas gratos non plus.}.
            \end{enumerate}
        \item[Théorème de Markov-Kakutani]\ref{ThoeJCdMP} qui donne un point fixe à une application continue d'un convexe fermé borné dans lui-même. Ce théorème donnera la mesure de Haar \ref{ThoBZBooOTxqcI} sur les groupes compacts.
        \item[Théorème de Schauder] \ref{ThovHJXIU} qui est une version valable en dimension infinie du théorème de Brouwer. 
    \end{description}

\item Pour les équations différentielles
    \begin{enumerate}
        \item 
            Le théorème de Schauder a pour conséquence le théorème de Cauchy-Arzela \ref{ThoHNBooUipgPX} pour les équations différentielles.
        \item
            Le théorème de Schauder \ref{ThovHJXIU} permet de démontrer une version du théorème de Cauchy-Lipschitz (théorème \ref{ThokUUlgU}) sans la condition Lipschitz, mais alors sans unicité de la solution. Notons que de ce point de vue nous sommes dans la même situation que la différence entre le théorème de Brouwer et celui de Picard : hors hypothèse de type «contraction», point d'unicité.
    \end{enumerate}
\item
    En calcul numérique
    \begin{itemize}
        \item
            La convergence d'une méthode de point fixe est donnée par la proposition \ref{PROPooRPHKooLnPCVJ}.
        \item
            La convergence quadratique de la méthode de Newton est donnée par le théorème \ref{THOooDOVSooWsAFkx}.
        \item
            En calcul numérique, section \ref{SECooWUVTooMhmvaW}
        \item
            Méthode de Newton comme méthode de point fixe, sous-section \ref{SUBSECooIBLNooTujslO}.
    \end{itemize}

\item
    D'autres utilisation de points fixes.
\begin{itemize}
    \item
        Processus de Galton-Watson, théorème \ref{ThoJZnAOA}.
    \item 
        Dans le théorème de Max-Milgram \ref{THOooLLUXooHyqmVL}, le théorème de Picard est utilisé.
\end{itemize}


\end{enumerate}

\InternalLinks{Méthode de Newton}
    \begin{enumerate}
        \item
            Nous parlons un petit peu de méthode de Newton en dimension \( 1\) dans \ref{SECooIKXNooACLljs}.
        \item
            La méthode de Newton fonctionne bien avec les fonction convexes par la proposition \ref{PROPooVTSAooAtSLeI}.
        \item
            La méthode de Newton en dimension $n$ est le théorème \ref{ThoHGpGwXk}.
       \item
            Un intervalle de convergence autour de \( \alpha\) s'obtient par majoration de \( | g' |\), proposition \ref{PROPooRPHKooLnPCVJ}.
       \item
           Un intervalle de convergence quadratique s'obtient par majoration de \( | g'' |\), théorème \ref{THOooDOVSooWsAFkx}.
       \item
           En calcul numérique, section \ref{SECooIKXNooACLljs}.
       \end{enumerate}

\InternalLinks{Enveloppes}
    \begin{enumerate}
        \item
            L'ellipse de John-Loewner donne un ellipsoïde de volume minimum autour d'un compact dans \( \eR^n\), théorème \ref{PropJYVooRMaPok}.
        \item
            Le cercle circonscrit à une courbe donne un cercle de rayon minimal contenant une courbe fermée simple, proposition \ref{PROPDEFooCWESooVbDven}.
    \item Enveloppe convexe du groupe orthogonal \ref{ThoVBzqUpy}.
        \end{enumerate}

\InternalLinks{Produit semi-direct de groupes}
    \begin{enumerate}
        \item
            Définition \ref{DEFooKWEHooISNQzi}.
        \item
            Le corollaire \ref{CoroGohOZ} donne un critère pour prouver qu'un produit \( NH\) est un produit semi-direct.
        \item
            L'exemple \ref{EXooHNYYooUDsKnm} donne le groupe des isométries du carré comme un produit semi-direct.
        \item
            Le théorème \ref{THOooQJSRooMrqQct} donne les isométries de \( \eR^n\) par \( \Isom(\eR^n)=T(n)\times_{\rho} O(n)\) où \( T(n)\) est le groupe des translations.
        \item
            La proposition \ref{PROPooDHYWooXxEXvl} donne une décomposition du groupe orthogonal \( \gO(n)=\SO(n)\times_{\rho} C_2\) où \( C_2=\{ \id,R \}\) où \( R\) est de déterminant \( -1\).
        \item
            La proposition \ref{PROPooTPFZooKtFxhg} donne \( \Aff(\eR^n)=T(n)\times_{\rho}\GL(n,\eR)\) où \( \Aff(\eR^n)\) est le groupe des applications affines bijectives de \( \eR^n\).
        \end{enumerate}

\InternalLinks{Racines de polynôme et factorisation de polynômes}
    \begin{enumerate}
        \item
            Si \( \eA\) est une anneau, la proposition \ref{PropHSQooASRbeA} factorise une racine.
        \item
            Si \( \eA\) est un anneau, la proposition \ref{PropahQQpA} factorise une racine avec sa multiplicité.
        \item
            Si \( \eA\) est un anneau, le théorème \ref{ThoSVZooMpNANi} factorise plusieurs racines avec leurs multiplicités.
        \item
            Si \( \eK\) est un corps et \( \alpha\) une racine dans une extension, le polynôme minimal de \( \alpha\) divise tout polynôme annulateur par la proposition \ref{PropXULooPCusvE}.
        \item
            Le théorème \ref{ThoLXTooNaUAKR} annule un polynôme de degré \( n\) ayant \( n+1\) racines distinctes.
        \item
            La proposition \ref{PropTETooGuBYQf} nous annule un polynôme à plusieurs variables lorsqu'il a trop de racines.
        \item
            En analyse complexe, le principe des zéros isolés \ref{ThoukDPBX} annule en gros toute série entière possédant un zéro non isolé.
        \item 
            Polynômes irréductibles sur \( \eF_q\).
        \end{enumerate}

\InternalLinks{Théorème de Bézout}
    \begin{enumerate}
        \item
            Pour \( \eZ^*\) c'est le théorème \ref{ThoBuNjam}.
        \item
            Théorème de Bézout dans un anneau principal : corollaire \ref{CorimHyXy}.
        \item
            Théorème de Bézout dans un anneau de polynômes : théorème \ref{ThoBezoutOuGmLB}.
        \item
            En parlant des racines de l'unité et des générateurs du groupe unitaire dans le lemme \ref{LemcFTNMa}. Au passage nous y parlerons de solfège.
        \end{enumerate}

\InternalLinks{Équations diophantiennes}
    \begin{enumerate}
        \item
            Équation \( ax+by=c\) dans \( \eN\), équation \eqref{EqTOVSooJbxlIq}.
        \item Dans \ref{subsecZVKNooXNjPSf}, nous résolvons \( ax+by=c\) en utilisant Bézout (théorème \ref{ThoBuNjam}).
        \item L'exemple \ref{ExZPVFooPpdKJc} donne une application de la pure notion de modulo pour \( x^2=3y^2+8\). Pas de solutions.
        \item L'exemple \ref{ExmuQisZU} résout l'équation \( x^2+2=y^3\) en parlant de l'extension \( \eZ[i\sqrt{2}]\) et de stathme.
        \item Les propositions \ref{PropXHMLooRnJKRi} et \ref{propFKKKooFYQcxE} parlent de triplets pythagoriciens.
        \item Le dénombrement des solutions de l'équation \( \alpha_1 n_1+\ldots \alpha_pn_p=n\) utilise des séries entières et des décomposition de fractions en éléments simples, théorème \ref{ThoDIDNooUrFFei}.
        \end{enumerate}

\InternalLinks{Application réciproque}
    \begin{enumerate}
        \item
            Définition \ref{DEFooTRGYooRxORpY}.
        \item
            Dans le cas des réels, des exemples sont donnés en \ref{EXooCWYHooLEciVj}.
        \end{enumerate}

\InternalLinks{Extension de corps et polynômes}
    \begin{enumerate}
        \item
            Définition d'une extension de corps \ref{DEFooFLJJooGJYDOe}.
        \item
            Pour l'extension du corps de base d'un espace vectoriel et les propriétés d'extension des applications linéaires, voir la section \ref{SECooAUOWooNdYTZf}.
        \item
            Extension de corps de base et similitude d'application linéaire (ou de matrices, c'est la même chose), théorème \ref{THOooHUFBooReKZWG}.
        \item
            Extension de corps de base et cyclicité des applications linéaires, corollaire \ref{CORooAKQEooSliXPp}.
        \item 
            À propos d'extensions de \( \eQ\), le lemme \ref{LemSoXCQH}.
    \end{enumerate}

\InternalLinks{Rang}
    \begin{enumerate}
        \item Définition \ref{DefALUAooSPcmyK}.
        \item Le théorème du rang, théorème \ref{ThoGkkffA}
        \item Prouver que des matrices sont équivalentes et les mettre sous des formes canoniques, lemme \ref{LemZMxxnfM} et son corollaire \ref{CorGOUYooErfOIe}.
        \item Tout hyperplan de \( \eM(n,\eK)\) coupe \( \GL(n,\eK)\), corollaire \ref{CorGOUYooErfOIe}. Cela utilise la forme canonique sus-mentionnée.
        \item Le lien entre application duale et orthogonal de la proposition \ref{PropWOPIooBHFDdP} utilise la notion de rang.
        \item Prouver les équivalences à être un endomorphisme cyclique du théorème \ref{THOooGLMSooYewNxW} via le lemme \ref{LEMooDFFDooJTQkRu}.
        \end{enumerate}

\InternalLinks{Topologie produit}
    \begin{enumerate}
        \item
            La définition de la topologie produit est \ref{DefIINHooAAjTdY}.
        \item
            Pour les espaces vectoriels normés, le produit est donné par la définition \ref{DefFAJgTCE}.
        \item
            L'équivalence entre la topologie de la norme produit et la topologie produit est le lemme \ref{LEMooWVVCooIGgAdJ}.
        \end{enumerate}

\InternalLinks{Produit de compact}
    \begin{enumerate}
    \item
        Les produits d'espaces métriques compacts sont compacts; c'est le théorème de Tykhonov. Nous verrons ce résultat dans le cas suivants.
        \begin{itemize}
            \item 
    \item
         \( \eR\), lemme \ref{LemCKBooXkwkte}.
    \item
        Produit fini d'espaces métriques compacts, théorème \ref{THOIYmxXuu}.
    \item
        Produit dénombrable d'espaces métrique compacts, théorème \ref{ThoKKBooNaZgoO}.
        \end{itemize}
    \end{enumerate}

\InternalLinks{Connexité}
    \begin{enumerate}
        \item
            Définition \ref{DefIRKNooJJlmiD}
        \item
            Le groupe \( \SL(n,\eK)\) est connexe par arcs : proposition \ref{PROPooALQCooLZCKrH}.
        \item
            Le groupe \( \GL(n,\eC)\) est connexe par arcs : proposition \ref{PROPooVJNIooMByUJQ}.
        \item
            Le groupe \( \GL(n,\eC)\) est connexe par arcs, proposition \ref{PROPooVJNIooMByUJQ}.
        \item
            Le groupe \( \GL(n,\eR)\) a exactement deux composantes connexes par arcs, proposition \ref{PROPooBIYQooWLndSW}.
        \item
            Le groupe \( \gO(n,\eR)\) n'est pas connexe, lemme \ref{LEMooIPOVooZJyNoH}.
        \item
            Les groupe \( \gU(n)\) et \( \SU(n)\) sont connexes par arcs, lemme \ref{LEMooQMXHooZQozMK}.
        \item
            Le groupe \( \SO(n)\) est connexe mais ce n'est pas encore démontré, proposition \ref{PROPooYKMAooCuLtyh}.
        \item 
            Connexité des formes quadratiques de signature donnée, proposition \ref{PropNPbnsMd}.
        \end{enumerate}

\InternalLinks{Norme matricielle et rayon spectral}
    \begin{enumerate}
        \item
            Définition du rayon spectral \ref{DEFooEAUKooSsjqaL}.
        \item
            Lien entre norme matricielle et rayon spectral, le théorème \ref{THOooNDQSooOUWQrK} assure que $\|A\|_2=\sqrt{\rho(A{^t}A)}$.
        \item
            Pour toute norme algébrique nous avons \( \rho(A)\leq \| A \|\), proposition \ref{PROPooCCPRooZLJiRs}.
        \item
            Dans le cadre du conditionnement de matrice. Voir en particulier la proposition \ref{PROPooNUAUooIbVgcN} qui utilise le théorème \ref{THOooNDQSooOUWQrK}.
        \end{enumerate}

\InternalLinks{Diagonalisation}
    Des résultats qui parlent diagonalisation
    \begin{enumerate}
        \item
            Définition d'un endomorphisme diagonalisable : \ref{DefCNJqsmo}.
        \item
            Conditions équivalentes au fait d'être diagonalisable en termes de polynôme minimal, y compris la décomposition en espaces propres : théorème \ref{ThoDigLEQEXR}.
        \item
            Diagonalisation simultanée \ref{PropGqhAMei}, pseudo-diagonalisation simultanée \ref{CorNHKnLVA}.
        \item
            Diagonalisation d'exponentielle \ref{PropCOMNooIErskN} utilisant Dunford.
        \item
            Décomposition polaire théorème \ref{ThoLHebUAU}. \( M=SQ\), \( S\) est symétrique, réelle, définie positive, \( Q\) est orthogonale.
        \item
            Décomposition de Dunford \ref{ThoRURcpW}. \( u=s+n\) où \( s\) est diagonalisable et \( n\) est nilpotent, \( [s,n]=0\).
        \item 
            Réduction de Jordan (bloc-diagonale) \ref{ThoGGMYooPzMVpe}.
        \item 
            L'algorithme des facteurs invariants \ref{PropPDfCqee} donne \( U=PDQ\) avec \( P\) et \( Q\) inversibles, \( D\) diagonale, sans hypothèse sur \( U\). De plus les éléments de \( D\) forment une chaîne d'éléments qui se divisent l'un l'autre.
        \end{enumerate}
        Le théorème spectral et ses variantes :
        \begin{enumerate}
            \item
                Théorème spectral, matrice symétrique, théorème \ref{ThoeTMXla}. Via le lemme de Schur.
            \item
                Théorème spectral autoadjoint (c'est le même, mais vu sans matrices), théorème \ref{ThoRSBahHH}
            \item
                Théorème spectral hermitien, lemme \ref{LEMooVCEOooIXnTpp}.
            \item
                Théorème spectral, matrice normales, théorème \ref{ThogammwA}.
            \end{enumerate}
        Pour les résultats de décomposition dont une partie est diagonale, voir le thème \ref{DECooWTAIooNkZAFg} sur les décompositions.

\InternalLinks{Sous-groupes}
\begin{enumerate}
    \item 
        Théorème de Burnside sur les sous groupes d'exposant fini de \( \GL(n,\eC)\), théorème \ref{ThooJLTit}.
    \item 
        Sous-groupes compacts de \( \GL(n,\eR)\), lemme \ref{LemOCtdiaE} ou proposition \ref{PropQZkeHeG}.
\end{enumerate}

\InternalLinks{Mesure et intégrale}
    \begin{enumerate}
        \item
            Mesure de Lebesgue, définition \ref{DefooYZSQooSOcyYN}
        \item
            Intégrale associée à une mesure, définition \ref{DefTVOooleEst}
        \end{enumerate}

\InternalLinks{Équivalence de normes}
    \begin{enumerate}
\item
    La proposition \ref{PropLJEJooMOWPNi} sur l'équivalence des normes dans \( \eR^n\).
\item
    Montrer que le problème \( a-b\) est stable dans l'exemple \ref{ExooXJONooTAYZVc}.
\end{enumerate}

\InternalLinks{Espace \( L^2\) (L2)}
    \begin{enumerate}
        \item
            Définition de \( L^2(\Omega,\mu)\), \ref{DEFooSVCHooIwwuIx}.
        \item
            L'espace \( L^2\)  est discuté en analyse fonctionnelle, en \ref{subSecCKZSrZK} parce que l'étude de \( L^2\) utilise entre autres l'inégalité de Hölder \ref{ProptYqspT}.
\end{enumerate}

\InternalLinks{Espaces \( L^p\) (Lp)}
\begin{enumerate}
    \item
        Dual de \( L^p\big( \mathopen[ 0 , 1 \mathclose] \big)\) pour \( 1<p<2\), proposition \ref{PropOAVooYZSodR}.
\end{enumerate}

\InternalLinks{Théorème de Stokes, Green et compagnie}
    \begin{enumerate}
        \item
            Forme générale, théorème \ref{ThoATsPuzF}.
        \item
            Rotationnel et circulation, théorème \ref{THOooIRYTooFEyxif}.
        \end{enumerate}

\InternalLinks{Invariants de similitude}
    \begin{enumerate}
        \item
            Théorème \ref{THOooDOWUooOzxzxm}.
        \item
            Pour prouver que la similitude d'applications linéaires résiste à l'extension du corps de base, théorème \ref{THOooHUFBooReKZWG}.
        \item
            Pour prouver que la dimension du commutant d'un endomorphisme de \( E\) est de dimension au moins \( \dim(E)\), lemme \ref{LEMooDFFDooJTQkRu}.
        \item
            Nous verrons dans la remarque \ref{REMooPVLEooYDRXQI} à propos des invariants de similitude que toute matrice est semblable à la matrice bloc-diagonale constituées des matrices compagnon (définition \ref{DEFooOSVAooGevsda}) de la suite des polynômes minimals.
        \end{enumerate}

\InternalLinks{Endomorphismes cycliques}
    \begin{enumerate}
        \item
            Définition \ref{DEFooFEIFooNSGhQE}.
        \item
            Son lien avec le commutant donné dans la proposition \ref{PropooQALUooTluDif} et le théorème \ref{THOooGLMSooYewNxW}.
        \item
            Utilisation dans le théorème de Frobenius (invariants de similitude), théorème \ref{THOooDOWUooOzxzxm}.
        \end{enumerate}

\InternalLinks{Formes bilinéaires et quadratiques}
    \begin{enumerate}
\item
    Les formes bilinéaires ont été définies en \ref{DEFooEEQGooNiPjHz}.
\item
    Forme quadratique, définition \ref{DefBSIoouvuKR}
\end{enumerate}

\InternalLinks{Isométries}
\begin{enumerate}
    \item
        Les isométries de l'espace euclidien sont affines, \ref{ThoDsFErq}.
    \item
        Les isométries de l'espace euclidien comme produit semi-direct : $\Isom(\eR^n)\simeq T(n)\times_{\rho}\gO(n)$, théorème \ref{THOooQJSRooMrqQct}.
    \item
        Isométries du cube, section \ref{SecPVCmkxM}.
    \item 
        Générateurs du groupe diédral, proposition \ref{PropLDIPoZ}.
\end{enumerate}

\InternalLinks{Déterminant}
    \begin{enumerate}
    \item
        Les \( n\)-formes alternées forment un espace de dimension \( 1\), proposition \ref{ProprbjihK}.
    \item
        Déterminant d'une famille de vecteurs \ref{DEFooODDFooSNahPb}.
    \item
        Déterminant d'un endomorphisme \ref{DefCOZEooGhRfxA}.
        \item
            Des interprétations géométriques du déterminant sont dans la section \ref{SECooSQRDooGifgQi}.
        \end{enumerate}

\InternalLinks{Polynôme d'endomorphismes}
    \begin{enumerate}
    \item Endomorphismes cycliques et commutant dans le cas diagonalisable, proposition \ref{PropooQALUooTluDif}.
    \item Racine carré d'une matrice hermitienne positive, proposition \ref{PropVZvCWn}.
    \item Théorème de Burnside sur les sous groupes d'exposant fini de \( \GL(n,\eC)\), théorème \ref{ThooJLTit}.
    \item Décomposition de Dunford, théorème \ref{ThoRURcpW}. 
    \item Algorithme des facteurs invariants \ref{PropPDfCqee}.
    \end{enumerate}

\InternalLinks{Action de groupe}
    \begin{enumerate}
    \item Action du groupe modulaire sur le demi-plan de Poincaré, théorème \ref{ThoItqXCm}. 
    \end{enumerate}

\InternalLinks{Systèmes d'équations linéaires}
\begin{itemize}
    \item Algorithme des facteurs invariants \ref{PropPDfCqee}.
    \item Méthode du gradient à pas optimal \ref{PropSOOooGoMOxG}.
\end{itemize}

\InternalLinks{Classification de groupes}
\begin{enumerate}
    \item Structure des groupes d'ordre \( pq\), théorème \ref{ThoLnTMBy}.
    \item Le groupe alterné est simple, théorème \ref{ThoURfSUXP}.
    \item Théorème de Sylow \ref{ThoUkPDXf}. Tout le théorème, c'est un peu long. On peut se contenter de la partie qui dit que \( G\) contient un \( p\)-Sylow.
    \item Théorème de Burnside sur les sous groupes d'exposant fini de \( \GL(n,\eC)\), théorème \ref{ThooJLTit}.
    \item \( (\eZ/p\eZ)^*\simeq \eZ/(p-1)\eZ\), corollaire \ref{CorpRUndR}.
\end{enumerate}

\InternalLinks{Théorie des représentations}
\begin{enumerate}
    \item Table des caractères du groupe diédral, section \ref{SecWMzheKf}.
    \item Table des caractères du groupe symétrique \( S_4\), section \ref{SecUMIgTmO}.
\end{enumerate}

\InternalLinks{Décomposition de matrices}   \label{DECooWTAIooNkZAFg}
\begin{enumerate}
    \item 
        Décomposition de Bruhat, théorème \ref{ThoizlYJO}.
    \item 
        Décomposition de Dunford, théorème \ref{ThoRURcpW}. 
    \item 
        Décomposition polaire \ref{ThoLHebUAU}.
\end{enumerate}

\InternalLinks{Méthodes de calcul}
\begin{enumerate}
    \item 
        Théorème de Rothstein-Trager \ref{ThoXJFatfu}.
    \item 
        Algorithme des facteurs invariants \ref{PropPDfCqee}.
    \item 
        Méthode de Newton, théorème \ref{ThoHGpGwXk}
    \item 
        Calcul d'intégrale par suite équirépartie \ref{PropDMvPDc}.
\end{enumerate}

\InternalLinks{Équations différentielles}
L'utilisation des théorèmes de point fixe pour l'existence de solutions à des équations différentielles est fait dans le chapitre sur les points fixes.
\begin{enumerate}
    \item
            Le théorème de Schauder a pour conséquence le théorème de Cauchy-Arzela \ref{ThoHNBooUipgPX} pour les équations différentielles.
        \item
            Le théorème de Schauder \ref{ThovHJXIU} permet de démontrer une version du théorème de Cauchy-Lipschitz (théorème \ref{ThokUUlgU}) sans la condition Lipschitz
        \item
            Le théorème de Cauchy-Lipschitz \ref{ThokUUlgU} est utilisé à plusieurs endroits :
            \begin{itemize}
                \item 
                    Pour calculer la transformée de Fourier de \(  e^{-x^2/2}\) dans le lemme \ref{LEMooPAAJooCsoyAJ}.
            \end{itemize}
    \item
        Théorème de stabilité de Lyapunov \ref{ThoBSEJooIcdHYp}.
    \item
        Le système proie prédateurs, Lokta-Voltera \ref{ThoJHCLooHjeCvT}
    \item 
        Équation de Schrödinger, théorème \ref{ThoLDmNnBR}.
    \item 
        L'équation \( (x-x_0)^{\alpha}u=0\) pour \( u\in\swD'(\eR)\), théorème \ref{ThoRDUXooQBlLNb}.
    \item 
        La proposition \ref{PropMYskGa} donne un résultat sur \( y''+qy=0\) à partir d'une hypothèse de croissance.
    \item
        Équation de Hill \( y''+qy=0\), proposition \ref{PropGJCZcjR}.
\end{enumerate}

\InternalLinks{Dénombrements}
\begin{itemize}
    \item Coloriage de roulette (\ref{pTqJLY}) et composition de colliers (\ref{siOQlG}).
    \item Nombres de Bell, théorème \ref{ThoYFAzwSg}.
    \item Le dénombrement des solutions de l'équation \( \alpha_1 n_1+\ldots \alpha_pn_p=n\) utilise des séries entières et des décomposition de fractions en éléments simples, théorème \ref{ThoDIDNooUrFFei}.
\end{itemize}

\InternalLinks{Densité}
\begin{enumerate}
    \item 
        Densité des polynômes dans \( C^0\big( \mathopen[ 0 , 1 \mathclose] \big)\), théorème de Bernstein \ref{ThoDJIvrty}.
    \item
        Densité de \( \swD(\eR^d)\) dans \( L^p(\eR^d)\) pour \( 1\leq p<\infty\), théorème \ref{ThoILGYXhX}.
    \item
        Densité de \( \swS(\eR^d)\) dans l'espace de Sobolev \( H^s(\eR^d)\), proposition \ref{PROPooMKAFooKDNTbO}. 

    \item
        Densité de \( \swD(\eR^d)\) dans l'espace de Sobolev \( H^s(\eR^d)\), proposition \ref{PROPooLIQJooKpWtnV}. 

        Cela est utilisé pour le théorème de trace \ref{THOooXEJZooBKtXBW}.
\end{enumerate}
Les densités sont bien entendu utilisées pour prouver des formules sur un espace en sachant qu'elles sont vraies sur une partie dense. Mais également pour étendre une application définie seulement sur une partie dense. C'est par exemple ce qui est fait pour définir la trace \( \gamma_0\) sur les espaces de Sobolev \( H^s(\eR^d)\) en utilisant le théorème d'extension \ref{PropTTiRgAq}.

\InternalLinks{Injections}
\begin{enumerate}
        \item
            L'espace de Sobolev \( H^1(I)\) s'injecte de façon compacte dans \( C^0(\bar I)\), proposition \ref{ThoESIyxfU}.
        \item
            L'espace de Sobolev \( H^1(I)\) s'injecte de façon continue dans \( L^2(I)\), proposition \ref{ThoESIyxfU}.
        \item
            L'espace \( L^2(\Omega)\) s'injecte continument dans \( \swD'(\Omega)\) (les distributions), proposition \ref{PROPooYAJSooMSwVOm}.
\end{enumerate}

\InternalLinks{Dualité}
Parmi les théorèmes de dualité nous avons
\begin{enumerate}
    \item
        Le théorème de représentation de Riesz \ref{ThoQgTovL} pour les espaces de Hilbert.
    \item
        La proposition \ref{PropOAVooYZSodR} pour les espaces \( L^p\big( \mathopen[ 0 , 1 \mathclose] \big)\) avec \( 1<p<2\).
    \item
        Le théorème de représentation de Riesz \ref{ThoSCiPRpq} pour les espaces \( L^p\) en général.
\end{enumerate}
Tous ces théorèmes donnent la dualité par l'application \( \Phi_x=\langle x, .\rangle \).

\InternalLinks{Opérations sur les distributions}
\begin{enumerate}
    \item
        Convolution d'une distribution par une fonction, définition par l'équation \eqref{EQooOUXKooGHDSzL}.
    \item
        Dérivation d'une distribution, proposition-définition \ref{PropKJLrfSX}.
    \item
        Produit d'une distribution par une fonction, définition \ref{DefZVRNooDXAoTU}.
\end{enumerate}

\InternalLinks{Permuter des limites}
\begin{enumerate}
    \item 
        Les théorèmes sur les fonctions définies par des intégrales, section \ref{SecCHwnBDj}. Nous avons entre autres
        \begin{enumerate}
            \item
                \( \partial_i\int_Bf=\int_B\partial_if\), avec \( B\) compact, proposition \ref{PropDerrSSIntegraleDSD}.
            \item
                Si \( f\) est majorée par une fonction ne dépendant pas de \( x\), nous avons le théorème \ref{ThoKnuSNd}.
            \item
                Si l'intégrale est uniformément convergente, nous avons le théorème \ref{ThotexmgE}.
        \end{enumerate}
    \item 
        Théorème de la convergence monotone, théorème \ref{ThoRRDooFUvEAN}.
\end{enumerate}

\InternalLinks{Convolution}
\begin{enumerate}
    \item
        Définition pour \( f,g\in L^1\), théorème \ref{THOooMLNMooQfksn}.
    \item
        Inégalité de normes : si \( f\in L^p\) et \( g\in L^1\), alors \( \| f*g \|_p\leq \| f \|_p\| g \|_1\), proposition \ref{PROPooDMMCooPTuQuS}.
    \item
        \( \varphi\in L^1(\eR)\) et \( \psi\in\swS(\eR)\), alors \( \varphi * \psi\in \swS(\eR)\), proposition \ref{PROPooUNFYooYdbSbJ}.
    \item
        Les suites régularisantes : \( \lim_{n\to \infty} \rho_n*f=f\) dans la proposition \ref{PROPooYUVUooMiOktf}.
\end{enumerate}

\InternalLinks{Séries de Fourier}
\begin{itemize}
    \item Formule sommatoire de Poisson, proposition \ref{ProprPbkoQ}.
    \item Inégalité isopérimétrique, théorème \ref{ThoIXyctPo}.
    \item Fonction continue et périodique dont la série de Fourier ne converge pas, proposition \ref{PropREkHdol}.
\end{itemize}

\InternalLinks{Transformée de Fourier}
\begin{enumerate}
    \item
        Définition sur \( L^1\), définition \ref{DEFooJAIUooFbaRkR}.
    \item
        La transformée de Fourier d'une fonction \( L^1(\eR^d)\) est continue, proposition \ref{PropJvNfj}.
    \item
    L'espace de Schwartz est stable par transformée de Fourier. L'application $\TF\colon \swS(\eR^d)\to \swS(\eR^d)$ est une bijection linéaire et continue. Proposition  \ref{PropKPsjyzT}
\end{enumerate}

\InternalLinks{Applications continues et bornées}
\begin{enumerate}
    \item
        Une application linéaire non continue : exemple \ref{ExHKsIelG} de \( e_k\mapsto ke_k\). Les dérivées partielles sont calculées en \eqref{EQooWNLOooJNRUMQ}.
    \item
        Une application linéaire est bornée si et seulement si elle est continue, proposition \ref{PropmEJjLE}.
\end{enumerate}

\InternalLinks{Définie positive}
\begin{enumerate}
    \item
        Une application bilinéaire est définie positive lorsque \( g(u,u)\geq 0\) et \( g(u,u)=0\) si et seulement si \( u=0\) est la définition \ref{DEFooJIAQooZkBtTy}.
    \item
        Un opérateur ou une matrice est défini positif si toutes ses valeurs propres sont positives, c'est la définition \ref{DefAWAooCMPuVM}.
    \item
        Pour une matrice symétrique, définie positive implique \( \langle Ax, x\rangle >0\) pour tout \( x\). C'est le lemme \ref{LemWZFSooYvksjw}.
    \item
        Une application linéaire est définie positive si et seulement si sa matrice associée l'est. C'est la proposition \ref{PROPooUAAFooEGVDRC}.
\end{enumerate}
Remarque : nous ne définissons pas la notion de matrice définie positive dans le cas d'une matrice non symétrique.

\InternalLinks{Gaussienne}
\begin{enumerate}
    \item
        Le calcul de l'intégrale
        \begin{equation}
            \int_{\eR} e^{-x^2}dx=\sqrt{\pi }
        \end{equation}
        est fait dans les exemples \ref{ExrgMIni} et \ref{EXooLUFAooGcxFUW}.
    \item
        Le lemme \ref{LEMooPAAJooCsoyAJ} calcule la transformée de Fourier de $ g_{\epsilon}(x)=  e^{-\epsilon\| x \|^2}$ qui donne $\hat g_{\epsilon}(\xi)=\left( \frac{ \pi }{ \epsilon } \right)^{d/2} e^{-\| \xi \|^2/4\epsilon}$.
    \item
        Le lemme \ref{LEMooTDWSooSBJXdv} donne une suite régularisante à base de gaussienne.
    \item
        Elle est utilisée pour régulariser une intégrale dans la preuve de la formule d'inversion de Fourier \ref{PROPooLWTJooReGlaN}
\end{enumerate}

%--------------------------------------------------------------------------------------------------------------------------- 
\subsection*{Différentes versions du même théorème}
%---------------------------------------------------------------------------------------------------------------------------

\InternalLinks{Inégalités}
\begin{description}
    \item[Inégalité de Jensen] 
        \begin{enumerate}
            \item
                Une version discrète pour \( f\big( \sum_i\lambda_ix_i \big)\), la proposition \ref{PropXIBooLxTkhU}.
            \item
                Une version intégrale pour \( f\big( \int \alpha d\mu \big)\), la proposition \ref{PropXISooBxdaLk}.
            \item
                Une version pour l'espérance conditionnelle, la proposition \ref{PropABtKbBo}.
        \end{enumerate}
    \item[Inégalité de Minkowsky]
        \begin{enumerate}
            \item
                Pour une forme quadratique \( q\) sur \( \eR^n\) nous avons $\sqrt{q(x+y)}\leq\sqrt{q(x)}+\sqrt{q(y)}$. Proposition \ref{PropACHooLtsMUL}.
            \item
                Si \( 1\leq p<\infty\) et si \( f,g\in L^p(\Omega,\tribA,\mu)\) alors \(  \| f+g \|_p\leq \| f \|_p+\| g \|_p\). Proposition \ref{PropInegMinkKUpRHg}.
            \item
                L'inégalité de Minkowsky sous forme intégrale s'écrit sous forme débalée
                \begin{equation}
                    \left[ \int_X\Big( \int_Y| f(x,y) |d\nu(y) \Big)^pd\mu(x) \right]^{1/p}\leq \int_Y\Big( \int_X| f(x,y) |^pd\mu(x) \Big)^{1/p}d\nu(y).
                \end{equation}
                ou sous forme compacte
                \begin{equation}
                    \left\|   x\mapsto\int_Y f(x,y)d\nu(y)   \right\|_p\leq \int_Y  \| f_y \|_pd\nu(y)
                \end{equation}
        \end{enumerate}
\end{description}

\InternalLinks{Théorème central limite}
\begin{enumerate}
    \item
        Pour les processus de Poisson, théorème \ref{ThoCSuLLo}.
\end{enumerate}

\InternalLinks{Lemme de transfert}      \label{THEMEooJREIooKEdMOl}
Il s'agit du résultat \( \hat{f'}=i\xi \hat{f}\).
\begin{enumerate}
    \item
        Lemme \ref{LemQPVQjCx} sur \( \swS(\eR^d)\)
    \item
        Lemme \ref{LEMooAGBZooWCbPDd} pour \( L^2\).
\end{enumerate}

\InternalLinks{Déduire la nullité d'une fonction depuis son intégrale}
Des résultats qui disent que si \( \int f=0\) c'est que \( f=0\) dans un sens ou dans un autre.
\begin{enumerate}
    \item
        Il y a le lemme \ref{Lemfobnwt} qui dit ça.
    \item
        Un lemme du genre dans \( L^2\) existe aussi pour \( \int f\varphi=0\) pour tout \( \varphi\). C'est le lemme \ref{LemDQEKNNf}.
    \item
        Et encore un pour \( L^p\) dans la proposition \ref{PropUKLZZZh}.
    \item
        Si \( \int f\chi=0\) pour tout \( \chi\) à support compact alors \( f=0\) presque partout, proposition \ref{PropAAjSURG}.
    \item
        La proposition \ref{PropRERZooYcEchc} donne \( f=0\) dans \( L^p\) lorsque \( \int fg=0\) pour tout \( g\in L^q\) lorsque l'espace est \( \sigma\)-fini.
    \item
        Une fonction \( h\in C^{\infty}_c(I)\) admet une primitive dans \(  C^{\infty}_c(I)\) si et seulement si \( \int_Ih=0\). Théorème \ref{PropHFWNpRb}.
\end{enumerate}
