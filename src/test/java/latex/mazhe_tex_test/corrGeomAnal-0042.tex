\begin{corrige}{GeomAnal-0042}

    Nous suivons essentiellement les mêmes étapes et astuces que pour l'exercice \ref{exoGeomAnal-0041}. Soit \( x\in\eR^n\) avec \( \| x \|_1=1\) et \( b=Ax\). Alors nous avons
    \begin{subequations}
        \begin{align}
            \| Ax \|_1=\| b \|_1&=\sum_{i=1}^n| b_i |\\
            &=\sum_i\left| \sum_{j=1}^na_{ijx_j} \right| \\
            &\leq\sum_i\sum_j| a_{ij}x_j |\\
            &=\sum_j| x_j |\sum_i| a_{ij} |
        \end{align}
    \end{subequations}
    À ce moment nous ne pouvons pas remplacer \( \sum_j| x_j |\) par \( \| x \|_1=1\) parce que la somme sur \( j\) se poursuit dans le reste de la formule ! Nous devons donc faire une étape supplémentaire :
    \begin{subequations}
        \begin{align}
            \| Ax \|_1&\leq\sum_j| x_j |\sum_i| a_{ij} |\\
            &\leq\sum_j| x_j |\left( \max_{j\in\{ 1,\ldots,n \}}\sum_i| a_{ij} | \right)\\
            &=\max_j\sum_{i}| a_{ij} |.
        \end{align}
    \end{subequations}

    Pour l'inégalité inverse, nous considérons \( j^*\), le \( j\) qui réalise le maximum \( \max_j\sum_i| a_{ij} |\), c'est à dire
    \begin{equation}
        \max_j\sum_i| a_{ij} |=\sum_i| a_{ij^*} |,
    \end{equation}
    et nous considérons le vecteur \( x^*\) donné par
    \begin{equation}
        x^*_j=\begin{cases}
            1    &   \text{si \( a_{ij^*}\geq 0\)}\\
            -1    &    \text{si $a_{ij^*}<0$}.
        \end{cases}
    \end{equation}
    Dans ce cas nous avons
    \begin{subequations}
        \begin{align}
            \| Ax^* \|_1&=\sum_i\left| \sum_ja_{ij}x_j^* \right| \\
            &=\sum_i\sum_j| a_{ij} |\\
            &=\sum_{j}\sum_i| a_{ij} |\\
            &\geq\max_j\sum_i| a_{ij} |
        \end{align}
    \end{subequations}
    La dernière majoration est le fait qu'une somme d'éléments positifs est plus grande que le plus grands de ses éléments. Nous avons donc prouvé que
    \begin{equation}
        \| A \|_1=\sup_{\| x \|_1=1}\| Ax \|_1\geq\max_j\sum_i| a_{ij} |.
    \end{equation}

\end{corrige}
