% This is part of Exercices de mathématique pour SVT
% Copyright (c) 2011
%   Laurent Claessens and Carlotta Donadello
% See the file fdl-1.3.txt for copying conditions.

\begin{corrige}{SVT-0003}

    \begin{enumerate}
        \item
            En dérivant la fonction proposée par rapport à \( t\), nous trouvons
            \begin{equation}
                x'(t)=-t e^{\ln(c/2\pi)\cos(t)}\ln(c/2\pi)\sin(t)+ e^{\ln(c/2\pi)\cos(t)}.
            \end{equation}
            Nous vérifions maintenant l'équation différentielle
            \begin{equation}
                x'(t)=-x\ln(c/2\pi)\sin(t)+\frac{ x }{ t }
            \end{equation}
            en remplaçant \( x\) et \( x'\) par leurs valeurs.
            \begin{equation}
                -t e^{\ln(c/2\pi)\cos(t)}\ln(c/2\pi)\sin(t)+ e^{\ln(c/2\pi)\cos(t)}  \stackrel{?}{=}  -  t e^{\ln(c/2\pi)\cos(t)}  \ln(c/2\pi)\sin(t)+\frac{ t e^{\ln(x)\cos(t)} }{ t }
            \end{equation}
            Un peu de calcul montre que cela est correct.
        \item
            Lorsque \( t\to\infty\), le facteur exponentiel ne tend vers rien de précis (parce que \( \cos(t)\) oscille). Il est cependant toujours positif et inférieur à \( c/2\pi\). Le facteur \( t\) par contre tend vers l'infini.

            Nous avons par conséquent
            \begin{equation}
                \lim_{t\to \infty} x(t)=\infty.
            \end{equation}
            
        \item
            
            \( t=2\pi\).

    \end{enumerate}

\end{corrige}
