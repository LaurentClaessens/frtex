% This is part of Exercices et corrigés de CdI-1
% Copyright (c) 2011,2014
%   Laurent Claessens
% See the file fdl-1.3.txt for copying conditions.

\begin{corrige}{OutilsMath-0027}

	Pour trouver le coefficient directeur de la tangente, il faut étudier le rapport
	\begin{equation}
		\begin{aligned}[]
			\frac{ f(x)-f(2) }{ x-2 }&=\frac{ \frac{1}{ x }-\frac{1}{ 2 } }{ x-2 }\\
			&=\frac{ 2-x }{ 2x(x-2) }\\
			&=-\frac{1}{ 2x }.
		\end{aligned}
	\end{equation}
	En prenant la limite $x\to 2$ nous trouvons $-\frac{1}{ 4 }$ comme coefficient directeur.

	Il s'agit donc de trouver la droite de coefficient directeur $\frac{-1}{ 4 }$ et qui passe par le point $\big( 2,f(2) \big)=(2,\frac{1}{ 2 })$. Si nous écrivons la droite sous la forme $y=ax+b$, la première chose est que $a=-\frac{1}{ 4 }$. Ensuite, il faut que l'équation
	\begin{equation}
		y=-\frac{1}{ 4 }x+b
	\end{equation}
	soir vérifiée pour le point $(x,y)=(2,\frac{1}{ 2 })$. Nous trouvons l'équation pour $b$ :
	\begin{equation}
		-\frac{ 1 }{4}\cdot 2+b=\frac{ 1 }{2},
	\end{equation}
	donc $b=1$ et l'équation de la tangente recherchée est
	\begin{equation}
		y=-\frac{1}{ 4 }x+1.
	\end{equation}

\end{corrige}
