\begin{corrige}{LimiteContinue0011}

  \begin{enumerate}
  \item 

	  Lorsqu'on a le sinus d'une fonction qui tend vers zéro, il est souvent bon de multiplier et diviser par la fonction afin de faire apparaître la limite connue $\lim_{x\to 0} \frac{ \sin(x) }{ x }=1$. Cela permet de simplifier les expressions. Ici, on a
	  \begin{equation}
		  \sin(xy)=\frac{ xy\sin(xy) }{ xy }.
	  \end{equation}
	  Le calcul de la limite s'effectue donc de la façon suivante :
    \begin{equation}
	    \lim_{(x,y)\to (0,0)}\frac{\sin(xy)}{\sqrt{x^2+y^2}}=\lim_{(x,y)\to (0,0)}\frac{\sin(xy)}{xy}\frac{xy}{\sqrt{x^2+y^2}}=\lim_{(x,y)\to(0,0)}\frac{ xy }{ \sqrt{x^2+y^2} }=0.
    \end{equation}
    Le fait que la limite finale soit zéro peut être vu en passant aux coordonnées polaires. La fonction $f$ est donc continue à l'origine.

	\item
		La fonction est continue, pour le voir il suffit de passer aux coordonnées sphériques.  
  \end{enumerate}
  

\end{corrige}
