% This is part of Exercices et corrections de MAT1151
% Copyright (C) 2010
%   Laurent Claessens
% See the file LICENCE.txt for copying conditions.

\begin{corrige}{SerieUn0009}

	Nous devons d'abord calculer le supremum de
	\begin{equation}
		\begin{aligned}[]
			\frac{ |(d-1)^2-(d_0-1)^2| }{ | d-d_0 | }&=\frac{ | (d+d_0-2)(d-d_0) | }{ | d-d_0 | }\\
				&=| d+d_0-2 |.
		\end{aligned}
	\end{equation}
	Si $d-2>0$, alors le supremum est atteint en $d=d_0+\eta$ et alors on a
	\begin{equation}
		K_{\text{abs}}=2d_0+\eta-2
	\end{equation}
	et
	\begin{equation}
		K_{\text{rel}}=\frac{ d_0(2d_0+\eta-2) }{ (d_0-1)^2 }.
	\end{equation}

	Si on a $d-2<0$, alors le supremum est atteint en $d=d_0-\eta$ et alors
	\begin{equation}
		K_{\text{abs}}=| 2d_0-\eta-2 |.
	\end{equation}
	Notez que dans ce cas ci, nous n'avons pas de garanties sur le signe de ce qui se trouve dans la valeur absolue.

\end{corrige}
