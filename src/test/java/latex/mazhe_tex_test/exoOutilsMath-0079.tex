% This is part of Exercices et corrigés de CdI-1
% Copyright (c) 2011
%   Laurent Claessens
% See the file fdl-1.3.txt for copying conditions.

\begin{exercice}\label{exoOutilsMath-0079}

    Un fil électrique rectiligne suivant l'axe $z$ est parcouru d'un courant électrique $Ie_z$ où $I$ est une constante. Soit $p\in\eR^3$. Le valeur du champ magnétique en $p$ est donnée par
    \begin{equation}
        B(p)=\frac{ I\times d }{ \| d \|^2 }
    \end{equation}
    où $d$ est le vecteur qui relie le point $p$ au fil.

    \begin{enumerate}
        \item
            Faire un dessin de la situation; indiquer le vecteur $d$.
        \item 
            Calculer le la divergence et le rotationnel de $B$.
    \end{enumerate}

\corrref{OutilsMath-0079}
\end{exercice}
