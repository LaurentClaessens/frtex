% This is part of Exercices et corrigés de CdI-1
% Copyright (c) 2011
%   Laurent Claessens
% See the file fdl-1.3.txt for copying conditions.

\begin{exercice}\label{exo0039}

Donnez un exemple d'une fonction $f\colon \eR\to \eR$ partout discontinue et d'une fonction non constante et partout continue $g\colon \eR\to \eR$ telles que $g\circ f$ est partout continue (Aide~: considérez $g(x)=|x|$).

\corrref{0039}
\end{exercice}
