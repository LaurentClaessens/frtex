% This is part of (almost) Everything I know in mathematics and physics
% Copyright (c) 2013-2014
%   Laurent Claessens
% See the file fdl-1.3.txt for copying conditions.

\section{Characterization by angles in \texorpdfstring{$SO(l-1)$}{SOl-1}}
%++++++++++++++++++++++++++++++++++++++++++++++++++++++++++++++++++++++++++

Let $D[g]$ be the set of light-like directions (vectors in $\SO(n)$) for which the point $[g]$ falls into $\hS_{AN}$. Similarly, the set $\overline{D}[g]$ is the one of directions which fall into $\hS_{A \bar{N}}$. One can express $\overline{ D }$ in terms of $D$:
\[
\begin{split}
\overline{ D }[g]&=\{ k\in\SO(n)\tq\exists t\text{ for which }\pi\big( g e^{t\Ad(k)E_1} \big)\in\hS_{A\bar N} \}\\
		&=\{ k\in\SO(n)\tq\exists t\text{ for which }\pi\big( \theta(g)\theta( e^{tAd(k)E_1}) \big)\in\hS_{AN}\}\\
		&=\{ k\in\SO(n)\tq \pi(k)\in D\big( \theta[g] \big)\}\\
		&=\{ k\in\SO(n)\tq k\in\big( D(\theta[g]) \big)_{\theta}\},
\end{split}
\]
So
\begin{equation} \label{eq:DbarD}
\overline{D}[g]=(D\theta[g])_{\theta}
\end{equation}
where by definition, $k_{\theta}=Jk$ with $J$ being defined by $\theta=\Ad(J)$ ($\theta$ is the Cartan involution). It is easy to see that $\theta$ changes the sign of the spacial part of $k$, i.e. changes $w_i\to -w_i$.

\begin{probleme}
C'est la même chose qu'un autre problème que de voir l'involution de Cartan comme un automorphisme interne.
\label{propCrtadeux}
\end{probleme}

 A main property of $k_{\theta}$ is
\[
	\theta(\Ad(k)E_1)=\Ad(k_{\theta})E_1.
\]
Since $k_{\theta}$ only appears in the expression $\Ad(k)E_1$, that property is actually a sufficient characterization of $k_{\theta}$ for our purpose. In particular, $k_{\theta\theta}\neq k$, but $\Ad(k_{\theta\theta})E_1=\Ad(k)E_1$.

How to express the condition $g\in\hH$ in terms of $D[g]$ ? The condition to belong to the black hole is $D[g]\cup \overline{D}[g]=\SO(n)$. If the complementary of $D[g]\cup \overline{D}[g]$ has an interior (i.e. if it contains an open subset), then by continuity the complementary $D[g']\cup \overline{D}[g']$ has also an interior for all $[g']$ near from $[g]$. In this case, $[g]$ cannot belong to the horizon. So a characterization of $\hH$ is the fact that the boundary of $D[g]$ and $\overline{D}[g]$ coincide. Equation \eqref{eq:DbarD} expresses this condition under the form
\begin{equation}
  \Fr D[g]=\Fr \big( D(\theta[g])\big)_{\theta},
 \end{equation}
from which one immediately deduces that $\hH$ is $\theta$-invariant.

We have an expression of $D[\mu]$ for $\mu\in \SO(2)$ by examining equations \eqref{eq:tempssingul}. The set $D[\mu]$ is the set of $w_2\in [-1,1]$ such that $\cos \mu+w_2>0$:
\begin{equation}
  D[\mu]=]-\cos \mu,1[.
\end{equation}
So in order for $\mu$ to belong to $\hH$, the point $[\mu]$ must satisfy
\[
\overline{D}[\mu]=D[\theta \mu]_{\theta}=]-1,-\cos \mu[.
\]
Consequently, if $\mu'$ is the $K$-component of $\theta \mu$ in the $ANK$ decomposition, we impose $]-\cos \mu',1[=D[\theta \mu]\stackrel{!}{=}]-\cos \mu',1[$\,, and we can describe the horizon by
\begin{equation} \label{eq:caractcous}
\cos \mu=-\cos \mu'
\end{equation}
where $\mu'$ is the $K$-component of $\mu$ in the $A\bar{N}K$ decomposition.


\subsection{Another (useless) characterisation}
%----------------------------------------------

A way to express our characterization \eqref{eq:caractcous} is $ank=a'\overline{n}k'$ with $k'=e^{i\pi}k^{-1}$. We know\quext{Mais faudra lire Helgason hein.} that $NA\bar{N}$ is dense in $G$. Let $k_0\in \SO(2)$ and $m=k_0^2e^{i\pi}$. We define $n,n'\in N$, $a\in A$ such that $m=n^{-1} a\theta(n')$.\quext{Il faudra voir si le coup de la densit\'e fait quelque chose dans cette histoire}. For this $n$, the point $[k_0n]$ belongs to the horizon because
\begin{equation}
nk_0=a\theta(n')m^{-1} k_0
    =a\theta(n')e^{-i\pi}k_0^{-1}.
\end{equation}
Then this $nk$ reads in decomposition $A\bar{N}K$ with $k'=e^{-i\pi}k^{-1}$. Then (almost) all element in $\SO(2)$ give rise to an element in $\hH$.


%+++++++++++++++++++++++++++++++++++++++++++++++++++++++++++++++++++++++++++++++++++++++++++++++++++++++++++++++++++++++++++
					\section{Characterisation as orbit of group (by the equation)}
%+++++++++++++++++++++++++++++++++++++++++++++++++++++++++++++++++++++++++++++++++++++++++++++++++++++++++++++++++++++++++++
\label{SecHOrOrbEquation}

This section proves that, if we embed $AdS_3$ in $AdS_4$, one can express the horizon in $AdS_4$ as the result of the action of a one dimensional group on the horizon of $AdS_3$ (seen in $AdS_4$), theorem \ref{ThoEqHorQCoore}.

%---------------------------------------------------------------------------------------------------------------------------
					\subsection{The old three dimensional case}
%---------------------------------------------------------------------------------------------------------------------------

As mentioned in \cite{Keio}, the singularity of the three dimensional black hole in $AdS_3$ (seen as the group $\SL(2,\eR)$) accepts a nice description as lateral classes of $AN$ and $A\bar N$. That description is recalled in the proposition \ref{PropLatClassANSLdeuxR}. We want here to provide a similar description for the dimensional generalization $AdS_l=\SO(2,l-1)/\SO(1,l-1)$. 

Let us first make a simple remark. A lateral class in the description of proposition \ref{PropLatClassANSLdeuxR} is not guaranteed to be a lateral class in the description $AdS=G/H$. Moreover the ``$AN$'' of equation  \eqref{EqHorClassLatdeux} is not the ``$AN$'' of $\SO(2,2)$, but the one of $\SL(2,\eR)$. The results from the description $AdS_3=\SL(2,\eR)$ cannot be that simply translated into results in the description of $AdS_3=\SO(2,2)/\SO(1,2)$.

Let us begin by finding a group description of the horizon in $AdS_3$ in the description $AdS_3=\SO(2,2)/SO(1,2)$. The matricial expression of $ANJ$ in $AdS_3=\SL(2,\eR)$ is
\begin{equation}		\label{EqProSLJANexp}
\begin{pmatrix}
	e^a	&	le^a	\\ 
	0	&	 e^{-a}	
\end{pmatrix}
\begin{pmatrix}
	0	&	1	\\ 
	-1	&	0	
\end{pmatrix}
=
\begin{pmatrix}
	-le^a	&	e^a	\\ 
	- e^{-a}	&	0	
\end{pmatrix}
\end{equation}
The part of the hyperboloid described by these matrices is obtained by equating \eqref{EqProSLJANexp} with the matrix
\begin{equation}		\label{EqIdentMatriSLAdS}
	\begin{pmatrix}
	u+x	&	y+t	\\ 
	y-t	&	u-x	
\end{pmatrix}.
\end{equation}
The result is the vectors of the form
\begin{equation}		\label{EqVectoPotementSingAN}
\psi\big( Z(G)ANJ \big)\leadsto
	\begin{pmatrix}
	u	\\ 
	t	\\ 
	x	\\ 
	y	
\end{pmatrix}=
\pm
\begin{pmatrix}
	-\frac{ 1 }{2}e^al	\\ 
	\cosh(a)	\\ 
	-\frac{ 1 }{2}e^al	\\ 
	\sinh(a)	
\end{pmatrix}
=
\pm
\begin{pmatrix}
	\alpha	\\ 
	\cosh(a)	\\ 
	\alpha	\\ 
	\sinh(a)	
\end{pmatrix}
=\pm r_{AN}
\end{equation}
with $\alpha$, $a\in\eR$. This is a (almost\footnote{We did not compute the $A\bar N$ part of the horizon in $\SL(2,\eR)$.}) general vector of $AdS_3$ with $u^2-x^2=0$, which is coherent with the description \eqref{BTZSingHor}.

The same computation, using \eqref{EqGeneANbarSLdeuxR}, shows that the other part of the horizon in $AdS_3$ is given by
\begin{equation}		\label{EqVectoPotementSingANbar}
\psi\big( Z(G)A\bar NJ\big)
=
\pm\psi
\begin{pmatrix}
	0	&	e^a	\\ 
	- e^{-a}	&	l e^{-a}	
\end{pmatrix}
\leadsto
\begin{pmatrix}
	u	\\ 
	t	\\ 
	x	\\ 
	y	
\end{pmatrix}=
\pm
\begin{pmatrix}
	\frac{1}{ 2 } e^{-a}l	\\ 
	\cosh(a)	\\ 
	-\frac{1}{ 2 } e^{-a}l	\\ 
	\sinh(a)	
\end{pmatrix}
=
\pm
\begin{pmatrix}
	\alpha	\\ 
	\cosh(a)	\\ 
	-\alpha	\\ 
	\sinh(a)	
\end{pmatrix}
=\pm
r_{A\bar N}
\end{equation}
where $a$ and $\alpha$ are running over $\eR$.

From the equations \eqref{EqVectoPotementSingAN} and \eqref{EqVectoPotementSingANbar}, we are able to express the horizon in $AdS_3$ as union of lateral classes of the element
\begin{equation}
	b=\begin{pmatrix}
		0	\\ 
		1	\\ 
		0	\\ 
		0	
	\end{pmatrix}.
\end{equation}
It is, indeed, easy to see that $G_{ X_{(-1,1)},J_1}\cdot b =G_{ X_{(1,1)},J_1}\cdot b$ and $G_{ J_1,X_{(1,-1)},J_1}\cdot b=G_{ J_1,X_{(-1,-1)} ,J_1}\cdot b$. We can express the horizon $\hH_3$ in the following way :
\begin{equation}
	\begin{aligned}[]
		\hH_3	&=\pm G_{ X_{(-1,1)},J_1}\cdot b\cup \pm G_{ X_{(1,-1)},J_1}\cdot b  \\
			&=\pm G_{ \{J_1,X_{(1,1)}\}}\cdot b\cup \pm G_{ \{J_1,X_{(-1,-1)}\}}\cdot b,
	\end{aligned}
\end{equation}
and the two other combinations. Here, $G_{X,Y}$ is the group generated by $X$ and $Y$.

%---------------------------------------------------------------------------------------------------------------------------
\subsection{Characterization by induction on the dimension}
%---------------------------------------------------------------------------------------------------------------------------

From a computational point of view, it reveals to be more or less impossible to directly check that \eqref{EqVectoPotementSingAN} belongs to the singularity using the method of equation \eqref{EqhohnCondHOrExpl}, not even in dimension $4$. Here is the strategy to compute the horizon in higher dimension:
\begin{enumerate}
\item
The map $\psi\colon \SL(2,\eR)\to AdS_3$ given by \eqref{EqIdentMatriSLAdS} is an isometry which maps the singularity into the singularity. Thus it has to map the horizon to the horizon. If $\hH_{\SL(2,\eR)}$ denotes the horizon in $\SL(2,\eR)$, then the set $\psi\big( \hH_{\SL(2,\eR)} \big)$ is the horizon in $AdS_3=\SO(2,2)/SO(2,1)$.

\item
We consider the inclusion $\iota\colon \SO(2,n)\to \SO(2,n+1)$ given by $g\mapsto\begin{pmatrix}
	g	&	0	\\ 
	0	&	1	
\end{pmatrix}$ and its differential $d\iota\colon \so(2,n)\to \so(2,n+1)$, $X\mapsto\begin{pmatrix}
	X	&	0	\\ 
	0	&	0	
\end{pmatrix}$. Now, we are going to build the horizons of $AdS_l$ by induction over $l$, starting on $l=3$.
\end{enumerate}

We denote by $\hH_l$ and $\hS_l$ the horizon and the singularity in $AdS_l$. The structure of the algebras (equations \eqref{EqLeANEnDimAlg} and \eqref{EqTableSOIwa}) show immediately that
\begin{equation}
	(\sA\oplus\sN)_{\so(2,n+1)}=\Span\left\{   d\iota(\sA\oplus\sN)_{\so(2,n)},V_{n+2},W_{n+2}  \right\},
\end{equation}
so that the structure of one dimension is defined from the structure of the previous one by adding the two new vectors $V$ and $W$. The same holds for $\sA\oplus\bar\sN$.


Now, the work is to find what is \emph{added} to the horizon when one passes from one dimension to the higher one. From that point of view, the matrix $V_i$ has a wonderful property: $ e^{V}$ does not change the $t$ and $y$ component of the vector on which it acts. Thus we have the following.
\begin{lemma}		\label{LemHorpigeVDdeux}
We have
\begin{equation}
	\pi(g e^{-s\Ad(k)E_1})\in\hS
\end{equation}
if and only if
\begin{equation}
	 \pi(e^{V}g e^{-s\Ad(k)E_1})\in\hS.
\end{equation}
The same holds replacing $V$ by $X$.
\end{lemma}

\begin{proof}
The exponential of the matrix $V_5$ is given in equation \eqref{EqExpDeV}. The second and fourth column being the identity, $e^V1_t=1_y$ and $e^V1_y=1_y$. Thus the characterisation $t^2-y^2=0$ of the singularity is satisfied for one point $x\in AdS$ if and only if it is satisfied by the point $e^Vx$.
\end{proof}

	
We consider the following points in the horizon:
\begin{equation}		\label{EqPartewWrAN}
	\begin{aligned}[]
	r(a,\alpha,w)&= e^{wW}r_{AN}=
\frac{ 1 }{2}
\begin{pmatrix}
	2\alpha	\\ 
	e^{-a}w^2+2\cosh(a)\\ 
	2\alpha	\\ 
	e^{-a}w^2+2\sinh(a)	\\ 
	2 e^{-a}w	
\end{pmatrix},\\
	\bar r(a,\alpha,w)&= e^{wW}r_{A\bar N}=
\frac{ 1 }{2}
\begin{pmatrix}
	2\alpha	\\ 
	 e^{-a}w^2+2\cosh(a) \\ 
	-2\alpha	\\ 
	e^{-a}w^2+2\sinh(a)	\\ 
	 2e^{-a}w	
\end{pmatrix}.
	\end{aligned}
\end{equation}

The tangent vectors of that surface are given by
\begin{equation}
	\begin{aligned}[]
		(\partial_ar)(a,\alpha,w)&=
\begin{pmatrix}
	0	\\ 
	\frac{ - e^{-a}w^2+2\sinh(a) }{2}	\\ 
	0	\\ 
	\frac{ - e^{-a}w^2+2\cosh(a) }{2}	\\ 
	- e^{-a}w	
\end{pmatrix}
,&
		(\partial_{\alpha}r)(a,\alpha,w)&=
\begin{pmatrix}
	1	\\ 
	0\\ 
	1	\\ 
	0\\ 
	0	
\end{pmatrix}
,&
		(\partial_wr)(a,\alpha,w)&=
\begin{pmatrix}
	0	\\ 
	 e^{-a}w\\ 
	0	\\ 
	 e^{-a}w\\ 
	e^{-a}
\end{pmatrix}
	\end{aligned}.
\end{equation}
Notice that these three vectors are nowhere vanishing. It is immediate that the vector $\partial_{\alpha}r$ is linearly independent of $\partial_{a}r$ and of $\partial_wr$. It is also immediately apparent that $\partial_ar=-w\partial_wr$ is the worse possible situation. It is, however, not possible because it would imply that 
\begin{equation}
	\begin{aligned}[]
		-w^2 e^{-a}&=\frac{ - e^{-a}w^2+2\sinh(a) }{2}&\text{and}&&-w^2 e^{-a}&=\frac{ - e^{-a}w^2+2\cosh(a) }{2},
	\end{aligned}
\end{equation}
which is only possible when $\cosh(a)=\sinh(a)$, in other words : never. Thus, the part of $AdS_4$ described by \eqref{EqPartewWrAN} has dimension $3$.


\begin{proposition}
We have
\begin{equation}
	 G_W\cdot\iota(\hH_3)=
	\{ r(a,\alpha,w)\cup\bar r(a,\alpha,w) \}_{a,\alpha,w\in\eR}.	
\end{equation}
\end{proposition}

\begin{proof}
The facts that $G_W\cdot\iota(\hH_3)=\{ r(a,\alpha,w)\cup\bar r(a,\alpha,w) \}_{a,\alpha,w\in\eR}$ and that all the elements of that set are subject to $u^2-x^2=0$ are by construction.

We still have to prove that $\{ u^2-x^2=0\}\subseteq G_W\cdot\iota(\hH_3)$.

Let $v=(y,t,x,y,z)$ be a vector which satisfies $u^2-x^2=0$. Following the signs of $u$ and $t$, we are searching $v$ under the form $\pm r(\alpha,a,w)$ or $\pm \bar r(\alpha,a,w)$. In any case, the value of $u$ and $x$ fix $\alpha$ and we are left with the condition
\begin{equation}
\pm\frac{ 1 }{2}
	\begin{pmatrix}
	e^{-a}w^2+2\cosh(a)	\\ 
	e^{-a}w^2+2\sinh(a)	\\ 
	2 e^{-a}w	
\end{pmatrix}
=
\begin{pmatrix}
	t	\\ 
	y	\\ 
	z	
\end{pmatrix}
\end{equation}
with $t^2-y^2-z^2=1$.  If $t-y>0$, we choose the sign $+$ and the value of $t-y$ fixes $a$ because $t-y= e^{-a}$. In that situation, $w$ is given by $w= e^{a}\big(2y-2\sinh(a)\big)$. If $t-y<0$, we have $t-y=- e^{-a}$ and the same argument holds.

\end{proof}

In the sequel, we will use the following notations :
\begin{equation}
	\begin{aligned}[]
		G_W&=\{  e^{wW}\tq w\in\eR \}\\
		G_V&=\{  e^{\alpha V}\tq \alpha\in\eR \}\\
		G_X&=\{  e^{\beta X}\tq \beta\in\eR \}\\
		G_Y&=\{  e^{y Y}\tq y\in\eR \}
	\end{aligned}
\end{equation}
These are one parameter subgroups of $SO(2,3)$.

\begin{proposition}		\label{PropInclusionsTroisQuatreWVXY}
If $v\in AdS_4$ satisfies $u-t\neq 0$, then $v= e^{wW}v'$ for a certain $v'\in AdS_3$. In other words,
\begin{equation}
	\{ y-t\neq 0 \}_4\subset G_W\cdot\iota(AdS_3).
\end{equation}
In particular, every points outside the singularity $\hS_4$ are obtained by action of $G_W$ on a point of $AdS_3$. We also have
\begin{equation}
	\begin{aligned}[]
		\{ x-u\neq 0 \}_4&\subseteq G_V\cdot\iota(AdS_3)\\
		\{ x+u\neq 0 \}_4&\subseteq G_X\cdot\iota(AdS_3).
	\end{aligned}
\end{equation}
\end{proposition}

\begin{proof}
We have
\begin{equation}
	 e^{wW}
\begin{pmatrix}
	u'	\\ 
	t'	\\ 
	x'	\\ 
	y'	\\ 
	0	
\end{pmatrix}=
\begin{pmatrix}
	u'	\\ 
	\left( 1+\frac{ w^2 }{2} \right)t'-\frac{ w^2 }{2}y'	\\ 
	x'	\\ 
	\frac{ w^2 }{2}t'+\left( 1-\frac{ w^2 }{2} \right)y'	\\ 
	w(t'-y')	
\end{pmatrix}
=
\begin{pmatrix}
	u	\\ 
	t	\\ 
	x	\\ 
	y	\\ 
	z	
\end{pmatrix}
\end{equation}
when
\begin{equation}
	\begin{aligned}[]
		u'&=u,& t'&=\frac{ z^2+2ty-2t^2 }{ 2(y-t) },&x'&=x,&y'&=\frac{ z^2-2ty+2y^2 }{ 2(y-t) },&w&=-\frac{ z }{ y-t }.
	\end{aligned}
\end{equation}
In the same way, the equation
\begin{equation}
	 e^{\alpha V}\begin{pmatrix}
	u'	\\ 
	t'	\\ 
	x'	\\ 
	y'	\\ 
	0	
\end{pmatrix}=
	 \begin{pmatrix}
	u	\\ 
	t\\ 
	x	\\ 
	y\\ 
	z	
\end{pmatrix}
\end{equation}
is solved by
\begin{equation}
	\begin{aligned}[]
		u'&=\frac{ z^2+2ux-2u^2 }{ 2(x-u) },&x'&=\frac{ z^2-2ux+2x^2 }{ 2(x-u) },&\alpha&=-\frac{ z }{ x-u }.
	\end{aligned}
\end{equation}
Thus, $\{ x-u\neq 0 \}_4\subseteq G_V\cdot\iota(AdS_3)$. And, finally, the equation
\begin{equation}
	 e^{\beta X}\begin{pmatrix}
	u'	\\ 
	t'	\\ 
	x'	\\ 
	y'	\\ 
	0	
\end{pmatrix}=
	 \begin{pmatrix}
	u	\\ 
	t	\\ 
	x	\\ 
	y	\\ 
	z	
\end{pmatrix}
\end{equation}
is solved by
\begin{equation}
	\begin{aligned}[]
		u'&=\frac{ z^2-2ux-2u^2 }{ 2(x+u) },&x'&=\frac{ z^2+2ux+2x^2 }{ 2(x+u) },&\beta&=-\frac{ z }{ x+u }.
	\end{aligned}
\end{equation}
Thus, $\{ x+u\neq 0 \}_4\subseteq G_X\cdot\iota(AdS_3)$. 

\end{proof}

One interest of that proposition resides in the fact that every element of $AdS_4$ outside the singularity is the image of an element of $AdS_3$ by $G_W$.


\begin{proposition}		\label{PropSingQTiV}
We have 
\begin{equation}
	\hS_4=G_V\cdot\iota(\hS_3)
\end{equation}
where $G_V=\{  e^{vV}\tq v\in\eR \}$ is the group generated by $V$.
\end{proposition}

\begin{proof}
A point of $\iota(\hS_3)$ is of the form
$	\begin{pmatrix}
	u	\\ 
	\alpha	\\ 
	x	\\ 
	\epsilon\alpha	\\ 
	0	
\end{pmatrix}
$, while an element of $\hS_4$ is of the form
$
	\begin{pmatrix}
	u'	\\ 
	\alpha	\\ 
	x'	\\ 
	\epsilon\alpha	\\ 
	z'	
\end{pmatrix}
$ where $\epsilon=\pm 1$. So we have to solve the equation
\begin{equation}
	 e^{vV}
\begin{pmatrix}
	u	\\ 
	\alpha	\\ 
	x	\\ 
	\epsilon\alpha	\\ 
	0	
\end{pmatrix}=
\begin{pmatrix}
	u\left( \frac{ v^2 }{ 2 }+1 \right)+\frac{ v^2 }{2}x	\\ 
	\alpha	\\ 
	\left( 1-\frac{ v^2 }{2} \right)x+\frac{ v^2 }{2}u	\\ 
	\epsilon\alpha	\\ 
	v(u-x)	
\end{pmatrix}
\stackrel{!}{=}
\begin{pmatrix}
	u'	\\ 
	\alpha	\\ 
	x'	\\ 
	\epsilon\alpha	\\ 
	z'	
\end{pmatrix}
\end{equation}
with respect to $v$, $u$ and $x$. A solution is given by
\begin{equation}
	\begin{aligned}[]
		u&=\frac{ z'^2+2u'x'-2u'^2 }{ 2(x'-u') },&x&=\frac{ z'^2-2u'x'+2x'^2 }{ 2(x'-u') },&v&=-\frac{ z' }{ x'-u' }.
	\end{aligned}
\end{equation}
The condition $u'^2-x'^2-z'^2=1$ imposes $x'\neq u'$, so that that solution always makes sense: a point of $\hS_4$ is always obtained as the result of the action of an element of $G_V$ on an element of~$\hS_3$.

Since the operator $ e^{vV}$ does not touch the variables $t$ and $y$, it is obvious that $G_V\cdot \hS_3\subseteq\hS_4$.
\end{proof}

\begin{lemma}		\label{LemTNTroisIneq}
In $AdS_3$, the black hole is given by $u^2-x^2>0$
\end{lemma}

\begin{proof}
The black hole is the set of point from which every light ray intersect the singularity. The boundary of that set is given by the horizon (this is the definition of the horizon), and we already proved that $\hH_3\equiv u^2-x^2=0$. Thus the black hole is $u^2-x^2>0$, or $u^2-x^2<0$. Since the singularity (which is part of the black hole) is given by $t^2-y^2=0$, the singularity satisfies $u^2-x^2=1$, and is thus in the part $u^2-x^2>0$.
\end{proof}


Let $TN[g]$ be the subset of $\{ \Ad(k)E_1 \}_{k\in \SO(3)}$ of elements for which there exists a $s>0$ such that
\begin{equation}
	\pi(g e^{s\Ad(k)E_1})\in\hS.
\end{equation}
In other words, $TN[g]$ is the set of directions along which $[g]$ falls in the singularity. If the complementary $TN[g]^c$ has a non empty interior, the by continuity, the complementary $TN[g']$ will have an interior as well for every $[g']$ close enough from $[g]$. In that case, $[g]$ does not belongs to the horizon. So a point belongs to the horizon when the set of safe direction has no interior.


\begin{lemma}
We have
\begin{equation}
	G_V\cdot\iota(\hH_3)\equiv u^2-x^2-z^2=0,
\end{equation}
so that it is the good candidate to be the horizon.
\end{lemma}

\begin{proof}
An element of $\iota(\hH_3)$ has the form
$r=\begin{pmatrix}
	u'	\\ 
	t'	\\ 
	x'	\\ 
	\pm\sqrt{t'^2-1}	\\ 
	0	
\end{pmatrix}$,
so that we have to solve the equation
\begin{equation}
	 e^{vV}r=\begin{pmatrix}
	\left( \frac{ v^2 }{ 2 }+1 \right)u'-\frac{ v^2 }{ 2 }x'	\\ 
	t'	\\ 
	\left( 1-\frac{ v^2 }{ 2 } \right)x'+\frac{ v^2 }{ 2 }u'	\\ 
	\pm\sqrt{t'^2-1}	\\ 
	v(u'-x')	
\end{pmatrix}
=
\begin{pmatrix}
	u	\\ 
	t	\\ 
	x	\\ 
	\pm\sqrt{t^2-1}	\\ 
	z	
\end{pmatrix}.
\end{equation}
The solution is
\begin{equation}
	\begin{aligned}[]
		u'&=\frac{ z^2+2ux-2u^2 }{ 2(x-u) },&x'&=\frac{ z^2-2ux+2x^2 }{ 2(x-u) },&v&=-\frac{ z }{ x-u }.
	\end{aligned}
\end{equation}
Since $u^2-x^2-z^2=1$, we have $x-u\neq 0$, so that these solutions always make sense.
\end{proof}

\begin{lemma}
If $[g]=\begin{pmatrix}
	u	\\ 
	t	\\ 
	x	\\ 
	y	\\ 
	z	
\end{pmatrix}\in AdS_4$ with $u$ and $x$ not both vanishing, then 
\begin{equation}
	[g]\in G_V\cdot\iota(AdS_3)\cup G_X\cdot \iota(AdS_3).
\end{equation}
Notice that the union is not disjoint.
\end{lemma}

\begin{proof}
The proof is a simple computation. Following proposition \ref{PropInclusionsTroisQuatreWVXY}, we have $\{ x-u\neq 0 \}_4\subseteq G_V\cdot\iota(AdS_3)$ and $\{ x+u\neq 0 \}_4\subseteq G_X\cdot\iota(AdS_3)$.

So the only part of $AdS_4$ which is not included in $G_V\cdot \iota(AdS_3)\cup G_X\cdot\iota(AdS_3)$ is the part where $x+u=x-u=0$.
\end{proof}

Now, we want to study the horizon, that means the boundary of $BH_4$. If $v\in\partial\big(\Adh(BH_4) \big)$, there exists, in any neighbourhood of $v$, an element $\bar v$ and a direction following which the geodesic from $\bar v$ escapes the singularity.

Up to now, we studied the way $AdS_3$ embed in $AdS_4$. In particular, we proved that the horizon of $AdS_3$ is included in the horizon of $AdS_4$. We can propagate the results by $G_V$ and $G_X$ because, given a $v\in AdS_3$, the existence of a $\alpha$ such that $ e^{\alpha V}v\in\iota(AdS_3)$ or $ e^{\alpha X}v\in\iota(AdS_3)$ is related to the fact that $u^2-x^2\neq 0$, while that condition holds in a neighbourhood of $v$. 

%klklklmkmlkklmmlkkmlkmùlmklkll
%+++++++++++++++++++++++++++++++++++++++++++++++++++++++++++++++++++++++++++++++++++++++++++++++++++++++++++++++++++++++++++
\section{Organization of the next few pages}
%+++++++++++++++++++++++++++++++++++++++++++++++++++++++++++++++++++++++++++++++++++++++++++++++++++++++++++++++++++++++++++

\begin{abstract}
	This paper is a sequel of \emph{Solvable symmetric black hole in anti de Sitter spaces} \cite{lcTNAdS}. In the latter, we described the BTZ black hole in every dimension by defining the singularity as the closed orbits of the Iwasawa subgroup of $\SO(2,n)$. In this article, we study the horizon of the black hole and we show that it is expressed as lateral classes of one point of the space. The computation is given in the four-dimensional case, but it makes no doubt that it can be generalized to any dimension.

	The main idea is to define an ``inclusion map'' from $AdS_3$ into $AdS_4$ and to show that all the relevant structure pass trough the inclusion. We prove, for example, that the inclusion of the three dimensional horizon into $AdS_4$ belongs to the four dimensional horizon : $\iota(\hH_3)\subseteq\hH_4$ and then we deduce the expression of the horizon in $AdS_4$.
\end{abstract}

In section \ref{SecOldResults}, we describe some old results about BTZ black hole.

In subsection \ref{SubSecHorInThreeDimensionOld}, we recall how we proved the existence of the black hole structure in \cite{lcTNAdS} and how the horizon was described in the three dimensional case in \cite{Keio}. We adapt the latter result in our homogeneous space setting.

The subsection \ref{subSecTopoHor} gives some topological remarks about the black hole and the horizon. We point out that there are some light-like geodesics that are intersecting the singularity \emph{and then} the free part later in the future. We explain why that circumstance is very different from the situation of the most famous black holes in physics like the Schwarzschild's one.

Section \ref{SecNewWithMatrices} is devoted to the proof of our main result: the horizon of the BTZ black hole in $AdS_4$ is given by
\begin{equation}
	\hH_4=G_{X_{0+}}\cdot \iota(\hH_3)\cup G_{X_{0-}}\iota(\hH_3).
\end{equation}
where $\iota$ is the inclusion of $AdS_3$ in $AdS_4$ and $\hH_3$ is the horizon of the BTZ black hole in $AdS_3$.

%+++++++++++++++++++++++++++++++++++++++++++++++++++++++++++++++++++++++++++++++++++++++++++++++++++++++++++++++++++++++++++
\section{Some old results}
%+++++++++++++++++++++++++++++++++++++++++++++++++++++++++++++++++++++++++++++++++++++++++++++++++++++++++++++++++++++++++++
\label{SecOldResults}

From the results of section \ref{SecExistenceHor}, we know that a non trivial horizon exists. However, the question of the structure of the horizon was not yet addressed. This is what we are going to do now.

%---------------------------------------------------------------------------------------------------------------------------
\subsection{Horizon in the three dimensional case}
%---------------------------------------------------------------------------------------------------------------------------
\label{SubSecHorInThreeDimensionOld}

The structure of the horizon of $AdS_3$ was described in \cite{Keio} in the setting of $AdS_3=\SL(2,\eR)$. Our first job is to translate that result into the language of quotient of groups. This is done by the identification
\begin{equation}
	\begin{aligned}
		\psi\colon \SL(2,\eR)&\to AdS_3 \\
		\begin{pmatrix}
			u+x	&	y+t	\\ 
			y-t	&	u-x	
		\end{pmatrix}&\mapsto \begin{pmatrix}
			u	\\ 
			t	\\ 
			x	\\ 
			y	
		\end{pmatrix}.
	\end{aligned}
\end{equation}
We see that the points of the horizon are given by
\begin{equation}			\label{EqHOrAdSTroisVecteur}
	\begin{aligned}[]
		\pm\begin{pmatrix}
			\alpha	\\ 
			\cosh(a)	\\ 
			\alpha	\\ 
			\sinh(a)	
		\end{pmatrix}&&\text{and}&&\pm\begin{pmatrix}
			\alpha	\\ 
			\cosh(a)	\\ 
			-\alpha	\\ 
			\sinh(a)	
		\end{pmatrix},
	\end{aligned}
\end{equation}
which correspond to the points $(u,t,x,y)$ such that $u^2-x^2=0$. One should notice that these points can be expressed as lateral classes of the point $b=(0,1,0,0)$~:
\begin{equation}
	\hH_3=\pm G_{\{ J_1,X_{++} \}}b\cup\pm G_{\{ J_1,X_{--} \}}b
\end{equation}
where $G_{\{ X,Y \}}$ is the group of elements of the form $\exp(aX+bY)$. Notice that $G_{\{ J_1,X_{++} \}}b=G_{\{ J_1,X_{-+} \}}b$ and $G_{\{ J_1,X_{--} \}}b=G_{\{ J_1,X_{+-} \}}b$. For example,
\begin{equation}
	e^{aJ_2} e^{\alpha X_{++}}b=\begin{pmatrix}
		\alpha	\\ 
		\cosh(a)	\\ 
		\alpha	\\ 
		\sinh(a)	
	\end{pmatrix}.
\end{equation}
We are now intended to extend that result and express the horizon in $AdS_4$ as lateral classes of the horizon in $AdS_3$. Before to complete that work, we have to make a few remarks about the topology.

%---------------------------------------------------------------------------------------------------------------------------
\subsection{Topology and horizon}
%---------------------------------------------------------------------------------------------------------------------------
\label{subSecTopoHor}

The definition given in the previous sections produces a paradox. Let $x\in AdS$ and $l(s)$ be a light like geodesic trough $x$ which only intersects the singularity in past. We suppose that $l(0)=x$ and that $s_0<0$ is the biggest value of $s$ such that $l(s_0)\in \hS$. Thus, all points of the form $l(s)$ with $s_0<s<0$ are free. That form a sequence of free points which converges to the singularity, and then $l(s_0)$ belongs to the horizon.

This is however not possible in $AdS_3$ because the equation of the singularity is $t^2-y^2=0$ while the equation of the horizon is $u^2-x^2=0$. These two parts are really separated. 

The situation here is really different from the situation in the Schwarzschild's case. In the latter the singularity is well inside the horizon, and there are no geodesics reaching the infinity which have intersected the singularity in the past.

In our case, however, such geodesics do exist. The reason of such a difference resides in the fact that the causal structure (geodesics) are defined by the metric while, in our BTZ black hole, the singularity is not defined from metric considerations. There are thus no reasons to expect some compatibility relations like the fact to have a non naked singularity.

In order to correctly define the horizon, we have to introduce the space $BTZ=AdS\setminus\hS$ which in endowed with the induced topology. Then we define
\begin{equation}
	BH=\{ v\in BTZ\tq\forall k\in \SO(n),\, l_v^k(s)\in\hS\text{ has a solution with $s>0$} \}.
\end{equation}
Let us point out that the singularity itself is not part of the black hole, because it is not even part of $BTZ$. We define the free part of $BTZ$ as the set of points from which there exists a light-like geodesics which does not intersects the singularity in the future:
\begin{equation}
	\hF=\{ v\in BTZ\tq\exists k\in \SO(n),\, l_v^k(s)\in\hS\Rightarrow s<0 \}.
\end{equation}
The first definition makes that the black hole part is open by continuity and compactness of $\SO(n)$ : the minimum and the maximum of time to reach the singularity from one point of the black hole are both strictly positive numbers, and then can be maintained strictly positive in a neighborhood of the point.

\begin{proposition}		\label{PropBHouvertLibreFerme}
	The set of points in the black hole is open and set of free points is closed. In particular, the horizon is contained in the free set.
\end{proposition}

\begin{proof}
	The first point is the remark above. Now, the free part is closed in $BTZ$ as complementary of an open set.	
\end{proof}

The following theorem says that if the set of directions escaping the singularity from a point in $BTZ$ has an interior, then that point does not lies in the horizon. 
\begin{proposition}		\label{PropvFOsvghorvec}
	A point $v\in\hF_l$ such that there is an open set $\mO\subset S^{l-1}$ of directions for which $l^{w}_v(s)\in\hS$ has no solutions for $s\in\eR^+_0$ belongs to $\Int(\hF)$.
\end{proposition}

\begin{proof}
	Using the matricial representation \eqref{eq:AdkE}, we see that a point $v=[g]$ belongs to the singularity if the vector
	\begin{equation}
		g\cdot \begin{pmatrix}
			1	\\ 
			-s	\\ 
			s\bar w	
		\end{pmatrix}
	\end{equation}
	satisfies $t^2-y^2=0$. That equation is a second order polynomial in $s$ whose coefficients cannot be a constant for an open set with respect to $\bar w\in S^{l-1}$. From the assumptions, all the roots of that polynomial belong to $\eC\setminus\eR^+_0$. The latter being open, the roots of $l_{v'}^w(s)\in\hS$ are still in $\eC\setminus\eR^+_0$ when $v'$ runs over a small enough open set around $v$.

	We conclude that $v$ is in the interior of the free zone rather than on the horizon.
\end{proof}

An important characterisation of the horizon, pointed out in \cite{Keio}, is the following.
\begin{theorem}		\label{ThoHorIntDansS}
	A point belongs to the horizon if and only if the set of light-like directions for which the geodesics does not intersects the singularity has no interior in $S^{l-1}$.
\end{theorem}


%+++++++++++++++++++++++++++++++++++++++++++++++++++++++++++++++++++++++++++++++++++++++++++++++++++++++++++++++++++++++++++
\section{The horizon of the BTZ black hole}
%+++++++++++++++++++++++++++++++++++++++++++++++++++++++++++++++++++++++++++++++++++++++++++++++++++++++++++++++++++++++++++
\label{SecNewWithMatrices}

In this section, we show, that the horizon of the horizon of $AdS_4$ can be obtained using the action of a very simple group on the horizon of $AdS_3$, which is, itself, the orbit of one point under a known group. The result opens the possibility of describing the horizon in $AdS_l$ by induction on the dimension, and the possibility to compute the group which generates the horizon.  We define the inclusion map 
\begin{equation}
	\begin{aligned}
		\iota\colon AdS_3&\to AdS_4 \\
		\begin{pmatrix}
			u	\\ 
			t	\\ 
			x	\\ 
			y	
		\end{pmatrix}&\mapsto \begin{pmatrix}
			u	\\ 
			t	\\ 
			x	\\ 
			y	\\ 
			0	
		\end{pmatrix}.
	\end{aligned}
\end{equation}
At the matrix level, it corresponds to add a line and a column of zeros. We will denote by $\hF_l$ the free part of $AdS_l$. By definition, if $v\in\hF_l$, there exists a light like geodesic trough $v$ which does not intersect the singularity in the future. We also denote by $BH_l$ the set of elements of $AdS_l$ from which all the light-like geodesics intersect the singularity in the future.

Notice that $BH_l$ is open while $\hF_l$ is closed, as explained in proposition \ref{PropBHouvertLibreFerme}.

\begin{lemma}		\label{LemOouversttq}
	Let $v\in AdS_4$ and $g\in \SO(2,3)$ be a representative of $v$. If the set
	\begin{equation}
		\{ \begin{pmatrix}
			w_1	\\ 
			w_2	
		\end{pmatrix}\in S^2\tq
		\pi g\begin{pmatrix}
			1	\\ 
			-s	\\ 
			s\bar w	\\ 
			0	
		\end{pmatrix}\cap\hS_4=\emptyset\text{ with $s>0$}
				\}
	\end{equation}
	has an interior in $S^1$, then the set
	\begin{equation}
		\{ 
		\begin{pmatrix}
			w_1	\\ 
			w_2	\\ 
			w_3	
		\end{pmatrix}\in S^2\tq
		\pi g\begin{pmatrix}
			1	\\ 
			-s	\\ 
			s\bar w		
		\end{pmatrix}\cap\hS_4=\emptyset\text{ with $s>0$}
		\}
	\end{equation}
	has an interior in $S^2$.
\end{lemma}

\begin{proof}
The matrix $g$ in $\SO(2,3)$ representing the point $v$ has the form
\begin{equation}
	g=\begin{pmatrix}
 u	&	.	&	.	&	.	&	.\\ 
 t	&	a	&	b	&	c	&	d\\ 
 x	&	.	&	.	&	.	&	.\\ 
 y	&	a'	&	b'	&	c'	&	d'\\ 
z	&	.	&	.	&	.	&	. 
 \end{pmatrix}
\end{equation}
where the numbers $a,b,c,d,a',b',c',d'$ are not uniquely determined. We choose the representative in such a way to have $b\neq \pm b'$, which is always possible.

The assumption is that there exists an open set (with respect to $(w_1,w_2)\in S^1$) around $(w_1,w_2,0)$ such that the path
\begin{equation}		\label{EqPathgexpUTXYZ}
	\pi(g e^{s\Ad\left( k \right)E_1)})=
	\begin{pmatrix}
		U	\\ 
		T	\\ 
		X	\\ 
		Y	\\ 
		Z	
	\end{pmatrix}=
	\begin{pmatrix}
 u	&	.	&	.	&	.	&	.\\ 
 t	&	a	&	b	&	c	&	d\\ 
 x	&	.	&	.	&	.	&	.\\ 
 y	&	a'	&	b'	&	c'	&	d'\\ 
z	&	.	&	.	&	.	&	. 
 \end{pmatrix}
 \begin{pmatrix}
	 1	\\ 
	 -s	\\ 
	 sw_1	\\ 
	 sw_2	\\ 
	 0	
 \end{pmatrix}
\end{equation}
does not intersects the singularity in the future. In other words, we have $T\pm Y=0$ only with $s\leq 0$. Let
\begin{equation}
	\begin{aligned}[]
		T(w_1,w_2)&=t+s(bw_1+cw_2-a)\\
		Y(w_1,w_2)&=y+s(b'w_1+c'w_2-a')\\
		A_+(w_1,w_2)&=(b+b')w_1+(c+c')w_2-(a+a')\\
		A_-(w_1,w_2)&=(b-b')w_1+(c-c')w_2-(a-a').
	\end{aligned}
\end{equation}
We also denote by $\sigma_{\pm}$ the sign of $t\pm y$.

A simple computation shows that $T+Y=0$ when
\begin{equation}
	s=s_+=-\frac{ t+y }{ A_+(w_1,w_2) },
\end{equation}
and $T-Y=0$ when
\begin{equation}
	s=s_-=-\frac{ t-y }{ A_-(w_1,w_2) },
\end{equation}
The assumption is that the direction $(w_1,w_2,0)$ (and an open set in $S^1$ with respect to $(w_1,w_2)$) escapes the singularity, so that for every $(w_1',w_2')$ in a neighborhood of $(w_1,w_2)$, we have
\begin{equation}
	\begin{aligned}[]
		\sigma_{\pm}A_{\pm}(w_1',w_2')\geq 0,
	\end{aligned}
\end{equation}
which assures that the values of $s$ which annihilate $T+Y$ and $T-Y$ are negative or non existing. Since we choose $b\neq \pm b'$, the functions $A_{\pm}$ are nowhere constant, so we can find a direction $(w_1,w_2)$ such that $\sigma_{\pm}A_{\pm}(w_1,w_2)>0$. Notice that, by continuity, there exists a neighbourhood of $(w_1,w_2)$ in $S^1$ which escapes the singularity.

We are now studying what happens when one looks at a neighbourhood of $(w_1,w_2,0)$ in $S^3$. The path \eqref{EqPathgexpUTXYZ} is replaced by
\begin{equation}
	\pi(g e^{s\Ad(k)E_1})= 
	\begin{pmatrix}
 u	&	.	&	.	&	.	&	.\\ 
 t	&	a	&	b	&	c	&	d\\ 
 x	&	.	&	.	&	.	&	.\\ 
 y	&	a'	&	b'	&	c'	&	d'\\ 
z	&	.	&	.	&	.	&	. 
 \end{pmatrix}
\begin{pmatrix}
	1	\\ 
	-s	\\ 
	s(w_1+\epsilon_1)	\\ 
	s(w_2+\epsilon_2)	\\ 
	\epsilon_3	
\end{pmatrix},
\end{equation}
and we consider
\begin{equation}
	\begin{aligned}[]
		T(w_1,w_2,\bar\epsilon)&=t+s\big( b(w_1+\epsilon_1)+c(w_2+\epsilon_2)+d\epsilon_3-a \big)\\
		Y(w_1,w_2,\bar\epsilon)&=y+s\big( b'(w_1+\epsilon_1)+c'(w_2+\epsilon_2)+d'\epsilon_3-a' \big)
	\end{aligned}
\end{equation}
where $\bar\epsilon$ stands for $\epsilon_1$, $\epsilon_2$ and $\epsilon_3$. The same computations as before shows that $T+Y=0$ when
\begin{equation}
	s=s_+=-\frac{ t+y }{ A_+(w_1,w_2)+(b+b')\epsilon_1+(c+c')\epsilon_2+(d+d')\epsilon_3 },
\end{equation}
Since $\sigma_+A(w_1,w_2)>0$, there exists a $\delta$ such that $s_+$ remains negative for every choice of $\bar\epsilon<\delta$. The same holds with $T-Y$ which is zero when
\begin{equation}
	s=s_-=-\frac{ t-y }{ A_-(w_1,w_2)+(b-b')\epsilon_1+(c-c')\epsilon_2 +(d-d')\epsilon_3 }.
\end{equation}
Since $\sigma_-A_-(w_1,w_2)>0$, one can find a $\delta>0$ such that $\bar\epsilon<\delta$ implies that this fraction remains negative.

Thus, there exists a neighbourhood of $(w_1,w_2,0)$ in $S^2$ of directions escaping the singularity from the point $v$.
\end{proof}

\begin{lemma}		\label{LemIntTroisQueatr}
	With the notations defined before, we have
	\begin{equation}
		\iota\big( \Int(\hF_3) \big)\subseteq \Int\big( \hF_4 \big)
	\end{equation}
	where $\Int$ stands for the interior. In other words,
	\begin{equation}
		\Adh(BH_4)\cap\iota(AdS_3)\subset\iota\big( \Adh(BH_3) \big).
	\end{equation}
\end{lemma}

\begin{proof}

	Let $v=\iota(v')\notin\iota\big( \Adh(BH_3) \big)$, we also consider $g'$ a representative of $v'$ and $g=\iota(g')$, which is a representative of $v$. The element $v'$ is in the interior of the free zone: there exists an open set of directions which do not intersect the singularity of $AdS_3$ by theorem \ref{ThoHorIntDansS}. In other words, the set
\begin{equation}		\label{EqwwswswUn}
	\{ \begin{pmatrix}
	w_1	\\ 
	w_2	
\end{pmatrix}\in S^1\tq
\pi g'\begin{pmatrix}
	1	\\ 
	-s	\\ 
	sw_1	\\ 
	sw_2	
\end{pmatrix}\cap\hS_3 =\emptyset\}
\end{equation}
contains an open set of $S^1$. On the other hand, the $z$-component of the latter vector is obviously zero because $g=\iota(g')$ has the form
\begin{equation}
	g=\begin{pmatrix}
 .	&	.	&	.	&	.	&	0\\ 
 .	&	.	&	.	&	.	&	0\\ 
 .	&	.	&	.	&	.	&	0\\ 
 .	&	.	&	.	&	.	&	0\\ 
0	&	0	&	0	&	0	&	1 
 \end{pmatrix},
\end{equation}
thus equation \eqref{EqwwswswUn} can be ``extended'' and there exists an open set in $S^1$ such that
\begin{equation}
	\pi g\begin{pmatrix}
		1	\\ 
		-s	\\ 
		sw_1	\\ 
		sw_2	\\ 
		0	
	\end{pmatrix}\cap\iota(\hS_3)=\emptyset.
\end{equation}
Now, lemma \ref{LemOouversttq} shows that the set
\begin{equation}
	\{ 
		\begin{pmatrix}
			w_1	\\ 
			w_2	\\ 
			w_3	
		\end{pmatrix}\in S^2\tq
		\pi g\begin{pmatrix}
			1	\\ 
			-s	\\ 
			sw_1	\\ 
			sw_2	\\ 
			sw_3	
		\end{pmatrix}\cap\hS_4=\emptyset
	\}
\end{equation}
contains an open subset of $S^2$. That means that $\pi(g)=v$ belongs to the interior of $\hF_4$.
\end{proof}

\begin{proposition}		\label{PropFqTroisFt}
We have $\hF_4\cap\iota(AdS_3)\subset \iota(\hF_3)$.
\end{proposition}

\begin{proof}
Let $v\in\hF_4\cap\iota(AdS_3)$. With the same notations as above, we have
\begin{equation}		\label{EqRepresSOiotag}
	\iota(g')=
\begin{pmatrix}
 u	&	.	&	.	&	.	&	0\\ 
 t	&	a	&	b	&	c	&	0\\ 
 x	&	.	&	.	&	.	&	0\\ 
 y	&	a'	&	b'	&	c'	&	0\\ 
0	&	0	&	0	&	0	&	1 
 \end{pmatrix}
\end{equation}
The assumption is that, for every representative $g'$ of $v'$, there exists a direction $(w_1,w_2,w_3)\in S^2$ such that the path
\begin{equation}		\label{EqGedgpudt}
	\pi   \iota(g')\begin{pmatrix}
	1	\\ 
	-s	\\ 
	sw_1	\\ 
	sw_2	\\ 
	sw_3	
\end{pmatrix} 
\end{equation}
only intersects the singularity fore negative values of $s$. The values of $s$ that annihilate $t^2-y^2$ in the geodesic \eqref{EqGedgpudt} are
\begin{equation}
	\begin{aligned}[]
		s_+	&=-\frac{ t+y }{ -(a+a')+(b+b')w_1+(c+c')w_2 }\\
		s_-	&=-\frac{ t-y }{ -(a-a')+(b-b')w_1+(c-c')w_2 },
	\end{aligned}
\end{equation}
and these two values are either negative either non existing (vanishing denominator).

The work is now to find a direction $(w'_1,w'_2)\in S^1$ such that the geodesic
\begin{equation}
	\pi\big( g'\begin{pmatrix}
	1	\\ 
	-s	\\ 
	sw'_1	\\ 
	sw'_2	
\end{pmatrix} \big)
\end{equation}
does not intersect the singularity. The values of $s$ for which the latter geodesics intersects the singularity are
\begin{equation}
	\begin{aligned}[]
		s'_+	&=-\frac{ t+y }{ -(a+a')+(b+b')w'_1+(c+c')w'_2 }\\
		s'_-	&=-\frac{ t-y }{ -(a-a')+(b-b')w'_1+(c-c')w'_2 }.
	\end{aligned}
\end{equation}
If $w_3=0$, the proposition is true because one can choose $(w'_1,w'_2)=(w_1,w_2)$. If $w_3\neq 0$, the vector $(w_1,w_2)$ does not belong to $S^1$, and we have to find something else.

Let us consider the following two cases.
\begin{enumerate}
\item
there exists a representative \eqref{EqRepresSOiotag} with $a=a'=0$,
\item
there exists a representative \eqref{EqRepresSOiotag} with $c=c'=0$.
\end{enumerate}
In the first case, we have 
\begin{equation}		\label{EqDenoAAnnulerspm}
	s'_{\pm}=-\frac{ t\pm y }{ (b\pm b')w'_1+(c\pm c')w'_2 },
\end{equation}
and we can choose $(w'_1,w'_2)=N(w_1,w_2)$ with $N\in\eR$ fixed in such a way that $(w'_1,w'_2)\in S^1$. Thus we have $s'_{\pm}=\frac{1}{ N }s_{\pm}$ and it is sufficient to choose $N>0$ in order to leave the denominators of \eqref{EqDenoAAnnulerspm} of the right sign or zero.

In the second case, we have
\begin{equation}
	s'_{\pm}=-\frac{ t\pm y }{ -(a\pm a')+(b\pm b')w'_1 },
\end{equation}
thus one has to choose $w'_1=w_1$ and $w'_2=\sqrt{1-w_1^2}$.

Let us now discuss the values of $u$, $t$, $x$ and $y$ for which the first or the second cases are enforced. In order to be in the first case, we need to build a matrix of $\SO(2,2)$ of the form
\begin{equation}
	g'=\begin{pmatrix}
 u	&	\alpha	&	.	&	.	\\ 
 t	&	0	&	.	&	.	\\ 
 x	&	\beta	&	.	&	.	\\ 
 y	&	0	&	.	&	.	 
 \end{pmatrix}.
\end{equation}
That requires $\alpha^2-\beta^2=1$ and $u\alpha-x\beta=0$, while, for the second case, we need to build a matrix of $\SO(2,2)$ of the form
\begin{equation}
	g'=\begin{pmatrix}
 u	&	.	&	\alpha	&	.	\\ 
 t	&	.	&	0	&	.	\\ 
 x	&	.	&	\beta	&	.	\\ 
 y	&	.	&	0	&	.		 
 \end{pmatrix}.
\end{equation}
That requires $\alpha^2-\beta^2=-1$ and $u\alpha-x\beta=0$. 

In both cases, we have $\beta=\frac{ u }{ x }\alpha$ and $\alpha^2-\beta^2=\alpha^2\left( 1-\frac{ u^2 }{ x^2 } \right)$. If $| u |>| x |$, we can solve $\alpha^2-\beta^2=-1$, and if $| u |<| x |$, then we can solve $\alpha^2-\beta^2=1$. 

The last possible situation is $u=\pm x$. A point of $AdS_3$ in that situation belongs to the horizon by equation \eqref{EqHOrAdSTroisVecteur}, while one knows that point of horizon do have some directions which escape the singularity by corollary \ref{PropBHouvertLibreFerme}. Notice that in the latter situation, we do not use the assumption that $\iota(v')$ is free in $AdS_4$.
\end{proof}

\begin{corollary}		\label{CorBHBHHHHH}
	We have $\iota(BH_3)\subset BH_4$ and $\iota(\hH_3)\subset \hH_4$.
\end{corollary}

\begin{proof}
	If $\iota(v)\notin BH_4$, we have $\iota(v)\in \hF_4\cap\iota(AdS_3)\subset\iota(\hF_3)$, which is not possible if $v\in BH_3$.

	For the second part, we consider $v\in\hH_3\subset\hF_3$ (proposition \ref{PropBHouvertLibreFerme}). There is a direction $\begin{pmatrix}
		w_1	\\ 
		w_2	
	\end{pmatrix}\in S^1$ which escapes the singularity from $v$ in $AdS_3$. Of course, the direction $\begin{pmatrix}
		w_1	\\ 
		w_2	\\ 
		0
	\end{pmatrix}\in S^2$ escapes the singularity from $\iota(v)$ in $AdS_4$. Thus $\iota(v)\in\hF_4$.

	In every neighborhood of $v$, there exists a $\bar v\in BH_3$, and thus $\iota(\bar v)\in BH_4$. In other words, in every neighborhood of $\iota(v)$, there is that $\iota(\bar v)$ which belongs to $BH_4$. That proves that $\iota(v)$ belongs to $\hH_4$.
\end{proof}

\begin{lemma}		\label{LemHinteridansH}
	We have $\hH_4\cap\iota(AdS_3)\subset\iota(\hH_3)$.
\end{lemma}

\begin{proof}
	Let $v\in\hH_4\cap\iota(AdS_3)$. Since $\hH_4\subset\hF_4$, we have $v\in\hF_4\cap\iota(AdS_3)\subset\iota(\hF_3)$ (proposition \ref{PropFqTroisFt}), and then there exists a $v'\in\hF_3$ such that $v=\iota(v')$. Now, we have to prove that $v'\in\hH_3$. If $v'$ belongs to the interior of $\hF_3$, lemma \ref{LemIntTroisQueatr} implies that
	\begin{equation}
		v=\iota(v')\in\iota\big( \Int(\hF_3) \big)\subset\Int(\hF_4),
	\end{equation}
	which disagrees with the fact that $v\in\hH_4$.
\end{proof}

\begin{lemma}		\label{LemPresqueHOrQadp}
Let $v\in\hH_4$ such that $u$ and $x$ are not both vanishing. In that case, $v\in G_V\cdot \iota(\hH_3)\cup G_X\cdot\iota(\hH_3)$.
\end{lemma}

\begin{proof}
The assumption on $u$ and $x$ make that $v\in G_V\cdot(AdS_3)\cup G_X\iota(AdS_3)$. In order to fix ideas, let us suppose that $v= e^{\alpha V}\iota(v')$ with $v'\in AdS_3$. Since the set of directions $(w_1,w_2,w_3)\in S^2$ which save the points $v$, $ e^{\alpha V}v$ and $ e^{\beta X}v$ are the same, the assumption that $v\in\hH_4$ implies that $\iota(v')\in \hH_4$, which in turn proves that $v'\in \hH_3$ by lemma \ref{LemHinteridansH}. Thus $v\in G_V\cdot\iota(\hH_3)$.

The same being true with $X$ instead of $V$, the lemma is proved.
\end{proof}

\begin{proposition}		\label{PropovHhnonXYzero}
	Let $v'=(u',t',x',y',z')\in\hH_4$ with $u'$ and $x'$ not both vanishing. Then
	\begin{equation}
		v'\in G_{X_{0+}}\cdot \iota(\hH_3)\cup G_{X_{0-}}\cdot \iota(\hH_3).
	\end{equation}
\end{proposition}

\begin{proof}
	As a first step, we want to solve the equation
	\begin{equation}
		e^{\alpha X_{0+}}\begin{pmatrix}
			u	\\ 
			t	\\ 
			x	\\ 
			y	\\ 
			0	
		\end{pmatrix}=
		\begin{pmatrix}
			\frac{ \alpha^2(u-x) }{2}+u	\\ 
			t	\\ 
			\frac{ \alpha^2(u-x) }{2}+x	\\ 
			y	\\ 
			-\alpha(x-u)	
		\end{pmatrix}=\begin{pmatrix}
			u'	\\ 
			t'	\\ 
			x'	\\ 
			y'	\\ 
			z'	
		\end{pmatrix}
	\end{equation}
	with respect to $u$, $t$, $x$, $y$ and $\alpha$. The result is $t=t'$, $y=y'$ and
	\begin{equation}
		\begin{aligned}[]
			\alpha&=\frac{ z' }{ u'-x' },&u&=u'-\frac{ z'^2 }{ 2(u'-x') },&x&=\frac{ z'^2 }{ 2(u'-x') }-x'.
		\end{aligned}
	\end{equation}
	We conclude that, as long as $u'-x'\neq 0$, the point $v'$ belongs to $G_{X_{0+}}\cdot\iota(AdS_3)$. The same computation shows that $v'\in G_{X_{0-}}\cdot\iota(AdS_3)$ as long as $x'+u'\neq 0$. Let us observe that the actions of the matrices $ e^{\alpha X_{0+}}$ and $ e^{\beta X_{0-}}$ do not change the $t$ and $y$ component of a vector in $\eR^{2,l-1}$, so that the set of directions for which $v$ falls in the singularity is exactly the same as the set of directions for which $ e^{\alpha X_{0+}}v$ and $ e^{\beta X_{0-}}v$ fall in the singularity.
	
	Now, let us suppose that $v= e^{\alpha X_{0+}}\iota(v')\in\hH_4$ with $v'\in AdS_3$. We want to prove that $\iota(v')\in\hH_4$ (i.e. there is an element in the black hole in each neighbourhood of $\iota(v')$) because in that case, lemma \ref{LemHinteridansH} would conclude that $v'\in\hH_3$.

	Let $\mO$ be a neighbourhood of $\iota(v')$. The set $ e^{\alpha X_{0+}}\mO$ is a neighborhood of $v$, and thus there exists an element $\bar v\in e^{\alpha X_{0+}}\mO\cap BH_4$. Now the element $ e^{-\alpha X_{0+}}\bar v$ belongs to $\mO\cap BH_4$, so that $\iota(v')$ belongs to $\hH_4$.
\end{proof}

\begin{lemma}		\label{LemPasLEsDerniersAQ}\label{Lemuxznonsing}
The points of $AdS_4$ of the form $v=\begin{pmatrix}
	0	\\ 
	t	\\ 
	0	\\ 
	y	\\ 
	z	
\end{pmatrix}$ do not belong to the horizon.
\end{lemma}

\begin{proof}

Since the horizon is $A$-invariant, we can reduce the lemma to the case of any element of the form $ e^{\eta J_1}v$. We have
\begin{equation}
	 e^{\eta J_1}
\begin{pmatrix}
	0	\\ 
	t	\\ 
	0	\\ 
	y	\\ 
	z	
\end{pmatrix}=
\begin{pmatrix}
 1	&	0		&	0	&	0		&	0\\ 
 0	&	\cosh(\eta)	&	0	&	\sinh(\eta)	&	0\\ 
 0	&	0		&	1	&	0		&	0\\ 
 0	&	\sinh(\eta)	&	0	&	\cosh(\eta)	&	0\\ 
 0	&	0		&	0	&	0		&	1 
 \end{pmatrix}
\begin{pmatrix}
	0	\\ 
	t	\\ 
	0	\\ 
	y	\\ 
	z	
\end{pmatrix}=
\begin{pmatrix}
	0				\\ 
	\cosh(\eta)t+\sinh(\eta)y	\\ 
	0				\\ 
	\sinh(\eta)t+\cosh(\eta)y	\\ 
	z
\end{pmatrix}
\end{equation}
We annihilate the $y$ component by choosing $\eta=\ln\left( \frac{ t-y }{ t+y } \right)$. Notice that $t^2-y^2>0$, thus we have $| t |>| y |$ and the expression in the logarithm is always positive.

A representative of $(0,t,0,0,z)$ in $\SO(2,2)$ is easy to find, and the geodesic in the direction $\bar w\in S^2$ is given by
\begin{equation}
	\begin{pmatrix}
 0	&	1	&	0	&	0	&	0\\ 
 t	&	0	&	0	&	0	&	-z\\ 
 0	&	0	&	1	&	0	&	0\\ 
 0	&	0	&	0	&	1	&	0\\ 
z	&	0	&	0	&	0	&	-t 
 \end{pmatrix}
\begin{pmatrix}
	1	\\ 
	-s	\\ 
	sw_1	\\ 
	sw_2	\\ 
	sw_3	
\end{pmatrix}=
\begin{pmatrix}
	.	\\ 
	t-szw_3	\\ 
	.	\\ 
	sw_2	\\ 
	.	
\end{pmatrix}.
\end{equation}
It belongs to the singularity when $s$ takes one of the values
\begin{equation}
	s_{\pm}=\frac{ t }{ w_3z\pm w_2 }.
\end{equation}
As long as $|w_2|<|w_3z|$, the two values $s_{\pm}$ have the same sign, which can be decided by making $w_3$ positive or negative. That provides an open set in $S^2$ of directions which escape the singularity, so that $v\notin\hH_4$.
\end{proof}


\begin{theorem}			\label{ThoHorQuatreInclusionHorTrois}\label{ThoEqHorQCoore}
	The horizon of $AdS_4$ is given by
	\begin{equation}		\label{EqEqHOrGVGXQuatr}
		\hH_4=G_{X_{0+}}\cdot \iota(\hH_3)\cup G_{X_{0-}}\iota(\hH_3).
	\end{equation}
	i.e. an union of lateral classes of the horizon of $AdS_3$ by one dimensional subgroups of $N$ and $\bar N$.

	The equation in the ambient $\eR^5$ is $\hH_4\equiv u^2-x^2-z^2=0$.
\end{theorem}

\begin{proof}
	We begin by the direct inclusion. If $v=(u,t,x,y,z)\in\hH_4$ with $u\neq 0$ or $x\neq 0$, we proved in proposition \ref{PropovHhnonXYzero} that $v$ has the form \eqref{EqEqHOrGVGXQuatr}. Now, if $u=x=0$, the lemma \ref{Lemuxznonsing} shows that $v$ does not belongs to the horizon.

	For the reverse inclusion, we know that elements of $\iota(\hH_3)$ belong to $\hH_4$ by corollary \ref{CorBHBHHHHH}. If $v$ belong to $\hH_4$, then $ e^{\alpha X_{0+}}v$ and $ e^{\beta X_{0-}}v$ also belong to the horizon.
\end{proof}

%+++++++++++++++++++++++++++++++++++++++++++++++++++++++++++++++++++++++++++++++++++++++++++++++++++++++++++++++++++++++++++
\section{Conclusion}
%+++++++++++++++++++++++++++++++++++++++++++++++++++++++++++++++++++++++++++++++++++++++++++++++++++++++++++++++++++++++++++

The horizon of the BTZ black hole in $AdS_3$ was already expressed in \cite{Keio} as lateral classes of one point under the action of the Iwasawa component of the isometry group of $AdS_3$.

We proved that the simple inclusion map $\iota\colon AdS_3\to AdS_4$ transports the causal structure (free zone, black hole, horizon) from $AdS_3$ to $AdS_4$. We studied in particular the way the horizon changes when ones jumps from dimension $3$ to dimension $4$ and we obtained that the horizon in $AdS_4$ is expressed as lateral classes of the inclusion of the horizon of $AdS_3$ in $AdS_4$. In the same time, we obtained a simple equation for the horizon seen as a subset of $\eR^5$.

Although the results are quite satisfying, the method used here to prove them is quite unsatisfactory because we didn't used all the wealth structure of $\so(2,3)$ and of its reductive decompositions $\sG=\sH\oplus\sQ=\sK\oplus\sK$. We plan, in a future work, to get a much deeper understanding of the structure of $\sG$ and $\sQ$, in such a way to provide simpler proofs, in the same time as a dimensional generalization of the result of theorem \ref{ThoHorQuatreInclusionHorTrois}. We would also like to define a class of homogeneous spaces $G/H$ which accept a BTZ-like black hole.
%+++++++++++++++++++++++++++++++++++++++++++++++++++++++++++++++++++++++++++++++++++++++++++++++++++++++++++++++++++++++++++
\section{The algebras without matrices}
%+++++++++++++++++++++++++++++++++++++++++++++++++++++++++++++++++++++++++++++++++++++++++++++++++++++++++++++++++++++++++++
\label{SecRebuildStructRoot}

We have two decompositions
\begin{equation}
	\begin{aligned}[]
		\sG&=\sK\stackrel{\theta}{=}\sP\\
		\sG&=\sH\stackrel{\sigma}{=}\sQ
	\end{aligned}
\end{equation}
of $\sG=\so(2,n)$. From there, we will build the basis elements of $\sA$, $\sN$, $\bar\sN$ with all the properties we used so far. The explicit matrices \eqref{EqGeueuleVWXY} and \eqref{EqGeneRedQ} consist in a concrete realisation of what we are about to do.

%---------------------------------------------------------------------------------------------------------------------------
\subsection{The structure theorem by Pyatetskii-Shapiro}
%---------------------------------------------------------------------------------------------------------------------------

We are going to use the  Pyatetskii-Shapiro's decompositions of normal $j$-algebra \eqref{EqDecNormale} and \eqref{EqDecoEleJal}. 

\begin{lemma}
	We have
	\begin{equation}
		\| (X_{\alpha\beta})_{\sK} \|=\| (X_{\alpha\beta})_{\sP} \|.
	\end{equation}
\end{lemma}

\begin{proof}
	We use the invariance of the Killing form:
	\begin{equation}
		\begin{aligned}[]
			B\big( (X_{\alpha\beta})_{\sK},(X_{\alpha\beta})_{\sK} \big)&=\frac{1}{ \alpha }B\big( (X_{\alpha\beta})_{\sK},\ad(J_1)(X_{\alpha\beta})_{\sP} \big)\\
			&=-\frac{1}{ \alpha }B\big( \ad(J_1)(X_{\alpha\beta})_{\sK},(X_{\alpha\beta})_{\sP} \big)\\
				&=-B\big( (X_{\alpha\beta})_{\sP},(X_{\alpha\beta})_{\sP} \big).
		\end{aligned}
	\end{equation}
\end{proof}

\begin{lemma}
	we have
	\begin{equation}
		\| (X_{\alpha\beta})_{\sK} \|=\| (X_{\alpha,-\beta})_{\sK} \|.
	\end{equation}
\end{lemma}

\begin{proof}
	First, remark that $X_{\alpha,-\beta}=\sigma X_{\alpha\beta}$. We also know that $[\pr_{\sK},\sigma]=0$ because $[\sigma,\theta]=0$. The conclusion now comes from the fact that $\sigma$ is an isometry.
\end{proof}

%+++++++++++++++++++++++++++++++++++++++++++++++++++++++++++++++++++++++++++++++++++++++++++++++++++++++++++++++++++++++++++
\section{Characterisation of the horizon (vanishing norm)}
%+++++++++++++++++++++++++++++++++++++++++++++++++++++++++++++++++++++++++++++++++++++++++++++++++++++++++++++++++++++++++++
\label{SecVanNormChar}

In order to get the theorem \ref{ThoEqHorQCoore}, we used the equation of the singularity, $\hS\equiv t^2-y^2=0$, which was proved in proposition \ref{Proptcarrycarr}. But subsection \ref{SubSecTwoCharSing} provides an other characterisation of the singularity, namely the loci of points $[g]$ such that $\| J_1^* \|=0$.

\section{Conclusions and perspectives}		\label{SecConcPerspAd}
%++++++++++++++++++++++++++++++++++++

Higher-dimensional generalizations of the BTZ construction have been studied in the physics' literature, by classifying the one-parameter subgroups of $\Iso(AdS_l)=\SO(2,l-1)$, see \cite{Figueroa,AdSBH,Madden,BanadosIQxXuEh,Aminneborg,HolstPeldan}.  Nevertheless, the approach we adopt here is conceptually different. We first reinterpret the non-rotating BTZ black hole solution using symmetric spaces techniques and present an alternative way to express its singularity.  We saw the latter as the union of the closed orbits of Iwasawa subgroups of the isometry group.  As shown, this construction extends straightforwardly to higher dimensional cases, allowing to build a non trivial black hole on anti de Sitter spaces of arbitrary dimension $l\geq 3$.  From this point of view, all anti de Sitter spaces of dimension $l\geq 3$ appear on an equal footing.

A natural question arising from this analysis is the following: \emph{given a semisimple symmetric space, when does the set of closed orbits of the Iwasawa subgroups of the isometry group, seen as singularity, define a non-trivial causal structure ?} We answered this question in the case of anti de Sitter spaces, using techniques allowing in principle for generalization to any semisimple symmetric space.

We also proved that performing a discrete quotient along the orbits of $J_1$ makes the resulting space causally inextensible (closed space-like curves appear in the singular part of the space), but we did not address  questions like: are there other vector fields defining singularities (in the three dimensional case, we know that the answer is positive) ? Can we identify a mass and an angular momentum from these hypothetic vectors ? Are \emph{all} BTZ black holes obtainable in this way in higher dimensions ?
