% This is part of Analyse Starter CTU
% Copyright (c) 2014
%   Laurent Claessens,Carlotta Donadello
% See the file fdl-1.3.txt for copying conditions.

\begin{corrige}{autoanalyseCTU-33}

Dans cet exercice nous allons utiliser la formule \eqref{solgeneqlinordre1}. Si vous oubliez la formule vous pouvez vous en sortir sans problèmes en utilisant la méthode de résolution pour les équations à variables séparables, dont les équations linéaires du premier ordre sans second membre sont un cas particulier. 
\begin{enumerate}
\item Dans ce cas, $a(x) = 1$ et $b(x) = -x $ donc la formule nous donne 
\[
\mathcal{Y} = \left\{K e^{\int x\,dx} = K e^{x^2/2}, \qquad K \in \eR, \:x\in \eR\right\}.
\]
Pour déterminer la solution $\varphi$ nous utilisons la condition donnée dans l'énoncé, ce qui nous donne $K =  2e^{-1/2}$.
\item Ici $a(x) = x$ et $b(x) = -1 $ donc la formule nous donne 
\[
\mathcal{Y} = \left\{K e^{\int \frac{1}{x}\,dx} = K e^{\ln(x)} = Kx , \qquad K \in \eR, \:x\in ]0\,;\,+\infty[\right\}. 
\]
Pour déterminer la solution $\varphi$ nous utilisons la condition donnée dans l'énoncé, ce qui nous donne $K = 2$. Il faut observer que dans la formule nous avons pu intégrer $1/x$ sans crainte et ensuite omettre la valeur absolue parce que l'énoncé nous dit que l'intervalle sur lequel nos solutions sont définies est  $]0\,;\,+\infty[$.
\item Par la formule de résolution présentée dans le cours la solution générale de cette équation est 
\[
\mathcal{Y} =\left\{ K e^{\int \frac{2}{x^2}\,dx} = K e^{-2/x}, \qquad K \in \eR, \:x\in ]0\,;\,+\infty[ \right\}.
\]
\end{enumerate}


\end{corrige}   
