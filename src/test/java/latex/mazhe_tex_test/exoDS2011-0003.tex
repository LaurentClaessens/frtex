\begin{exercice}\label{exoDS2011-0003}

Quelques calculs sur les normes et les fonctions à valeurs vectorielles. 

\begin{enumerate}
\item Soit $v= (2,1,6)$. Calculer les normes $\|v\|_1$, $\|v\|_2$,  $\|v\|_\infty$.
\item Trouver un vecteur $w_1\in\eR^2$ de norme $1$ par rapport à la norme $\|\cdot\|_\infty$ qui a la même direction du vecteur $w=(2,3)$. Dessiner les vecteurs $w$, $w_1$ et la boule unitaire de $\|\cdot\|_\infty$. 
\item Soit $F: \eR\to \eR^3$ la fonction de composantes $f_1(x)=x$, $\displaystyle f_2(x)=\frac{1}{\ln(x+1)}$, $f_3(x)=\sqrt{x^2-4}$. Trouver le domaine de définition de $F$ et ses limites aux extrêmes du domaine, si elles existent. 
%\item Calculer l'intégrale de la fonction $G(x)=(x\sin(x), \ln(x))$ sur l'intervalle $[1,\pi/2]$. 
\end{enumerate}


\corrref{DS2011-0003}
\end{exercice}
