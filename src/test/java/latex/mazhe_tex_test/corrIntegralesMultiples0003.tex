\begin{corrige}{IntegralesMultiples0003}
  L'équation qui décrit le cercle $D$ en coordonnées polaires est $r=2\sin(\theta)$. Dans l'intégrale il faut donc faire varier $r$ entre $0$ et $2\sin(\theta)$ et $\theta$ entre $0$ et $\pi$.
L'intégrale à calculer devient
	\begin{equation}
		\begin{aligned}[]
			\int_0^{\pi}\int_{0}^{2\sin(\theta)}2r^2\sin(\theta)\,drd\theta&=\int_0^{\pi}\big[ \frac{ 2r^3 }{ 3 }\sin(\theta) \big]_{r=0}^{r=2\sin(\theta)}d\theta\\
			&=\int_0^{\pi}\big(\frac{ 16\sin^4(\theta) }{ 3 } \big)d\theta\\
			&=2\pi.
		\end{aligned}
	\end{equation}

\end{corrige}
