\begin{exercice}\label{exoMatlab0017}

	La production mondiale de pétrole, en milliers de barils par jours, de la dernière décennie est donnée par

	\begin{center}
	\begin{tabular}{|c|cccccccc|}
	\hline
	Année & 1997 & 1998 & 1999 & 2000 & 2001 & 2002 & 2003 & 2004  \\
	\hline
	Production & 72231 &	73588 &	72377 &	74916 &	74847 &	74478 &	77031 &	80326\\
	\hline
	\end{tabular}
	\end{center}

	\begin{center}
	\begin{tabular}{|c|ccc|}
	\hline
	Année & 2005 & 2006 & 2007 \\
	\hline
	Production & 81255 &	81659 & 81533\\
	\hline
	\end{tabular}
	\end{center}
	
	
	 \begin{enumerate}

		 \item
			 Donnez les valeurs correspondantes, en milliards de barils par an. Nous vous rappelons que le facteur de conversion est de $365$ jours par an, et de $10^6$ milliers par milliards. Enregistrez le résultat dans le vecteur \verb+consommation+.
		\item
			Trouvez les meilleurs constantes $a$ et $b$ telles que le vecteur \verb+consommation+ soit approximé par la droite
			\begin{equation}
				aT+b
			\end{equation}
			où $T$ est l'année.

		\item
			Tracez, dans un même diagramme, les données et la droite trouvée, et prolonger la droite jusqu'en $2050$. Quelle consommation mondiale serait atteinte selon cette prolongation linéaire ?

	 \end{enumerate}

	 Questions bonus (à ne faire que s'il vous reste du temps, ne comptent pas pour des points) :
	 \begin{enumerate}

		\item
			Comparez les résultats obtenus avec la réserve globale de pétrole qui reste sous nos pieds en $2009$~: environ $1240$ milliards de barils. En particulier, calculez la somme
			\begin{equation}
				\sum_{T=2009}^{2050}aT+b.
			\end{equation}
			 
		\item
			Où peut-on trouver des assiettes gyros spéciales avec frites à Louvain-la-Neuve ?
	 \end{enumerate}


\corrref{Matlab0017}
\end{exercice}
