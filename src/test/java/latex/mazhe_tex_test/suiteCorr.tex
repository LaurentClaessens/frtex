% This is part of Exercices et corrigés de CdI-1
% Copyright (c) 2011,2013-2014
%   Laurent Claessens
% See the file fdl-1.3.txt for copying conditions.

\section{Équations différentielles du premier ordre}

\subsection{Exercices}

\paragraph{Exercice 104.}
En remplaçant $y$ dans l'équation par $f(t) = t^4 e^{2t}$,
l'équation devient, après simplifications
\begin{equation*}
\left( \left( b+4\,a+4\right) \,{t}^{4}+\left( 8\,a+16\right)
\,{t}^{3}\right) = 0 
\end{equation*}
qui doit être vraie pour toute valeur de $t$. Un polynôme est nul si
et seulement chacun des coefficients est nul, donc l'équation se
ramène au système
\begin{equation*}
\begin{cases}
b + 4a + 4 = 0\\
8a + 16 = 0
\end{cases}
\end{equation*}
dont l'unique est solution est $(a,b) = (-2,4)$.

\paragraph{Exercice 105.}
\begin{enumerate}
\item $y(t) = \ln(\frac{t^4}{4} + \frac{t^2}{2} + K)$
\item $y(t) = \tan(t + K)$
\item L'intégration directe donne la relation
\begin{equation}\label{eqyplusexpy}
y(t) + e^{y(t)} = \sin(t) + K
\end{equation}
et il faut encore justifier l'éventuelle existence et/ou unicité
d'un tel $y(t)$. Pour ce faire, nous aurons besoin du lemme suivant.

\begin{lemma}
La fonction $f : \eR \to \eR : z \mapsto z + e^z$ est une
bijection.
\end{lemma}
\begin{proof}
La dérivée de $f$ est strictement positive, donc $f$ est strictement
croissante, donc $f$ est injective.

Par ailleurs, étant donné que
\begin{equation*}
\limite z {-\infty} f(z) = -\infty \quad\text{et}\quad \limite z
{+\infty} f(z) = +\infty,
\end{equation*}
le théorème des valeurs intermédiaires (qui affirme que l'image
d'une fonction continue est un intervalle) dit que l'image de $f$
contient n'importe quel intervalle arbitrairement grand, donc
contient $\eR$ entier. Ceci prouve la surjectivité de $f$.
\end{proof}

Ce lemme montre qu'il existe une (unique) fonction $g : \eR \to \eR$
qui est réciproque de $f$. On en déduit, en récrivant l'équation
(\ref{eqyplusexpy}) sous la forme
\begin{equation*}
f(y(t)) = \sin(t) + K
\end{equation*}
et en lui appliquant $g$, que $y(t) = g(\sin(t) + K)$ existe et est
univoquement définie.

\item L'équation $y^\prime = y^2$ pourrait poser un problème pour
trouver des solutions $y$ pour lesquelles il existe $t$ tel que
$y(t) = 0$, car on ne peut alors pas diviser par $y^2$.

On commence par remarquer que $y(t) = 0$ (fonction identiquement
nulle) est une solution de l'équation.

Si $y$ ne s'annule pas sur un certain intervalle fixé, l'équation
s'y écrit
\begin{equation*}
\frac{y^\prime}{y^2} = 1
\end{equation*}
dont les solutions sont de la forme $y(t) = \frac{-1}{t + K}$ (où
$K$ est une constante). Une telle solution ne peut pas tendre vers
$0$, dès lors une solution définie sur un intervalle est soit
identiquement nulle, soit ne s'annule pas du tout.

Si on ne s'intéresse qu'à des fonctions définies sur des
intervalles, il n'y a donc que ces solutions : soit $y(t) = 0$, soit
$y(t)$ est de la forme $\frac{-1}{t + K}$ pour une certaine
constante $K$.

La solution générale sur un domaine quelconque s'obtient en prenant
l'union sur des intervalles disjoints de solutions du type
précédent.

\item L'équation $y^\prime = y^{\frac13}$ pose le même problème que
l'équation précédente, mais la solution est différente.

On remarque à nouveau que la fonction nulle est solution, et on
s'intéresse aux autres solutions.

Si $y$ est une solution qui ne s'annule pas sur un certain
intervalle fixé, elle satisfait
\begin{equation*}
\frac{y^\prime}{y^{\sfrac13}} = 1
\end{equation*}
et donc est de la forme $y(t) = \pm (\frac{2x}{3} +
K)^{\sfrac23}$. Une telle solution est définie pour $x \geq
\sfrac{3K}2$ et tend vers $0$ lorsque $x \to \sfrac{3K}2$, et donc
peut se recoller avec une solution nulle \og sur le bord gauche de
l'intervalle\fg{} (sur le bord droit, la solution tend vers $\pm
\infty$).

La solution générale sur un intervalle est donc une solution de la
forme
\begin{equation*}
y(t) =
\begin{cases}
0& \text{si $x \leq \sfrac{3K}2$}\\
\pm (\frac{2x}{3} + K)^{\sfrac23}& \text{si $x >
\sfrac{3K}2$}\end{cases}
\end{equation*}
pour un certaine constante $K$, ou alors $y(t)$ est identiquement
nulle.

\item L'équation est équivalente à
\begin{equation*}
\frac{yy^\prime}{y^2+1} = - \sin(t)
\end{equation*}
ce qu'on intègre pour obtenir
\begin{equation*}
\frac12\ln(y^2 + 1) = - \cos(t) + K
\end{equation*}
c'est-à-dire $y^2 = -1 + K e^{-2 \cos(t)}$ pour une certaine
constante (positive) $K$. Selon la valeur de $K$, ces solutions sont
définies ou non sur tout $\eR$ :
\begin{enumerate}
\item Si $K > e^{2}$, le membre de droite est strictement positif
pour tout $t$ et on peut en prendre la racine (solution sur $\eR$)
\item Si $K < e^{2}$, le membre de droite est négatif pour
certaines valeurs de $t$ et on ne peut pas en prendre la racine.
\item Si $K = e^{2}$, le choix d'une racine du membre de droite
(positive ou négative) donne une fonction qui n'est pas dérivable
aux points où elle est nulle (car la racine n'est pas dérivable en
ces points). Par contre, si on change de choix de signe pour la
racine à chaque fois que le membre de droite s'annule, la fonction
obtenue est dérivable.
\end{enumerate}

Ces trois cas seraient plus clairs sur une illustration.
% % \psset{unit=1pt,yunit=20pt,plotpoints=300,plotstyle=curve}
% \psaxes{<->}(2,2)(0,0)(4,3)
% \psplot[linecolor=green]{-3.14}{3.14}{x cos}
% % \psplot[linecolor=red]{-360}{360}{-1 9 2.73 2 x cos mul exp mul
% %   add sqrt}%

% $y^2 = -1 + 6*exp(2 cos(x))$ $y^2 = -1 + 9*exp(2 cos(x))$ $y^2 =
% -1 + exp(2 + 2 cos(x))$

% % \begin{mfpic}

% % \end{mfpic}
\end{enumerate}

\paragraph{Exercice 106.}
Il s'agit ici de reprendre les solutions générales de l'exercice
ci-dessus en sélectionnant la ou les solutions qui satisfont au
problème de Cauchy. Dans les cas agréables cela revient simplement à
déterminer la constante. Dans les autres cas, il faut vérifier
l'existence et l'unicité de la solution.


\paragraph{Exercice 116.}
Dans chacun de ces exercices, il s'agit d'intégrer la fonction donnée
sur un domaine précisé.

\begin{enumerate}
\item Le domaine d'intégration est un rectangle, le choix des bornes
est donc simple~:
\begin{equation*}\begin{split}
\int_0^2 \left(\int_0^1 (4 - x^2 - y^2) d x\right) d y%
& = \int_0^2 \left[ 4x - \frac{x^3}3 - y^2x \right]_{x=0}^{x=1}
d y\\%
& = \int_0^2 4 - \frac13 - y^2 d y\\
& =\left[\left(4 - \frac13 \right)y - \frac{y^3}3 \right]_0^2\\
& = 8 - \frac{10}3
\end{split}\end{equation*}
\item
\end{enumerate}

\section{Théorème de la fonction implicite}




\subsection{Exercices}

\paragraph{Exercice 129.}
\begin{enumerate}
\item Une telle fonction $Z$ doit vérifier $F(1,1,Z(1,1)) = 0$, donc
en particulier, en notant $z_0 = Z(1,1)$, il faut $z_0 + \ln(z_0) =
1$. Une solution évidente est $z_0 = 1$.

Montrons que cette solution est unique : la fonction auxiliaire $g :
\eR_0^+ \to \eR : z \mapsto z + \ln(z)$ est strictement croissante
(sa dérivée est strictement positive sur son domaine) et en
particulier injective. L'équation en $z_0$ se réécrit sous la forme
$g(z_0) = 1$, et l'injectivité nous assure l'unicité de la solution.

Au point $(x_0,y_0,z_0) = (1,1,1)$ on peut appliquer le théorème de
la fonction implicite puisque $\pder F z(1,1,1) = 1 + 1 = 2 \neq 0$,
et donc on a un voisinage $U$ de $(1,1)$ et une unique fonction $Z :
U \to \eR$ satisfaisant à la condition énoncée.

\item Notons $Z = Z(x,y)$ pour la simplicité. Pour tout $(x,y)$ dans
$U$, nous avons donc $Z + \ln(Z) - xy = 0$. En particulier on peut
dériver cette identité par rapport à $x$ et à $y$, d'où
\begin{equation*}
\begin{split}
\pder Z x + \frac 1Z \pder Z x - y = 0\\
\pder Z y + \frac 1Z \pder Z y - x = 0
\end{split}
\end{equation*}
pour tout $(x,y) \in U$, et on en tire (on note $\partial_x$ au lieu
de $\pder {} x$)
\begin{equation*}
\partial_x Z = \frac y {1+\frac1Z} = \frac{y Z}{1+Z}
\quad\text{et}\quad   \partial_y Z = \frac x {1+\frac1Z} = \frac {x Z} {1+Z}
\end{equation*}

On peut enfin dériver l'une ou l'autre de ces égalités pour obtenir
les dérivées secondes, en remplaçant ensuite les occurrences de
$\partial_x Z$ et $\partial_y Z$ par leur expression ci-dessus, par
exemple
\begin{equation*}
\begin{split}
\partial^2_{xy} Z &= \frac{(x \partial_x Z + Z)(1+Z) - x
Z \partial_x Z}{(1+Z)^2}\\
&= \frac{(x y Z + Z (1+Z)^2)}{(1+Z)^3}
\end{split}
\end{equation*}
où $\partial^2_{xy}$ désigne la dérivée partielle seconde par
rapport à $y$ puis par rapport à $x$.
\end{enumerate}
\paragraph{Exercice 130.}
Si $x = 1$, l'équation nous donne $1 = y$, donc la fonction doit
vérifier $Y(1) = 1$. Montrons d'abord qu'une telle fonction existe :
on considère
\begin{equation*}
F : \eR_0^+ \to \eR_0^+ : (x,y) \mapsto x^y - y^x
\end{equation*}
et on vérifie que $\partial_y F(1,1) = - 1 \neq 0$. Le théorème de la
fonction implicite s'applique, et fournit effectivement un voisinage
$U$ de $1$ et une unique fonction $Y : U\subset \eR \to \eR$
vérifiant $Y(1) = 1$ et vérifiant l'équation $x^{Y(x)} - {Y(x)}^x$.

Pour calculer la dérivée, on peut ré-écrire l'équation (en notant $Y =
Y(x)$ pour simplifier la notation) sous la forme
\begin{equation*}
Y \ln(x) = x \ln(Y)
\end{equation*}
et en dérivant par rapport à $x$ on obtient
\begin{equation*}
Y^\prime \ln(x) + Y\frac1x = \ln(Y) + x \frac {Y^\prime} Y
\end{equation*}
d'où on tire en réarrangeant les termes
\begin{equation*}
Y^\prime(x) =   \frac Y x \frac{x \ln(Y) - Y}{Y \ln(x) - x} =
\frac{\ln(Y) - \frac Y x}{\ln(x) - \frac x Y}
\end{equation*}

\paragraph{Exercice 131.}
L'équation implicite pour $(x,y) = (0,0)$ devient $z e^z = 0$ dont
l'unique solution est $z = 0$, donc une telle fonction $Z$ doit
vérifier $Z(0,0) = 0$. Pour vérifier l'existence et l'unicité de la
fonction $Z$, on considère la fonction
\begin{equation*}
F : \eR^3 \to \eR : (x,y,z) \mapsto z e^z - x - y.
\end{equation*}
On calcule $\partial_z F(0,0,0) = 1 \neq 0$, de sorte que le théorème
de la fonction implicite s'applique et fournit une unique fonction $Z$
telle que demandée.

Pour écrire le polynôme de Taylor il suffit de calculer les dérivées
de $Z$, ce qu'on fait en utilisant la relation $Z e^Z = x+y$~:
\begin{equation*}
\begin{split}
\partial_x Z (1 + Z) e^Z = 1 \Rightarrow \partial_x Z = \frac{e^{-Z}}{1+Z}\\
\partial_y Z (1 + Z) e^Z = 1 \Rightarrow \partial_y Z = \frac{e^{-Z}}{1+Z}\\
\end{split}
\end{equation*}
où on note comme toujours $Z = Z(x,y)$ pour simplifier l'écriture. On
calcule également les dérivées secondes~:
\begin{equation*}
\partial^2_{xx} Z = \partial^2_{yx} Z = \partial^2_{yy} Z =%
-\frac{Z+2}{(1+Z)^3} e^{-2 Z}
\end{equation*}
et donc le polynôme de Taylor à l'ordre $2$ s'écrit
\begin{equation*}
Z(x,y) = x + y - x^2 - 2 xy - y^2 + o(\norme{(x,y)}^2)
\end{equation*}

\paragraph{Exercice 132.}
Pour $x = \sfrac34$, l'équation $F(\sfrac34,y,z) = 0$ devient le
système
\begin{equation*}
\begin{cases}
\frac9{16} + y^2 + z^2 - 1 = 0\\
\frac9{16} + y^2 - \frac3{4} = 0
\end{cases}
\end{equation*}
qui a exactement les quatre solutions $(y,z) = (\pm
\frac{\sqrt3}4,\pm\frac12)$.

Le jacobien partiel de $F$ par rapport à $(y,z)$ est donné par
\begin{equation}\label{exo132-jacobienpartiel}\tag{*}
\begin{vmatrix}
2 y & 2z\\
2 y & 0
\end{vmatrix} = -4 y z
\end{equation}
et donc est non-nul en chacun des quatre points $(\pm
\frac{\sqrt3}4,\pm\frac12)$, ce qui permet d'appliquer le théorème de
la fonction implicite. Ceci prouve l'existence des quatre fonctions
$\varphi$ demandées, correspondant chacune à un des points ci-dessus.

Pour calculer les dérivées, on sait que ces fonctions vérifient
\begin{equation*}
\begin{cases}
x^2 + Y(x)^2 + Z(x)^2 = 1\\
x^2 + Y(x)^2 = x
\end{cases}
\end{equation*}
pour tout $x$ dans un voisinage de $\sfrac34$. En particulier on peut
dériver ces deux équations pour obtenir (on note $Y = Y(x)$ et $Z =
Z(x)$ pour alléger la notation)~:
\begin{equation*}
\begin{cases}
2 x + 2 Y Y^\prime + 2 Z Z^\prime = 0\\
2 x + 2 Y Y^\prime = 1
\end{cases}
\end{equation*}
d'où on tire
\begin{equation*}
\begin{cases}
Z^\prime = \frac {-1}{2Z}\\
Y^\prime = \frac{1 - 2 x}{2 Y}
\end{cases}
\end{equation*}
où on remarque que la division par $Y$ et par $Z$ est bien définie si
$x$ est assez proche de $\sfrac34$ puisque le jacobien partiel
(\ref{exo132-jacobienpartiel}) est non nul.

\begin{remark}
La formule donnant la dérivée ne dépend pas explicitement du point
autour duquel on fait le calcul, mais dépend bien sûr encore de la
valeur de $Z$ en ce point.
\end{remark}

\paragraph{Exercice 133.}
Étant donnée la relation, on vérifie que la fonction $Y$ définie
implicitement par
\begin{equation*}
e^{yx} - 1 = x^2 + y
\end{equation*}
doit satisfaire à $Y(0) = 0$. Une première tentative montre que la
limite demandée est donc du type indéterminé \og $\frac00$\fg{}. Afin
d'appliquer la \emph{règle de L'hospital} on veut d'abord vérifier que
la fonction $Y$ est bien dérivable autour de $0$.

Pour ce faire, on considère
\begin{equation*}
F : \eR^2 \to \eR : (x,y) \mapsto e^{xy} - 1 - x^2 - y
\end{equation*}
et on a $\partial_y F(0,0) = - 1 \neq 0$, donc le théorème de la
fonction implicite s'applique et assure l'existence d'une fonction $Y$
de classe $C^\infty$ autour de $0$ vérifiant
\begin{equation*}
e^{x Y} - 1 = x^2 + Y
\end{equation*}
où on note $Y = Y(x)$ pour alléger la notation. En particulier sa
dérivée doit satisfaire à l'équation
\begin{equation*}
e^{x Y} (Y + x Y^\prime) = 2 x + Y^\prime
\end{equation*}
et donc, pour $x = 0$ on obtient
\begin{equation*}
Y(0) = Y^\prime(0)
\end{equation*}
ce qui montre que la dérivée s'annule en $0$.

La limite devient donc
\begin{equation*}
\limite x 0 \frac{Y(x)}{\cos(x) - 1} = \limite x 0 \frac{Y^\prime(x)}{-\sin(x)}
\end{equation*}
et est à nouveau du type indéterminé \og $\frac00$\fg{}.

Dérivons à nouveau l'équation satisfaite par $Y^\prime$ pour obtenir
\begin{equation*}
e^{x Y} (Y + x Y^\prime)^2 + e^{x Y} (Y^\prime + Y^\prime + x
Y^{\prime\prime}) = 2 + Y^{\prime\prime}
\end{equation*}
ce qui, pour $x = 0$, donne
\begin{equation*}
Y(0)^2 + 2 Y^\prime(0) = 2 + Y^{\prime\prime}(0)
\end{equation*}
et donc $Y^{\prime\prime}(0) = -2$ puisque $Y(0) = Y^\prime(0) = 0$.

La limite devient donc
\begin{equation*}
\limite x 0 \frac{Y(x)}{\cos(x) - 1} = \limite x 0
\frac{Y^{\prime\prime}(x)}{-\cos(x)} = 2
\end{equation*}
et la réponse attendue est donc $2$.

\paragraph{Exercice 134.}
Au voisinage du point $(0,1)$ la courbe peut s'écrire sous la forme $y
= Y(x)$ par application du théorème de la fonction implicite (le
vérifier !). La tangente à la courbe est alors donnée par l'équation
\begin{equation*}
y - 1 = Y^\prime(0) (x - 0)
\end{equation*}
ce qui implique de calculer $Y^\prime(0)$.

Par définition $Y(x)$ vérifie $Y(x)^2 + \sin(x Y(x))  = 1$ et donc sa
dérivée vérifie
\begin{equation*}
2 Y(x) Y^\prime(x) + \cos(x Y(x)) (Y(x) + x Y^\prime(x)) = 0
\end{equation*}
et donc en $x = 0$, on obtient
\begin{equation*}
2 Y(0) Y^\prime(0) + Y(0) = 0
\end{equation*}
ce qui donne $Y^\prime(0) = -\sfrac{1}{2}$ puisque sachant que $Y(0) =
1$.

\paragraph{Exercice 135.}

Nous savons que pour une surface donnée sous forme implicite $F(x,y,z) =
k$, le vecteur gradient $\nabla F$ en un point de la surface est normal
à cette surface.

Dans le cadre de cet exercice, on est donc ramené à chercher les
points sur la surface dont le gradient est un multiple de $(1,-2,2)$
(qui est le vecteur normal au plan donné).

Le gradient au point $(x,y,z)$ est donné par $(8x,32y,16z)$, et on
veut qu'il existe $\lambda \neq 0$ tel que $(8x,32y,16z) =
(\lambda,-2\lambda,2\lambda)$. Cette condition combinée à la condition
d'appartenance à la surface fournit une équation pour $\lambda$ qu'il
suffit de résoudre.

La fin de l'exercice dépend de la manière de compléter l'énoncé,
puisqu'il faut expliciter l'équation des plans tangents.

\paragraph{Exercice 136.}
\begin{enumerate}
\item Étant donné que $ M = F^{-1}(0,0)$ et que $\{(0,0)\}$ est un
ensemble fermé (ne contient qu'un seul point !), on en déduit que
$ M$ est l'image réciproque par une fonction continue d'un
fermé, donc $ M$ est un ensemble fermé (vérifier !).  Par
ailleurs, $ M$ est complètement contenue dans la sphère de rayon
$1$, donc est bornée. Ces deux propriétés fournissent la compacité.

L'ensemble $ M$ est donné sous forme implicite par l'annulation
de deux fonctions, dont les gradients sont $(1,1,1)$ et $(2x,
2y,2z)$. Ces deux vecteurs linéairement dépendant si et seulement si
$x = y = z$, ce qui n'est pas possible pour un point de $ M$. On
en déduit que les gradients sont indépendants sur $ M$, et donc
que $ M$ est une variété $C^1$ de dimension $3 - 2 = 1$.

En fait, c'est l'intersection d'une sphère et d'un plan, c'est donc
un cercle.

\item Pour $y = 0$, on observe qu'il faut que $X(0) + Z(0) = 0$ et
$X(0)^2 + Z(0)^2 = 1$. On en déduit que $(X(0),Z(0))$ vaut soit
$\frac{\sqrt2}{2},-\frac{\sqrt2}{2})$, soit
$(-\frac{\sqrt2}{2},\frac{\sqrt2}{2})$. Pour chacun de ces points le
théorème de la fonction implicite s'applique et fournit donc deux
paires de fonctions $(X_1,Z_1)$ (avec $X_1(0) = \frac{\sqrt2}{2}$) et
$(X_2,Z_2)$ (avec $X_2(0) = -\frac{\sqrt2}{2}$).

\item La meilleure approximation polynômiale de degré $1$ est donnée
par le polynôme de Mc Laurin d'ordre $1$, donc on calcule la dérivée
première des fonctions $X_1$ et $X_2$.

En dérivant les équations qui définissent $X_1$ et $Z_1$ (et $X_2$
et $Z_2$ également, ce sont les mêmes !), on obtient les relations
\begin{equation*}
\begin{cases}
X_1^\prime + 1 + Z_1^\prime = 0\\
2 X_1 X_1^\prime + 2 y + 2 Z_1 Z_1^\prime = 0
\end{cases}
\end{equation*}
ce qui, pour $y = 0$, fournit $X_1^\prime(0) = Z_1^\prime(0) =
-\sfrac12$. Et en mettant un indice ${}_2$ partout, on obtient la
même chose $X_2^\prime(0) = Z_2^\prime(0) = -\sfrac12$.

Le polynôme de Mc Laurin pour $X_1$ et $X_2$ s'écrit donc
\begin{equation*}
\begin{split}
X_1(y) &= \frac{\sqrt2}{2} - \frac12 y + o(\abs y)\\
X_2(y) &= -\frac{\sqrt2}{2} - \frac12 y + o(\abs y)\\
\end{split}
\end{equation*}
\end{enumerate}

\paragraph{Exercice 137.}
\begin{enumerate}
\item Considérons l'application $F : \eR\times\eR^2 \to \eR^2 :
(t,v) \mapsto v - \varphi_t(v)$.  On sait que $F(0,v_0) = 0$ par
définition de $v_0$. On sait également que pour $t = 0$, la
différentielle de l'application partielle $\eR^2 \to \eR^2 : v
\mapsto F(v)$ vaut $\id - d\varphi_0$ et est donc inversible par
hypothèse sur le spectre. On en déduit que le théorème de la
fonction implicite s'applique, et qu'il existe un voisinage
$\mathopen\rbrack-\epsilon,\epsilon\mathclose\lbrack$ de $t = 0$, un
voisinage $U$ de $v = v_0$ et une unique application
\begin{equation*}
V : \mathopen\rbrack-\epsilon,\epsilon\mathclose\lbrack \to U
\end{equation*}
telle que $F(t,V(t)) = 0$ pour tout $t \in
\mathopen\rbrack-\epsilon,\epsilon\mathclose\lbrack$. C'est-à-dire
$V(t) = \varphi_t(V(t))$ pour $t \in
\mathopen\rbrack-\epsilon,\epsilon\mathclose\lbrack$, et donc $V(t)$
est un point fixe de $\varphi_t$.

\item Les applications $\varphi_t : (x,y) \mapsto (x+t,y)$ n'ont pas
de point fixe, sauf pour $t = 0$. A titre d'information, on dit que
c'est une action (la variable $t$ \og agit\fg{} puisqu'elle
translate vers la droite) \emph{libre} (sans point fixe autre que $t
= 0$) de la droite $\eR$ sur le plan $\eR^2$.
\end{enumerate}

\section{Intégrales curvilignes}

\subsection{Exercices}
\paragraph{Exercice 144$^\prime$}
\begin{enumerate}
\item Une paramétrisation est donnée, il reste à intégrer
\begin{equation*}
\int_0^{2\pi} \norme{\gamma^\prime(t)} d t
\end{equation*}
où $\gamma^\prime(t) = (a (1 - \cos(t)), a \sin(t))$, c'est-à-dire
\begin{equation*}
\int_0^{2\pi} a \sqrt{2 - 2 \cos(t)} d t = 8a
\end{equation*}
où on a utilisé l'égalité trigonométrique
\begin{equation*}
1 - \cos(t) = 2\sin^2\left(\frac t2\right).
\end{equation*}

\item On sait qu'on peut paramétriser cet astroïde par
\begin{equation*}
\begin{cases}
\sqrt[3]x = \sqrt[3]a \cos(t)\\
\sqrt[3]y = \sqrt[3]a \sin(t)
\end{cases}
\end{equation*}
ce qui donne le chemin
\begin{equation*}
\gamma(t) = (a \cos^3(t), a \sin^3(t)) \donc \gamma^\prime(t) = (-
3 a \cos^2(t) \sin(t), 3 a \sin^2(t) \cos(t))
\end{equation*}
et l'intégrale devient, grace aux relations trigonométriques,
\begin{equation*}
\int_0^{2\pi} 3 a \sqrt{\sin^2(t)\cos^2(t)} d t = \frac{3a}2 \int_0^{2\pi} \abs{\sin(2t)} d t
\end{equation*}
où il faut encore faire attention au signe. Par périodicité et par
symétrie, on se ramène à 
\begin{equation*}
6a \int_0^{\frac\pi2} \sin(2t) d t = 6 a
\end{equation*}

\item Le chemin peut être paramétrisé de la manière suivante
\begin{equation*}
\gamma(t) =%
\begin{cases}
(t+1,0) & -1 \leq t \leq 0\\
(1-t,t) & 0 \leq t \leq 1\\
(0,2-t) & 1 \leq t \leq 2
\end{cases}
\end{equation*}
c'est un chemin $C^1$ par morceaux% , où les morceaux sont
%   $\mathopen\rbrack-1,0\mathopen\lbrack$,
%   $\mathopen\rbrack0,1\mathopen\lbrack$ et
%   $\mathopen\rbrack1,2\mathopen\lbrack$
. On a
\begin{equation*}
\gamma^\prime(t) =%
\begin{cases}
(1,0) & -1 < t < 0\\
(-1,1) & 0 < t < 1\\
(0,-1) & 1 < t < 2
\end{cases}%
\quad\text{et donc}\quad%
\norme{\gamma^\prime(t)} =%
\begin{cases}
1 & -1 < t < 0\\
\sqrt2 & 0 < t < 1\\
1 & 1 < t < 2
\end{cases}%
\end{equation*}
et l'intégrale devient donc
\begin{equation*}
\int_{-1}^0 (t+1) d t + \int_0^1 \sqrt2 d t + \int_1^2 (2-t) \d
t = 1+\sqrt2
\end{equation*}
\end{enumerate}


\paragraph{Exercice 143 = 145$^\prime$.}
\begin{enumerate}
\item Il faut calculer
\begin{equation*}
\int_0^{\frac{1}{2}} \sqrt{1 + \frac{4 x^2}{(1-x^2)^2}} d x
\end{equation*}

\item Calculer
\begin{equation*}
\int_0^{5} \sqrt{1 + \frac94 x} d x
\end{equation*}

\item Calculer
\begin{equation*}
\int_0^{\frac\pi4} \sqrt{1 + \tan^2(x)}\ d x = \int_0^{\frac\pi4} \frac{1}{\cos(x)} d x
\end{equation*}
\end{enumerate}

\paragraph{Exercice 146$^\prime$}
\begin{enumerate}
\item On paramétrise ce cercle de la façon usuelle mais en prenant
attention au sens
\begin{equation*}
\gamma(t) = (\cos(-t),\sin(-t)) = (\cos(t),-\sin(t)) \qquad t \in [0,2\pi]
\end{equation*}
d'où on tire $\gamma^\prime(t) = (-\sin(t),-\cos(t))$ et l'intégrale
devient
\begin{equation*}
\int_0^{2\pi} (-\sin^3(t) - \cos^3(t)) d t = 0.
\end{equation*}
L'intégrale est nulle parce que ces deux fonctions ($\sin^3$ et
$\cos^3$) possèdent un centre de symétrie, et qu'on les intègre sur
une période complète.

\item Le chemin se paramétrise par 
\begin{equation*}
\gamma(t) = (\cos(t),\sin(t)) \donc \gamma^\prime(t) = (-\sin(t),\cos(t))
\end{equation*}
et donc l'intégrale devient
\begin{equation*}
\int_0^{2\pi} \scalprod{G(\gamma(t))}{\gamma^\prime(t)}d t =
\int_0^{2\pi} \left(-\sin^3(t) + \cos^3(t)\right)d t = 0
\end{equation*}
Cette intégrale est nulle pour les même raisons que ci-dessus.

\item
La paramétrisation est donnée, et on a 
\begin{equation*}
	\gamma^\prime(t) = (-a \sin(t), a \cos(t), b)
\end{equation*}
donc l'intégrale devient
\begin{equation}
	\begin{aligned}[]
		\int_0^{2\pi} \Big[ \big(a \sin(t) - b t\big) (-a\sin(t)) &+ \big(bt - a \sin(t)\big) a \cos(t) \\
		&+ ab(\cos(t)-\sin(t)) \Big] dt
	\end{aligned}
\end{equation}
et vaut $- \pi a (a+ 2b)$ après calculs.

\item
Sur le chemin donné, l'intégrale vaut
\begin{equation*}
	\int_\gamma -x d x - y d y - z d z = \int_\gamma - \frac12 d f
\end{equation*}
où $f(x,y,z) = x^2+y^2+z^2$. Cette intégrale est donc nulle puisque le chemin $\gamma$ est fermé (c'est un cercle).

On peut même aller plus loin : sur la sphère $f \equiv 1$ donc $df= 0$.
\end{enumerate}

\paragraph{Exercice 144 = 147$^\prime$.}
Ayant $y = x^3 + x^2 + x + 1$ on a $d y = (3 x^2 + 2 x + 1)d x$ donc
il faut calculer
\begin{equation*}
\int_0^1 (x (3 x^2 + 2 x + 1) + x^3 + x^2 + x + 1)d x
\end{equation*}

\paragraph{Exercice 145 = 148$^\prime$.}
Il faut calculer
\begin{enumerate}
\item
\begin{equation*}
\int_0^2 (x^2 + \frac{x^2}{2}) d x
\end{equation*}
\item
\begin{equation*}
\int_0^2 (\frac{x^3}{2} + \frac{x^3}{2}) d x
\end{equation*}
\item 
\begin{equation*}
\int_0^1  (16 y^4 + 4 y^4)d y
\end{equation*}
\item La ligne brisée est un peu inutile...
\end{enumerate}
Ces quatres résultats sont identiques, ce qui laisse penser qu'en
réalité la $1$-forme $\omega = 2 x y d x + x^2 d y$ est une forme
exacte. En effet, on vérifie aisément que $\omega = d (x^2 y)$.

\paragraph{Exercice 146 = 149$^\prime$.}
On peut paramétriser l'hélice par
les équations
\begin{equation*}
\begin{cases}
x = \cos(2 \pi t)\\
y = \sin(2 \pi t)\\
z = t
\end{cases}\qquad t \in [0;1]
\end{equation*}
et l'intégrale devient
\begin{equation*}
\int_0^1 \left(-2\pi \sin^2(2\pi t) + 2\pi \cos^2(2\pi t) +
t^2\right) d t = \frac13  
\end{equation*}

On pouvait aussi remarquer que
\begin{equation*}
y d x + x d y + z^2 d z = d f \quad\text{où $f(x,y,z) = xy + \frac13 z^3$}
\end{equation*}
et donc l'intégrale vaut bien $f(1,0,1) - f(1,0,0) = \sfrac13$.

\section{Intégrales de surface}





\paragraph{Exercice 147.}
\begin{enumerate}
\item La sphère est paramétrisée en coordonnées sphériques (avec un
rayon $r$ constant) par
\begin{equation*}
\begin{cases}
x = r \cos(u)\sin(v)\\
y = r \sin(u)\sin(v)\\
z = r \cos(v)
\end{cases}\qquad\text{avec}\quad
\begin{cases}
u \in \llbrack{0;2\pi}\\
v \in \llbrack{0;\pi}
\end{cases}
\end{equation*}
c'est-à-dire la paramétrisation est
\begin{equation*}
F : \llbrack{0;2\pi} \times \llbrack{0;\pi} \to \eR^3 : (u,v)
\mapsto (r \cos(u)\sin(v), r\sin(u)\sin(v), r\cos(v))
\end{equation*}
et on calcule que l'élément de surface vaut
\begin{equation*}
\norme{\pder F u \wedge \pder F v} d u d v = r^2 \sin (v) d u d v
\end{equation*}
Dès lors, la surface de la sphère vaut
\begin{equation*}
\int_0^\pi \int_0^{2\pi} r^2 \sin(v) d u d v = 2r^2 \pi \crochets{-\cos(v)}_0^\pi = 4 \pi r^2
\end{equation*}

\item En coordonnées cylindriques $(\rho,\theta,z)$, le cylindre plein
de rayon $r$ tel qu'il est décrit a pour équation
\begin{equation*}
\rho \leq r \cos(\theta)
\end{equation*}
et la sphère s'écrit $\rho^2 + z^2 = r^2$. On peut donc paramétriser
la surface demandée via
\begin{equation*}
\begin{split}
F_1 \equiv x &= \rho \cos \theta\\
F_2 \equiv y &= \rho \sin \theta\\
F_3 \equiv z &= \pm \sqrt{r^2 - \rho^2}
\end{split}
\end{equation*}
où le signe $\pm$ indique qu'il y a deux morceaux de surface.

On peut calculer le produit vectoriel
\begin{equation*}
\pder F \theta \wedge \pder F \rho =
\begin{vmatrix}
\vect{e_x} & \vect{e_y} & \vect{e_z}\\
-\rho\sin\theta & \rho\cos\theta & 0\\
\cos\theta & \sin\theta & \pm \frac{-\rho}{r^2-\rho^2}
\end{vmatrix} = (\pm\frac{-\rho^2 \cos\theta}{r^2-\rho^2}, \pm
\frac{- \rho^2 \sin\theta}{r^2-\rho^2}, -\rho)
\end{equation*}
dont la norme donne l'élément de surface
\begin{equation*}
d \sigma = \frac{r \rho}{\sqrt{r^2-\rho^2}} d \rho d \theta
\end{equation*}
ce qui permet de calculer l'intégrale sur l'un des morceaux :
\begin{equation*}
\begin{split}
\int_{-\frac\pi2}^{\frac\pi2} \int_0^{r\cos\theta}\frac{r
\rho}{\sqrt{r^2-\rho^2}} d \rho d \theta %
&= r \int_{-\frac\pi2}^{\frac\pi2} \crochets{- \sqrt{r^2-\rho^2}}_0^{r\cos\theta} \theta\\
&= r \int_{-\frac\pi2}^{\frac\pi2} (- r \abs{\sin\theta} + r)\\
	&=d\theta = \pi r^2 - 2 r^2 = r^2 (\pi - 2)
\end{split}
\end{equation*}

\item Le morceau de cône se accepte la paramétrisation suivante en coordonnées cylindriques :
\begin{equation*}
\rho = \abs z \quad 0 < z < b
\end{equation*}
et puisque $z > 0$, on a $z = \rho$. On peut donc expliciter le
changement de coordonnées~:
\begin{equation*}
\begin{cases}
F_1 \equiv x = \rho \cos \theta\\
F_2 \equiv y = \rho \cos \theta\\
F_3 \equiv \rho = \rho
\end{cases} \qquad \text{avec $0 < \rho < b$}
\end{equation*}
l'élément de surface peut alors se calculer comme suit
\begin{equation*}
\pder F \rho \wedge \pder F \theta = %
\begin{vmatrix}
\vect{e_x} & \vect{e_y} & \vect{e_z}\\
\cos\theta & \sin\theta & 1\\
-\rho \sin\theta & \rho \cos \theta & 0
\end{vmatrix} = (- \rho \cos\theta, - \rho \sin\theta, \rho),
\end{equation*}
donc $d\sigma = \abs \rho \sqrt2 = \rho \sqrt 2$, et l'intégrale devient
\begin{equation*}
\int_0^b \int_0^{2\pi} \rho^2 \sqrt 2 d \theta \rho = \frac {2\sqrt2
\pi}3 b^3
\end{equation*}

\item La sphère se paramétrise en coordonnées sphériques par
\begin{equation*}
\begin{cases}
x = \cos u \sin v\\
y = \sin u \sin v\\
z = \cos v
\end{cases} \qquad\text{avec }%
\begin{cases}
u \in \llbrack {0, 2\pi}\\
v \in \llbrack {0, \pi}
\end{cases}
\end{equation*}
dont l'élément de volume a déjà été calculé et vaut $\sin(v) d u \d
v$.

L'équation du cône devient
\begin{equation*}
\cos^2 v = \sin^2 v
\end{equation*}
c'est-à-dire $\sin v = \abs{\cos v}$ (car $\sin v \geq 0$), donc les
choix possibles sont
\begin{equation*}
v = \frac\pi4 \qquad\text{ou}\qquad v = \frac{3\pi}4
\end{equation*}
Étant donné qu'on veut être \emph{dans} le cône, il faut
\begin{equation*}
v \in \lrbrack{0, \frac\pi4}\cup \llbrack{\frac{3\pi}4, \pi}
\end{equation*}
et donc l'intégrale devient
\begin{equation*}
\int_0^{2\pi} \left(\int_0^{\frac\pi4} \sin(v)d v +
\int_{\frac{3\pi}4}^\pi \sin(v) d v\right)d u =
\int_{0}^{2\pi} (2 - \sqrt 2) d u = 2 \pi (2 - \sqrt 2).
\end{equation*}

\item En coupant le cylindre le long de la droite
\begin{equation*}
\begin{cases}
x = 0\\ y = 1
\end{cases}
\end{equation*}
on remarque que la surface limitée par l'hélice
\begin{equation*}
\begin{cases}
x = \sin t\\
y = \cos t\\
z = \frac t {2\pi}
\end{cases}
\end{equation*}
est un triangle dont la base est la circonférence du cylindre et la
hauteur est le pas de l'hélice, c'est-à-dire son aire vaut $\frac12
(2\pi \cdot 1) = \pi$.
\end{enumerate}

\paragraph{Exercice 150.}
\begin{enumerate}
\item Si $P(x,y) = 2 (x^2 + y^2)$ et $Q(x,y) = (x+y)^2$, l'intégrale
demandée rentre dans les conditions du théorème de Green, puisque le
domaine est le périmètre $\gamma$ d'un triangle plein $T$, dont le bord
admet clairement en chaque point un vecteur normal extérieur. Dès
lors
\begin{equation*}
\begin{split}
\int_\gamma 2 (x^2 + y^2) d x + (x+y)^2 d y &= \iint_T (2 x -
2 y )d x d y\\
&= \int_1^2\int_1^y 2 (x-y) d x d y\\
&= \int_1^2 \crochets{x^2 - 2 x y}_1^y d y\\
&= \int_1^2 (- y^2 - 1 + 2 y) d y = - \frac13
\end{split}
\end{equation*}

\item Le théorème de Green s'applique, et il s'agit donc d'intégrer
\begin{equation*}
- \iint_S 4 x y d x d y
\end{equation*}
où $S$ est le disque de rayon $R$. Par passage en coordonnées polaires, on trouve
\begin{equation*}
- \int_0^{2\pi} \int_0^R 4\rho^3 \sin\theta \cos\theta d \rho d\theta = - R^4 \frac12 \crochets{\sin^2(\theta)}_0^{2\pi} = 0
\end{equation*}
ce qu'on aurait pu deviner en utilisant les symétries du problème.

\item D'après le théorème de Green, on a
\begin{equation*}
\int_\gamma d x + x d y = \int_0^1 \int_{y^2}^{\sqrt y} d x
d y = \frac13
\end{equation*}

\item Tout le monde sait que l'ellipse $E$ a pour aire $\pi a b$. On
peut aussi le calculer en utilisant le théorème de Green~:
\begin{equation*}
\iint_E d x d y = \int_\gamma (d x + x d y)
\end{equation*}
mais c'est plus compliqué.

\item Par le théorème de Green, cette intégrale vaut
\begin{equation*}
-\iint_D x^2 + y^2 = -\int_0^{2\pi}\int_0^1 \rho^3 d\rho d \theta
= -\frac\pi2
\end{equation*}
où $D$ est le disque unité. Ne pas oublier le signe qui vient du fait que l'on demande de tourner dans le sens horloger, alors que $\int_{\partial D}\omega$ n'est égal à $\int_{\gamma}\omega$ que lorsque $\gamma$ parcours $\partial D$ dans le sens trigonométrique, c'est à dire dans l'autre sens.
\end{enumerate}

\paragraph{Exercice 151.}
\begin{enumerate}

\item Si on applique le théorème de Stokes dans le plan $z = 0$, cela revient en fait à appliquer le théorème de Green, en effet le rotationnel vaut $(0, 0, 2 x - 2 y)$ et le vecteur normal vaut $(0,0,1)$. L'intégrale devient donc une intégrale sur le disque $D$ unité dans le plan $z = 0$
\begin{equation*}
\iint_D \nabla\times G \cdot d S = \iint_D (2x - 2 y) d x d y
\end{equation*}
et elle vaut zéro par les symétries (ou par calcul).

Si on applique le théorème de Stokes à la demi sphère unité, on se rappelle que le rotationnel vaut
\begin{equation*}
\nabla\times G = (0, 0, 2x - 2 y)
\end{equation*}
et que pour calculer le flux de $\nabla\times G$ au travers de la demi sphère $S$, on peut utiliser le théorème de la divergence. En effet, celui-ci établit que
\begin{equation*}
\iint_D \nabla\times G \cdot d S + \iint_S \nabla\times G \cdot d S = \iiint_V
\nabla\cdot \nabla\times G
\end{equation*}
où $V$ est la demi sphère unité pleine. Or $\nabla\cdot\nabla\times G = 0$. On en
déduit que l'intégrale est nulle. D'après le calcul ci-dessous, nous
avons donc
\begin{equation*}
\iint_S \nabla\times \cdot d S = 0 - 0 = 0.
\end{equation*}

\item Soit $\gamma(t) = (\cos(t), \sin(t), 0)$. L'intégrale demandée
est
\begin{equation*}
\int_0^{2\pi} (\cos^3(t) \sin(t) + \sin^3(t) \cos(t)) d t = 0
\end{equation*}
ou, par la formule de stokes~:
\begin{equation*}
\iint_D \scalprod{(0,2,0)}{(0,0,1)} d x d y = 0
\end{equation*}
où $(0,0,1)$ représente le vecteur normal au disque unité $D$ dans
le plan $z = 0$.

\item Utilisons le théorème de Stokes. Le rotationnel de $G =
(y+z,z+x,x+y)$ vaut $\nabla\times G = (0,0,0)$. Voilà qui est réglé.

\item 
Le rotationnel de $G = (y,z,x)$ vaut $(-1,-1,-1)$. Le vecteur normal au disque $x^2 + z^2 = a^2$ dans le plan $y = 0$ (ordre des paramètres~: $(x,z)$)vaut $(1,0,0) \wedge (0,0,1) = (0,-1,0)$. On peut donc calculer l'intégrale de
\begin{equation*}
\iint_D d x d y = \pi a^2
\end{equation*}
puisque $\iint_D d x d y$ est l'aire du disque de rayon $a$. On en conclut que
\begin{equation*}
\int_\gamma y d x + z d y + x d z = \pi a^2
\end{equation*}
où $\gamma$ est le cercle donné, orienté par le vecteur $(0,0,1)$ au point $(1,0,0)$.
\end{enumerate}

