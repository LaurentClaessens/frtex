% This is part of Un soupçon de physique, sans être agressif pour autant
% Copyright (C) 2006-2009
%   Laurent Claessens
% See the file fdl-1.3.txt for copying conditions.


\begin{corrige}{INGE11140023}

	Étant donné que l'inconnue $x$ apparaît seulement dans la combinaison $e^{2x}$, il est naturel de poser $y= e^{2x}$. Nous avons alors $ e^{-2x}=\frac{1}{ y }$ et l'équation à résoudre devient
	\begin{equation}
		y-\frac{ 2 }{ y }-1=0
	\end{equation}
	où nous savons que $y\neq 0$ parce qu'une exponentielle ne s'annule jamais. En multipliant par $y$, nous trouvons une équation du second degré $y^2-y-2=0$, dont les solutions sont
	\begin{equation}
		y=\frac{ 1\pm 3 }{2},
	\end{equation}
	c'est à dire $y=2$ ou $y=-1$. La seconde possibilité doit être rejetée parce qu'elle ne correspond à aucune solution en $x$ : $ e^{2x}=-1$ est impossible. La seconde solution, $y=2$, donne
	\begin{equation}
		\begin{aligned}[]
			e^{2x}&=2\\
			2x&=\ln(2)\\
			x&=\frac{ \ln(2) }{ 2 }.
		\end{aligned}
	\end{equation}
	La solution de l'exercice est donc $x=\ln(2)/2$. Cette solution peut aussi être exprimée sous la forme $\ln(\sqrt{2})$.

	\begin{remark}
		Une erreur classique est de prendre le logarithme des deux membres de l'équation et d'écrire
		\begin{equation}
			\ln( e^{2x})-2\ln( e^{-2x})=\ln(1).
		\end{equation}
		La fonction logarithme n'est pas une fonction linéaire. Il n'est donc pas vrai que $\ln(a+2b)=\ln(a)+2\ln(b)$.

		Parmi les variantes de cette erreur, notons l'oubli de prendre le logarithme du second membre~: laisser le $1$ inchangé est une circonstance aggravante. Si on prend le logarithme du membre de gauche, il faut aussi prendre celui du membre de droite. Il est également faux de croire que $\ln(1)=1$ ou que $\ln(1)=e$. Que vaut le logarithme de $1$ ?
	\end{remark}
	

\end{corrige}
