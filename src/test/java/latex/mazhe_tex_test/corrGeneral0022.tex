% This is part of the Exercices et corrigés de mathématique générale.
% Copyright (C) 2009-2011,2015
%   Laurent Claessens
% See the file fdl-1.3.txt for copying conditions.
\begin{corrige}{General0022}

Les courbes s'intersectent en $x=0$ et $x=4$. La surface entre les deux est
\begin{equation}
	S=\int_0^4y_1(x)-\int_0^4y_2(x)=\int_0^4(8x-2x^2)=\left[ \frac{ 8x^2 }{ 2 }-2\frac{ x^3 }{ 3 } \right]_0^4=64-\frac{ 128 }{ 3 }=\frac{ 64 }{ 3 }.
\end{equation}
Note : nous avons calculé $\int(y_1-y_2)$, mais en réalité, c'est un coup de bol. Si la fonction $y_2$ avait été au dessus de $y_1$, nous aurions dû calculer $\int(y_2-y_1)$. Heureusement, il y a un truc : il suffit de voir si le résultat final est positif ou négatif. Si nous avions eut un résultat négatif, alors c'est qu'il fallait calculer l'autre possibilité (ça ne change qu'un signe, donc il ne faut même pas refaire le calcul).

\end{corrige}
