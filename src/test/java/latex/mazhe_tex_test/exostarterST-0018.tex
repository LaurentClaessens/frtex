% This is part of Analyse Starter CTU
% Copyright (c) 2014
%   Laurent Claessens,Carlotta Donadello
% See the file fdl-1.3.txt for copying conditions.

\begin{exercice}\label{exostarterST-0018}

 
Déterminer les ensembles de  primitives suivants :
\begin{multicols}{3}
  \begin{enumerate}
  \item $F_{1}(x)=\displaystyle\int x(x^2+3) \,\mathrm dx$ 
  \item $F_{2}(x)=\displaystyle\int x\sqrt{1+x^2} \,\mathrm dx$ 
  \item $F_{3}(x)=\displaystyle\int \dfrac{x^2}{1+x^3} \,\mathrm dx$ 
  \item $F_{4}(x)=\displaystyle\int x e^x  \,\mathrm dx$ 
  \item $F_{5}(x)=\displaystyle\int\arcsin (x) \,\mathrm dx$ 
  \item $F_{6}(x)=\displaystyle\int x^2 e^x  \,\mathrm dx$ 
  \item $F_{7}(x)=\displaystyle\int\ln (x) \,\mathrm dx$ 
  \item $F_{8}(x)=\displaystyle\int\arctan (x) \,\mathrm dx$ 
  \item $F_{9}(x)=\displaystyle\int x\sin (x) \,\mathrm dx$  
  \item $F_{10}(x)=\displaystyle\int\sin^2 (x)\,\mathrm dx$  
  \item $F_{11}(x)=\displaystyle\int\dfrac{\ln(x)}{x}\,\mathrm dx$
  \item $F_{12}(x)=\displaystyle\int\dfrac{x}{1+x^4}\,\mathrm dx$ 
  \item $F_{13}(x)=\displaystyle\int\dfrac{1}{x^2+4x+5}\,\mathrm dx$
  \item $F_{14}(x)=\displaystyle\int\dfrac{8x}{x^2+4x+5}\,\mathrm dx$
  \item $F_{15}(x)=\displaystyle\int \dfrac{1}{x\sqrt{1-\ln (x)}}\,\mathrm dx$  (poser $t=1-\ln(x)$).



  \end{enumerate}
\end{multicols}


\corrref{starterST-0018}
\end{exercice}
