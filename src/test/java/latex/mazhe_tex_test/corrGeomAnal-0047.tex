\begin{corrige}{GeomAnal-0047}

    \begin{enumerate}
        \item
            Il n'y a pas de fonctions répondant aux conditions parce qu'elle devrait satisfaire
            \begin{equation}
                \frac{ \partial^2f  }{ \partial y\partial y }=2y
            \end{equation}
            et
            \begin{equation}
                \frac{ \partial^2f  }{ \partial x\partial y }=0.
            \end{equation}
            Ces deux fonctions n'étant pas égales, il n'y a pas de fonctions \( f\).

            \begin{remark}
                Beaucoup d'étudiants se sont lancés avec plus ou moins de succès dans l'intégration des deux condition. On obtient alors
                \begin{subequations}
                    \begin{align}
                        f(x,y)&=-\sin(x)+xy^2+\alpha(y)\\
                        f(x,y)&=y\cos(x)+2xy+\beta(x).
                    \end{align}
                \end{subequations}
                Il faut alors justifier qu'il n'existe pas de fonctions \( \alpha\) et \( \beta\) qui rendent égales ces deux expressions.

                Dans ce cas ci, cela n'était pas la méthode la plus simple.
            \end{remark}

        \item
            Ici le «truc» est de calculer les dérivées partielles de \( g\) dans les directions \( u=(1,2)\) et \( v=(1,-1)\). D'abord \( g(0,0)=0\), ensuite
            \begin{equation}
                \frac{ \partial g }{ \partial u }(0,0)=\lim_{t\to 0} \frac{ g(t,2t)-g(0,0) }{ t }=\lim_{t\to 0} \frac{ 2t }{ t }=2
            \end{equation}
            et
            \begin{equation}
                \frac{ \partial g }{ \partial v }=\lim_{t\to 0} \frac{ g(t,-t) }{ t }=-2.
            \end{equation}
            Les dérivées directionnelles étant les combinaisons linéaires des dérivées partielles nous obtenons le système d'équations
            \begin{subequations}
                \begin{numcases}{}
                    \frac{ \partial f }{ \partial x }(0,0)+2\frac{ \partial f }{ \partial y }(0,0)=2
                    \frac{ \partial f }{ \partial x }(0,0)-\frac{ \partial f }{ \partial y }(0,0)=-2
                \end{numcases}
            \end{subequations}
            pour les \emph{nombres} \( \partial_xf(0,0)\) et \( \partial_yf(0,0)\). Le résultat est
            \begin{equation}
                \begin{aligned}[]
                    \frac{ \partial f }{ \partial x }&=-\frac{ 2 }{ 3 }&&\frac{ \partial f }{ \partial y }&=\frac{ 4 }{ 3 }.
                \end{aligned}
            \end{equation}
            Nous avons donc
            \begin{equation}
                \nabla f(0,0)=-\frac{ 2 }{ 3 }\begin{pmatrix}
                    1    \\ 
                    0    
                \end{pmatrix}+\frac{ 4 }{ 3 }\begin{pmatrix}
                    0    \\ 
                    1    
                \end{pmatrix}=\frac{ 2 }{ 3 }\begin{pmatrix}
                    -1    \\ 
                    3    
                \end{pmatrix}.
            \end{equation}
    \end{enumerate}

\end{corrige}
