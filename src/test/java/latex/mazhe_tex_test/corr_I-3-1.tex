% This is part of the Exercices et corrigés de CdI-2.
% Copyright (C) 2008, 2009
%   Laurent Claessens
% See the file fdl-1.3.txt for copying conditions.


\begin{corrige}{_I-3-1}

Nous vérifions facilement que toutes les hypothèses de la proposition  \ref{PropDerrFnAvecBornesFonctions} sont satisfaites avec $\partial f/\partial x=0$, $\varphi(x)=a-x$ et $\psi(x)=a+x$. La formule donne alors
\begin{equation}
	\frac{ dF }{ dx }=f(a+x)+f(a-x).
\end{equation}
Une autre possibilité est de séparer l'intégrale en deux parties :
\begin{equation}
	F(x)=\int_{a-x}^af(t)dt + \int_{a}^{a+x}f(t)dt
\end{equation}
et d'appliquer le théorème fondamental de l'analyse :
\begin{equation}
	\begin{aligned}[]
	\frac{ dF }{ dx }	&=\frac{ d }{ d(a+x) }\int_a^{a+x}f(t)dt + \frac{ d }{ d(a-x) }\left( -\int_a^{a-x}f(t)dt \right)(-1)\\
				&=f(a+x)+f(a-x).
	\end{aligned}
\end{equation}
Le dernier signe moins vient du changement de variable $x\mapsto a-x$ fait pour calculer la dérivée.

\end{corrige}
