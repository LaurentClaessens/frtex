% This is part of the Exercices et corrigés de CdI-2.
% Copyright (C) 2008, 2009,2013
%   Laurent Claessens
% See the file fdl-1.3.txt for copying conditions.


\begin{corrige}{_I-3-8}

Grâce à l'exponentielle décroissante, la fonction tend vers zéro à l'infini assez vite pour que l'intégrale converge de ce côté. En ce qui concerne l'éventuel problème en $\pi$, nous calculons
\begin{equation}
	\lim_{t\to\pi}\frac{  e^{-xt}\sin(t) }{ \sqrt{t-\pi} }= e^{-\pi x}\lim_{t\to\pi}\frac{\sin(t) }{ \sqrt{t-\pi} }=\lim_{t\to\pi}2\sqrt{t-\pi}\cos(t)=0,
\end{equation}
il n'y a donc pas de singularité en $\pi$, contrairement à ce que l'on aurait pu craindre à première vue. 

La fonction  $f(x,t)$ sous le signe intégral est une fonction continue et bornée (comme toute fonction continue sur un compact) sur le compact $x\in[0,1]$ et $t\in[\pi,\pi+1]$. Nous la majorons par une constante sans importance. L'important est que pour l'intervalle non compact $t\in[\pi+1,\infty[$, nous pouvons faire la majoration 
\begin{equation}
	\left| \frac{  e^{-xt}\sin(t) }{ (t-\pi)^{\frac{1}{ 2 }} } \right| \leq e^{-xt}.
\end{equation}
Sur $x>0$, nous avons $\int_{\pi}^{\infty} e^{-xt}dt=\frac{  e^{-\pi x} }{ x }$, qui est un nombre bien défini tant que $x>0$. Le théorème \ref{ThoCritWeiIntUnifCv} (critère de Weierstrass) conclut que $F(x)$ converge uniformément sur tout compact de $]0,1]$. Elle y est donc continue.

\end{corrige}
