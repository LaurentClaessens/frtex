% This is part of Exercices de mathématique pour SVT
% Copyright (C) 2010
%   Laurent Claessens et Carlotta Donadello
% See the file fdl-1.3.txt for copying conditions.

\begin{corrige}{TD3-0003}

	\begin{enumerate}
		\item
			Mettons le plus haut degré en évidence au numérateur et au dénominateur :
			\begin{equation}
				u_n=\frac{ n^2\left( 1+\frac{ 2n }{ n^2 }+\frac{ 1 }{ n^2 } \right) }{ n^3\left( 1+\frac{ 1 }{ n^3 }+\frac{ n^2 }{ n^3 } \right) }=\frac{ 1+\frac{ 3 }{ n }+\frac{1}{ n^2 } }{ n\left( 1+\frac{1}{ n^3 }+\frac{1}{ n } \right) }.
			\end{equation}
			Cette dernière expression montre que le limite est zéro lorsque $n\to \infty$.
		\item
			C'est le produit d'une suite bornée ($|\sin(n)|$) par une suite qui tend vers zéro ($\frac{1}{ n+1 }$). Cette suite converge donc vers zéro.
		\item
			Il s'agit d'une exponentielle en base deux ($2^n$) divisée par un polynôme; la limite est donc $\infty$ parce que l'exponentielle va toujours plus vite.
		\item
			Il faut se rappeler que $ e^{-n}=\frac{1}{  e^{n} }$. La formule donnée est donc un polynôme \emph{divisée} par une exponentielle. Le résultat tend vers zéro.
		\item
			Ici encore il s'agit d'une limite remarquable. Le logarithme au numérateur tend vers l'infini moins vite que le polynôme au dénominateur. Le résultat est que la limite est zéro.
		\item
			Il faut multiplier et diviser par le binôme conjugué. Nous avons
			\begin{equation}
				\begin{aligned}[]
						u_n&=\frac{ \left( \sqrt{n^2+n+1}-\sqrt{n^2-n+1} \right)\left( \sqrt{n^2+n+1}+\sqrt{n^2-n+1} \right) }{  \sqrt{n^2+n+1}+\sqrt{n^2-n+1}  }\\
						&=\frac{ (n^2+n+1)-(n^2-n+1) }{    \sqrt{n^2+n+1}+\sqrt{n^2-n+1}  }\\
						&=\frac{ 2n }{   \sqrt{n^2+n+1}+\sqrt{n^2-n+1}   }.
				\end{aligned}
			\end{equation}
			Maintenant il faut mettre la plus haute puissance de $n$ en évidence dans le dénominateur. La manipulation à faire est
			\begin{equation}
				\begin{aligned}[]
					\sqrt{n^2+n+1}&=\sqrt{n^2\left( 1+\frac{1}{ n }+\frac{1}{ n^2 } \right)}\\
					&=| n |\sqrt{1+\frac{1}{ n }+\frac{1}{ n^2 }}\\
					&= n \sqrt{1+\frac{1}{ n }+\frac{1}{ n^2 }}
				\end{aligned}
			\end{equation}
			où nous avons pu enlever la valeur absolue parce que nous considérons que $n\to\infty$. Ce $n$ est donc certainement positif. Nous avons donc
			\begin{equation}
				\begin{aligned}[]
					u_n&=\frac{ 2n }{   \sqrt{n^2+n+1}+\sqrt{n^2-n+1}   }\\
					&=\frac{ 2n }{ n \sqrt{1+\frac{1}{ n }+\frac{1}{ n^2 }}+n\sqrt{1-\frac{1}{ n }+\frac{1}{ n^2 }} }.
				\end{aligned}
			\end{equation}
			À ce moment nous pouvons simplifier par $n$. Il reste $2$ au numérateur tandis que le dénominateur tend vers $2$. La limite est donc $1$.
		\item
			Ici il faut commencer par multiplier et diviser par le binôme conjugué, c'est à dire par
			\begin{equation}
				\sqrt{n+\sqrt{n^2+1}}+\sqrt{n+\sqrt{n^2-1}}.
			\end{equation}
			Nous trouvons
            \begin{equation}        \label{Eqtdtzzztsqg}
                \sqrt{n+\sqrt{n^2+1}}-\sqrt{n+\sqrt{n^2-1}}=\frac{  \sqrt{n^2+1}-\sqrt{n^2-1} }{ \sqrt{n}\sqrt{1+\sqrt{ 1+\frac{1}{n} }  }+\sqrt{n}\sqrt{1+\sqrt{ 1-\frac{1}{n} }} }.
			\end{equation}
            Maintenant le dénominateur tend clairement vers l'infini tandis que le numérateur est encore une forme indéterminée. Il faut donc le traiter encore une fois avec le coup du binôme conjugué :
            \begin{equation}
                \sqrt{n^2+1}-\sqrt{n^2-1}=\frac{ n^2+1-n^2+1 }{ \sqrt{n^2+1}+\sqrt{n^2-1} }.
            \end{equation}
            Par conséquent le numérateur de \eqref{Eqtdtzzztsqg} tend vers zéro lorsque \( n\to\infty\). Au final, nous avons
            \begin{equation}
                \lim_{n\to \infty} \sqrt{n+\sqrt{n^2+1}}-\sqrt{n+\sqrt{n^2-1}}=0
            \end{equation}
		\item
			Nous multiplions et divisons par le binôme conjugué des du numérateur \emph{et} du dénominateur. Nous trouvons alors
			\begin{equation}
				\begin{aligned}[]
					u_n&=\frac{ ( n-\sqrt{n^2+1} )( n+\sqrt{n^2+1} )( n+\sqrt{n^2-1} ) }{ (n-\sqrt{n^2-1})(n+\sqrt{n^2+1})(n+\sqrt{n^2-1}) }\\
					&=\frac{ \big( n^2-(n^2+1) \big)\big( n+\sqrt{n^2-1} \big) }{ \big( n+\sqrt{n^2+1} \big)\big( n^2-(n^2-1) \big) }\\
					&=\frac{ -n-\sqrt{n^2-1} }{ n+\sqrt{n^2+1} }\\
					&=\frac{ -n-n\sqrt{1-\frac{1}{ n^2 }} }{ n+n\sqrt{1+\frac{1}{ n^2 }} }\\
					&=\frac{ -1-\sqrt{1-\frac{1}{ n^2 }} }{ 1+\sqrt{1+\frac{1}{ n^2 }} },
				\end{aligned}
			\end{equation}
			et la dernière ligne tend vers $1$ lorsque $n\to\infty$.

			Remarquez qu'on a effectué la manipulation
			\begin{equation}
				\sqrt{n^2-1}=\sqrt{n^2\left( 1-\frac{1}{ n^2 } \right)}=n\sqrt{1-\frac{1}{ n^2 }}
			\end{equation}
			avant de simplifier par $n$.
	\end{enumerate}
	

\end{corrige}




