% This is part of Analyse Starter CTU
% Copyright (c) 2015
%   Laurent Claessens,Carlotta Donadello
% See the file fdl-1.3.txt for copying conditions.

\begin{corrige}{exoanalyse-0104}

  \begin{enumerate}
  \item Nous allons utiliser la variable $t= \sqrt{x}$, donc $dt = \frac{1}{2\sqrt{x}} dx$. On a alors  
    \begin{equation*}
      \begin{aligned}
        \int &\frac{1+e^{\sqrt{x}}}{\sqrt{x}}\, dx = 2\int (1+e^{\sqrt{x}})\frac{1}{2\sqrt{x}}\, dx \\ 
&= 2 \int 1+e^t \, dt = 2\left[t + e^t\right]_{t=\sqrt{x}} + C = 2\left(\sqrt{x} + e^{\sqrt{x}}\right) + C.
      \end{aligned}
    \end{equation*}
  \item Nous pouvons prendre $t= \sqrt{(\ln(x))^2-1}$, 
\[
dt = \frac{\ln(x)}{x\sqrt{(\ln(x))^2-1}}\, dx.  
\]
On a 
\begin{equation*}
  \begin{aligned}
    \int &\frac{\ln(x)}{x}\sqrt{(\ln(x))^2-1}\, e^{\sqrt{(\ln(x))^2-1}}\, dx = \int ((\ln(x))^2-1) e^{\sqrt{(\ln(x))^2-1}}\,\frac{\ln(x)}{x\sqrt{(\ln(x))^2-1}}\, dx \\
    &= \int t^2 e^t\, dt= \left[(t^2-2t +2) e^t\right]_{t=\sqrt{(\ln(x))^2-1}} + C \\
    & = \left((\ln(x))^2+1 -2 \sqrt{(\ln(x))^2-1} \right) e^{\sqrt{(\ln(x))^2-1}} + C.
  \end{aligned}
\end{equation*}
  \item Soit $t= x^2-1$, $dt = 2x dx$. Alors $\displaystyle \int \frac{x}{\sqrt{x^2-1}}\, dx = \int \frac{1}{2\sqrt{t}} \, dt$.  Therefore the primitive is $\sqrt{x^2-1} +C$.
  \item Le point pr\'ec\'edent nous permet d'effectuer une int\'egration par parties 
    \begin{equation*}
      \begin{aligned}
        \int&  \frac{x}{\sqrt{x^2-1}} \arctan{\sqrt{x-1}}\, dx = \sqrt{x^2-1} \arctan{\sqrt{x-1}} - \int \frac{\sqrt{x^2-1}}{2x\sqrt{x-1}}\, dx\\
&=\sqrt{x^2-1} \arctan{\sqrt{x-1}} - \int \frac{\sqrt{x+1}}{2x}\, dx.
      \end{aligned}
    \end{equation*}
Pour int\'egrer le second terme nous utilisons le changement de variable $t= \sqrt{x+1}$, $dt = (2\sqrt{x+1})^{-1} dx$ 
\begin{equation*}
  \begin{aligned}
     \int& \frac{\sqrt{x+1}}{2x}\, dx= \int \frac{x+1}{x}\,\frac{1}{2\sqrt{x+1}} dx\\
&= \int \frac{t^2}{t^2-1} \,dt = \int 1+\frac{1}{t^2-1} \,dt\\
&=\int 1+\frac{1}{2(t-1)}-\frac{1}{2(t+1)} \,dt\\
&= \left[t + \frac{1}{2}\ln\left(\frac{t-1}{t+1}\right)\right]_{t=\sqrt{x+1}} + C.
      \end{aligned}
    \end{equation*}
Au final nous avons 
\begin{equation*}
      \begin{aligned}
        \int&  \frac{x}{\sqrt{x^2-1}} \arctan{\sqrt{x-1}}\, dx = \\
&=\sqrt{x^2-1} \arctan{\sqrt{x-1}} - \left[\sqrt{x+1} + \frac{1}{2}\ln\left(\frac{\sqrt{x+1}-1}{\sqrt{x+1}+1}\right)\right] + C.
      \end{aligned}
    \end{equation*}
  \item Nous prenons $t = \sqrt{x}$ comme dans le premier point de cet exercice. \[
\int \frac{1}{x+\sqrt{x}}\, dx = \int \frac{2}{\sqrt{x} +1} \frac{1}{2\sqrt{x}}\, dx = \int  \frac{2}{t +1}  \, dt, 
\]
Donc la primitive est $2\ln(|\sqrt{x} +1|) +C$.
  \end{enumerate}

\end{corrige}

