% This is part of the Exercices et corrigés de CdI-2.
% Copyright (C) 2008, 2009
%   Laurent Claessens
% See the file fdl-1.3.txt for copying conditions.


\begin{corrige}{_II-1-06}

L'équation homogène est $y''+y=0$, dont la solution est $y_H=A\cos(t)+B\sin(t)$. Afin de trouver une solution particulière de l'équation non homogène, nous essayons
\begin{equation}
	y=\alpha\cos(\omega t),
\end{equation}
ce qui mène à l'équation $-\omega^2\alpha+\alpha=1$, donc
\begin{equation}
	\alpha=\frac{1}{ 1-\omega^2 }.
\end{equation}
Si $\omega\neq 1$ (on a supposé que $\omega\geq0$), les solutions sont donc
\begin{equation}
	y(t)=A\cos(t)+B\sin(t)+\frac{1}{ 1-\omega^2 }\cos(\omega t).
\end{equation}
Étant donné que $y(0)=A+\frac{1}{ 1-\omega^2 }$ et $y'(0)=B$, nous trouvons
\begin{equation}
	y_A=\left( y_0-\frac{1}{ 1-\omega^2 } \right)\cos(t)+y'_0\sin(t)+\frac{1}{ 1-\omega^2 }\cos(\omega t).
\end{equation}

Par contre, si $\omega=1$, cette solution ne fonctionne pas, et nous cherchons une solution particulière de l'équation non homogène sous la forme $y(t)=\alpha t\cos(t)+\beta t\sin(t)$. En dérivant deux fois et en remplaçant dans l'équation départ, nous trouvons
\begin{equation}
	\sin(t)(-2\alpha-\beta t+\beta t)+\cos(t)(-\alpha t+2\beta+\alpha t)=\cos(t),
\end{equation}
donc $\alpha=0$ et $\beta=\frac{ 1 }{2}$. Dans le cas $\omega=1$, la solution est donc
\begin{equation}
	y=A\cos(t)+B\sin(t)+\frac{ t }{ 2 }\sin(t).
\end{equation}
Dans ce cas-ci, nous avons $y(0)=A$ et $y'(0)=B$, donc
\begin{equation}
	y_B=y_0\cos(t)+y'_0\sin(t)+\frac{ t }{2}\sin(t).
\end{equation}

Prouver la convergence de $\lim_{\omega\to 0}y_A=y_B$, nous calculons $y_A-y_B$ :
\begin{equation}
	y_A-y_B=\frac{ \cos(\omega t)-\cos(t) }{ 1-\omega^2 }-\frac{ t }{2}\sin(t).
\end{equation}
Nous calculons donc (avec la règle de l'Hospital par rapport à $\omega$)
\begin{equation}
	\begin{aligned}[]
		\lim_{\omega\to 1}\frac{ \cos(\omega t)-\cos(t) }{ 1-\omega^2 }-\frac{ t }{2}\sin(t)=\lim_{\omega\to 1}\frac{ -t\sin(\omega t) }{2}-\frac{ t }{2}\sin(t)=0.
	\end{aligned}
\end{equation}

\end{corrige}
