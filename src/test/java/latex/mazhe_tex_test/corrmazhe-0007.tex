% This is part of Analyse Starter CTU
% Copyright (c) 2014
%   Laurent Claessens,Carlotta Donadello
% See the file fdl-1.3.txt for copying conditions.


\begin{corrige}{mazhe-0007}
  L'ensemble de définition de $g$ est $\eR$ tout entier. Cela arrive parce que $1+x^2 >0$ pour tout $x$, $\sqrt{1+x^2}>1$ pour tout $x$ et 
\[
\left|\frac{x}{\sqrt{1+x^2}}\right| \leq 1, \qquad \text{ pour tout } x\in\eR . 
\]
Cette dernière inégalité est facile à démontrer si on utilise le fait que $|x| = \sqrt{(x)^2}$ pour tout $x\in\eR$.

La dérivée de $g$ est 
\begin{equation*}
  \begin{aligned}
    g' (x) = &\frac{1}{\sqrt{1-\left(\frac{x}{\sqrt{1+x^2}}\right)^2}} \cdot \frac{\sqrt{1+x^2} -\frac{x^2}{\sqrt{1+x^2}}}{1+x^2} = \\
    &= \frac{\sqrt{1+x^2}}{\sqrt{(1+x^2)-x^2}}\cdot\frac{\frac{(1+x^2) -x^2}{\sqrt{1+x^2}}}{1+x^2} =\frac{1}{1+x^2}.
  \end{aligned}
\end{equation*}

Les fonctions $g$  et $\arctan$ ont donc la m\^eme dérivée et le m\^eme ensemble de définition. On peut en conclure que la fonction différence entre $g$ et $\arctan$  est une constante.  On trouve $g(0)-\arctan(0) = 0$. Les deux fonctions sont donc égales. 
 
\end{corrige}
