% This is part of the Exercices et corrigés de CdI-2.
% Copyright (C) 2008, 2009
%   Laurent Claessens
% See the file fdl-1.3.txt for copying conditions.


\begin{corrige}{_II-1-15}


Nous posons $z=y'$, c'est à dire
\begin{equation}
	y'(t)=z\big( y(t) \big).
\end{equation}
La dérivée seconde est donnée par
\begin{equation}
	y''(t)=z'\big( y(t) \big)y'(t)=z'\big( y(t) \big)z\big( y(t) \big),
\end{equation}
qui peut encore être écrit sous la forme ingénieuse
\begin{equation}
	y''(t)=\frac{ 1 }{2}\frac{ d }{ dy }\Big[ z^2(y) \Big]_{y(t)}.
\end{equation}
Si nous notons $u$ la variable de $z$, nous avons
\begin{equation}
	z^2(y)=2\frac{ k^2 }{ u }+C,
\end{equation}
et donc
\begin{equation}
	z(u)=\pm\sqrt{ 2\frac{ k^2 }{ u }+C }.
\end{equation}
Donc, pour refaire le passage vers les anciennes variables
\begin{equation}		\label{EqEqPourPrimeII115}
	z\big( y(t) \big)=\pm\left( \frac{ 2k^2 }{ y(t) }+C \right)^{1/2}=y'(t).
\end{equation}
L'équation (très implicite) pour $y$ prend alors la forme (voir l'astuce de la page \pageref{SecFairedzdt})
\begin{equation}		\label{EqGeneII115IntCsol}
	t-t_0=\pm\int_{y(t_0)}^y\left( \frac{ 2k^2 }{ \xi }+C \right)^{-1/2}d\xi.
\end{equation}


Le jeu est maintenant de trouver le $C$, ainsi que le signe $\pm$ à choisir et effectuer l'intégrale dans les différents cas proposés. Calculons donc cette intégrale :
\begin{equation}
	I=\int\sqrt{\frac{ \xi }{ C\xi+k^2 }}d\xi.
\end{equation}
Commençons par le changement de variable 
\begin{equation}
	\begin{aligned}[]
		u&=\sqrt{\xi/(C\xi+k^2)},	&	d\xi&=\frac{ du }{ u }\frac{ k^2 }{ \sqrt{2(C\xi+k^2)^2} }=\frac{ 2 }{ k^2 }u(C\xi+k^2)^2du.
	\end{aligned}
\end{equation}
L'intégrale à calculer devient alors
\begin{equation}
	I=2k^2\int\frac{ u^2 }{ (1-Cu^2)^2 }du,
\end{equation}
qui se traite avec le changement de variable
\begin{equation}
	\begin{aligned}[]
		v&=\frac{ 1 }{ 1-Cu^2 },	&du&=\frac{ (1-Cu^2)^2 }{ 2uC }dv.
	\end{aligned}
\end{equation}
Ce sur quoi nous tombons est
\begin{equation}		\label{IntII115avecC}
	I=\frac{ k^2 }{ C }\int udv=\frac{ k^2 }{ C }\left( \frac{ u }{ 1-Cu^2 } -\int\frac{ du }{ 1-Cu^2 } \right).
\end{equation}
Cette intégrale dépend de la valeur de $C$.


Nous cherchons la valeur de $C$ en reprenant la formule \eqref{EqEqPourPrimeII115} pour $y'(t)$. Nous trouvons
\begin{equation}
	y'(0)=\sqrt{2}k=\pm\left( \frac{ 2k^2 }{ y(0) }+C \right)^{1/2}.
\end{equation}
Étant donné que $y(0)=1$, nous en déduisons $C=0$. Le signe à choisir est $+$ parce que $k>0$. Les conditions initiales données étant en zéro, nous avons $t_0=0$, et nous nous retrouvons à calculer
\begin{equation}
	t=\int_1^y\sqrt{\frac{ \xi }{ 2k^2 }}d\xi=\frac{ 2 }{ 3 }\frac{1}{ \sqrt{2}k }\big( y^{3/2}-1 \big).
\end{equation}
De là, sortir $y(t)$ est une opération algébrique simple :
\begin{equation}
	y(t)=\left[ \frac{ 3\sqrt{2}k }{ 2 }\Big( t+\frac{ 2 }{ 3 }\frac{ 1 }{ \sqrt{2}k } \Big)    \right]^{2/3}=\left( \frac{ 3k }{ \sqrt{2} }t+1 \right)^{2/3}.
\end{equation}

Cherchons maintenant les valeurs de $C$ pour les autres données de Cauchy.
\begin{enumerate}
\setcounter{enumi}{1}

\item
$y(0)=1$, $y'(0)=2k$.
L'équation \eqref{EqEqPourPrimeII115} nous donne $C$ :
\begin{equation}
	2k=\pm\left( \frac{ 2k^2 }{ 1+C } \right)^{1/2},
\end{equation}
donc $C=2k^2$. Il faut prendre le signe $+$ parce que $k$ est positif.

\item
$2k^2+C=k^2$, donc $C=-k^2$,
\item
$2k^2+C=0$, donc $C=-2k^2$.

\end{enumerate}
Dans ces trois cas, la formule générale \eqref{EqGeneII115IntCsol} devient
\begin{equation}		\label{EqtIyII115}
	t=[I]_0^y
\end{equation}
parce que $t_0=0$ dans les trois cas.

Passons à la résolution du second problème de Cauchy. L'intégrale \eqref{IntII115avecC} peut être calculée :
\begin{equation}
	I=\frac{ k^2 }{ C }\left( \frac{ u }{ 1-k^2u^2 }  -\int\frac{ du }{ 1-k^2u^2 } \right)=\frac{ u }{ 1-k^2u^2 }-\frac{1}{ 2k }\ln\left| \frac{ ku+1 }{ ku-1 } \right|.
\end{equation}
Il faut maintenant faire les changements de variables  inverse :
\begin{equation}		\label{EqufracII115cici}
	u=\frac{1}{ k }\sqrt{\frac{ \xi }{ \xi+1 }}.
\end{equation}
L'équation \eqref{EqtIyII115} donne
\begin{equation}
	\left[\frac{ k^2 }{ C }\left( \frac{ u }{ 1-k^2u^2 }  -\int\frac{ du }{ 1-k^2u^2 } \right)=\frac{ u }{ 1-k^2u^2 }-\frac{1}{ 2k }\ln\left| \frac{ ku+1 }{ ku-1 } \right|\right]_0^y
\end{equation}
dans laquelle il faut remettre \eqref{EqufracII115cici}. Après quelque calculs, nous trouvons
\begin{equation}
	-\frac{ \sqrt{2} }{ k }-\frac{1}{ 2k }\ln\left| \frac{ \sqrt{2}+1 }{ \sqrt{2}-1 } \right|.
\end{equation}

Passons au second problème de Cauchy. Cette fois,
\begin{equation}
	I=\frac{ -u }{ 1+k^2u^2 }+\frac{1}{ k }\arctan(ku),
\end{equation}
et 
\begin{equation}
	u=\frac{1}{ k }\sqrt{\frac{ \xi }{ 1-\xi }}.
\end{equation}
Encore une fois, il faut utiliser la formule $t=[I]_0^t$, et puis remplacer.


Faisons le troisième problème de Cauchy. Cette fois, $C=-2k^2$. Étant donné que $y'(0)=0$ et $y(0)=1$, l'équation de départ devient 
\begin{equation}
	y''(0)=-k^2,
\end{equation}
qui est négative. Mais comme $y'(0)=0$, la dérivée $y'(t)$ est négative, en tout cas pour les petits $t$. Par conséquent, $0<y(t)<1$ sur un voisinage de $0$ (parce que $y(0)=1$), donc la fraction dans
\begin{equation}
	y'(t)=\pm\left( \frac{ 2k^2-2k^2y(t) }{ y(t) } \right)
\end{equation}
est positive, et il faut choisir le signe $-$ afin que $y'(t)$ soit négatif. Nous devons donc, cette fois, utiliser la formule
\begin{equation}
	t=-[I]_0^{y(t)},
\end{equation}
 et tout ce qui s'ensuit.


\end{corrige}
