\begin{corrige}{LimiteContinue0010}

%Nous considérons la fonction $g$ qui associe au couple $(w, y)$ le couple $(x, y)$ , où $w$ est la différence entre $x$ et $y$ : $w=x-y$. La fonction composée $F = f \circ g (w,y)$ est 
%\[
%F(w,y)=\frac{\sin(y+w)-\sin(y)}{w}.
%\]
%La limite de $F$ pour $(w, y)$ qui tend vers $(0, a)$, cela veut dire que le couple original $(x,y)$ tend vers $(a,a)$,  est $\cos a$.   

Le but de l'exercice est de déterminer la limite $\lim_{(x,y)\to(a,a)}f(x,y)$ pour tout $a$. Le théorème des accroissements finis en une dimension nous enseigne que pour chaque $x$ et $y$, il existe un nombre $\xi(x,y)$ tel que
\begin{equation}
	\sin(y)=\sin(x)+\cos\big( \xi(x,y) \big)(y-x).
\end{equation}
Cela est le théorème \ref{ThoAccFinis}. Donc la fonction $f$ peut être écrite sous la forme
\begin{equation}
	f(x,y)=\frac{ -\cos\big( \xi(x,y) \big)(y-x) }{ x-y }=\cos\big( \xi(x,y) \big).
\end{equation}
Ceci est une composée de fonctions. Il nous faut trouver la limite $\lim_{(x,y)\to(a,a)}\xi(x,y)$. Nous savons que pour tout $x$ et $y$, le nombre $\xi(x,y)$ est strictement compris entre $x$ et $y$, donc
\begin{equation}
	\big| \xi(x,y)-a \big|\leq\max\big\{ | x-a |,| y-a | \big\}.
\end{equation}
Par conséquent, $\lim_{(x,y)\to(a,a)}\xi(x,y)=a$, et
\begin{equation}
	\lim_{(x,y)\to(a,a)}f(x,y)=\cos(a).
\end{equation}

Nous pouvons donc prolonger $f$ par continuité en posant
\begin{equation}
	f(x,y)=\begin{cases}
		\displaystyle\frac{ \sin(x)-\sin(y) }{ x-y }	&	\text{si $x\neq y$}\\
		\cos(x)	&	 \text{si $x=y$.}
	\end{cases}
\end{equation}

\end{corrige}
