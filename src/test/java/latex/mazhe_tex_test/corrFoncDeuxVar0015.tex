% This is part of the Exercices et corrigés de mathématique générale.
% Copyright (C) 2009-2010
%   Laurent Claessens
% See the file fdl-1.3.txt for copying conditions.


\begin{corrige}{FoncDeuxVar0015}

	Lorsqu'un a un sinus de quelque chose qui tend vers zéro et qu'on ne sait pas très bien quoi faire, une bonne idée est toujours de multiplier et diviser par ce qui est dans le sinus. De cette façon, le sinus disparaît. Ici, nous avons
	\begin{equation}
		f(x,y)=\frac{ \sin\sqrt{x^2+y^2} }{ \sqrt{x^2+y^2} }\cdot\sqrt{x^2+y^2}\cdot\ln\sqrt{x^2+y^2}.
	\end{equation}
	Lors de la limite, le premier morceau est la limite bien connue qui fait $1$. Il reste à trouver la limite de la fonction
	\begin{equation}
		\sqrt{x^2+y^2}\ln\sqrt{x^2+y^2}.
	\end{equation}
	Cette fonction est la composée $f(x,y)=(g\circ r)(x,y)$ où
	\begin{equation}
		\begin{aligned}[]
			r(x,y)&=\sqrt{x^2+y^2}\\
			g(t)&=t\ln(t).
		\end{aligned}
	\end{equation}
	Évidement, la limite de $r$ lorsque on tend vers $(0,0)$ est zéro. Voyons la limite de $g(t)$ lorsque $t\to 0$. Cela se règle en utilisant la règle de l'Hopital :
	\begin{equation}
		\lim_{t\to 0}t\ln(t)=\lim_{t\to 0}\frac{ \ln(t) }{ 1/t }=\lim_{t\to 0}\frac{ 1/r }{ -1/r^2 }=\lim_{t\to0}-t=0.
	\end{equation}
	Donc
	\begin{equation}
		\lim_{(x,y)\to(0,0)}f(x,y)=0.
	\end{equation}

\end{corrige}
