\begin{exercice}\label{exoSC_serie4-0001}

	La chaleur spécifique $C_v$ (en \joule\per\kelvin\per\mole) d'un solide monoatomique varie en fonction de la température absolue $T$ suivant la loi
	\begin{equation}
		C_V=\frac{ 9R }{ x_m^3 }\int_0^{x_m}\frac{  e^{x}x^4 }{ (e^x-1)^2 }dx
	\end{equation}
	où $R=8.314\joule\per\kelvin\per\mole$ et $x_m=\Theta_D/T$, $\Theta_D$ étant la température de Debye, qui dépend du solide considéré. Déterminer la chaleur spécifique pour $T=300\kelvin$, dans le cas du cuivre, pour lequel $\Theta_D=313\kelvin$.

\corrref{SC_serie4-0001}
\end{exercice}
