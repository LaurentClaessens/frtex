% This is part of Exercices de mathématique pour SVT
% Copyright (c) 2010
%   Laurent Claessens et Carlotta Donadello
% See the file fdl-1.3.txt for copying conditions.

\begin{corrige}{TD5-a-0001}

	\begin{enumerate}
		\item
			Lorsqu'on a une somme à intégrer, on peut prendre l'intégrale terme à terme :
			\begin{equation}
				\begin{aligned}[]
					\int_{0}^1x^5+3x^2+3\,dx&=\int_0^1x^5+3\int_0^1x^2+3\int_0^1 1dx\\
					&=\left[ \frac{ x^6 }{ 6 } \right]_0^1+3\left[ \frac{ x^3 }{ 3 } \right]_0^1+3\left[ x \right]_0^1\\
					&=\frac{ 25 }{ 6 }.
				\end{aligned}
			\end{equation}
			Nous avons utilisé le fait que $\int x^n=\frac{ x^{n+1} }{ n+1 }$. Par exemple, une primitive de $x^5$ est $\frac{ x^6 }{ 6 }$.
		\item
			Ici nous utilisons la même technique, par exemple $\int x^{1/3}dx=\frac{ x^{4/3} }{ 4/3 }=\frac{ 3 }{ 4 }x^{4/3}$. La réponse est
			\begin{equation}
				-\frac{2^{1/3}}{ 24 }\big( 412^{2/3}-128\cdot 2^{1/6}-36 \big).
			\end{equation}
		\item
			Une primitive de $\cos(x)$ est $\sin(x)$. Donc nous avons
			\begin{equation}
				\int_{-\pi}^0\cos(x)dx=\left[ \sin(x) \right]_{-\pi}^0=\sin(0)-\sin(-\pi)=0.
			\end{equation}
		\item
			Une primitive de $e^x$ est la fonction $e^x$ elle-même. Nous avons donc
			\begin{equation}
				\int_{-2}^1 e^{x}dx=\left[ e^x \right]_{-2}^1=e-e^{-2}.
			\end{equation}
		\item
			Étant donné que $\frac{1}{ \sqrt{x} }=x^{-1/2}$, nous pouvons utiliser la formule usuelle d'intégration de puissance de $x$ :
			\begin{equation}
				\int_0^1\frac{1}{ \sqrt{x} }dx=\left[ \frac{ x^{1/2} }{ 1/2 } \right]_0^1=2.
			\end{equation}
			
	\end{enumerate}

\end{corrige}
