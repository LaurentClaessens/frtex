% This is part of Exercices de mathématique pour SVT
% Copyright (C) 2010-2011
%   Laurent Claessens et Carlotta Donadello
% See the file fdl-1.3.txt for copying conditions.

\begin{corrige}{interro-0005}

    \begin{enumerate}
        \item
            Ce qui se trouve dans le logarithme doit être positif (strictement), donc ici $x-2>0$, et par conséquent $x>2$. L'ensemble de définition est donc $\mathopen] 2 , \infty \mathclose[$.
        \item
            Le dénominateur doit être non nul, donc $\pi x\neq 0$, c'est à dire $x\neq 0$. Le domaine de définition est donc $\eR_0$.
        \item
            Ici nous avons deux éléments à problèmes. D'abord nous avons un dénominateur qui demande $\ln(x)\neq 0$, et ensuite nous avons un logarithme qui demande $x>0$. Le domaine de la fonction sont les $x$ qui satisfont aux deux conditions en même temps. Nous avons $\ln(x)=0$ si $x=1$, donc le domaine sera tous les $x$ positifs différents de $1$ : $\mathopen] 0 , \infty \mathclose[\setminus\{ 1 \}$.
        \item
            Nous avons une racine, donc ce qui se trouve dedans doit être positif (ou nul). Nous posons donc $x+1\geq 0$, et par conséquent $x\geq -1$. Le domaine de définition est par conséquent le domaine est \( \mathopen[ 1 , \infty [\).
    \end{enumerate}

\end{corrige}
