\begin{corrige}{SC_serie1-0005}

	En ce qui concerne l'équation chimique, on a

	\begin{equation}
		P(C_3H_8) + \cdots O_2 + A(0.21 O_2 + 0.79 N_2) \to C (CO_2) + W(H_2O) + \frac{ N }{ 0/79 } (0.21 O_2 + 0.79 N_2)
	\end{equation}

Le $N/0.79$ sert à faire qu'il y ait $N$ moles de $N_2$ qui sortent, comme demandé. À partir de là, il faut comprendre que le nombre de moles de $O_2$ qui sortent est $(0.21)*(N/0.79)$. Cela est $X$.

 Les trois petits points signifie que ce $O_2$ est «virtuel». En réalité il est inclus dans l'air, et donc dans le $A$

\begin{itemize}
	\item 
bilan carbone : $3P = C$
\item
 bilan hydrogène : $8P = 2W$
 \item
 bilan azote : $0.79 A = N$
\end{itemize}

En ce qui concerne le bilon d'oxygène, il y a deux choses à faire.
\begin{itemize}
	\item  D'abord le $5O_2$ doit venir de l'air, et on sait qu'il en fait $5P$, donc on peut croire que $0.21A = 5P$. Hélas, les choses ne sont pas aussi simple : le $A$ fournit un \emph{excès} d'air. Donc le $0.21A$ n'est en réalité pas $5P$, mais $125\%$ de $5P$. Nous avons donc $0.21A = (1.25)*(5P)$
	\item Le bilan oxygène (à compter en atomes, et non en molécules $O_2$ !!) s'écrit $0.21 A = C+(W/2)+X$
\end{itemize}
Enfin, on impose qu'il y ait exactement 100 moles qui sortent, c'est à dire $C+W+B+X=100$.

Le reste du problème est pour Matlab.

\lstinputlisting{SC_exo_1-5.m}

\end{corrige}
