% This is part of Mes notes de mathématique
% Copyright (c) 2012
%   Laurent Claessens
% See the file fdl-1.3.txt for copying conditions.


\begin{exercice}\label{exoreserve0006}

    Montrer que pour tout \( x\in\mathopen] -1 , 1 \mathclose[\) nous avons
    \begin{equation}        \label{EqweEZnV}
        -\ln(1-x)=\sum_{n=1}^{\infty}\frac{ x^n }{ n }.
    \end{equation}
    Calculer 
    \begin{equation}    \label{EqKUQmOZ}
        \sum_{n=1}^{\infty}\frac{ (-1)^n }{ n }.
    \end{equation}

\corrref{reserve0006}
\end{exercice}
