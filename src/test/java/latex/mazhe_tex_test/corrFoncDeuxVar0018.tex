% This is part of the Exercices et corrigés de mathématique générale.
% Copyright (C) 2009-2010
%   Laurent Claessens
% See the file fdl-1.3.txt for copying conditions.

\begin{corrige}{FoncDeuxVar0018}

	\begin{enumerate}

		\item	%2
			Le degré du dénominateur est plus grand que celui du numérateur. Nous nous attendons donc à avoir une limite qui n'existe pas. Prenons le chemin $(t,t)$ par exemple :
			\begin{equation}
				f(t,t)=\frac{1}{ t },
			\end{equation}
			et dont la limite le long de ce chemin n'existe pas.
		\item 	%3
			Le point $(2,3)$ est à l'intérieur du domaine de continuité de la fonction, donc sa limite est égale à la valeur de la fonction au point donné, c'est à dire $-5$. En pratique, il s'agit juste de remplacer $x$ et $y$ par les valeurs $2$ et $3$.

		\item	%4
			Un bon chemin à tenter est toujours $(t,kt,kt)$. Nous avons
			\begin{equation}
				f(t,kt,kt)=\frac{ k^2+2k }{ 1+2k^2 },
			\end{equation}
			et donc la limite dépend de $k$. Deux $k$ différents correspondent donc à deux chemins le long desquels la limite est différente.

			La limite n'existe donc pas.
		\item	%5
			Le domaine de continuité de cette fonction est $\eR^2$ tout entier parce que c'est une fraction de fonctions continues dont le dénominateur ne s'annule pas. Il n'y a donc aucune indéterminations et la limite s'obtient en remplaçant par les valeurs : $(\pi+1)/2$.
		\item	%6
			C'est une limite de polynômes dont le numérateur et le dénominateurs ont le même degré. Utiliser le chemin $(t,kt,kt)$ est en général une bonne idée : le résultat a de bonnes chances de dépendre de $k$ et donc la limite de ne pas exister. Ici nous avons
			\begin{equation}
				f(t,kt,kt)=\frac{ 2 }{ 1+2k^2 }.
			\end{equation}
			Cette expression dépend effectivement de $k$ et donc deux $k$ différents correspondent à deux chemins le long desquels la limite de $f$ est différente.
		\item	%7
			Le domaine de continuité de cette fonction est tous les points de $\eR^2$ sauf les points avec $x=1$ ou $y=-2$. Étant donné que la limite que l'on demande n'est pas vers un de ces points, nous la calculons en remplaçant les valeurs :
			\begin{equation}
				\lim_{(x,y)\to(-1,1)}\frac{ y^2+x }{ (x-1)(y+2) }=\frac{ 0 }{ -6 }=0.
			\end{equation}
			J'insiste : $0/6$ n'est pas une indétermination.
		\item	%7
			Si nous prenons le chemin $(t,kt,kt)$, nous trouvons
			\begin{equation}
				\lim_{t\to 0}f(t,kt,kt)=\lim_{t\to 0}\frac{ 1+3k^3t }{ k^3t^2 }=\infty.
			\end{equation}
			La limite de la fonction n'existant pas le long de ces chemins, la fonction n'a pas de limite en $(0,0,0)$.
		\item	%9
			Grâce au produit remarquable $x^4-y^4=(x^2-y^2)(x^2+y^2)$, nous pouvons simplifier la fraction et obtenir $f(x,y)=x^2-y^2$, dont la limite vaut zéro.
		\item	%12
			Le degré du numérateur ($3$) est plus grand que celui du dénominateur, donc nous partons avec l'idée que la limite va exister et sera zéro. Prouvons cela. D'abord
			\begin{equation}
				0\leq | f(x,y) |=\frac{ | x | |y |^2 }{ x^2+2y^2 }.
			\end{equation}
			Ensuite nous utilisons les majorations quasi classiques (mais un peu adaptées au $2y^2$) $| x |\leq\sqrt{x^2+y^2}$ et $| y |\leq \sqrt{x^2+2y^2}$. Donc
			\begin{equation}
				0\leq \frac{ \sqrt{x^2+2y^2}\big( \sqrt{x^2+2y^2} \big)^2 }{ \big( \sqrt{x^2+y^2} \big)^2 }=\sqrt{x^2+2y^2},
			\end{equation}
			mais la limite de l'expression à droite vaut manifestement zéro. Donc la limite de la fonction est zéro.
		\item	%13
			Prouvons que la limite est nulle (parce que le degré du numérateur est plus grand que celui du dénominateur). Nous majorons la valeur absolue du numérateur par
			\begin{equation}
				| x^3-2x^2y+3y^2x-y^3 |\leq | x |^3+2x^2| y |+3y^2| x |+| y |^3.
			\end{equation}
			Ensuite nous utilisons la majoration $| x |\leq\sqrt{x^2+y^2}$ (idem pour $y$) et après simplification nous trouvons
			\begin{equation}
				0\leq| f(x,y) |\leq \frac{ 9 }{2}\sqrt{x^2+y^2},
			\end{equation}
			dont la limite vaut zéro.
		\item	%18
			ATTENTION : il ne suffit pas de dire que sur le chemin $(t,t)$ la limite n'existe pas. En effet, dans la technique des chemins, on demande que les chemins soient compris dans le domaine de définition de la fonction. Or aucun point de la forme $(t,t)$ n'est dans le domaine de définition de $f$.

			Pour cette fonction, les droites $(t,kt)$ fonctionnent très bien (penser à rejeter $k=1$). En effet,
			\begin{equation}
				f(t,kt)=\frac{ k^3t^5 }{ t^5(k^5-1) }=\frac{ k^3 }{ k^5-1 },
			\end{equation}
			qui dépend de $k$. Nous avons donc tout plein de chemins qui ont des limites différentes.
		\item	%19
			En utilisant une division euclidienne, nous factorisons le numérateur en
			\begin{equation}
				x^3-y^3=(x-z)(x^2+xz+z^2),
			\end{equation}
			tandis que le numérateur se factorise en $(x+z)(x-z)$. Après simplification nous trouvons
			\begin{equation}
				f(x,y)=\frac{ x+z }{ x^2+xz+z^2 },
			\end{equation}
			dont le domaine de continuité contient maintenant le point $(1,1,1)$. La limite se trouve donc en remplaçant $x$, $y$ et $z$ par leurs valeurs. La réponse est $2/3$.
		\item	%20
			Nous voyons cette fonction comme composée $f=h\circ g$ avec
			\begin{equation}
				\begin{aligned}[]
					g(x,y)&=-xy^2\\
					h(t)= e^{t}.
				\end{aligned}
			\end{equation}
			Ce sont deux fonctions qui sont partout continues, donc la fonction $f$ est partout continue. De ce fait, la limite se calcule simplement en remplaçant les valeurs. Nous avons $ e^{-1}$.
		\item	%22
			Le domaine d'existence de la fonction est donné par $x^2+y^2<1$. Le point proposé est bien à l'intérieur de ce domaine, donc il suffit de remplacer par les valeurs, on obtient donc
			\begin{equation}
				\ln\sqrt{1}=\ln(1)=0.
			\end{equation}
		\item	%23
			Le chemin $(t,0)$ donne $\frac{ t^2 }{ t^2 }\to 1$. Mais le chemin $(t,t)$ donne $\frac{ t^2 }{ 3t^2 }\to \frac{1}{ 3 }$. La limite de la fonction n'existe donc pas.
		\item	%24
			C'est la composée de $g(x,y)=(1+x+3y)/(3y^2-x)$ et de $h(t)=\ln(t)$. Nous avons que $\lim_{(x,y)\to(2,-1)}g(x,y)=0$, et que la limite du logarithme pour $t\to 0$ est moins l'infini. La limite n'existe donc pas.

		\item	%26
			Cette fonction est la composée $f(x,y)=h\big( g(x,y) \big)$ où $g(x,y)=xy$ et
			\begin{equation}
				h(t)=\frac{  e^{t}\sin^2(t) }{ t }.
			\end{equation}
			Il faut donc calculer la limite
			\begin{equation}
				\lim_{t\to 0}\frac{  e^{t}\sin^2(t) }{ t }.
			\end{equation}
			Pour ce faire, nous faisons la manipulation suivante :
			\begin{equation}
				\frac{  e^{t}\sin^2(t) }{ t }=\frac{  e^{t}\sin^2(t) }{ t }\frac{ t }{ t }=t e^{t}\left( \frac{ \sin(t) }{ t } \right)^2.
			\end{equation}
			Lorsque nous faisons la limite $t\to 0$, la grande parenthèse tend vers $1$, tandis que $t e^{t}\to 0$.

			Nous avons donc que la limite de $f$ vaut zéro.

			Une autre façon de résoudre cet exercice est de multiplier et diviser par $xy$, de telle façon à faire apparaître le terme
			\begin{equation}
				\left( \frac{ \sin(xy) }{ xy } \right)^2
			\end{equation}
			dont la limite est $1$.
		\item	%27
			En remplaçant les valeurs de $x$ et $y$ nous trouvons $16/0$, ce qui n'existe pas. Afin de prouver que la limite n'existe effectivement pas, prenons un chemin. Le chemin $(t-2,1+t)$ fait l'affaire :
			\begin{equation}
				f(t-2,1+t)=\frac{ (t-2)^2-4(t-2)+4 }{ (t-2)(1+t)+2(1+t)-(t-2)-2 }=\frac{ t^2-8t+16 }{ t^2 },
			\end{equation}
			dont la limite $t\to 0$ est $\infty$. La limite de cet exercice n'existe donc pas.
		\item	%28
			Nous pouvons voir la fonction comme composée $f=h\circ g$ avec
			\begin{equation}
				\begin{aligned}[]
					g(x,y)&=\frac{1}{ x^2+y^2 }
					h(t)&= e^{-t}.
				\end{aligned}
			\end{equation}
			Dans ce cas, la limite de $g$ vaut $\infty$ et ensuite, $ e^{-\infty}=0$. Cela fournit la bonne réponse mais la théorème \ref{ThoLimiteCompose} sur lequel se base la méthode des composées ne permet pas de passer l'infini de $g$ à $h$ aussi facilement. La solution pour être plus rigoureux est de poser
			\begin{equation}
				\begin{aligned}[]
					g(x,y)&=x^2+y^2
					h(t)&= e^{-1/t}.
				\end{aligned}
			\end{equation}
			D'ailleurs, lorsque $x$ et $y$ n'arrivent que dans la combinaison $x^2+y^2$, c'est toujours une bonne idée de faire une composée de cette façon.

			Maintenant, la limite de $g$ est zéro et $\lim_{x\to 0} e^{-1/t}=0$.

	\end{enumerate}

\end{corrige}
