% This is part of Exercices et corrigés de CdI-1
% Copyright (c) 2011
%   Laurent Claessens
% See the file fdl-1.3.txt for copying conditions.

\begin{exercice}\label{exo0021}

Soient $a$ et $b$ deux nombres réels fixés. Définissons la suite $\left(x_n\right)$ comme suit~: $x_1=a$, $x_2=b$ et, pour $n>2$, $x_n$ est la moyenne des deux termes précédents, c'est-à-dire, $x_n=\frac{1}{2}\left(x_{n-2}+x_{n-1}\right)$.
\begin{enumerate}
	\item Faites un dessin (Aide : supposez $a<b$ et pensez aux points milieux).
	\item Prouvez (par induction) que $$x_{n+1}-x_n=\left(-\frac{1}{2}\right)^{n-1}\left(b-a\right)$$
	\item Montrez que $$x_{n+1}-x_1=\left[\sum_{k=1}^n\left(-\frac{1}{2}\right)^{k-1}\right]\left(b-a\right)$$
	\item Prouvez que $x_n\rightarrow \frac{\left(a+2b\right)}{3}$.
\end{enumerate}

\corrref{0021}
\end{exercice}
