% This is part of Exercices et corrigés de CdI-1
% Copyright (c) 2011
%   Laurent Claessens
% See the file fdl-1.3.txt for copying conditions.

\begin{exercice}\label{exoImplicite0008}

Soit la fonction 
\begin{equation}
	F:\eR^3\rightarrow\eR^2:(x,y,z)\mapsto(x+y+z,\; x^2+y^2+z^2-1).
\end{equation}
On définit $M\subset \eR^3$ comme l'ensemble
\begin{equation}
	M=\{(x,y,z)\in \eR^3 \mbox{ tels que } F(x,y,z)=(0,0)\}.
\end{equation}

\begin{enumerate}
\item Montrer que $M$ est compact. Montrer que $M$ est une variété $C^1$ dans $\eR^3$. Quelle est sa dimension?
\item Prouver qu'il existe des fonctions $\phi:\eR\rightarrow \eR^2 : y \mapsto (X(y), Z(y))$ définies dans un voisinage $V$ de $y=0$ telles que $F(X(y), y, Z(y))=(0,0) \; \forall y \in V$. 
\item Donner une approximation des   fonctions $y\rightarrow X(y)$ autour de $y=0$ par un polynôme  du premier degré en une indéterminée.   
\end{enumerate}

\corrref{Implicite0008}
\end{exercice}
