% This is part of Outils mathématiques
% Copyright (c) 2011,2012
%   Laurent Claessens
% See the file fdl-1.3.txt for copying conditions.

\begin{exercice}\label{exoOutilsMath-0128}

    \begin{enumerate}
        \item
            Vérifier que \( y\ln(y)-y\) est une primitive de \( \ln(y)\).
        \item
            Calculer l'intégrale double
            \begin{equation}
                \iint_Dx\ln(y)\,dxdy
            \end{equation}
            où
            \begin{equation}
                D=\{ (x,y)\tq x\in\mathopen[ 0 , 1 \mathclose], e^{x}\leq y\leq  e^{2x} \}.
            \end{equation}
        \item   \label{ImtemzuhtE}
            Soient \( 0<a<b\) donnés. Dessiner le \emph{quart de couronne} de la partie \( \{ (x,y)\tq x\geq 0,y\geq 0 \}\) du plan et situé entre les cercles de rayon \( a\) et \( b\).
        \item
            Calculer, en passant aux coordonnées polaires, l'intégrale double
            \begin{equation}
                \iint_E\ln(x^2+y^2)\,dxdy
            \end{equation}
            où \( E\) est l'ensemble dessiné au point \ref{ImtemzuhtE}.
    \end{enumerate}

\corrref{OutilsMath-0128}
\end{exercice}
