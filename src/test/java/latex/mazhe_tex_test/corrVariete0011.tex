% This is part of Exercices et corrigés de CdI-1
% Copyright (c) 2011
%   Laurent Claessens
% See the file fdl-1.3.txt for copying conditions.

\begin{corrige}{Variete0011}

	\begin{enumerate}

		\item
		\item
		\item
		\item
			Cet exercice est intéressant parce qu'il est difficile de trouver une bonne expression pour le chemin. Heureusement, cela n'est pas nécessaire. En effet, dans le plan, $P\equiv x+y+z=0$, nous avons
			\begin{equation}
				\omega|_{P}(x,y,z)=-xdx-ydy-zdz.
			\end{equation}
			Si maintenant nous considérons la forme $\omega'=-xdx-ydy-zdz$, il est facile de voir qu'elle est exacte et que $\omega'=df$ avec
			\begin{equation}
				f(x,y,z)=-\frac{ x^2 }{2}-\frac{ y^2 }{2}-\frac{ z^2 }{2}.
			\end{equation}
			Étant donné que le chemin $\gamma$ considéré est contenu dans le plan $P$, nous avons
			\begin{equation}
				\int_{\gamma}\omega=\int_{\gamma}\omega'=0
			\end{equation}
			parce que l'intégrale d'une forme exacte sur un chemin fermé est nulle.
	\end{enumerate}
\end{corrige}
