% This is part of Exercices et corrigés de CdI-1
% Copyright (c) 2011
%   Laurent Claessens
% See the file fdl-1.3.txt for copying conditions.

\begin{corrige}{OutilsMath-0077}

    La fonction $\tilde g$ est donnée par
    \begin{equation}
        \tilde g(\rho,\theta,\varphi)=\rho^2\sin\theta\big( \cos\varphi+\sin\varphi \big).
    \end{equation}
    Les dérivées partielles sont
    \begin{equation}
        \begin{aligned}[]
            \frac{ \partial \tilde g }{ \partial \rho }&=2\rho\sin(\theta)\big( \cos(\varphi)+\sin(\varphi) \big)\\
            \frac{ \partial \tilde g }{ \partial \theta }&=\rho^2\cos(\theta)\big( \cos(\varphi)+\sin(\varphi) \big)\\
            \frac{ \partial \tilde g }{ \partial \varphi }&=\rho^2\sin(\theta)\big( \cos(\varphi)-\sin(\varphi) \big)
        \end{aligned}
    \end{equation}

\end{corrige}
