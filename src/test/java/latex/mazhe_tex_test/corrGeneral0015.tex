% This is part of the Exercices et corrigés de mathématique générale.
% Copyright (C) 2009-2011
%   Laurent Claessens
% See the file fdl-1.3.txt for copying conditions.
\begin{corrige}{General0015}

La droite $D$ passe par $(3,5)$, par $(0,b)$ et par $(a,0)$. Étant donné que deux point fixent la droite, la connaissance de $b$ fixe $a$. Nous commençons par chercher la surface du triangle qui passe par $(0,b)$ et par $(3,5)$.

La droite est
\begin{equation}
	y=\frac{ 5-b }{ 3 }x+b,
\end{equation}
et elle coupe l'axe $Ox$ en $x=-\frac{ 3b }{ 5-b }$. Le triangle a donc une base de longueur $-3b/(5-b)$ et de hauteur $b$, donc la surface est
\begin{equation}
	S(b)=\frac{ -3b^2 }{ 5-b },
\end{equation}
qu'on va essayer de maximiser. La dérivée est
\begin{equation}
	S'(b)=\frac{ 3b^2-30b }{ b^2-10b+25 },
\end{equation}
et s'annule pour $b=10$.

\end{corrige}
