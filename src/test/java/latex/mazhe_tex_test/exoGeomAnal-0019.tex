\begin{exercice}\label{exoGeomAnal-0019}

  \begin{enumerate}
  \item  Donner la  formule qui permet de calculer la longueur de la courbe $\gamma$ paramétrée pas la fonction $r: [3,5]\to \eR^3$, $r: t \mapsto (x(t), y(t), z(t))$.   
\item Soit $\mathcal{D}$ le cube de sommets $A=(0,0,0)$, $B=(0,1,0)$, $C=(1,1,0)$, $D=(1,0,0)$, $E=(0,0,1)$, $F=(0,1,1)$, $G=(1,1,1)$ et $H=(1,0,1)$. 
\begin{enumerate}
\item Dessiner $\mathcal{D}$ dans l'espace. 
\item Paramétrer les trois courbes suivantes d'extrêmes $A$ et $G$ :
  \begin{enumerate}
  \item $\gamma_1$ est l'union des deux morceaux de droite entre $A$ et $B$ et entre $B$ et $G$ ;
  \item $\gamma_2$ est l'union des trois morceaux de droite entre $A$ et $B$, entre $B$ et $F$ et entre $F$ et $G$ ;
  \item $\gamma_3$ est l'union des deux morceaux de droite entre $A$ et $P=(0,1,1/2)$ et entre $P$ et $G$.
  \end{enumerate}
\item Calculer la longueur de la courbe qui vous parait la plus courte (motiver votre choix). Vous pouvez donner la valeur de l'intégrale sans calculer des intégrales si vous n'avez pas pu trouver la bonne paramétrisation au point précédent, mais il faudra la motiver à l'aide d'un dessin pour qu'elle soit prise en compte.

\end{enumerate}

  \end{enumerate}

\corrref{GeomAnal-0019}
\end{exercice}
