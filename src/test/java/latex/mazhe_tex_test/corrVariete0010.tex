% This is part of Exercices et corrigés de CdI-1
% Copyright (c) 2011
%   Laurent Claessens
% See the file fdl-1.3.txt for copying conditions.

\begin{corrige}{Variete0010}

    Il faut diviser le chemin en trois morceaux. Attention à l'ordre : nous intégrons sur le chemin parcourant le triangle dans le sens $ABC$ et non dans le sens inverse $ACB$.

    Le chemin $\sigma_1$ joint $A$ à $B$ : $\sigma_1(t)=(1-t)A+tB=\begin{pmatrix}
        1-t    \\ 
        t    
    \end{pmatrix}$. L'intégrale à effectuer est donc
    \begin{equation}
        \int_{\sigma_1}(x+y)d\sigma_1=\int_0^1(1-t)+t\,dt=1.
    \end{equation}
    Le chemin qui joint $B$ à $C$ est $\sigma_2(t)=(0,1-t)$ et l'intégrale le long de ce chemin vaut
    \begin{equation}
        \int_{\sigma_2}(x+y)d\sigma_2=\int_0^1(1-t)dt=\frac{ 1 }{2}.
    \end{equation}
    La troisième intégrale est sur $\sigma_3(t)=(t,0)$ :
    \begin{equation}
        \int_{\sigma_3}(x+y)d\sigma_3=\int_0^1t\,dt=\frac{ 1 }{2}.
    \end{equation}
    L'intégrale de $x+y$ le long de tout le triangle vaut donc $1+\frac{ 1 }{2}+\frac{ 1 }{2}=2$.

\end{corrige}
