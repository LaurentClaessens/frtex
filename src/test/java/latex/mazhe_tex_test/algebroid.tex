\section{Lie groupoids and algebroids}
%++++++++++++++++++++++++++++++++++++
The reference for this section are \cite{WeinGroupoids,WeinGpoidsSymple}.

Let $M$ be a manifold and $\cA$, a vector bundle over $M$. Let $[.,.]$ be a Lie algebra structure on $\cA$. Then $\cA$ becomes a \defe{Lie algebroid}{Lie!algebroid} when we endow it by a homomorphism $\rho\colon \cA\to TM$ such that
\begin{enumerate}
\item\label{DefAloidItemi} it induces a Lie algebra homomorphism on the sections, i.e. a $\rho\colon \Gamma(M,\cA)\to \Gamma(TM)$ such that $\rho[\psi,\varphi]=[\rho(\psi),\rho(\varphi)]$,
\item for all $f\in C^{\infty}(M)$ and $\psi,\varphi\in\Gamma(M,\cA)$,
\begin{equation}  \label{DefAlgoidCondc}
[f\psi,\varphi]=f[\psi,\varphi]=f[\psi,\varphi]-\big( \rho(\varphi)f \big)\psi.
\end{equation}

\end{enumerate}
The map $\rho$ is the \defe{anchor}{anchor}. In order to be more precise for item \ref{DefAloidItemi}, the anchor function $\rho\colon \cA\to TM$ induces the homomorphism  $\rho'\colon \Gamma(\cA)\to \Gamma(TM)$ by formula
\begin{equation}  \label{EqInduitAncra} 
  \rho'(\psi)(x)=\rho(\psi(x)).
\end{equation}
We will omit the prime and simply write $\rho$ for both functions.

The algebroid $\cA$ admit standard coordinates\label{PgStandCoord} $(q,\lambda)$ where $q$ are coordinates on $M$ and $\lambda$ some coordinates on the fibres (whose are vector spaces). These coordinates depends on a choice of a local base $\xi$ of sections of $\cA$: the set of sections $\{\xi_{i}(x)\}$ is a basis of the fibre $\cA_{x}$ for each $x\in M$. IN terms of these coordinates, the Lie algebra structure of $\cA$ reads
\[ 
  [\xi_{i},\xi_{j}]=c_{ijk}\xi_{k}
\]
where the $c_{ijk}\in C^{\infty}(M)$ are the \emph{structure constant}. In the same way, the anchor is given in terms of functions $a_{ij}\in C^{\infty}(M)$ by
\[ 
  \rho(\xi_{i})=a_{ij}\frac{ \partial }{ \partial q_{j} }.
\]

\subsection{Example: tangent bundle \texorpdfstring{$TM$}{TM}}
%---------------------------------------------------------------

If we consider $\cA=TM$, the anchor is $\rho=\id$. Then formula \eqref{EqInduitAncra} is $\rho(\psi)(x)=\psi(x)$, and the condition \eqref{DefAlgoidCondc}, with more familiar notations, reduces to
\[ 
  [fX,Y]=f[X,Y]-(Yf)X.
\]

\subsection{Example: a Lie algebra}
%-----------------------------------

Any Lie algebra $\mG$ can be seen as an algebroid on a manifold $M=\{ p \}$ is a manifold containing only one point. In this case, any path in $M$ is constant and the tangent space $TM$ reduces to only the zero vector. The identically vanishing map $\rho\colon \mG\to TM$ is an anchor.

\subsection{Example: gauge algebroid}
%-------------------------------------

We consider the following principal bundle:
\[
\xymatrix{
    G \ar@{~>}[r] & P\ar[d]^{\displaystyle \pi}\\ &M 
  }
\]
Let's begin to give a structure of vector bundle on the quotient $TP/G$. What is clear is that $TP$ is a vector bundle $p\colon TP\to M$. The action of $G$ on $TP$ is
\[ 
  X_{\xi}\cdot g=d\tau_{g}X_{\xi}
\]
where $\tau\colon P\to P$ is the action. So the quotient $TP/G$ is taken with respect to the equivalence $X_{\xi}\sim d\tau_{g}X_{\xi}$. Notice that $X_{\xi}\in T_{\xi}P$ and $d\tau_{g}X_{\xi}\in T_{\xi\cdot g}P$; in fact in each class $[X_{\xi}]$, there is one and only one vector at each point of the fibre of $\xi$.

The differential $d\pi$ of the projection passes to quotient: $d\pi(X_{\xi})=d\pi(X_{\xi}\cdot g)$. It follows that the next definition is correct:
\begin{equation}
\begin{aligned}
 d\pi\colon TP/G&\to TM \\ 
d\pi[X_{\xi}]&= d\pi X_{\xi}\in T_{\pi(\xi)}M. 
\end{aligned}
\end{equation}

Now let us study the sections $\psi\colon M\to TP/G$.

\begin{lemma}
The sections of $TP/G$ are the $G$-equivariant vector fields on $P$\index{equivariant!vector field on principal bundle}. (see definition \ref{DefEqVectPrinc}.)
\end{lemma}

\begin{proof}
 We consider the map $L\colon \Gamma(M,TP/G)\to \Gamma(P,TP)$, 
\[ 
  (L\psi)(\xi)=\psi(\pi(\xi))|_{\xi}
\]
where, when $q\in TP/G$, the symbol $q|_{\xi}$ denotes the element of $T_{\xi} P$ which belongs to $q$. In particular, $[X_{\xi}]|_{\xi}=X_{\xi}$. The section $L\psi$ is $G$-equivariant, i.e.
\begin{equation}
 d\tau_{g}(L\psi)(\xi)=(L\psi)(\xi\cdot g).
\end{equation}
Indeed, $d\tau_{g}X_{\xi}\in [X_{\xi}]$, thus
\[ 
  d\tau_{g}\big[ \psi(\pi\xi) \big]|_{\xi}\in [\psi(\pi\xi)]\cap T_{\xi\cdot g}P,
\]
and there is only one element which is $\big[ \psi(\pi\xi) \big]|_{\xi\cdot g}=(L\psi)(\xi\cdot g)$.
\end{proof}

Now, taking $d\pi$ as anchor, this construction gives an algebroid structure to $TP/G$. In order to prove that we have to prove that 
\[ 
  [f\psi,\varphi]=f[\psi,\varphi]-\big( \rho(\varphi)f \big)\psi
\]
for any choice of sections $\psi,\varphi\colon M\to TP/G$ and of function $f\colon M\to \eR$. Here we have
\begin{equation}
\begin{aligned}
 \rho\colon \Gamma(M,TP/G)&\to \Gamma(M,TM) \\ 
\rho(\psi)(x)& =d\pi\big( \psi(x) \big). 
\end{aligned}
\end{equation}
The first think to be remarked is that a good choice of local section $a\colon M\to P$ and vector fields $X$, $Y\in\Gamma(P,TP)$ on $P$, one can express $\psi$ and $\varphi$ under the form
\begin{equation}  \label{EqXcorrapsi}
\psi(x)=\big[ (X\circ a)(x) \big],\qquad \varphi(x)=\big[ (Y\circ a)(x) \big]
\end{equation}
with the same $a$. On $TP/G$, we put the Lie bracket inherited from the one of $TP$:
\begin{equation}
[\psi,\varphi](x)=\Big[ [X,Y]\circ a(x) \Big].
\end{equation}
If $X$ corresponds to $\psi$ by \eqref{EqXcorrapsi}, we have
\begin{align*}
  (f\psi)(x)&=\Big[ f(x)(X\circ a)(x) \Big]\\
		&=\Big[ \big( (f\circ\psi)X \big)\big( a(x) \big) \Big],
\end{align*}
thus $(f\circ \pi)X$ corresponds to $f\psi$ in the sense of
\[ 
  \Big( (f\circ \pi)X \Big)(\xi)=f(\pi\xi)X(\xi).
\]
Then we have
\begin{align*}
  [f\psi,\varphi](x)&=\Big[ \big[ (f\circ\pi)X,Y \big]\circ a(x) \Big]\\
		&=\Big[ \big( (f\circ\pi)[X,Y]-Y(f\circ\pi)X \big)\circ a(x) \Big]\\
		&=\big[ f(x)[X,Y]\circ a(x) \big]-\big[ Y(f\circ\pi)X\circ a(x) \big].
\end{align*}
What lies in the bracket of the second term reads better under the form:
\[ 
  \big( Y(f\circ\pi)X \big)\circ a(x)=Y_{a(x)}(f\circ\pi)X_{a(x)}.
\]
We want this thing to be equal to
\begin{align*}
\Big( \big( \rho(\varphi)f \big)\psi \Big)(x)&=\big( \rho(\varphi)(x)f \big)\psi(x)\\
		&=d\pi\big( \varphi(x) \big)f\big[ (X\circ a)(x) \big]\\
		&=d\pi\big[ (Y\circ a)(x) \big]f\big[ (X\circ a)(x) \big].
\end{align*}
In the latter expression, $(Y\circ a)(x)\in T_{a(x)}P$ while  $d\pi\big[ (Y\circ a)(x)\Big]\in T_{x}M$. Since $d\pi\colon TP\to TM$ fulfils $d\pi(X_{\xi}\cdot g)=d\pi X_{\xi}$, the differential of $\pi$ passes to the classes (this is the reason for which we chose it a anchor) and
\[ 
  d\pi[(Y\circ a)(x)]=d\pi(Y\circ a)\in TM.
\]
It remains to be proved that 
\[ 
  \Big[ \big( Y(f\circ\pi)X \big)\circ a(x) \Big]=\big( d\pi(Y\circ a)(x) \big)f\big[ (X\circ a)(x) \big],
\]
which is true because
\begin{align*}
Y_{a(x)}(f\circ\pi)&=(Y\circ a)(x)\\
		&=(df\circ d\pi)(Y\circ a)(x)\\
		&=d\pi(Y\circ a)(x)f.
\end{align*}

\subsection{Poisson structure}
%------------------------------

Let us describe a Poisson structure on the dual $\cA^*$. For this, we put a Poisson bracket\index{Poisson structure!on dual algebroid} on each $\cA^*_{x}$, $x\in M$. Since the bracket only depends on differential, we just have to define it on affine functions on the fibres. Two remarks: firstly, the functions which are constant on fibres can be seen as functions on $M$ and second, linear functions can be seen as sections of $\cA$ because a linear function on a vector space is equivalent to the data of a single vector. So let $f,g\colon M\to \eR$ be two functions and $\psi,\varphi\colon M\to \cA$ be sections. The bracket on $ C^{\infty}(\cA^*)$ is defined as
\begin{equation}
\{ f,g \}=0,\quad \{ f,\psi \}=\rho(\psi)f,\quad\{ \psi,\varphi \}=[\psi,\varphi].
\end{equation}

\section{Lagrangian formalism}
%+++++++++++++++++++++++++++++

Let $\cA$ be a Lie algebroid on a manifold $M$ and a function $L\colon \cA\to \eR$ which we will call \defe{Lagrangian}{Lagrangian}. The \defe{Legendre mapping}{legendre mapping} is the fibre derivative $FL\colon \cA\to \cA^*$\nomenclature{$FL$}{Legendre mapping, fibre derivative of the Lagrangian $L$} given by
\begin{equation}
FL(a)\in\cA^*_{\pi(a)},\quad FL(a)(b)=dL_{a}(b).
\end{equation}
More precisely, the differential of $L$ at $a\in \cA$ is a map from $T_{a}\cA$. In order to define $FL(a)\in\cA^*$, we begin to consider the restriction $L|_{a}$ of $L$ to the fibre of $a$. The differential of $L|_{a}\colon \cA_{\pi(a)}\to \eR$ is
\[ 
  (dL|_{a})_{a}\colon T_{a}\cA_{\pi(a)}\to \eR,
\]
 but the fibre $\cA_{\pi(a)}$ is a vector space which can therefore be identified with its dual space. Then we have
\[ 
  (dL|_{a})_{a}\in\cA^*_{\pi(a)}\subset\cA^*.
\]

When one has a Lagrangian on $\cA$, one define the \defe{action}{action!associated with Lagrangian} as the function $A\colon \cA\to \eR$,
\begin{equation}
   A(v)=\langle FL(v),\,v\rangle,
\end{equation}
i.e. the action of $FL(v)\in\cA^*_{\pi(v)}$ on $v\in\cA_{\pi(v)}$. The \defe{energy}{energy!associated with a Lagrangian} is the function
\[ 
  E=A-L.
\]
The Lagrangian $L$ is a \defe{regular Lagrangian}{regular!Lagrangian} if $FL$ is a local diffeomorphism. In this case, one can bring the Poisson structure of $\cA^*$ on $\cA$. The resulting Poisson structure on $\cA$ is the \defe{Lagrange-Poisson structure}{Lagrange-Poisson structure}\index{Poisson structure!on algebroid}. For this, we have to define $\{ \xi,\eta \}$ when $\xi$, $\eta\in C^{\infty}(\cA)$. The natural definition is
\begin{equation}
\{ \xi,\eta \}_{\cA}:=\{ FL(\xi),FL(\eta) \}_{\cA^*}
\end{equation}
with the following definition of $FL(\xi)\in  C^{\infty}(\cA^*)$:
\begin{equation}
FL(\xi)(\omega)=\xi(FL^{-1}(\omega)).
\end{equation}
The latter definition says that when $\omega\in\cA^*_{\pi(a)}$ the element $FL^{-1}(\omega)$ is the element $a$ such that $(dL|_{a})_{a}=\omega$ (this is an equality in $\cA^*_{\pi(a)}$).

Since we have a Poisson structure, we can consider the Hamiltonian field (see definition \eqref{EqDefHamVect}) corresponding to the energy function $E\colon \cA\to \eR$. This is a vector field on $\cA$. This field is the \defe{Lagrangian vector field}{Lagrangian!vector field}

In standard coordinates (see page \pageref{PgStandCoord}), the Lagrangian is a function $(q,\lambda)\mapsto L(q,\lambda)$. In order to build $FL$, we first restrict $F$ to a fibre; so $L$ becomes a function $\lambda\mapsto L(q_{0},\lambda)$ and thus the derivatives which appear in $FL$ are 
\[ 
  \mu_{i}=\frac{ \partial L }{ \partial\lambda_{i} }.
\]
Now we try to express the Poisson structure on $\cA$ in the standard coordinates. Fist, $q_{i}$ is a  function on $\cA$ which associates to one point the component $i$ of its projection. If $\omega$ and $\eta$ both belong to $\cA^*_{x}$, the elements $FL^{-1}(\omega)$ and $FL^{-1}(\eta)$ both belongs to the same fibre $\cA_{x}$. Thus $FL(q_{i})$ is a function that is constant on the fibres. So
\[ 
  \{ q_{i},q_{j} \}=0.
\]





%%%%%%%%%%%%%%%%%%%%%%%%%%
%
   \section{Groupoids}
%
%%%%%%%%%%%%%%%%%%%%%%%%

A set $\Gamma$ is a \defe{groupoid}{groupoid} when we consider some maps
\begin{itemize}
\item $\alpha,\beta\colon \Gamma\to \Gamma_{0}\subset\Gamma$,
 \item $m\colon \Gamma_{2}\to \Gamma$ where $\Gamma_{2}=\{ (x,y)\in\Gamma\times\Gamma\tq \beta(y)=\alpha(x)) \}$,
\item $i\colon \Gamma\to \Gamma$
\end{itemize}
such that
\begin{enumerate}
\item $m(x,m(y,z))$ is defined if and only if $m(m(x,y),z)$ is defined, and in this case, they are equal,
\item $m(\beta(x),x)=m(x,\alpha(x))=x$,
  \item $m(x,i(x))$ and $m(i(x),x)$ are defined to be respectively equal to $\beta(x)$ and $\alpha(x)$.
\end{enumerate}
Notice that the fact for $m(x,y)$ to be defined means that $\alpha(x)=\beta(x)$. As notations and terminology, we adopt the following conventions. The map $i$ is the \emph{inversion}, $m$ is the \emph{multiplication} and is denoted by a dot: $m(x,y)=x\cdot y$. We also write $i(x)=i^{-1}$. 

The first hypothesis is called \emph{associativity}.

\begin{lemma}
$\forall\,x\in\Gamma$, we have 

\begin{subequations}
\begin{align}
\alpha(\beta(x))&=\beta(x)\\
\beta(\alpha(x))&=\alpha(x).
\end{align}
\end{subequations}
\end{lemma}

\begin{proof}
From definition of a groupoid, $\beta(x)\cdot x=x$, so $\alpha(\beta(x))=\beta(x)$. The other statement comes in the same way.

Notice that the conclusion does not come from the equality $\beta(x)\cdot x=x$, but only from the existence of the product $\beta(x)\cdot x$.
\end{proof}

\begin{lemma}
If $(x,y)\in\Gamma_{2}$, we have
\begin{subequations}
\begin{align}
\beta(x\cdot y)&=\beta(x)\\
\alpha(x\cdot y)&=\alpha(y)
\end{align}
\end{subequations}

\end{lemma}

\begin{proof}
Since $(x,y)\in \Gamma_{2}$, one can write $x\cdot y$. Using $x=\beta(x)\cdot x$, and associativity we find :
\[ 
  x\cdot y=\big( \beta(x)\cdot x \big)\cdot y=\beta(x)\cdot(x\cdot y).	
\]
Existence of the last product implies $\alpha(\beta(x))=\beta(x\cdot x)$. Using the equality $\alpha(\beta(x))=\beta(x)$, we find the second relation. The first one is proven by the same :
\[ 
  x\cdot y=x\cdot \big( y\cdot\alpha(y) \big)=(x\cdot y)\cdot \alpha(x).
\]

\end{proof}

\begin{lemma}
For all $x\in\Gamma$, we have
\begin{subequations}
\begin{align}
  \alpha(\alpha(x))&=\alpha(x)\\
\beta(\beta(x))&=\beta(x).
\end{align}
\end{subequations}
\end{lemma}

\begin{proof}
Using formula $x\cdot\alpha(x)=x$ in itself, and using associativity,
\[ 
  (x\cdot\alpha(x))\cdot\alpha(x)=x\cdot\big( \alpha(x)\cdot\alpha(x) \big)=x.
\]
The existence of the product $\alpha(x)\cdot\alpha(x)$ and the fact that $\beta\circ\alpha=\alpha$ give the result.
\end{proof}

\begin{lemma}
Let us mention the following other properties :
\begin{align}
\alpha(x^{-1})&=\beta(x)&\beta(x^{-1})&=\alpha(x)\\
\alpha(x)\cdot\alpha(x)&=\alpha(x)&\beta(x)\cdot\beta(x)&=\beta(x)
\end{align}
the \emph{simplification} rule :
\begin{align}
x\cdot y_{1}=x\cdot y_{2}&\Rightarrow y_{1}=y_{2}\\
x_1\cdot y=x_{2}\cdot y&\Rightarrow x_1-x_2,
\end{align}
and the corollary
\begin{equation}
(x^{-1})^{-1}=x.
\end{equation}

\end{lemma}
\begin{proof}
No proof.
\end{proof}

\begin{proposition}
The set $\Gamma_0$ is the set of fixed points of $\alpha$ and $\beta$.
\end{proposition}

\begin{proof}
We proof that $\Gamma_0=\{ x\in\Gamma\tq \alpha(x)=x \}$; the same is true for $\beta$. The definition of $\Gamma_0$ is to be the image of $\alpha$. Let $x\in\Gamma_0$ : there exists a $y\in\Gamma$ such that $x=\alpha(y)$. So $\alpha(x)=\alpha(\alpha(y))=\alpha(y)=x$.

For the reciprocal, let $x=\alpha(x)$. Then $x\in \Gamma_0$ because $x$ belongs to the image of $\alpha$ --- for instance, the image of itself.
\end{proof}

\subsection{Example: when \texorpdfstring{$\Gamma_0=\{ e \}$ }{G=e}}
%---------------------------------------------------------------------

Let us prove that in the case when $\Gamma_0$ reduces to only one point $\{ e \}$, the groupoid $\Gamma$ is a group whose unit is $e$. 

First, remark that for all $x\in\Gamma$, we have $\alpha(x)=\beta(x)=e$, so that $\Gamma_2=\Gamma$ and the multiplication is everywhere defined. Associativity is not a problem. The map $x\mapsto i(x)$ is the inverse because $x\cdot i(x)=\beta(x)=e$ and $i(x)\cdot x=\alpha(x)=e$. We also have $x\cdot e=e\cdot x=x$ because $e=\alpha(x)$ and $x\cdot \alpha(x)=x$.

\subsection{Example: the null  groupoid}
%----------------------------------------

A \defe{null groupoid}{null!groupoid} is a groupoid in which $\Gamma_0=\Gamma$. In this case, since $\Gamma_0$ is the set of fix points of $\alpha$ and $\beta$, we have 
\[ 
  \alpha=\beta=\id.
\]
In order for $x\cdot y$ to exist, we need $\beta(y)=\alpha(x)$, which in the case of the null groupoid gives $x=y$. So the only products that are defined are
\[ 
  x\cdot x=x.
\]

\subsection{The case \texorpdfstring{$\alpha=\beta$}{a=b}}
%----------------------------------------------------------

The case $\alpha=\beta$ regroup the two preceding cases. We will prove that for each $u\in\Gamma_0$, the set $\alpha^{-1}(u)$ is a group with $u$ as unit. Indeed let $x$, $y\in\alpha^{-1}(u)$; it is clear that $x\cdot y$ exists because $\beta(y)=\alpha(y)=x=\alpha(x)$. So the product is defined everywhere. Proof of the fact that $u$ is the unit is easy :
\begin{align*}
x\cdot u=x=x\cdot\alpha(x)=x\\
u\cdot x=\beta(x)\cdot x=x,
\end{align*}
and
\begin{align*}
x\cdot i(x)=\beta(x)=u\\
i(x)\cdot x=\alpha(x)=u.
\end{align*}

\subsection{An example on a vector bundle}
%-----------------------------------------

We consider a vector bundle $\pi\colon E\to M$ and
\[ 
  \Gamma_0=\{ o_{x}\tq x\in M \},
\]
the set of the zero of each fibre. As groupoid law, we choose the addition in the fibres: when $\pi(v)=\pi(w)$, we define $v\cdot w=v+w$, and as map $\alpha=\beta$, we naturally choose
\[ 
  \alpha(v)=\beta(v)=o_{x}
\]
if $v\in E_{x}$. In this case, $\alpha^{-1}(o)$ is the fibre of $o$ which is a group for the addition.


\subsection{Orbits}
%------------------

Let $\Gamma$ be a groupoid and $u\in\Gamma_0$. The \defe{isotropy group}{isotropy!group}\index{group!isotropy} of $u$ is 
\[ 
  \Gamma_{u}=\alpha^{-1}(u)\cap\beta^{-1}(u)
\]
In order to prove that it is a group, remark that if $x$, $y\in\Gamma_{u}$,
\[ 
  \alpha(x)=\alpha(y)=\beta(x)=\beta(y)=u,
\]
in particular, $x\cdot y$ exists and $\Gamma_{u}$ has a law. It is easy to prove that $u$ is th unit.

Notice that $\beta(\alpha^{-1}(x))=\alpha(\beta^{-1}(x))$. Indeed if $y\in\beta(\alpha^{-1}(x))$, there exists a $z$ such that $y=\beta(z)$ and $\alpha(z)=x$. We have to find a $z'$ such that $y=\alpha(z')$ and $\beta(z')=x$. The element $z'=\beta(z)$ works.

The set $\beta(\alpha^{-1}(x)=\alpha(\beta^{-1}(x))$ is the \defe{orbit}{orbit in a groupoid} of $\Gamma$ trough $x$.

\begin{proposition}
The set of orbits of $\Gamma$ is a partition of $\Gamma_0$.
\end{proposition}

\begin{proof}
The proof is as easy as I want not to give you.
\end{proof}

\subsection{Morphism}
%--------------------

If $\Gamma$ and $\Gamma'$ are two groupoids, the map $f\colon \Gamma\to \Gamma'$ is a \defe{morphism}{morphism!of groupoid} if for each existing product $x\cdot y$ in $\Gamma$, we have
\[ 
  f(x)\cdot f(y)=f(x\cdot y).
\]
We see that automatically 
\[ 
 \beta'(f(x))\cdot f(x)=f(x)=f(\beta(x)\cdot x)=f(\beta(x))\cdot f(x);
\]
so that the simplification rule gives
\begin{subequations}
\begin{align}
\beta'\circ f&=f\circ\beta\\
\alpha'\circ f&=f\circ\alpha.
\end{align}
\end{subequations}

\section{Lie groupoid}
%+++++++++++++++++++++

A \defe{Lie groupoid}{Lie!groupoid} is a (maybe non Hausdorff) manifold endowed with a groupoid structure such that
\begin{enumerate}
\item the set $\Gamma_0$ is a Hausdorff manifold,
\item the maps $\alpha$ and $\beta$ are differentiable submersions,
\item the multiplication $m\colon \Gamma_2\to \Gamma$ is differentiable,
\item the inversion $x\to x^{-1}$ is a diffeomorphism.
\end{enumerate}

