% This is part of Agregation : modélisation
% Copyright (c) 2011
%   Laurent Claessens
% See the file fdl-1.3.txt for copying conditions.

\begin{corrige}{Model-0008}

    Le risque de première espèce est donné par
    \begin{equation}    \label{EqwxFBMu}
        \alpha=P\left( \frac{1}{ n }\sum_iX_i>\frac{ \theta_0+\theta_1 }{2} \right)
    \end{equation}
    où les \( X_i\) sont des variables aléatoires indépendantes et identiquement distribuées de loi parente \( \dN(\theta_0,1)\). Cela est la probabilité d'être dans la région de rejet alors que l'hypothèse \( H_0\) est vraie. La formule \eqref{EqwxFBMu} se transforme en
    \begin{subequations}
        \begin{align}
        \alpha&=P\left( \frac{ \frac{1}{ n }\sum X_i-\theta_0 }{ 1/\sqrt{n} }>\frac{ \frac{ \theta_0+\theta_1 }{2}-\theta_0 }{ 1/\sqrt{n} } \right)\\
        &=P(T>\sqrt{n}\frac{ \theta_1-\theta_0 }{2}).
        \end{align}
    \end{subequations}
    En termes de la fonction de répartition nous avons alors
    \begin{equation}
        \alpha=1-F\big( \sqrt{n}\frac{ \theta_1-\theta_0 }{2} \big)
    \end{equation}
    Il s'agit maintenant de trouver le nombre \( n\) qui réalise cette égalité. Pour cela nous utilisons l'inverse \( F^{-1}\) de la fonction de répartition de la normale :
    \begin{equation}    \label{EqQLFsJpi}
        n=\left( \frac{ 2 }{ \theta_1-\theta_0 }F^{-1}(1-\alpha) \right)^2.
    \end{equation}
    Le risque de seconde espèce est la possibilité d'accepter \( H_0\) lorsque \( H_1\) est vraie, c'est à dire
    \begin{equation}        \label{EqBKQoyL}
        \beta=P\left( \frac{1}{ n }\sum_i X_i<\frac{ \theta_0+\theta_1 }{2} \right)
    \end{equation}
    sous l'hypothèse \( H_1\). Dans le calcul de \eqref{EqBKQoyL} nous prenons donc \( X_i\sim\dN(\theta_1,1)\). Le résultat est que
    \begin{equation}
        \beta=F\big( \sqrt{n}\frac{ \theta_0-\theta_1 }{2} \big).
    \end{equation}
    
    \begin{remark}
        L'expression \eqref{EqQLFsJpi} diminue lorsque \( \theta_0\) et \( \theta_1\) s'éloignent, ce qui est normal : plus les nombres à discerner sont éloignés, moins l'échantillon à prendre pour réaliser le travail doit être grand.

        Notons aussi que \( \theta_0-\theta_1<0\), par conséquent augmenter \( n\) diminue la valeur de 
        \begin{equation}
            \beta=F\big( \sqrt{n}\frac{ \theta_0-\theta_1 }{2} \big)
        \end{equation}
    \end{remark}

\end{corrige}
