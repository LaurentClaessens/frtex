% This is part of Exercices de mathématique pour SVT
% Copyright (C) 2010
%   Laurent Claessens et Carlotta Donadello
% See the file fdl-1.3.txt for copying conditions.

\begin{corrige}{interro-0003}

	\begin{enumerate}
		\item
			En remplaçant, nous trouvons $\frac{ 0 }{ 0 }$, donc il y a une indétermination à lever. Pour cela, nous factorisons le numérateur et le dénominateur. Le numérateur est un produit remarquable : $a^2-b^2=(a+b)(a-b)$, et donc $x^2-9=(x+3)(x-3)$. En ce qui concerne le dénominateur, nous avons $x^2-2x-3$ pour
			\begin{equation}
				x=\frac{ 2\pm\sqrt{4-4\cdot 1\cdot (-3)} }{ 2 }=\frac{ 2\pm\sqrt{16} }{ 2 }=\frac{ 2\pm 4 }{ 2 },
			\end{equation}
			et donc $x=3$ ou $x=-1$.

			Par conséquent le dénominateur s'écrit
			\begin{equation}
				x^2-2x-3=(x-3)(x+1),
			\end{equation}
			et $(x-3)$ se simplifie dans la fraction. Par conséquent nous devons calculer la limite
			\begin{equation}
				\lim_{x\to 3} \frac{ x+3 }{ x+1 }=\frac{ 6 }{ 2 }=3.
			\end{equation}
		\item
			Nous mettons $x^2$ en évidence au numérateur et $x$ au dénominateur. Cela donne
			\begin{equation}
				\frac{ x^2-2x }{ 2x-3 }=\frac{ x^2\left( 1-\frac{ 2 }{ x } \right) }{ x\left( 2-\frac{ 3 }{ x } \right) }.
			\end{equation}
			Lorsque nous prenons la limite pour $x\to \infty$, les fractions $\frac{ 2 }{ x }$ et $\frac{ 3 }{ x }$ tendent vers zéro, tandis qu'après simplification par $x$, il reste
			\begin{equation}
				\lim_{x\to \infty} \frac{ x }{ 2 }=\infty.
			\end{equation}
			
		\item
			En remplaçant, nous trouvons $\frac{ 0 }{ 0 }$, ce qui est une indétermination à lever. Ici le truc est de multiplier et diviser par le binôme conjugué $(1+\cos(x))$ :
			\begin{equation}
				\begin{aligned}[]
					\frac{ 1-\cos(x) }{ \sin^2(x) }&=\frac{ \big( 1-\cos(x) \big)\big( 1+\cos(x) \big) }{ \sin^2(x)\big( 1+\cos(x) \big) }\\
					&=\frac{ 1-\cos^2(x) }{ \sin^2(x)\big( 1+\cos(x) \big) }\\
					&=\frac{1}{ 1+\cos(x) }
				\end{aligned}
			\end{equation}
			où nous avons utilisé le fait que $1-\cos^2(x)=\sin^2(x)$, et simplifié par $\sin^2(x)$. Maintenant nous pouvons conclure le calcul :
			\begin{equation}
				\lim_{x\to 0} \frac{1}{ 1+\cos(x) }=\frac{ 1 }{2}.
			\end{equation}
		\item

			Lorsque nous posons $y=\frac{ x }{ 3 }$, nous remplaçons $\frac{ 3 }{ x }$ par $\frac{1}{ y }$ et $x$ par $3y$. D'autre part, lorsque $x$ tend vers l'infini, la nouvelle variable $y$ tend aussi vers l'infini, donc nous avons
			\begin{equation}
				\lim_{y\to \infty} \left( 1+\frac{ 1 }{ y } \right)^{3y}=\left[ \left( 1+\frac{1}{ y } \right)^y \right]^3.
			\end{equation}
			La limite de ce qu'il y a dans le crochet est une limite connue : elle vaut $e$. Donc nous avons
			\begin{equation}
				\lim_{y\to \infty} \left[ \left( 1+\frac{1}{ y } \right)^y \right]^3=e^3
			\end{equation}
		\item
			Ici encore le truc est de multiplier et diviser par la binôme conjugué de $\sqrt{x+4}-2$. Nous avons
			\begin{equation}
				\frac{ x }{ \sqrt{x+4}-2 }=\frac{ x\big( \sqrt{x+4}+2 \big) }{\big(  \sqrt{x+4}-2  \big)\big(  \sqrt{x+4}+2 \big)  }
				=\frac{ x\big(  \sqrt{x+4}+2  \big) }{ x+4-4 }.
			\end{equation}
			Après simplification par $x$, nous restons avec le calcul
			\begin{equation}
				\lim_{x\to 0} \sqrt{x+4}+2=4.
			\end{equation}
			
	\end{enumerate}

\end{corrige}
