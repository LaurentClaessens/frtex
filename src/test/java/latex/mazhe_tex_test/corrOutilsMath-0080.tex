% This is part of Exercices et corrigés de CdI-1
% Copyright (c) 2011
%   Laurent Claessens
% See the file fdl-1.3.txt for copying conditions.

\begin{corrige}{OutilsMath-0080}

    La difficulté est d'exprimer les vecteurs de la base locale $\{ e_r,e_{\theta} \}$ en termes des vecteurs de la base cartésienne $\{ e_x,e_y \}$. Nous savons que les vecteurs de la base locale des coordonnées polaires sont donnés par
    \begin{subequations}
        \begin{numcases}{}
            e_r=\cos(\theta)e_x+\sin(\theta)e_y\\
            e_{\theta}=-\sin(\theta)e_x+\cos(\theta)e_y.
        \end{numcases}
    \end{subequations}
    En utilisant les formules $x=r\cos(\theta)$ et $y=r\sin(\theta)$, nous avons immédiatement
    \begin{equation}
        \cos(\theta)=\frac{ x }{ r }=\frac{ x }{ \sqrt{x^2+y^2} }
    \end{equation}
    et
    \begin{equation}
        \sin(\theta)=\frac{ y }{ r }=\frac{ y }{ \sqrt{x^2+y^2} }.
    \end{equation}
    Par conséquent
    \begin{subequations}
        \begin{numcases}{}
            e_r=\frac{1}{ \sqrt{x^2+y^2} }(xe_x+ye_y)\\
            e_{\theta}=\frac{1}{ \sqrt{x^2+y^2} }(-ye_x+xe_y).
        \end{numcases}
    \end{subequations}
    En remplaçant cela dans la formule de $F$,
    \begin{equation}
        \begin{aligned}[]
            F(x,y)&=ye_r+2e_{\theta}\\
            &=\frac{ y }{ \sqrt{x^2+y^2} }(xe_x+ye_y)  +2\frac{ 1 }{ \sqrt{x^2+y^2} } (-ye_x+xe_y)\\
            &=\frac{ 1 }{ \sqrt{x^2+y^2} }  \big( (yx-2y)e_x+(y^2+2x)e_y \big).
        \end{aligned}
    \end{equation}

\end{corrige}
