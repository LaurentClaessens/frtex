\begin{exercice}[\minsyndical]\label{exoEspVectoNorme0001}

On considère l'espace vectoriel $\eR^n$ muni de la norme euclidienne.
\begin{enumerate}
	\item
		Montrer que la norme euclidienne est \defe{convexe}{convexe!norme} c'est à dire que quelque soient  $ x,y  \in \eR^n $ et $ t \in [0,1]$ on a 
		 \begin{equation}
	 		\|tx + (1-t)y\| \le t\|x\| + (1-t)\|y\|.
		 \end{equation}
	 
	\item
		 Montrer que la norme est lipschitzienne de constante $1$,  c'est à dire
	        quelque soient  $ x,y  \in \eR^n$ on a 
		\begin{equation}
		         \left| \ \|x\| - \|y\| \ \right| \le \|x-y\|. 
		\end{equation}
 \item
	 Montrer l'\defe{identité du parallélogramme}{identité!parallélogramme} :
        quelque soient  $ x,y  \in E $ on a 
	\begin{equation}
	        \|x-y\|^2 + \|x+y\|^2 = 2\|x\|^2 + 2\|y\|^2 .
	\end{equation}
		
\end{enumerate}

% TODO : Préciser lesquels de ces trois points sont valables pour une norme quelconque.

\corrref{EspVectoNorme0001}
\end{exercice}
