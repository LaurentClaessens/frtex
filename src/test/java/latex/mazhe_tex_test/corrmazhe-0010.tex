% This is part of Analyse Starter CTU
% Copyright (c) 2014
%   Laurent Claessens,Carlotta Donadello
% See the file fdl-1.3.txt for copying conditions.

\begin{corrige}{mazhe-0010}

    Il suffit d'appliquer la formule \eqref{Thfondcalc} et d'utiliser la linéarité de l'intégrale et le tableau des primitives des fonctions fondamentales. 
  \begin{enumerate}
  \item $\displaystyle \int_0^1 x^5+3x^2+3 \, dx = \left[\frac{1}{6} x^6 + x^3 +3x \right]_0^1 = \frac{1}{6}+4 = \frac{25}{6} $ ;
  \item $\displaystyle \int_1^2 x^{1/3}+4x^{1/2} \, dx  = \left[\frac{3}{4} x^{\frac{4}{3}} + 4 \frac{2}{3} x^{\frac{3}{2}} \right]_1^2 =\frac{3}{2} 2^{\frac{1}{3}} + \frac{16}{3} 2^{\frac{1}{2}} - \frac{41}{12} $ ;
  \item $\displaystyle \int_{-\pi}^{0} \cos(x) \, dx  = \left[\sin(x) \right]_{-\pi}^0  = 0$ ;
  \item $\displaystyle \int_{-2}^{1} e^x \, dx = \left[e^x \right]_{-2}^{1} = e - e^{-2} $ ;
  \item $\displaystyle \int_{0}^{1} \frac{1}{\sqrt{x}} \, dx =\int_{0}^{1} x^{-1/2} \, dx  = \left[2x^{\frac{1}{2}}  \right]_0^1  = 2$.
  \end{enumerate}
\end{corrige}
