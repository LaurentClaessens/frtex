\begin{corrige}{012}
The chart $\dpt{ f }{ \eR^2 }{ S^2 }$ of $S^2$ that we chose is
\[ 
  f(\theta,\varphi)=(\cos\theta\sin\varphi,\cos\theta\cos\varphi,\sin\theta).
\]
In these coordinates, a basis of $T_{(\theta_0,\varphi_0)}S^2$ is given by
\[ 
  \Dsdd{ f(\theta_0+t,\varphi_0) }{t}{0}\quad\text{and}\quad\Dsdd{ f(\theta_0,\varphi_0+t) }{t}{0}.
\]
Computations give
\begin{subequations}
\begin{align}
  \partial_{\theta}&=-
\begin{pmatrix}
\sin\theta_0\sin\varphi_0\\
\sin\theta_0\cos\varphi_0\\
-\cos\theta_0
\end{pmatrix}
,\\
\partial_{\varphi}&=
\begin{pmatrix}
\cos\theta_0\cos\varphi_0\\
-\cos\theta_0\sin\varphi_0\\
0
\end{pmatrix}
.
\end{align}
\end{subequations}
One can check that $\partial_{\theta}\cdot f(\theta_0,\varphi_0)=\partial_{\varphi}\cdot f(\theta_0,\varphi_0)=0$: tangent vectors are orthogonal to radius. It is easy to see that $\partial_{\varphi}\cdot\partial_{\theta}=0$, $\partial_{\theta}\cdot\partial_{\theta}=1$ and $\partial_{\varphi}\cdot\partial_{\varphi}=\cos^2\theta$.

\end{corrige}
