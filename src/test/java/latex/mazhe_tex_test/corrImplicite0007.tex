% This is part of Exercices et corrigés de CdI-1
% Copyright (c) 2011
%   Laurent Claessens
% See the file fdl-1.3.txt for copying conditions.

\begin{corrige}{Implicite0007}

	La relation $4x^2+16y^2+8z^2=1$ définit bien $z(x,y)$ pourvu que $z\neq 0$, en effet
	\begin{equation}
		\frac{ \partial F }{ \partial z }=16z.
	\end{equation}
	En procédant comme d'habitude, nous pouvons trouver les dérivées partielles $\partial_xz$ et $\partial_yz$. Nous pouvons même utiliser la formule explicite
	\begin{equation}
		z(x,y)=\pm\sqrt{\frac{ 1-4x^2-16y^2 }{ 8 }}.
	\end{equation}
	Ce faisant, nous avons oublié une série de points : ceux où $z=0$. Pour traiter ces points, nous savons que si $y\neq 0$, alors nous avons l'expression
	\begin{equation}
		y(x,z)=\sqrt{\frac{ 1-4x^2-8z^2 }{ 16 }}
	\end{equation}
	qui permet de trouver les plans tangents. Il reste enfin les points avec $y=z=0$ donnés par $x=\pm1/4$ qui doivent être traités séparément.

\end{corrige}
