\section{Identifications in \texorpdfstring{$AdS_3$}{AdS3}}\label{secBTZ}
%--------------------------------------------------------

In this section, we recall for the reader convenience the definition and construction of the non-rotating BTZ black hole \cite{BTZ_un,BTZ_deux}, emphasizing on some geometrical properties put forwards in \cite{BTZB_un,BDRS,Keio,Clement}. To lighten the presentation, the proofs will essentially be omitted and referred to the existing literature.

This situation will serve us as a guideline in defining black holes in general anti de Sitter spaces (see section \ref{SecCausal}).

Ba\~nados, Henneaux, Teitelboim and Zanelli observed that taking the quotient of (a part of) the three-dimensional anti de Sitter space ($AdS_3$) under the action of well-chosen discrete subgroups of its isometry group gives rise to solutions which correspond to axially symmetric and static black hole solutions of (2+1)-dimensional Einstein gravity with negative cosmological constant, characterized by their mass $M$ and angular momentum $J$.

The space $AdS_3$ is defined as the (universal covering of the) simple Lie group $\SL(2,R)$
\begin{equation}\label{AdS3}
AdS_3 \cong \SL(2,R) = \{g \in GL(2,\eR) \, | \, \mbox{det} g = 1 \}:= G
\end{equation}
endowed with its Killing metric $\dpt{B}{\sG\times\sG}{\eR}$, which can be extended to the whole group by
\begin{equation}
B_g(X,Y)=B(dL_{g^{-1}}X,dL_{g^{-1}}Y).
\end{equation}
Here, $\sG$ stands for the Lie algebra of $\SL(2,R)$. See section \ref{SecToolSL} for a review of $\SL(2,\eR)$ and its Lie algebra. We define the following one-parameter subgroups of $\SL(2,R)$:
\begin{equation}\label{IwasawaSubgroups}
A = \exp(\eR H), \quad  N = \exp(\eR E), \quad \bar{N}=\exp(\eR
F), \quad K=\exp(\eR T),
\end{equation}
with $T=E-F$. They are the building blocks of the Iwasawa decomposition
\begin{equation}
K \times A \times N \longrightarrow \SL(2,R): (k,a,n)
\longrightarrow kan \quad \mbox{or} \quad ank.
 \end{equation}
 The $\SL(2,R)$ subgroups $AN$ and $A\bar{N}$ are called \emph{Iwasawa subgroups}; they are minimal parabolic subgroups.

 We have now dispose of all necessary ingredients to make the definition of BTZ black holes more precise.

 \begin{definition}
 The one-parameter subgroup of $Iso(G)$ defined by
 \begin{equation}
 \psi_t (g) = \exp(t\, a\, H) \, g \exp(-t\, a\, H), \quad a \in \eR_0,\quad g \in G \,
 \end{equation}
 is called the \defe{BHTZ subgroup}{BHTZ!subgroup}. Its generator $ \Xi = a(H,H)$ is called the \defe{identification vector}{identification vector}.  The \defe{BHTZ action}{BHTZ!action} associated with $\Xi$ is $\psi_{\eZ}: G \rightarrow G$. It is an action of $\eZ$ on $G$.
 \label{defBHTZ}
 \end{definition}

 \begin{definition}
 A \defe{safe region}{safe region} in $AdS_3$ is defined as an open connected domain in which
 \begin{equation}
 \|\Xi\|^2:= \beta_z (\Xi,\Xi) > 0 \quad .
 \end{equation}
 \label{safe}
 \end{definition}

 \begin{definition}
 A \defe{non-rotating massive BTZ black hole}{non-rotating massive BTZ black hole} is defined as the quotient of a safe region in $AdS_3$ under the BHTZ action.
 \label{NRMassive}
 \end{definition}

 This definition deserves some comments. First, the restriction to a safe region in $AdS_3$ ensures that the resulting quotient space be free of closed time-like curves. This means that the other parts of $AdS_3$ have to be ``cut out''{} from the original space.  Furthermore, due to the identifications, one may restrict to a fundamental domain of the BHTZ action. Secondly, the \defe{black hole singularity}{black hole!singularity} $\hS$ is defined as the surfaces where the identification vector becomes light-like: 
 \begin{equation}  \label{Singularities}
 \hS = \{z \in AdS_3 \tq \beta_z(\Xi,\Xi) = 0\}.
 \end{equation}
 Thus, the BTZ singularity represents singularities in the causal structure, not curvature ones. The resulting space is causally inextensible, i.e. trying to extend it would produce closed time-like curves. Finally, the BTZ space-time exhibits all characteristic features of a black hole. Namely, it has \emph{event horizons}, that is, surfaces hiding a region (\emph{the interior region}, see hereafter) causally disconnected from spatial infinity.

 Note that it is the choice of identification vector which dictates the nature (rotating, extremal, vacuum or non-rotating massive) of the resulting black hole. Moreover, not all choices give rise to black holes.

 The reason why we here focus on the non-rotating massive case lies in the peculiar geometrical properties of its horizons and singularity.  

 We are now ready to define the horizons. The following definition is adapted to the present case, but cannot be used in general, see \ref{SecCausal} for more general definitions:

 \begin{definition}
 A point $g$ will be said to lie in the \defe{future interior region}{future!interior region}, denoted by ${\cal M}^{\mbox{int},+}$, if all future-directed light rays issued from $g$ necessarily fall into the black hole singularity, that is
 \begin{equation}\label{interioreq}
 g \in {\cal M}^{\mbox{int},+} \Leftrightarrow \forall k \in K, \exists s \in
 \eR^+  \mbox{s.t.}  \|\underline{H} - \overline{H}\|^2_{l^k_g (s)}=0.
 \end{equation}
 The \defe{future horizon}{future!horizon} $\hH^+$ is defined as the boundary of ${\cal M}^{\mbox{int},+}$.
 \label{interior-horizons}
 \end{definition}
 Equation \eqref{interioreq} simply expresses that any future-directed causal signal necessarily falls into the black hole singularity and cannot escape it. The \defe{past interior region}{past!interior region} and \defe{past horizon}{past!horizon} are defined in a similar way.

Using the embedding \eqref{hyperboloide} of $AdS_3$ into $\eR^{2,2}$, one finds, from \eqref{Singularities} and \eqref{interioreq}, that
\begin{equation}		\label{BTZSingHor}
	\begin{aligned}[]
		\hS	&\equiv t^2 - y^2 = 0,\\
		\hH	&\equiv u^2 - x^2 = 0,
	\end{aligned}
\end{equation}
 where $\hH = \hH^+ \cup \hH^-$.

As pointed out in \cite{Keio}, these results can be stated more intrinsically as follows:
\begin{proposition}
	In $G=AdS_3$, the non-rotating BTZ black hole singularity is given by a union of minimal parabolic subgroups of G:
	\begin{equation}
	\hS = Z(G) A N \cup Z(G) A \overline{N},
	 \end{equation}
	 where $Z(G)=\{e,-e\}$ denotes the center of $G=\SL(2,R)$.
	\label{BTZSing}
\end{proposition}

\begin{proposition}		\label{PropLatClassANSLdeuxR}
	In $G=AdS_3$, the non-rotating BTZ black hole horizon corresponds to a union of lateral classes of minimal parabolic subgroups of G:
	\begin{equation}		\label{EqHorClassLatdeux}
		\hH = Z(G) A N J \cup Z(G) A \overline{N} J \quad,
	\end{equation}
	 where $J=\exp(\frac{3\pi}{2}T) \in K$ satisfies $J^2 = e$.
	\label{BTZHor}
\end{proposition}

These two propositions follow directly from \eqref{BTZSingHor}, using the parametrization \eqref{hyperboloide}. Note that the small difference between the formula given here and the one in the paper \cite{Keio} is simply due to a different choice of parametrization \eqref{hyperboloide}.

They show that the black hole structure is closely related to the minimal parabolic subgroup of $\SL(2,R)$. Of course, this construction cannot be generalized in a straightforward way to higher-dimensional anti de Sitter spaces, because of the peculiar nature of the three-dimensional case, being the only to enjoy a group manifold structure. Rather, we will reconsider in the next section the case treated here in a more general framework, putting on an equal footing all anti de Sitter spaces. Again, a minimal parabolic subgroup will reveal crucial in the construction.

\subsection{Global description of the black hole}
%------------------------------------------------

In this appendix, we use results and techniques of \cite{BTZB_un,BTZB_deux,Keio,Clement} to derive the equation of the non-rotating BTZ black holes horizons. We will begin by stating some results which will be useful in describing the global geometry of the black hole.

\begin{proposition} 
Let $\sigma$ be the unique exterior automorphism of $G$ fixing pointwise the Cartan subgroup $A$ and consider the following \emph{twisted} action of $G$ on itself:
\begin{equation}
 \tau: G \times G \longrightarrow G: (g,x) \rightarrow \tau_g(x):=
 g\,x\,\sigma(g^{-1}).
 \end{equation}
Then, the BHTZ action  can be rewritten as
\begin{equation}
 \psi_n = \tau_{\exp(n \sqrt{M} H)}, \quad n\in \eZ.
 \end{equation}
\end{proposition}
The proof follows from the fact that $\sigma$ fixes the generator $H$. Using the action $\tau$, one finds the following global decomposition of $G$:
\begin{proposition}
 The map
\begin{equation}
\begin{aligned}
 \phi\colon A\times G/A&\to G \\ 
(a,[g])&\mapsto \tau_{g}(a) 
\end{aligned}
\end{equation}
 is well-defined as a global diffeomorphism.
 \end{proposition}

 This follows from the observation that the application
 \begin{equation} \label{twistedI}
 \phi: K \times A \times N \rightarrow G: (k,a,n) \rightarrow
 \phi(k,a,n) = \tau_{kn}(a) \quad
 \end{equation}
is a global diffeomorphism on $G$ (``twisted Iwasawa decomposition''). As a consequence, the space $G$ appears as the total space of a trivial fibration over $A = \SO(1,1) \simeq \eR$ whose fibers are the $\tau_G-$orbits, i.e. the $\sigma-$twisted conjugacy classes. As a homogeneous $G-$space, every fiber is isomorphic to $G/A = AdS_2$. Moreover, the BHTZ action is fiberwise, because
\begin{equation} \label{fiberwise}
 \tau_h (\phi(a,[g])) = \phi(a,h.[g]) = \phi(a,[hg]).
 \end{equation}
The Killing metric on $G$ turns out to be globally diagonal with respect to the twisted Iwasawa decomposition \cite{BTZB_un}:
\begin{equation}
ds^2_G = da_A^2 - \frac{1}{4} \cosh^2 (a) ds^2_{G/A},
\end{equation}
where $ds^2_{G/A}$ denotes the canonical projected $AdS_2-$metric on $G/A$. The study of the quotient space $G/\eZ$ therefore reduces to the study of $(G/A)/\eZ$.

The space $G/A$ can be realized as the $G-$equivariant universal covering space of the adjoint orbit ${\cal O}:= \mbox{Ad}(G)H$ in $\sldr$, where it corresponds to a one sheet hyperboloid. In this picture, we may identify the part of the hyperboloid corresponding to a safe region (see definition \ref{safe}) in~$G$.

\begin{lemma}
In ${\cal O}$, a connected region where the orbits of the BHTZ action are space-like is given by
\begin{equation}
 \{ X = x^H H+x^E E + x^F F \in {\cal O} \,\, | \,\, -1 < x^H <
 1\}.
\end{equation}
Furthermore, it can be parameterized as
\begin{equation}
 X = Ad\left(\exp(\frac{\theta}{2}H) \, \exp(-\frac{\tau}{2}
 T)\right)\, H, \, 0<\tau<\pi \, ,\,  -\infty<\theta<+\infty
 .
\end{equation}
\end{lemma}
This has been proven in \cite{BTZB_un}. From this and the preceding proposition, we find a global description of a safe region in $G$ well adapted to the BHTZ identifications.

\begin{proposition}
A global description of a safe region in $G$ is given by
\begin{equation}\label{CoordGlob}
 z(\rho,\theta,\tau) =
\tau_{\exp(\frac{\theta}{2}H) \,
 \exp(-\frac{\tau}{2}
 T)}(\exp(\rho H)).
 \end{equation}
 Furthermore, the action of the BHTZ subgroup reads in these coordinates
\begin{equation}
 (\tau , \rho, \theta) \rightarrow (\tau, \rho, \theta + 2 n
a).
\end{equation}
\end{proposition}


\subsection{Derivation of the horizons}
%--------------------------------------

Now we have to study the equation of \eqref{interioreq}. Using the bi-invariance of the Killing metric and the Ad-invariance of the Killing form, it reduces to
\begin{equation}\label{EqHoriz}
B(H,H) - B(H,Ad(\mbox{e}^{-s Ad(k)E}) Ad(x) H) = 0.
\end{equation}

\begin{lemma}
${\cal M}^{\mbox{int},+}$ is A bi-invariant.
\label{BiInv}
\end{lemma}
 \begin{proof}
 This equation is clearly invariant under $x \to x\cdot a$ for each $a\in A$. In order to see the invariance under $x \to a\cdot x$, one uses the cyclicity of the trace to bring the second term to
\[
 B(H,Ad(Ad(a^{-1})\mbox{e}^{u Ad(k)E}) Ad(x) H).
\]
But $Ad(a^{-1})\mbox{e}^{-s Ad(k)E} = \mbox{e}^{-\tilde{s} Ad(\tilde{k})E}$, with $\tilde{s} = s (\mbox{e}^{-2a} \cos^2\theta + \mbox{e}^{2a} \sin^2 \theta)$ and $\cot t = \mbox{e}^{-2a} \cot \theta$, where $k = \mbox{e}^{\theta T}$ and $\tilde{k}=\mbox{e}^{t T}$. The net result is thus simply a relabelling of the parameters (note that $s$ and $\tilde{s}$ have the same signs!)
\end{proof}

Let us now consider a light ray starting from a safe region in $G$. Because of \eqref{CoordGlob} and lemma \ref{BiInv}, we may restrict our study to
\begin{eqnarray}
 z &=& \mbox{e}^{-\tau/2 T} \mbox{e}^{\rho H} \sigma(\mbox{e}^{\tau/2 T}) \\
 &=& \mbox{e}^{-\tau/2 T} \mbox{e}^{\rho H} \mbox{e}^{-\tau/2 T}.
 \end{eqnarray}
The equation to study reduces to
\begin{equation}\label{EqHoriz2}
B(H,H) - B(H,Ad(\mbox{e}^{-s Ad(k)E}) Ad(\mbox{e}^{-\tau/2 T}
\mbox{e}^{\rho H} \mbox{e}^{-\tau/2 T}) H) = 0
\end{equation}
with $\tau\in \, ]0,\pi[$ and  $\rho \in \eR$.

Let us focus on the points in $Ad(G)H$ corresponding to
\[
{\cal B}:= Ad(\mbox{e}^{-\tau/2 T} \mbox{e}^{\rho H} \mbox{e}^{-\tau/2 T})H,
\]
with $\tau \in \, ]0,\pi[ \,\, , \,\, \rho \in \eR$. First note that $Ad(\mbox{e}^{\rho H} \mbox{e}^{-\tau/2 H}) H$ precisely corresponds to a safe region on the hyperboloid. Thus ${\cal B}$ is the region swept out by the a safe region when rotating it counterclockwise around the $T$-axis with an angle~$\pi$.  

It can be seen that the domain ${\cal B}$ can be decomposed into three regions:
\begin{equation}
 {\cal B} = {\cal B}_1 \cup {\cal B}_2 \cup {\cal B}_3,
 \end{equation}
with
\begin{subequations}
\begin{align}
{\cal B}_1 &=Ad(A)Ad(\mbox{e}^{-\beta/2 T})H &&   \beta \in \, ]0,2\pi[,\\
{\cal B}_2 &=Ad(A)Ad(\mbox{e}^{t (E+F)})H    &&    t\in \eR, \\
{\cal B}_3 &=Ad(A)(-H \pm E) \textrm{ or } Ad(A)(-H \pm F).
\end{align}
\end{subequations}
Thanks to the A bi-invariance, we may forget about the $Ad(A)$ in the above equations. We are thus led to analyze the existence of solutions of \eqref{EqHoriz2} with $X \in {\cal B}$ of the form $X_1 = Ad(\mbox{e}^{-\beta/2 T})H$, $X_2 =Ad(\mbox{e}^{t (E+F)})H$ and $X_3 =-H \pm E \, , \, -H \pm F$.

Consider the first case. With $ Ad(\mbox{e}^{-\tau/2 T} \mbox{e}^{\rho H} \mbox{e}^{-\tau/2 T}) H$ of the form $Ad(\mbox{e}^{-\beta/2 T})H$, equation \eqref{EqHoriz2} becomes 
\begin{equation}  \label{EqU}
 \frac{1}{4} s^2 (\cos \beta - \cos(\beta + 4\theta)) + s \sin
 \beta  + 2 \sin^2 \beta = 0.
\end{equation}
We are looking for the values of $\beta$ for which this equation admits a solution for $s>0$, for all $\theta \in \, [0,\pi]$ --this range for $\theta$ originates from the fact that $G/A$ is a $\eZ_2$ covering of $\Ad(G)H$.  By considering the particular case $\theta = 0$, we find $s=-\tan \frac{\beta}{2}$, thus the allowed values of $\beta$ have to lie in the range $]\pi,2\pi[$. Let us look at the constrains imposed by other values of $\theta$. If we denote by $s_1$ and $s_2$ the two roots of \eqref{EqU}, we have
\begin{eqnarray}
 s_1\cdot s_2 &=& \frac{4 \sin^2 \beta/2}{\sin 2\theta \sin(\beta +
 2 \theta)}, \label{PrRac} \\
 s_1 + s_2 &=& \frac{- 2 \sin \beta}{\sin 2\theta \sin(\beta +
 2\theta)}.
 \end{eqnarray}
First note that, $\forall \beta \in ]0,2\pi[$, the quantity $\sin 2\theta \sin(\beta + 2\theta)$ may be positive or negative as $\theta$ varies in the range $[0,\pi]$. If $\sin \beta < 0$, then there are two positive roots when $\sin 2\theta \sin(\beta + 2\theta)>0$, and one positive and one negative when $\sin 2\theta \sin(\beta + 2\theta)<0$. Thus there always exist a positive solution for $u$, for any $\theta$. If $\sin \beta > 0$, there are two negative roots when $\sin 2\theta \sin(\beta + 2\theta)>0$. Consequently, the interior region will correspond to points $X_1 = Ad(\mbox{e}^{-\beta/2 T})H \, \, , \, \, \beta \in \,]\pi , 2\pi[ $ on the adjoint orbit.
\begin{probleme}
On parle d'une solution pour tout $u$. Quel $u$ ?
\end{probleme}

For the second case, $X_2 =Ad(\mbox{e}^{t (E+F)})H$, the equation we get is
\begin{equation}
 \frac{1}{4} s^2 (\cosh 2t - \cos 4\theta \cosh 2t + 2 \sin
 2\theta \sinh 2t) + s \cos 2\theta \sinh 2t + (1 - \cosh 2t) = 0.
 \end{equation}
 By considering two special cases, it is easy to see that this equation does not admit a positive solution in $u$ for all $\theta$. Indeed, for $\theta = \pi/2$, one finds $s =-\tanh t$, while for $\theta = 0$, one gets $s=\tanh t$. Thus there is no $t\ne 0$ satisfying both conditions. The last case yields no positive solution for all $\theta$ neither.

 As a conclusion we find that the interior region is given by
 \begin{equation}
x \in {\cal M}^{\mbox{int},+} \Leftrightarrow Ad(x)H =
Ad(A)Ad(\mbox{e}^{-\beta/2 T})H, \quad \mbox{with} \,\,
\beta \in \, ]\pi , 2\pi[.
\end{equation}
The boundaries of the corresponding region in $Ad(G)H$ are given by $-H + r^2 E$ and $-H + r^2 F$ or
\begin{equation}
 Ad(N^-)(-H) \cup  Ad(\bar{N}^+)(-H),
 \end{equation}
 with $N^- = \{\mbox{e}^{tE}\}_{t\leq 0}$ and $\bar{N}^+ = \{\mbox{e}^{tF}\}_{t\geq 0}$.

The horizons can be deduced as
\begin{equation}
 x \in {\cal H^+} \Leftrightarrow Ad(x)H = Ad(N^-)(-H) \,\,
 \mbox{or} \,\, Ad(x)H = Ad(\bar{N}^+)(-H).
 \end{equation}
Because of the A-invariance, we may write $x=\tau_{\mbox{e}^{-\frac{\tau}{2} T}}(\mbox{e}^{\rho H})$ and look for the relation between $\tau$ and $\rho$ such that
\begin{equation}
 Ad(\tau_{\mbox{e}^{-\tau/2
T}}(\mbox{e}^{\rho H}))H = Ad(N^-)Ad(\mbox{e}^{\pi/2 T}) H.
\end{equation}
This amounts to require that
\begin{equation}\label{CondHor}
 \left(\mbox{e}^{-\tau/2 T} \mbox{e}^{\rho H} \mbox{e}^{-\tau/2 T}\right)^{-1} \,
 (\mbox{e}^{-t^2 E} \mbox{e}^{\pi/2 T}) \in A \cup Z(G).
 \end{equation}

This condition gives $\cos\tau = \tanh \rho$, $\rho<0$, $\tau\in \, ]\pi/2,\pi[$. By replacing $\mbox{e}^{-t^2 E}$ with $\mbox{e}^{t^2 F}$, one gets $\cos\tau = -\tanh \rho$, $\rho>0$, $\tau\in \, ]\pi/2,\pi[$.

The domain ${\cal M}^{\mbox{int},-}$ is of course defined as
\begin{equation}\label{interior}
 x \in {\cal M}^{\mbox{int},-} \Leftrightarrow \forall k \in K, \exists u \in
 \eR^-  \mbox{s.t.}  \|\underline{H} - \overline{H}\|^2_{l^k_x (u)}=0.
 \end{equation}
The \emph{past horizon} ${\cal H^-}$ is defined as the boundary of ${\cal M}^{\mbox{int},-}$. By proceeding the same way, we find that
\begin{equation}
x \in {\cal M}^{\mbox{int},-} \Leftrightarrow Ad(x)H =
Ad(A)Ad(\mbox{e}^{-\beta/2 T})H, \quad \mbox{with} \,\,
\beta \in \, ]\pi,2 \pi[,
\end{equation}
and
\begin{equation}
 x \in {\cal H^-} \Leftrightarrow Ad(x)H = Ad(N^+)(-H) \,\,
 \mbox{or} \,\, Ad(x)H = Ad(\bar{N}^-)(-H),
 \end{equation}
or in coordinates: $\tau\ \in \, ]0,\pi/2[$, $\cos \tau = \tanh \rho$ for $\rho > 0$ and $\cos \tau =-\tanh \rho$ for $\rho < 0$.

We thus established the following
\begin{proposition}
In a safe region in $G$ parameterized by
\[
z(\rho,\theta,\tau) =\tau_{\exp(\frac{\theta}{2}H) \, \exp(-\frac{\tau}{2} T)}(\exp(\rho H)),
\]
 the horizons ${\cal H}:= {\cal H}^+ \cup {\cal H}^-$ of the non-rotating BTZ black hole are given by
\begin{equation}
 \cos \tau = \pm \tanh \rho.
 \end{equation}
\end{proposition}
As a direct consequence, we have the
\begin{corollary}		\label{CorHorClassLat}
In terms of the embedding coordinates \eqref{hyperboloide} of $G$ in
$\eR^{2,2}$, the horizons of the non-rotating BTZ black hole are
\begin{equation}
 {\cal H} \equiv u^2 - x^2 = 0.
 \end{equation}
 \end{corollary}

\subsection{General considerations}
%----------------------------------

\subsubsection{Construction of the black hole}
%////////////////////////////////////////////

Let $\Xi$ be a Killing vector field on $G=\SL(2,R)$, and $\dpt{\psi_t}{\SL(2,R)}{\SL(2,R)}$ be its flow, i.e 
\[
\Dsdd{\psi_t(g)}{t}{s}=\Xi_{\psi_s(g)}
\]
for all $g\in\SL(2,R)$. We will discuss later why we need $\Xi$ to be a Killing vector field, and the reason why we will choice\footnote{Here, $\uvH$ is the left invariant vector field of $H$ and $\ovH$ the right one: $\uvH_z=(dL_z)_eH$ and $\ovH_z=(dR_z)_eH$, and $\alpha$ is a real constant related to some underlying physics.} 
\begin{equation}\label{eq:def_Xi}
\Xi=\uvH+\alpha\ovH.
\end{equation}

\begin{theorem}
There is a diffeomorphism $A\times N\times K\to \SL(2,R)$ given by
\[
 (a,n,k)\to anka^{\alpha}
\]
if $|\alpha|< 1$.
\end{theorem}
\begin{proof}
No proof.
\end{proof}

This theorem suggests us a coordinate system $(\theta,n,\tau)$ given by $a=e^{\theta H}$, $n=e^{n E}$, $k=e^{\tau T}$ where $T+E-F$:
\[
g(\theta,\tau,u)=e^{\theta H}e^{uE}e^{\tau T}e^{-\theta\alpha H}.
\]

We consider the space $BTZ=AdS_3/\psi_n=AdS_3/\sim$, where $\sim$ is the equivalence relation on $\SL(2,R)$ given by $g\sim g'$ if and only if there exists a $n\in\eZ$ such that $\psi_n(g)=g'$.

If $\psi_t(g)$ is a time-like curve, then the identification creates some closed time-like curves; so the region of $AdS_3$ where $\|\Xi\|^2<0$ must be banned of our theory. More precisely, we put
\[
\scrS=\{z\in\SL(2,R)\tq \|\Xi_z\|^2=0\},
\]
and we cut the space by $\scrS$: $AdS_3'=\SL(2,R)\setminus \{z\in\SL(2,R)\tq \|\Xi_z\|^2\leq 0\}$. So it is valuable to choice $\Xi$ in order to be not everywhere time-like or everywhere space-like. The choice \eqref{eq:def_Xi} corresponds to 
\begin{equation}
\psi_t(g)=e^{tH}ge^{-\alpha tH},
\end{equation}
and it is easy to see that $\psi_tg(\theta,\tau,u)=g(\theta+t,\tau,u)$. In these coordinates, $\Xi=\partial_{\theta}$.

\subsubsection{Light rays and causal structure}
%//////////////////////////////////////////////

Let us point out that a geodesic $exp_x(t\Xi)$ has a constant velocity (definition of a geodesic); then the norm of $\Xi$ along an identification curve is constant, figure \ref{LabelFigNEtAchr}. The part of space in which this is positive is the physical part and the part in which it is negative is the singular part.

\newcommand{\CaptionFigNEtAchr}{The norm (in particular its sing) of \( \Xi\) is contant along a geodesic.}
\input{Fig_NEtAchr.pstricks}

A light ray in $AdS_3$ passing through the point $g$ is given by curves of the form
\begin{equation}
 g(s)=ge^{s\Ad(e^{\kappa T})E}.
\end{equation}
The light ray passing through $z$ is the direction $\kappa$ is written $\ell_g\hkappa(s)$. We say that $g_0$ is in the \defe{black hole}{black hole} if for any $\kappa$, there exists a $s<\infty$ such that $\ell\hkappa_{g_0}(s)\in\scrS$. The \defe{horizon}{horizon} is the boundary of the black hole. In order to describe it, we have to solve 
\begin{equation}
 B_{\ell_{g_0}\hkappa(s)}(\Xi,\Xi)=0.
\end{equation}
