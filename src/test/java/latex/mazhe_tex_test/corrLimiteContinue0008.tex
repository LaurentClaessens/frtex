\begin{corrige}{LimiteContinue0008}

  \begin{enumerate}
  \item  Comme pour les  exercices \ref{exoLimiteContinue0003} et \ref{exoLimiteContinue0004}
    \begin{equation}
      \lim_{x\to 0}\left(\lim_{y\to 0}\frac{x^2 y }{x^4 + y^2} \right)=\lim_{x\to 0}\frac{0}{x^4} = 0,
    \end{equation}
    de même pour l'autre limite. 
    \item Nous calculons la limite de la fonction sur chaque droite passant par l'origine. Soit $(v_1, v_2)$ un vecteur fixé, alors
      \begin{equation}
        \lim_{t\to 0}f(tv_1, tv_2)=\lim_{t\to 0}\frac{v_1^2v_2 t^3}{t^2(v_1^4t^2+v_2^2)}=\lim_{t\to 0} t \underbrace{\frac{v_1^2v_2}{v_1^4t^2+v_2^2}}_{A},
      \end{equation} 
      cette limite existe et sa valeur est zéro, parce que $A$ est une fonction bornée de $t$.
      \item Il est suffisant de remarquer que la limite  pour $t \to 0$ de $f$ restreinte à  la parabole $(t, t^2)$ n'est pas zéro. En fait
        \begin{equation}
        \lim_{t\to 0}f(t, t^2)=\lim_{t\to 0}\frac{t^4}{t^4+t^4}=\frac{1}{2}.
        \end{equation}    
  \end{enumerate}

\end{corrige}
