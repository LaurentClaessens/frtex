% This is part of the Exercices et corrigés de mathématique générale.
% Copyright (C) 2009-2011
%   Laurent Claessens
% See the file fdl-1.3.txt for copying conditions.
\begin{corrige}{General0017}

Il y a deux types d'intégrales remarquables. Le premier est
\begin{equation}
	\int\frac{ f' }{ f }=\ln(f)
\end{equation}
pour toute fonction $f$.

Le second est
\begin{equation}
	\int ff'=\frac{ f^2 }{2}
\end{equation}
pour toute fonction $f$. Cette formule se prouve simplement faisant le changement de variable $u=f$, donc $du=f'dx$. Avec ce changement de variable, nous trouvons $\int ff'dx=\int udu=\frac{ u^2 }{2}$.

Afin de se ramener à une de ces deux, il faut soit faire un changement de variable, soit essayer de voir si une partie de la fonction à intégrer ne ressemble pas à une autre partie.

\renewcommand{\theenumi}{\arabic{enumi}}
\begin{enumerate}

\item
Utiliser la formule $\int x^n=\frac{ x^{n+1} }{ n+1 }$, et trouver
\begin{equation}
	\int(x+\sqrt{x})dx=\frac{ x^2 }{ 2 }+\frac{ 2 }{ 3 }x^{3/2}.
\end{equation}

\item
\item

\item
Faire le changement de variable $u=3x-7$, $du=2dx$, ce qui amène
\begin{equation}
	\int \frac{dx}{ 3x-7 }=\frac{1}{ 3 }\int \frac{ du }{ u }=\frac{ \ln(u) }{ 3 }=\frac{1}{ 3 }\ln(3x-7).
\end{equation}

\item\label{ItemIntCinq}
Après avoir effectué le changement de variable $u=2x$, il nous reste à faire
\begin{equation}
	I=\int\tan(2x)=\frac{1}{ 2 }\int\frac{ \sin(u) }{ \cos(u) }du.
\end{equation}
La subtilité est de voir que le numérateur $\sin(u)$ est la dérivée (au signe près) du dénominateur $\cos(u)$, donc nous sommes dans le cas de l'intégrale remarquable $f'/f$ avec $f(x)=\cos(x)$ :
\begin{equation}
	I=-\frac{ 1 }{2}\ln\big( \cos(u) \big)=-\frac{ 1 }{2}\ln\big( \cos(2x) \big).
\end{equation}

\item
Nous savons que $\big( \tan(x) \big)'=1/\cos^2(x)$, donc nous posons $u=7x$, $du=du/7$ et nous trouvons
\begin{equation}
	I=\int\frac{ dx }{ \cos^2(7x) }=\frac{1}{ 7 }\int\frac{ du }{ \cos^2(u) }=\frac{1}{ 7 }\tan(7x).
\end{equation}

\item Même idée que \ref{ItemIntCinq}.

\item
Ici, la bonne idée est de poser $u(x)=-\sin(x)$, de telle manière à avoir $dx=-du/\cos(x)$. L'intégrale devient
\begin{equation}
	\int e^{-\sin(x)}\cos(x)dx=\int e^u(-du)=- e^{-\sin(x)}.
\end{equation}

\item
Nous savons que $\big( \tan(x) \big)'=\sec^2(x)$, donc nous sommes dans le cas remarquable $\int ff'$ avec $f=\tan(x)$. La réponse est donc
\begin{equation}
	I=\frac{ \tan(x)^2 }{2}.
\end{equation}

\item
Le truc à voir, c'est que $\big( \sqrt{2x^2+3} \big)'=\frac{ 2x }{ \sqrt{2x^2+3} }$, donc
\begin{equation}
	I=\frac{ 1 }{2}\int \frac{ 2x }{ \sqrt{2x^2+3} }=\frac{ 1 }{2}\int \big( \sqrt{2x^2+3} \big)'=\frac{ \sqrt{2x^2+3} }{2}.
\end{equation}

\item
La formule à repérer est $\big(\arctan(x)\big)'=1/(1+x^2)$, donc nous sommes encore dans le cas remarquable $\int ff'$ avec $f(x)=\arctan(x)$. La réponse est donc tout de suite
\begin{equation}
	I=\frac{ f(x)^2 }{ 2 }=\frac{ 1 }{2}\arctan(x)^2.
\end{equation}

\item
Lorsque nous avons des combinaisons simples de fonctions trigonométriques, effectuer un changement de variable où $u$ est une des fonctions. Ici, le changement $u=\sin(x)$ fait l'affaire:
\begin{equation}
	\int\frac{ \cos(x) }{ 1+\sin^2(x) }=\int\frac{ du }{ 16+u^2 }=\frac{1}{ 4 }\arctan\left( \frac{ u }{ 4 } \right)=\frac{1}{ 4 }\arctan\left( \frac{ \sin(x) }{ 4 } \right).
\end{equation}

\item
Cette intégrale est un cas remarquable $\int (f'/f)$ avec $f(x)=\ln(x)$.

\item\label{ItemdixseptUnQuatre}
	
	\begin{verbatim}
		sage: var('x')
		sage: f(x)=1/sqrt(9-x**2)
		sage: f.integrate(x)
		x |--> arcsin(1/3*x)
	\end{verbatim}


\item
	\begin{verbatim}
		sage: f(x)=1/(4-9*x**2)
		sage: f.integrate(x)   
		x |--> -1/12*log(3*x - 2) + 1/12*log(3*x + 2)
	\end{verbatim}
\item

	\begin{verbatim}
		sage: f(x)=1/sqrt(x**2+9)
		sage: f.integrate(x)     
		x |--> arcsinh(1/3*x)
	\end{verbatim}
	Noter ici le sinus hyperbolique dû au signe différent par rapport au point \ref{ItemdixseptUnQuatre}.
\item\label{Item001717}
Ici, c'est le changement de variable $u=\ln(x)$ qui fonctionne parce qu'il amène gratuitement $du=dx/x$, c'est à dire le $x$ du dénominateur :
\begin{equation}
	\int\frac{ dx }{ x\sqrt{1-\ln^2(x)} }=\int\frac{ du }{ \sqrt{1-u^2} }=\arcsin(u)=\arcsin\big( \ln(x) \big).
\end{equation}

\item

	\begin{verbatim}
		sage: f(x)=sin(2*x)/sqrt(1+cos(x)**2)
		sage: f.integrate(x)                 
		x |--> -2*sqrt(cos(x)^2 + 1)
	\end{verbatim}
	

\item
Ici, afin de faire la même chose que dans le numéro \ref{Item001717}, on peut être tenté de poser $u= e^{2x}$ parce que c'est ce qui arrive dans la racine. Hélas, ça ne fonctionne pas (essayez pour voir). Le bon changement est $u=e^x$, et $du= e^{x}dx$, de telle manière à obtenir
\begin{equation}
	\int\frac{  e^{x}dx }{ \sqrt{1- e^{2x}} }=\int\frac{ du }{ \sqrt{1-u^2} }=\arcsin(e^x).
\end{equation}

\item
	\begin{verbatim}
		sage: f(x)=acos(x)**2/sqrt(1-x**2)
		sage: f.integrate(x)              
		x |--> -1/3*arccos(x)^3
	\end{verbatim}

\item
Le dénominateur ressemble à la formule connue de l'intégrale $\int\frac{1}{ \sqrt{1-x^2} }$. Hélas, nous avons $x^4$ au lieu de $x^2$. Du coup, nous allons essayer un changement de variable qui fait que $u^2=x^4$, comme ça nous aurons $\sqrt{1-u^2}$ au dénominateur, comme dans la formule connue. Le changement est $u=x^2$, $du=2xdx$. Il vient :
\begin{equation}
	\int\frac{ xdx }{ \sqrt{1-x^4} }=\frac{ 1 }{2}\int\frac{ du }{ \sqrt{1-u^2} },
\end{equation}
que nous savons faire.

\item
\item
Ici, il s'agit de trouver un changement de variable qui amène sur $u^2-1$, pour cela, nous commençons par récrire le dénominateur de la façon suivante :
\begin{equation}
	a^2x^2-b^2=b^2(\frac{ a^2 }{ b^2 }x^2-1),
\end{equation}
et puis nous faisons le changement de variable $u=ax/b$, qui donne
\begin{equation}
	I=\frac{1}{ b^2 }\int\frac{ (b/a)du }{ u^2-1 }=-\frac{1}{ ab }\frac{ 1 }{2}\ln\left| \frac{ 1+u }{ 1-u } \right| .
\end{equation}

\item
Ici, il faut voir que $1/\cos^2(x)$ est la dérivée de $\tan(x)$, donc le changement de variable $u=\tan(x)$ amène l'intégrale
\begin{equation}
	\int\frac{ du }{ \sqrt{u-1} }
\end{equation}

\item
Le $x$ au dénominateur et la présence d'un logarithme indiquent le changement de variable $u=\ln(x)$, qui amène à
\begin{equation}
	\int\frac{ \cos\big( \ln(x) \big)dx }{ x }=\int\cos(u)du=\sin(u)=\sin\big( \ln(x) \big).
\end{equation}

\item
\item
En écrivant l'intégrale sous la forme
\begin{equation}
	\int\frac{  e^{x/2}dx }{ \sqrt{1- e^{x}} },
\end{equation}
le changement de variable $u= e^{x/2}$ se propose tout seul. Nous avons $dx=2 e^{-x/2}du$ et il reste à intégrer
\begin{equation}
	\int\frac{ 2du }{ \sqrt{1-u^2} }.
\end{equation}

\item
\item
Encore une fois, la présence d'un $x$ au dénominateur et d'un logarithme suggère de poser $u=\ln(x)$. Il reste
\begin{equation}
	I=\int\frac{ \sqrt{1+\ln(x)} }{ x }dx=\int\sqrt{1+u}du,
\end{equation}
cette dernière intégrale se règle par le changement de variable $v=1+u$, et nous trouvons
\begin{equation}
	I=\frac{ v^{3/2} }{ 3/2 }=\frac{ 2 }{ 3 }(1+u)^{3/2}=\frac{ 2 }{ 3 }\big(1+\ln(x)\big)^{3/2}.
\end{equation}

\item
\item
Chose à remarquer : $\Big( \ln\big( \sin(x) \big) \Big)'=\cotg(x)$, donc on est dans le cas $\int(f'f)$ avec $f=\ln\big( \sin(x) \big)$, et donc
\begin{equation}
	I=\frac{ f(x)^2 }{2}=\frac{ 1 }{2}\ln^2\big( \sin(x) \big).
\end{equation}

\item
Le changement de variable $u=1-x^3$ amène
\begin{equation}
	\frac{1}{ 3 }\int u^2du=\frac{1}{ 3 }\frac{ u^3 }{ 3 }=\frac{1}{ 9 }(1-x^3)^3.
\end{equation}
\end{enumerate}



\end{corrige}
