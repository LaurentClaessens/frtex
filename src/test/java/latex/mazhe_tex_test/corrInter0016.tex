% This is part of the Exercices et corrigés de mathématique générale.
% Copyright (C) 2009
%   Laurent Claessens
% See the file fdl-1.3.txt for copying conditions.
\begin{corrige}{Inter0016}

La surface générée par la révolution de la courbe $y=1/x$ autour de l'axe $Ox$ entre $1$ et l'infini est donnée par
\begin{equation}
	S=2\pi\int_0^{\infty}\sqrt{1+\left( -\frac{1}{ x^2 } \right)^2}\frac{1}{ x }dx=2\pi\int_0^{\infty}\frac{ \sqrt{x^4+1} }{ x^3 }dx.
\end{equation}
Lorsque $x$ est grand, $x^4+1$ est à peu près comme $x^4$, et donc le numérateur est à peu près comme $x^2$. Donc, lorsque $x$ est grand, la fonction sous l'intégrale se comporte comme $1/x$, dont l'intégrale diverge.

Nous concluons que la surface est infinie.

En ce qui concerne le volume, le calcule est simple :
\begin{equation}
	V=\pi\int_0^{\infty}\frac{1}{ x^2 }dx=\pi\left[ -\frac{1}{ x } \right]_1^{\infty}=\pi,
\end{equation}
donc il y a un volume fini, entouré par une surface infinie.

\end{corrige}
