% This is part of Exercices et corrigés de CdI-1
% Copyright (c) 2011
%   Laurent Claessens
% See the file fdl-1.3.txt for copying conditions.

\begin{corrige}{OutilsMath-0083}

    Nous avons $F_{\rho}=0$, $F_{\theta}=\sin(\varphi)$ et $F_{\varphi}=1$. Par conséquent,
    \begin{equation}
            \begin{aligned}[]
        \nabla\cdot u&=\frac{1}{ \rho^2\sin(\theta) }\left( \frac{ \partial  }{ \partial \theta }(\rho\sin\theta\sin\varphi)+\frac{ \partial  }{ \partial \varphi }(\rho) \right)\\
        &=\frac{ \sin\varphi\cos\theta }{ R\sin\theta }.
            \end{aligned}
    \end{equation}
    Notez qu'on a le droit de remplacer $\rho$ par $R$ \emph{après} avoir fait les dérivées.

    Pour le rotationnel,
    \begin{equation}
        \nabla\times u=\frac{1}{ R\sin\theta }(\cos\theta-\cos\varphi)e_{\rho}+\frac{1}{ R }e_{\theta}+\frac{ \sin\varphi }{ R }e_{\varphi}.
    \end{equation}
    Étant donné que le rotationnel de $u$ n'est pas nul, il n'existe pas de fonction dont le gradient est $u$.

\end{corrige}
