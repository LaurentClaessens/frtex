% This is part of Un soupçon de physique, sans être agressif pour autant
% Copyright (C) 2006-2009
%   Laurent Claessens
% See the file fdl-1.3.txt for copying conditions.


\begin{corrige}{INGE11140036}

	\begin{enumerate}

		\item

			Étant donner que multiplier une suite par un nombre constant ou l'additionner à un nombre constant ne change par le fait qu'elle soit bornée ou non, la suite $s_n=2-\frac{ n-1 }{ 10 }$ sera bornée si et seulement si $\frac{ n-1 }{ 10 }$ est bornée, ce qui arrivera si et seulement si $n-1$ est bornée. Or on sait bien que la suite $n-1$ n'est pas bornée.

			Comme la suite n'est pas bornée, elle ne peut pas être convergente. Étudions la (dé)croissance en calculant la différence entre deux termes consécutifs~:
			\begin{equation}
				s_{n+1}-s_n=2-\frac{ (n+1)-1 }{ 10 }-\left( 2-\frac{ n-1 }{ 10 } \right)=-\frac{ 1 }{ 10 }.
			\end{equation}
			La différence étant négative, la suite est décroissante.

		\item
			La convergence de cette suite se règle par l'étau~:
			\begin{equation}
				-\frac{ 1 }{ n+1 }<\frac{ (-1)^{n-1} }{ n+1 }<\frac{ 1 }{ n+1 }.
			\end{equation}
			La suite proposée converge donc vers zéro.

		\item
			Cette suite n'est ni croissante ni décroissante (parce qu'elle est positive et négative un élément sur deux). Elle ne converge pas non plus parce qu'elle n'est pas bornée.

	\end{enumerate}

\end{corrige}
