% This is part of Exercices et corrigés de CdI-1
% Copyright (c) 2011
%   Laurent Claessens
% See the file fdl-1.3.txt for copying conditions.

\begin{corrige}{Variete0009}

	Pour l'hélice, il faut quelque chose qui tourne dans le plan $xy$, et qui monte dans la direction $z$. La paramétrisation
	\begin{equation}
		\gamma(t)=\big( \cos(t),\sin(t),t/2\pi \big)
	\end{equation}
	est parfaite. Nous avons
	\begin{equation}
		\omega_{\gamma(t)}=\sin(t)dx+\cos(t)dt+\frac{ t^2 }{ 4\pi^2 }dz,
	\end{equation}
	et donc
	\begin{equation}
		\int_{\gamma}\omega=\int_0^{2\pi}\big( -\sin(t)^2+\cos(t)^2+\frac{ t^2 }{ 8\pi^3 } \big)=\frac{ 1 }{ 3 }.
	\end{equation}
	Notons que la forme $\omega$ est exacte. En effet, il faut une fonction $f$ telle que
	\begin{equation}
		\begin{aligned}[]
			\frac{ \partial f }{ \partial x }=y,\\
			\frac{ \partial f }{ \partial y }=x,\\
			\frac{ \partial f }{ \partial z }=z^2.
		\end{aligned}
	\end{equation}
	La dernière nous dit que $f(x,y,z)=\frac{ z^3 }{ 3 }+c(x,y)$. La seconde équation dit donc que $c(x,y)=xy+h(y)$, et enfin la première équation montre que $h=0$. En résumé, $\omega=df$ avec
	\begin{equation}
		f(x,y,z)=xy+\frac{ z^3 }{ 3 }.
	\end{equation}
	Maintenant, on a
	\begin{equation}
		\int_{\gamma}\omega=(f\circ\gamma)(2\pi)-(f\circ\gamma)(0)=f(1,0,1)-f(1,0,0)=\frac{1}{ 3 }.
	\end{equation}
	

\end{corrige}
