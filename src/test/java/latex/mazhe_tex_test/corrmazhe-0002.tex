% This is part of (almost) Everything I know in mathematics
% Copyright (c) 2016
%   Laurent Claessens
% See the file fdl-1.3.txt for copying conditions.

\begin{corrige}{mazhe-0002}

Dans le cas du calcul à la main, il faut en faire un seul parce que, algénriquement, \( a=b\).

Nous avons \( x=1.403\), \( y=0.0004112\) et \( z=-0.0004111\). Et la somme donne :
\begin{equation}
    a=b=1.4030001=0.14030001\times 10^1.
\end{equation}

Faisons d'abord la normalisation de \( x\), c'est à dire \( \fl(x)\).
\begin{equation}
    \fl(x)=0.1403\times 10^1.
\end{equation}
et \( y\) est déjà normalisé :
\begin{equation}
    \fl(y)=0.4112\times 10^{-3}.
\end{equation}
Il n'y a pas d'erreurs d'assignation pour ces deux nombres. 

Pour faire la somme, il faudra déjà un peu casser les nombres pour les écrire de façon à pouvoir les sommer. En effet, il faut écrire les deux nombres avec le même exposant de \( 10\) (le plus grand), pour pouvoir les mettre en colonne :
\begin{subequations}
    \begin{align}
        0.1404\times 10^1&\to 0.1404\times 10^1\\
        0.4112\times 10^{-3}&\to 0.00004\times 10^1.
    \end{align}
\end{subequations}
La somme donne \( 0.14034\times 10 ^1\). Et ça, c'est à nouveau arrondi. Le premier chiffre supprimé est un \( 4\), donc 
\begin{equation}
    x\oplus y=0.1403\times 10^1.
\end{equation}
Et là on remarque que nous avons la même chose que \( x\). C'est un classique du calcul numérique.

Nous avons aussi
\begin{equation}
    \fl(z)=0.4111\times 10^{-3}.
\end{equation}
Et pour faire la somme de cela avec \( x\oplus y\) nous devons le remettre sous la forme d'un \( 10^1\) :
\begin{equation}
    \fl(z)\to -0.00004\times 10^1
\end{equation}
(erreur de conversion), et en sommant on trouve
\begin{equation}
    (x\oplus y)\oplus z=0.140216\times 10^1,
\end{equation}
qui est encore arrondi. Le premier chiffre supprimé est un \( 6\), donc
\begin{equation}
    \fl(a)=0.1403\times 10^1,
\end{equation}

Le nom de l'erreur qui consiste à avoir \( x\oplus y=x\) est ``relation annirmale''.

Calculons \( b\).

Les nombres \( y\) et \( z\) ont même ordre de grandeur, donc pas d'erreur au moment de les mettre sous forme sommable.
\begin{subequations}
    \begin{align}
        \fl(x)+\fl(y)=0.00010\times 10^{-3}.
    \end{align}
\end{subequations}
Cela est renormalisé et arrondi : \( \fl(x)\oplus\fl(y)=0.1000\times 10^{-6}\).

Notons que nous avons ici commis potentiellement une erreur de cancellation parce que entre \( y\) et \( z\), il y a \( 3\) chiffres sur \( 4\) qui sont identiques. Seul le chiffre \( 1\) est significatif en réalité.

Il faut maintenant ajouter \( x\) à cela. D'abord
\begin{equation}
    \fl(x)=0.1403\times 10^1.
\end{equation}
Pour cette somme, il faudra remettre notre \( 0.1000\times 10^{-6}\) avec une puissance \( 10^1\). Et là, nous obtenons zéro parce que vraiment ce nombre est trop petit pour être écrit avec \( 10^1\). Résultat des courses :
\begin{equation}
    \fl(b)=0.1403\times 10^1.
\end{equation}


Dans le premier calcul nous avons deux ``relations anormales'' et dans le second nous en avons une plus une cancellation.

Nous préférons avoir deux relations anormales, parce que l'erreur de cancellation est plus grave : elle consiste à une perte de chiffre significatifs. Le fait est que faisant la différence à l'ordinateur nous avons obtenu \( 0.1\) qui est certes exact, mais qui est un coup de bol : la différence aurait aussi bien pu être \(0.19\) avec d'autres nombres, machinement égaux.

Note : avec les données ici, il n'y a en fait pas d'erreur de cancellation. Mais il y a une erreur potentielle de cancellation, potentiellement grave.

En ce qui concerne l'erreur relative. Dans la formule
\begin{equation}
    \epsilon_r=\frac{ | a-a^* | }{ | a | },
\end{equation}
la différence ne peut pas être calculée à la calculatrice justement parce qu'elle est très potentiellement sujette à erreur de cancellation.
\begin{equation}
    \epsilon_r=\frac{ | 0.1030001\times 10^1-0.1403\times 10^1 | }{ 0.14030001\times 10^1 }=\frac{ 0.1\times 10^{-6} }{ 0.14030001\times 10^1 }\simeq 0.712758\times 10^{-7}.
\end{equation}
En passant à \( 3\) chiffres significatifs, \( 0.713\times 10^{-7}\) (le premier chiffre supprimé est un \( 7\)).

\end{corrige}
