% This is part of the Exercices et corrigés de CdI-2.
% Copyright (C) 2008, 2009
%   Laurent Claessens
% See the file fdl-1.3.txt for copying conditions.


\begin{corrige}{_I-4-2}

Il suffit de trouver une suite de fonctions qui s'écrase sur zéro tout en gardant une surface constante. Par exemple
\begin{equation}
	f_n(x)=
\begin{cases}
	0	&	\text{si $| x |>n^2$}\\
	\frac{ 1 }{ n }	&	 \text{sinon}
\end{cases}
\end{equation}
La convergence uniforme vers zéro est évidente parce que $\| f_n \|_{\infty}=\frac{1}{ n }$. Cette suite ne converge toutefois pas en moyenne quadratique parce que
\begin{equation}
	\| f_n \|_{L^2}=\int_{-\infty}^{+\infty}f_n^2(x)dx=\int_{-n^2}^{n^2}\frac{1}{ n^2 }dx=2.
\end{equation}

Il est intéressant de noter que ce contre-exemple ne tient pas si on demande d'étudier la convergence sur $[a,b]$ au lieu de $\eR$. En réalité, la convergence uniforme sur $[a,b]$ implique la convergence $L^2$ sur $[a,b]$ parce que la convergence uniforme $f_n\to f$ dit que dès que $n\geq N$, $| f_n(x)-f(x) |<\epsilon$, et donc
\begin{equation}
	\| f_n-f \|_{L^2}=\int_a^b| f_n(x)-f(x) |^2dx\leq\epsilon(b-a).
\end{equation}

\end{corrige}
