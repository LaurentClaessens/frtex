\begin{exercice}\label{exoSC_serie2-0005}

	Dans un problème d'écoulement turbulent d'un fluide, l'équation
	\begin{equation}		\label{EqReyn}
		1=\sqrt{c_f}\big( -0.4+1.74\ln(Re\sqrt{c_f}) \big)
	\end{equation}
	relie le coefficient de friction $c_f$ au nombre de Reynolds $Re$. Pour $Re=10^4$, représenter la fonction qui, à $c_f$, associe le second membre de l'équation \eqref{EqReyn}, $c_f$ allant de $0$ à $0.05$.

\corrref{SC_serie2-0005}
\end{exercice}
