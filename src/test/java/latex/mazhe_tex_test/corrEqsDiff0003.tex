% This is part of Exercices et corrigés de CdI-1
% Copyright (c) 2011
%   Laurent Claessens
% See the file fdl-1.3.txt for copying conditions.

\begin{corrige}{EqsDiff0003}

\begin{enumerate}

\item
Nous avons vu dans l'exercice \ref{exoEqsDiff0002}\ref{ItemaEqsDiff00002} que si il existe une solution, elle doit s'écrire sous la forme
\begin{equation}
	y(t)=\ln\left( \frac{ t^4 }{ 4 }+\frac{ t^2 }{ 2 }+C \right)
\end{equation}
où le domaine est déterminé par la valeur de la constante par l'inégalité
\begin{equation}
	 \frac{ t^4 }{ 4 }+\frac{ t^2 }{ 2 }+C >0.
\end{equation}
Nous demandons $y(1)=1$, c'est à dire
\begin{equation}
	y(1)= \ln\left(\frac{ 1 }{ 4 }+\frac{ 1 }{ 2 }+C\right) =1,
\end{equation}
ce qui permet de fixer la constante $C=e-\frac{ 3 }{ 4 }$.

\item
Nous avons déjà trouvé que $y(t)=\tan(t+C)$. La condition de Cauchy dit que $y(0)=\tan(C)=0$, donc $C\in\{ 0,\pi \}$. Il y a donc les deux solutions
\begin{equation}
	\begin{aligned}[]
		y_1(t)&=\tan(t)\\
		y_2(t)&=\tan(t+\pi).
	\end{aligned}
\end{equation}

\item
La même avec $y(0)=1$. La condition $1=\tan(C)$ demande $C\in\{ \frac{ \pi }{ 4 },\frac{ 5\pi }{ 4 } \}$, et donne deux solutions
\begin{equation}
	\begin{aligned}[]
		y_1(t)&=\tan(t+\frac{ \pi }{ 4 })\\
		y_2(t)&=\tan(t+\frac{ 5\pi }{ 4 }),
	\end{aligned}
\end{equation}
dont les domaines de définitions respectifs sont
\begin{equation}
	\begin{aligned}[]
		I'_1&=\mathopen]-\pi-\frac{ \pi }{ 4 },\pi-\frac{ \pi }{ 4 }\mathclose[\\
		I'_2&=\mathopen]-\pi-\frac{ 5\pi }{ 4 },\pi-\frac{ 5\pi }{ 4 }\mathclose[.
	\end{aligned}
\end{equation}
Une bonne question à se poser est de comprendre pourquoi cela ne viole pas le théorème d'unicité de la solution.

\item
Nous savons déjà que la solution est donnée par la formule implicite
\begin{equation}
	y+e^y=\sin(t)+C.
\end{equation}
La condition de Cauchy se traduit en $3+ e^{3}=\sin(\frac{ \pi }{ 2 })+C$, ce qui donne $C=2+ e^{3}$. La fonction $z\mapsto z+e^z$ étant bijective sur tout $\eR$, cette formule définit bien $y(t)$ pour tout $t$.

\item
Nous avons deux solutions mutuellement exclusives : $y(t)=-1/(t+C)$ et $y\equiv 0$. La condition de Cauchy $y(1)=2$ demande d'utiliser la première. Le calcul détermine que $C=-3/2$.

Cette solution n'existe pas en $t=\frac{ 3 }{ 2 }$. Donc on peut mettre n'importe quelle solution sur $\mathopen]-\infty,-\frac{ 3 }{ 2 }\mathclose[$, et puis la solution $y(t)=\frac{ -1 }{ t-\frac{ 3 }{ 2 } }$ sur $\mathopen]\frac{ 3 }{ 2 },\infty\mathclose[$.

\item
Seule la solution identiquement nulle satisfait à la condition.

\item
La condition de Cauchy amène la condition $y(0)=C^{3/2}=1$, donc $C=1$. La solution continue maximale est donc
\begin{equation}
	y(t)=\begin{cases}
	0	&	\text{si $x\leq\frac{ 3 }{ 2 }$}\\
	\left( \frac{ 2x }{ 3 }+1 \right)^{3/2}	&	 \text{si $x>\frac{ 3 }{ 2 }$}
\end{cases}
\end{equation}

\item
La solution $y\equiv 0$ fait l'affaire, mais il y en a une autre : celle avec $C=0$ qui est
\begin{equation}
	y(t)=\begin{cases}
	\left( \frac{ 2x }{ 3 } \right)^{3/2}	&	\text{si $x>0$}\\
	0	&	 \text{si $x\leq 0$.}
\end{cases}
\end{equation}
La seconde ligne de la définition est importante parce qu'on donne une condition de Cauchy en $t=0$, donc nous voulons des solutions qui soient au moins dérivables en zéro, sinon le problème n'a pas de sens. Le fait que la fonction $y$ ainsi définie soit continue est évident. Il faut vérifier qu'elle est aussi dérivable en $t=0$. Pour cela, il suffit de voir que
\begin{equation}
	\lim_{x\to 0^+} \frac{ y(0)-y(x) }{ x }=0,
\end{equation}
ce qui n'est pas très compliqué.

\item
En vertu de la solution \eqref{EqSolGeneRacExpCis} déjà trouvée , nous devons résoudre l'équation $\pm\sqrt{K e^{2}-1}=1$ par rapport à $K$, donc nous devons faire le choix du signe positif et prendre $K=2 e^{-2}$, qui est plus petit que $e^2$, mais plus grand que $e^{-2}$, donc la solution existe, mais n'est pas partout définie.

\end{enumerate}

\end{corrige}
