\begin{corrige}{GeomAnal-0006}

Soit $A$ une partie d'un espace vectoriel normé. Un point $P$ est dans la frontière de $A$ si pour tout $r> 0$ dans $\eR^+$ la boule $B(P, r)$ de centre $P$ et rayon $r$ a intersection non nulle avec $A$ et avec son complémentaire $\complement A$. La boule $B(P, r)$ contient des points de $A$ et des points de $\complement A$ 

Cela veut dire qu'il n'existe pas un seul $\bar r$ dans $\eR^+_{>0}$ tel que $B(P,\bar r)\subset \Int A$, parce que l'intersection entre $ \Int (A)$ et $\complement A$ est nulle. Le point $P$ est donc dans le complémentaire de $\Int (A)$. De même on voit que  $P$ est  dans le complémentaire de $\Int(\complement A)$. Le premier lemme dans l'annexe A nous permet de conclure 
\[
\complement\left(\Int(\complement A)\right)\cap\complement\left(\Int(A)\right)=\complement\left(\Int(\complement A)\cup\Int( A)\right). 
\] 
\end{corrige}
