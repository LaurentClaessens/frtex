\begin{corrige}{CalculDifferentiel0013}

	Nous effectuons le changement de variables
	\begin{equation}
		\begin{aligned}[]
			u&=x+2y\\
			v&=x+\frac{ 1 }{2}y,
		\end{aligned}
	\end{equation}
	avec les dérivées partielles
	\begin{equation}
		\begin{aligned}[]
			\frac{ \partial u }{ \partial x }&=1&\frac{ \partial u }{ \partial y }&=2\\
			\frac{ \partial v }{ \partial x }&=1&\frac{ \partial v }{ \partial y }&=\frac{ 1 }{2}.
		\end{aligned}
	\end{equation}
	Nous posons 
	\begin{equation}
		f(x,y)=\tilde f\big( u(x,y),v(x,y) \big),
	\end{equation}
	et puis nous écrivons l'équation demandée en termes de $\tilde f$. Nous avons
	\begin{equation}		\label{EqeCDzzutEqderuncx}
		\frac{ \partial f }{ \partial x }=\partial_u\tilde f+\partial_v\tilde f,
	\end{equation}
	donc
	\begin{equation}
		\begin{aligned}[]
			\frac{ \partial^2f }{ \partial x^2 }&=\partial^2_{uu}\tilde f\frac{ \partial u }{ \partial x }+\partial^2_{vu}\tilde f\frac{ \partial v }{ \partial x }+\partial^2_{uv}\tilde f\frac{ \partial u }{ \partial x }+\partial^2_{vv}\tilde f\frac{ \partial v }{ \partial x }\\
			&=\partial^2_{uu}\tilde f+2\partial^2_{uv}\tilde f+\partial^2_{vv}\tilde f.
		\end{aligned}
	\end{equation}
	Pour calculer $\partial^2_{xy}f$, nous pouvons calculer $\partial^2_{yx}f$ et donc dériver par rapport à $y$ l'équation \eqref{EqeCDzzutEqderuncx} :
	\begin{equation}
		\begin{aligned}[]
			\frac{ \partial^2f }{ \partial x\partial y }&=\frac{ \partial  }{ \partial y }\big( \partial_u\tilde f+\partial_v\tilde f \big)\\
			&=2\partial^2_{uu}\tilde f+\frac{ 5 }{2}\partial^2_{uv}\tilde f+\frac{ 1 }{2}\partial^2_{vv}\tilde f
		\end{aligned}
	\end{equation}
	où nous avons utilisé le fait que $\partial^2_{uv}\tilde f=\partial^2_{vu}\tilde f$. En ce qui concerne les dérivées par rapport à $y$,
	\begin{equation}
		\frac{ \partial f }{ \partial y }=2\partial_u\tilde f+\frac{ 1 }{2}\partial_v\tilde f
	\end{equation}
	et
	\begin{equation}
		\begin{aligned}[]
			\frac{ \partial^2f }{ \partial y^2 }&=2\partial^2_{uu}\tilde f\frac{ \partial u }{ \partial y }+2\partial_{vu}\tilde f\frac{ \partial v }{ \partial y }+\frac{ 1 }{2}\partial^2_{uv}\tilde f\frac{ \partial u }{ \partial y }+\frac{ 1 }{2}\partial^2_{vv}\tilde f\frac{ \partial v }{ \partial y }\\
			&=4\partial^2_{uu}\tilde f+2\partial^2_{uv}\tilde f+\frac{1}{ 4 }\partial^2_{vv}\tilde f.
		\end{aligned}
	\end{equation}
	En écrivant l'équation à résoudre avec ces expressions, nous trouvons $-\frac{ 9 }{ 12 }\partial^2_{uv}\tilde f=0$, c'est à dire
	\begin{equation}
		\frac{ \partial  }{ \partial u }\left( \frac{ \partial \tilde f }{ \partial v } \right)=0.
	\end{equation}
	Étant donné que la dérivée de $\partial_v\tilde f$ par rapport à $u$ est nulle, nous savons qu'il existe une fonction $\psi$ de $v$ uniquement telle que
	\begin{equation}
		\frac{ \partial \tilde f }{ \partial v }=\psi(v).
	\end{equation}
	En intégrant par rapport à $v$, et en se souvenant que la constante d'intégration peut être une fonction de $u$, nous trouvons qu'il existe une fonction $\varphi$ de $u$ telle que
	\begin{equation}
		\tilde f(u,v)=\int\psi(v)dv+\varphi(u).
	\end{equation}
	La fonction $\psi$ étant arbitraire, sa primitive est arbitraire. Quitte à redéfinir $\psi$, nous écrivons
	\begin{equation}
		\tilde f(u,v)=\psi(v)+\varphi(u).
	\end{equation}
	Nous exprimons maintenant la réponse en termes de $x$ et $y$ :
	\begin{equation}
		f(x,y)=\tilde f\big( u(x,y),v(x,y) \big)=\psi(x+\frac{ 1 }{2}y)+\varphi(x+2y).
	\end{equation}

\end{corrige}
