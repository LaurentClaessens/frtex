\begin{abstract}
This chapter contains two directions that were explored during my thesis and that were not finished for different reasons. 

In the first section, we state a result of Unterberger in \cite{UnterD} which provides a deformation of the complex half-plane, and we show how to translate it as a new noncommutative product on the group $ax+b$, i.e. the Iwasawa subgroup of $\SL(2,\eR)$. The technique of deformation by group action described in appendix \ref{SecDefAction} then induces a deformed product on the dual of its Lie algebra. We do not study the properties (symmetries, maximal functional space of convergence, symplectic condition to be a true quantization, \ldots) of this product, but we show that Unterberger's result assures the existence of at least one good functional space. Unfortunately the formula reveals not to be universal; we show the lack of universality on two examples of actions of the group $ax+b$ on $AdS_2$. The failure is due to divergences of the derivatives of the functions $z_i$ (see equation \eqref{EqDefziDefA}). 

This study is motivated by the fact that recent work (not published yet) of P. Bieliavsky provides an universal deformation of the $AN$ of $\SL(2,\eR)$. We are thus allowed to say that the latter new product is ``better'' than the one of Unterberger. We do not address the question to know the precise point that makes the lack of universality in Unterberger.

The second section is an application of the extension lemma (lemma \ref{EXT}). We show that all the ingredients needed to deform the $AN$ of $\SO(2,n)$ are present. The idea was to deform the $AdS$ black hole using the action of the so-deformed $AN$. That should provide an alternative way to deform $AdS$ to the one presented in chapter \ref{ChDefoBH}, and a quantization of $AdS_l$ using the same group as the group which defines a black hole. That method would use the deformation by group action machinery described in appendix \ref{SecDefAction}. The arising question is naturally to know if that quantization is in some sense equivalent to the one given in chapter \ref{ChDefoBH} or not. That question is not answered yet.

\end{abstract}

\section{Formula of Unterberger on \texorpdfstring{$\SL(2,\eR)$}{SL2R} }	\label{SecEplolUnter}
%++++++++++++++++++++++++++++++++++++++++++++++

The following results come from \cite{UnterD} (from page 1219) and provide a deformation\footnote{or, at least, a new noncommutative product.} of the half-plane
\[
  D=\{ (\xi,\eta)\tq \eta>0 \}\subset\eR^2.
\]

Before to give the precise statement that will be used, we need some definitions. A first product is defined by (we will precise the functional space later):
\begin{equation}
(f\circ g)(\xi,\eta)=\sum_{\alpha,\beta}\frac{ (-1)^{\alpha} }{ \alpha !\beta! }(4i\pi)^{-\alpha-\beta}(\partial^{\alpha}_q\partial^{\beta}_p\tilde f)(0,0)(\partial^{\beta}_q\partial^{\alpha}_p\tilde g)(0,0)
\end{equation}
where
\[ 
  \tilde f(p,q)=f\Big( p+\xi\big(q+\sqrt{1+q^2}\big),\eta\big( q+\sqrt{1+q^2} \big) \Big),
\]
and the same for $\tilde g$. In particular,
\[
  (f\circ g)(0,1)=4\int f\big( \Psi(q_{1},p_{1}) \big)g\big( \Psi(q_{2},p_{2}) \big) e^{ -4i\pi(-q_{1}p_{2}+q_{2}p_{1})}\,dq_{1}\,dp_{1}\,dq_{2}\,dp_{2}.
\]
with
\[
  \Psi(p,q)=\big(p,q+\sqrt{1+q^{2}}\big)
\]

\begin{definition}
Let $r_1$, $r_2$ and $n$ be real numbers with $r_1\geq 0$. We denote by $\Sigma^{n}_{r_1,r_2}$ the space of functions $f\in  C^{\infty}(D)$ such that for all  $(j,k)\in\eN\times\eN$, there exists a $C>0$ such that
\begin{equation} 
\Big| \left( \frac{ \partial }{ \partial\xi } \right)^{j}\left( \eta\frac{ \partial }{ \partial\eta } \right)^{k}f(\xi,\eta)  \Big|
\leq C\eta^{r_1}(1+\eta)^{r_2}(1+| \xi |)^{n-j}.
\end{equation} 
\end{definition}

Now, theorem 8.2 in \cite{UnterD} states
\begin{theorem}
Let $f\in\Sigma ^n_{r_1,r_2}$ and $g\in\Sigma^{n'}_{r_1'nr_2'}$. For each $N\in\eN_0$, the function
\begin{equation}
h_N=f\circ g-\sum_{\alpha+\beta\leq N-1} \frac{ (-1)^{\alpha} }{ \alpha !\beta ! }(4i\pi)^{-\alpha-\beta}\sum_{j,k,j',k'}C_{\beta,\alpha}^{j,k}C_{\alpha,\beta}^{j',k'}(e_1^je_2^kf)(e_1^{j'}e_2^{k'}g)
\end{equation}
belongs to the space $\Sigma_{r_1+r_1',r_2+r'_2}^{n+n'-N}$ if constants $C_{\alpha,\beta}^{j,k}$ are defined by the requirement that
\[ 
  (\epsilon_2^{\beta}\epsilon_1^{\alpha}f)(\xi_0,\eta_0)=\sum_{j,k}C_{\alpha,\beta}^{j,k}(e_1^je_2^kf)(\xi_0,\eta_0)
\]
for every smooth function $f$ and $(\xi_0,\eta_0)\in D$ when $j+k\leq \alpha+\beta$ and $j\geq\alpha$ and $C_{\alpha,\beta}^{j,k}=0$ otherwise.  The operators $\epsilon_i$ are defined by $\epsilon_1=e_1=\partial_{\xi}$ and $\epsilon_2=2\big[ 1+\big( \frac{ \eta_0 }{ \eta } \big)^2 \big]^{-1}(\xi_0\partial_{\xi}+\eta_0\partial_{\eta})$.
				\label{ThoUnterSigmaStable}
\end{theorem}


For our purpose, the point is that there exists a product on $\Fun(D)$ and that theorem \ref{ThoUnterSigmaStable} provides a functional space stabilized by the product. We are now going to translate this result in terms of the Iwasawa subgroup $R=AN$ of $\SL(2,\eR)$ that is parametrized (see \eqref{EqParmalSL}) by
\[
  (a,l)=\begin{pmatrix}
 e^{a}&l e^{a}\\0& e^{-a}
\end{pmatrix}.
\]
 The map 
\begin{equation}
\begin{aligned}
 j\colon R&\to R' \\ 
(a,l)&\mapsto ( e^{2a},l e^{2a}) 
\end{aligned}
\end{equation}
provides an isomorphism between $R$ and the group
\[
  R'=\left\{ (\alpha,\beta)= \begin{pmatrix}
\alpha&\beta\\0&1
\end{pmatrix},\,\alpha>0 \right\}.
\]
The inverse of $j$ is  $j^{-1}(\alpha,\beta)=(\ln\alpha^{1/2},\beta\alpha^{-1})$.  The group $R'$ acts on $D$ by
\begin{equation}
(\alpha,\beta)\cdot(\xi,\eta)=(\xi+\beta\alpha^{-1}\eta,\alpha^{-1}\eta)
\end{equation}
which is a freely transitive action. For each choice of ``reference point'' $(\xi_0,\eta_0)\in D$ we  build an identification $i\colon D\to R'$ by the requirement $ i(\xi,\eta)\cdot(\xi_0,\eta_0)=(\xi,\eta)$, that is
\begin{equation}
i(\xi,\eta)=\left( \frac{ \eta_0 }{ \eta },\frac{ \xi-\xi_0 }{ \eta } \right).
\end{equation}

Now we can identify $D$ to $R$ by $k\colon D\to R$, $k=j^{-1}\circ i$. For the choice $(\xi_0,\eta_0)=(0,1)$, we find $k(\xi,\eta)=(k_a(\xi,\eta),k_l(\xi,\eta))$ where
\begin{align}\label{Eqkaklexp}
	k_a(\xi,\eta)&=-\frac{ 1 }{2}\ln\eta,
  &k_l(\xi,\eta)&=\xi
\end{align}
and the function $f$ on $R$ corresponds to the function $\tilde f=f\circ k$ on $D$.

The result of Unterberger is that the function $f$ ``can be quantized'' if
\begin{equation}  \label{eq_condeUnterD}
\left|  \left( \frac{ \partial }{ \partial\xi } \right)^j\left( \eta\frac{ \partial }{ \partial\eta } \right)^k  \tilde f(\xi,\eta)  \right|
\leq C\eta^{r_1}(1+\eta)^{r_2}(1+| \xi |)^{n-j}
\end{equation}
where $n$, $r_1$ and $r_2$ are real numbers and $r_1\geq0$. We want to see what condition has to be imposed on $f$  in order for $\tilde f$ to fulfil this condition. In other words, we want to express the operator
\[ 
  A_{ij}=\left( \frac{ \partial }{ \partial\xi } \right)^j\left( \eta\frac{ \partial }{ \partial\eta } \right)^k
\]
in terms of the coordinates on $R$. For that we compute $\partial_{\xi}\tilde f$ and $(\eta\partial_{\eta})\tilde f$ in terms of $\partial_lf$ and $\partial_af$. 


Let us precise that, when we write expressions like $\eta\partial_{\eta}$, we mean for example
\[
  (\eta\partial_{\eta}\tilde f)(\xi,2)=2(\partial_{\eta}f)(\xi,2).
\]
For $\partial_{\xi}\tilde f$ we have:
\[ 
\begin{split}
  (\partial_{\xi}\tilde f)(\xi,\eta)&=(\partial_lf)\circ k(\xi,\eta)(\partial_{\xi}k_k)(\xi,\eta)\\
				&\quad+(\partial_af)\circ k(\xi,\eta)(\partial_{\xi}k_a)(\xi,\eta),
\end{split}  
\]
using the formula \eqref{Eqkaklexp}, we find $(\partial_{\xi}\tilde f)(\xi,\eta)=(\partial_lf)\circ k(\xi,\eta)$ and we conclude that
\begin{equation}
\partial_{\xi}\tilde f=(\partial_lf)\circ k.
\end{equation}
For $(\eta\partial_{\eta})\tilde f$, we find
\[ 
\begin{split}
(\eta\partial_{\eta})(f\circ k)(\xi,\eta)&=\eta\big( \partial_{\eta}(f\circ k) \big)(\xi,\eta)\\
		&=\eta(\partial_af)\circ k(\xi,\eta)(\partial_{\eta}k_a)(\xi,\eta)\\
		&\quad+\eta(\partial_lf)\circ k(\xi,\eta)(\partial_{\eta}k_l)(\xi,\eta)\\
		&=-\frac{ 1 }{2}(\partial_af)\circ k(\xi,\eta),
\end{split}  
\]
and we conclude that
\begin{equation}
  (\eta\partial_{\eta})(f\circ k)=-\frac{ 1 }{2}(\partial_af)\circ k.
\end{equation}
So the operator $A_{ij}$, expressed on $R$, reads
\begin{equation}
  A_{ij}(f\circ k)=\left( -\frac{ 1 }{2} \right)^j(\partial_a^j\partial_l^if)\circ k,
\end{equation}
and condition \eqref{eq_condeUnterD}, with $(\xi,\eta)=k^{-1}(a,l)=(l, e^{-2a})$ reads now
\begin{equation}  \label{eq_condUR}
 \left|  \frac{1}{ 2^k }(\partial_a^k\partial_l^jf)(a,l)   \right|\leq C e^{-2r_1a}(1+ e^{-2a})^{r_2}(1+| l |)^{n-j}
\end{equation}
with $r_1\geq 0$ and $r_2$, being any real number.
From now on this regularity condition will be referred as the \emph{Unterberger's condition}. That condition characterises a stable functional space for the Unterberger product on $R$.

We want now  to test the deformation of manifold by action of $R$. A somewhat deceiving result that will be shown is that Unterberger's deformation of $R$ is not an universal deformation in the sense that we will find some action of $R$ on manifold for which the action deformation does not provide a deformation of the manifold.

\subsection{Action on the dual of its Lie algebra}
%--------------------------------------------------

The action if given by
\[ 
  (a,l)\cdot\xi=(y_H+2y_El)H^*+y_E e^{-2a}E^*
\] 
where $\xi=y_HH^*+y_EE^*$ is any point in $\sR^*$.  The question is to know if the product $(u\star_{\sR^*}v)$ makes sense when $u$ and $v$ are compactly supported smooth functions on $\sR^*$. In order to address this question, we have to check if for every $\xi$ in $\sR^*$, the function
\[ 
   (\alpha^{\xi}u)(a,l)=u\big( (a,l)^{-1}\cdot\xi \big)
		=u\big( (y_H-2y_Ee^{-2a}l),y_E e^{2a} \big)
\]
fulfils condition \eqref{eq_condUR}. So we consider 
\[ 
  f(a,l)=u\big( \underbrace{(y_H-2y_Ee^{-2a}l)}_{z_H(a,l)},\underbrace{y_Ee^{-2a}}_{z_E(a,l)} \big),
\]
and we compute
\[ 
\begin{split}
  (\partial_lf)(a,l)&=(\partial_Hu)(z_H,z_E)\partial_l(y_H-2y_Ee^{-2a}l)\\
		&\quad+(\partial_Eu)(z_H,z_E)\partial_l(y_Ee^{-2a})\\
		&=(\partial_Hu)(z_H,z_E)(-2y_Ee^{-2a}),
\end{split}  
\]
so
\begin{equation}
(\partial_l^jf)(a,l)=(\partial_H^ju)(z_H,z_E)(-2y_Ee^{-2a})^j.
\end{equation}
The combination $y_E e^{-2a}$ which goes out is precisely $z_E$ which remains in the derivative of $u$. But the derivative of $u$ has compact support. Hence, in fact, the coefficient $y_E e^{-2a}$ remains constrained in the domain where the derivative of $u$ does not vanishes. The point is that the coefficient which go out with derivatives is exactly made of $z_H$ and $z_E$.

So $\sR^*$ is as deformable as $D$. More precisely, a deformation of $\sR^*$ by action of $R$ is induced by the deformation of $D$ by Unterberger.


\subsection{First action on the two dimensional anti de Sitter space}
%---------------------------------------------------------------------

We see $AdS_2$ as in \ref{SubsecGpAdsDeux} and we consider the following action of $AN$ on $AdS_2$:
\[ 
  r\cdot\Ad(g)H=\Ad(gr^{-1})H.
\]
It is easy to see what does this action become in terms of the cylinder:
\[ 
  \big(  e^{y_AH} e^{y_NE} \big)\cdot \Ad\big(  e^{x_KT} e^{x_NE} \big)
	=\Ad\big(   e^{x_KT} e^{x_NE-y_NE} e^{y_AH}  \big)H
\]
where the adjoint action of $ e^{y_AH}$ on $H$ is of course trivial. Thus we have
\begin{equation}
(y_A,y_N)\cdot (x_K,x_N)=(x_K,x_N-y_N).
\end{equation}
Notice that only one dimension of $AN$ really acts. This action is thus not a natural one, and has to be seen as an interesting warm up. Using the notations of coordinates \eqref{EqCylAdSDeux}, we consider $x=\phi(\theta,h)\in AdS_2$ and $u\in  C^{\infty}_c(Cyl)$, a compact supported function on $AdS_2$ and we compute
\[ 
(\alpha^xu)(a,l)=u\big( (a,l)^{-1}\cdot x \big)\\
		=u\big( (-a,-l  e^{2a})\cdot x \big)\\
		=u\big( \theta,h+l e^{2a} \big),
\]
so that if we pose $f(a,l)=u\big( \theta,h+l e^{2a} \big)$, we have
\[ 
  (\partial_lf)(a,l)= e^{2a}(\partial_2u)(\theta,l e^{2a}).
\]
When one makes $a\to\infty$ and $l\to 0$ in such a way that $l e^{2a}$ remains constant, the function $(\partial_lf)$ diverges in an exponential way with respect to $a$. It contradicts Unterberger's condition \eqref{eq_condUR}.
  
\subsection{Second action on the two dimensional anti de Sitter space}
%---------------------------------------------------------------------

Let us now study the more natural action
\begin{equation}
  r\cdot \Ad(g)H=\Ad(rg)H.
\end{equation}
It is in general very difficult to find, for given $y_A$, $y_N$, $x_K$ and $x_N$, the numbers (unique by construction) $z_K$ and $z_N$ such that 
\[ 
  \Ad( e^{y_AH} e^{y_NE} e^{x_KT} e^{x_NE})H=\Ad( e^{z_KT} e^{z_NE})H.
\]
In order to simplify the computations, we use the lemma \ref{LemUnPtParOrbite} which states that we only have to perform the computation for one $(x_K,x_N)$ in each orbit. We begin by $x_K=x_N=0$, i.e. the orbit of $H$ itself. First, computations show that 
\[ 
\begin{split}
\Ad( e^{z_KT} e^{z_NE})H=&\big( \cos(2z_K)-\sin(2z_K)z_N \big)H\\
			&-2\big( \cos(2z_K)z_N+\sin(2z_K) \big)E\\
			&+\Big( \big( \cos(2z_K)-1 \big)z_N+\sin(2z_K) \Big)T.
\end{split}  
\]
Next, 
\[ 
  \Ad( e^{aH} e^{lE})H=\begin{pmatrix}
1&-2 e^{2a}l\\
0& -1
\end{pmatrix}
=H-2 e^{2a}lE.
\]
Comparing with the general form, we find that
\begin{equation}
\Ad( e^{aH} e^{lE})H=\Ad( e^{l e^{2a}E})H,
\end{equation}
or $(a,l)\cdot (0,0)=(0,l e^{2a})$. What is important in our deformation problem is the function
\[ 
  (\alpha^Hu)(a,l)=u\big( (-a,-l e^{2a}))\cdot H \big)=u(0,-l).
\]
This function of course satisfies the Unterberger condition when $u$ has a compact support.

The second orbit that we study is the one of $V=\Ad( e^{\pi T/4})H=-2E+T$. One has
\[ 
\begin{split}
\Ad( e^{aH} e^{lE})V=-lH + e^{-2a}( e^{4a}l^2- e^{4a}-1)E+ e^{-2a}T.
\end{split}  
\]
If we pose $c=\cos(2z_K)$, $s=\sin(2z_K)$ and $b= e^{2a}$, we have to solve the system
\begin{subequations}
\begin{numcases}{}
	c-sz_N=-l  \\
	-2cz_N-2s=\frac{1}{ b }(b^2l^2-b^2-1)\\
	(c-1)z_N+s=\frac{1}{ b }\\
	c^2+s^2=1
	\end{numcases} 
\end{subequations}
with respect to $c$, $s$ and $z_N$. One can check that the following is a solution:
\begin{subequations}
\begin{align}
 c&=\frac{ b^2l^2-2b^2l+b^2-1 }{ b^2l^2-2b^2l+b^2+1 }\\
s&=\frac{ 2b(1-l) }{ b^2l^2-2b^2l+b^2+1 }\\
z_N&=\frac{ b^2(1-l^2)-1 }{ 2b }.
\end{align}
\end{subequations}
If we pose $\bar{z}_K(a,l)=z_K(-a,-l e^{2a})$ and $\bar{z}_N(a,l)=z_N(-a,-l e^{2a})$, we have
\begin{align}
\bar{z}_K(a,l)&=\frac{ 1 }{2}\arcsin\left(\frac{ 2 e^{-2a}(1+l e^{2a}) }{ l^2+2l e^{-2a}+ e^{-4a}+1 }\right)\\
2\bar{z}_N(a,l)&= e^{-2a}- e^{2a}(l^2+1).
\end{align}
The principle of deformation by action of group leads us to  deal with the function
\[ 
  f(a,l)=u(\bar{z}_K(a,l),\bar{z}_N(a,l)),
\]
which should satisfies Unterberger's condition when $u$ is compactly supported. Notice that $z_K$ is a compact variable, so that $u$ can be non vanishing for all values of $z_K$ without violate the compact support requirement.  The derivative of $f$ with respect to $a$ uses the chain rule, and it is apparent the higher order derivatives have to use the Leibnitz formula:
\[ 
\frac{ \partial f }{ \partial a }(a,l)=(\partial_1u)(\bar z_K,\bar{z}_N)\frac{ \partial \bar{z}_K }{ \partial a }(a,l)
		+(\partial_2u)(\bar{z}_K,\bar{z}_N)\frac{ \partial \bar{z}_N }{ \partial a }(a,l).
\]
In order to give an idea of what is going on, here is the first derivative of $\bar{z}_K$ with respect to $a$:
\[ 
  (\partial_a\bar{z}_K)(a,l)=\frac{ 2 e^{2a} }{  e^{4a}l^2+2 e^{2a}l+ e^{4a}+1 }.
\]
Let us look at the limit $a\to-\infty$ on the line $l= e^{-2a}$. If one performs multiple derivatives of $f(a,l)$ with respect to $a$, Leibnitz rules yields a lot of terms of the form
\begin{equation}		\label{EqTermGeederrau}
  (\partial_1^p\partial_2^qu)\big(\bar{z}_K(a,l),\bar{z}_N(a,l)\big)(\partial_a^i\bar{z}_K)(a,l)^j(\partial_a^k\bar{z}_N)(a,l)^m.
\end{equation}
On the line $l= e^{-2a}$, the numerator of $(\partial_a^i\bar{z}_K)(a,l)$ is $( e^{4a}+4)^{2i}$ while the numerator is a sum and product of monomials of the form $( e^{4a}+N)$ with $N > 0$. At the limit, this factor in \eqref{EqTermGeederrau} goes to a finite number. The factor $(\partial_a^k\bar{z}_N)(a,l)$ is very different because  
\[ 
  (\partial_a^k\bar{z}_N)(a,l)=(-1)^k2^{k-1} e^{-2a}-2^{k-1} e^{2a}(l^2+1).
\]
which becomes
\[ 
  2^{k-1} e^{-2a}\big( (-1)^k-1 \big)-2^{k-1} e^{2a}
\]
on $l= e^{-2a}$. It goes to zero when $a\to-\infty$ and $k$ is even, but is goes to $-\infty$ at the same limit when $k$ is odd. The highest divergence in all the terms of type \eqref{EqTermGeederrau} in $(\partial^n_af)$ is expected for maximal $m$, so when $i=j=0$. This is a divergence as
\[ 
  x\mapsto e^{2(n-1)x}.
\]
Notice that this divergence increases when the order of derivative increases. Hence it contradicts Unterberger's condition which works with parameters $r_1$ and $r_2$ who are \emph{constant} with respect to the order of the derivative.
