% This is part of the Exercices et corrigés de mathématique générale.
% Copyright (C) 2009
%   Laurent Claessens
% See the file fdl-1.3.txt for copying conditions.
\begin{corrige}{EquaDiff0010}

\begin{enumerate}

\item
	Le polynôme caractéristique est $x^2-2x-3=0$, et ses solutions sont
	\begin{equation}
		\begin{aligned}[]
			\lambda_1&=3\\
			\lambda_2&=-1,
		\end{aligned}
	\end{equation}
	donc les solutions de l'équation sont données par
	\begin{equation}
		y(x)=A e^{3x}+B e^{-x}
	\end{equation}
	avec $A$, $B\in\eR$.
	
\item
	Le polynôme caractéristique est $x^2+4x+4=0$, et sa racine est
	\begin{equation}
		\lambda_1=\lambda_2=-2.
	\end{equation}
	C'est une racine double. Dans le cas d'une racine double, les solutions \og de base\fg{}  de l'équation différentielle sont $y_1(x)= e^{\lambda x}$ et $y_2(x)=x e^{\lambda x}$. Dans notre cas,
	\begin{equation}
		y(x)=A e^{-2x}+Bx e^{-2x}.
	\end{equation}
	
\item
	Cette fois, les racines du polynôme caractéristique $\lambda^2-4\lambda+5=0$ sont complexes :
	\begin{equation}
		\frac{ 4\pm\sqrt{16-20} }{ 2 }=2\pm i.
	\end{equation}
	Les solutions de base de l'équation différentielle sont donc $y_1(x)= e^{(2+i)x}$ et $y_2(x)= e^{(2-i)x}$. Ce sont des solutions complexes, et il existe une astuce pour en extirper les solutions réelles. D'abord, nous séparons les parties réelles et imaginaires dans les exponentielles :
	\begin{equation}
		e^{(2+i)x}= e^{2x} e^{ix},
	\end{equation}
	et ensuite, nous utilisons la formule d'Euler $ e^{i\omega x}=\cos(\omega x)+i\sin(\omega x)$ :
	\begin{equation}
		\begin{aligned}[]
			y_1(x)= e^{2x}\big( \cos(x)+i\sin(x) \big)\\
			y_2(x)= e^{2x}\big( \cos(x)-i\sin(x) \big).
		\end{aligned}
	\end{equation}
	Nous savons que toutes les combinaisons de $y_1$ et $y_2$ sont des solutions de l'équation différentielle. Les combinaisons qui nous intéressent sont les combinaisons qui ne contiennent plus de nombres complexes. Il n'est pas très difficile de les trouver :
	\begin{equation}
		y_1(x)+y_2(x)=2 e^{2x}\cos(x),
	\end{equation}
	et
	\begin{equation}
		i\big( y_1(x)-y_2(x) \big)=-2 e^{2x}\sin(x).
	\end{equation}
	Les solutions réelles de l'équation sont donc 
	\begin{equation}
		y(x)=A e^{2x}\cos(x)+B e^{2x}\sin(x).
	\end{equation}
			

\item
	\begin{equation}
		y(x)=A e^{3x}+B e^{x}
	\end{equation}

\item
	Les racines du polynôme caractéristique sont $\lambda_1=1/2$ et $\lambda_2=0$. Pour rappel, $e^0=1$, donc la seconde racine ne donne pas lieu à une exponentielle dans la solution :
	\begin{equation}
		y(x)=A e^{x/2}+B
	\end{equation}

\item
	Les solutions complexes sont
	\begin{equation}
		\begin{aligned}[]
			y_1(x)= e^{-2i}\\
			y_2(x)= e^{2i}
		\end{aligned}
	\end{equation}
	Nous en extirpons les solutions réelles en en prenant la somme et la différence multipliée par $i$. Nous avons donc la solution réelle générale
	\begin{equation}
		y(x)=A\cos(2x)+B\sin(2x).
	\end{equation}
	

\item
	Cet exemple est justement traité dans une introduction à Maxima\footnote{\url{http://michel.gosse.free.fr/documentation/introduction/node6.html}}. Les solutions complexes sont données par
	\begin{equation}
		\begin{aligned}[]
			y_1(x)&=\exp\big( \frac{ -1+\sqrt{3}i }{ 2 }x \big)\\
			y_2(x)&=\exp\big( \frac{ -1-\sqrt{3}i }{ 2 }x \big)
		\end{aligned}
	\end{equation}
	Les solutions réelles sont
	\begin{equation}
		\begin{aligned}[]
			y_1(x)&= e^{-x/2}\cos(\frac{ \sqrt{3} }{ 2 }x)\\
			y_2(x)&= e^{-x/2}\sin(\frac{ \sqrt{3} }{ 2 }x)
		\end{aligned}
	\end{equation}
	
\item
	Le polynôme caractéristique admet la racine double $\lambda=-1$. La solution générale est donc
	\begin{equation}
		y(x)=A e^{-x}+Bx e^{-x}.
	\end{equation}
	
\item
	Nous avons $\lambda_1=5$ et $\lambda_2=2$, donc
	\begin{equation}
		y(x)=A e^{5x}+B e^{2x}.
	\end{equation}
	
\item
	Racine double $\lambda=-3/2$, donc
	\begin{equation}
		y(x)=A e^{-3x/2}+Bx e^{-3x/2}.
	\end{equation}

\end{enumerate}
\end{corrige}
