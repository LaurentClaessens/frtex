% This is part of Analyse Starter CTU
% Copyright (c) 2014
%   Laurent Claessens,Carlotta Donadello
% See the file fdl-1.3.txt for copying conditions.

\begin{corrige}{session1-0005}

Nous trouvons facilement par substitution que une solution particulière de l'équation différentielle est $\bar y (x)= 4x + 1$. 

Pour trouver la solution générale de l'équation homogène associée, $\mathcal{Y}_h$,  nous écrivons d'abord le polynôme caractéristique de l'équation, qui est 
\begin{equation*}
  9r^2 + 1 , 
\end{equation*}
et nous cherchons les racines de $9r^2 + 1 = 0$ dans $\mathbb{C}$. On a $r_{1,2} = \pm\frac{i}{3}$, donc la solution générale est 
\begin{equation*}
  \mathcal{Y}_h = \left\{A\cos\left(\frac{x}{3}\right)+ B\sin\left(\frac{x}{3}\right), \: A,\,B\in\mathbb{R}\right\}. 
\end{equation*}
La solution générale $\mathcal{Y}$ de l'équation différentielle d'origine est alors 
\begin{equation*}
  \mathcal{Y} = \left\{A\cos\left(\frac{x}{3}\right)+ B\sin\left(\frac{x}{3}\right)+ \bar y(x), \: A,\,B\in\mathbb{R}\right\}. 
\end{equation*}
La solution particulière $y_p$ qui vérifie $y(0) = 2$, $y'(0)=1$ correspond aux valeurs des coefficients $A = 1$ et  $B=-9$, en fait 
\begin{align*}
 A\cos\left(0\right)+& B\sin\left(0\right)+ \bar y(0) = A +1 \\
&\text{ donc la condition } y(0) = 2 \text{ devient } A+1 = 2 \\
 -\frac{A}{3}\sin\left(0\right)+ &\frac{B}{3}\cos\left(0\right)+ \bar y'(0) = \frac{B}{3} + 4\\
 &\text{ donc la condition } y'(0) = 1 \text{ devient } \frac{B}{3} + 4 = 1 .
\end{align*}
 L'unique solution du système 
 \begin{equation*}
   \begin{cases}
     A+1 = 2,\\
\frac{B}{3} + 4 = 1,
   \end{cases}
 \end{equation*}
est $A = 1$, $B=-9$.
\end{corrige}
