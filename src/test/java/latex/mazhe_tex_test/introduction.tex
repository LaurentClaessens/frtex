% This is part of (almost) Everything I know in mathematics
% Copyright (c) 2013-2014
%   Laurent Claessens
% See the file fdl-1.3.txt for copying conditions.

%+++++++++++++++++++++++++++++++++++++++++++++++++++++++++++++++++++++++++++++++++++++++++++++++++++++++++++++++++++++++++++
\section{Introduction}
%+++++++++++++++++++++++++++++++++++++++++++++++++++++++++++++++++++++++++++++++++++++++++++++++++++++++++++++++++++++++++++

The question arises when one watches movies such as Star Trek: what is a black hole ? One knows from special relativity that light speed cannot be exceeded. So, as a first attempt to define the notion of black hole, we just say that it is a region of the space from which even light cannot escape. Such an object causes a scientific problem because it is by assumption not observable. This fact allows science-fiction writers to invent whatever with no risk of contradiction. That is a Star Trek black hole.

Physical black holes are much more interesting because they are the signal of a general relativity failure.

The Newtonian gravitational field is given by a potential which increases as $1/r$ when you get closer to a massive object. At $r=0$, this potential makes no sense and physics is in trouble. One can avoid the problem by postulating that there exist no pointwise masses and that particles cannot penetrate each other. From these assumptions, the fact that Newtonian mechanics does not impose any limit speed makes the divergence at $r=0$ unimportant.

In general relativity, the divergence at small distances is much more problematic because there is a limit speed; hence a pointwise mass always creates a whole region from which nothing (not even light) can escape. Worse: even a homogeneous ball produces a divergence in the metric when it is too dense, and such objects may exist in the real world. Stated in a more mathematical way: solutions of Einstein's equations for the real world may be singular. From this point of view, black holes are nothing else than a feature in the mathematical framework of relativity which indicates that this is not a final theory. That is the notion of black hole in general relativity and in cosmology.

The transfer of concept from physics to mathematics always consists in taking the key features of the mathematical framework of a physical theory and posing them as definition of a new mathematical object. What are the main mathematical points in the concept of black hole in general relativity ? 
First, we retain the notion of pseudo-Riemannian manifold. The sign of the norm of a vector is the crucial property which allows one to define causality (the light cone).

The second main feature that we extract from the physical situation is the fact that a general relativity black hole has a non empty interior. We saw that this is the key difference between the Newtonian case in which all points are equivalent except the unique point where the mass lies, and the general relativistic case in which a whole region was causally disconnected from the rest of the space.

More precisely, as mathematicians, we ask a black hole to separate the pseudo-Riemannian manifold into two causally disconnected parts in the sense that no light-like geodesics can reach the second region from the first one. Notice that we do not include metric singularity in our mathematical black hole notion. In cosmology, in contrast, black holes always take root in a divergence of some metric invariant such as the curvature.

The anti de Sitter space is a solution of Einstein's equations with constant negative curvature. We consider this space as our framework. First, we define as \emph{singular} the closed orbits of the action of some subgroup of the isometry group $\SO(2,n)$ of anti de Sitter. This is done in such a way to generalize to any dimensions the celebrated BTZ black hole. Then we prove that the resulting structure is a black hole in the sense that it cuts the space into two parts : an interior region from which every light-like geodesic intersects the singularity and an exterior region in which every point accepts at least one light-like geodesics which does not intersect the singularity.  Notice that our black hole does not present any curvature singularity.

The second theme of this thesis is deformation quantization. The key ingredient of quantum mechanics is the noncommutativity of quantum observables. When one tries to measure the velocity and the position of a classical particle (such as a tennis ball or a planet), one can choose the order of measurement. It does not matter which of velocity or position is measured first. Quantum mechanics (the mechanics which governs subatomic particles) is very different. If you measure the position of an electron and then you measure its velocity, you do not get the same result as if you had measured the velocity first and then the position. That noncommutativity in measurements is the very foundation of the quantum mechanics. In the usual mathematical framework, it is implemented by describing each measurable quantity by an operator acting on a Hilbert space. The eigenvalues of these operators correspond to physical measurements. The position and momentum operators for example are respectively $f(x)\mapsto xf(x)$ and $f(x)\mapsto -i\hbar(\partial_xf)(x)$. These two operators obviously do not commute.

In a more abstract way, we say that noncommutativity of quantum mechanics is implemented by considering some noncommutative algebra of operators acting on a Hilbert space, while the classical mechanics deals with observables that are usual functions that form a commutative algebra. The procedure to pass from commutative function algebras to noncommutative operator algebras is the so-called \emph{quantization} in physics.

In our sense\footnote{Quantization is a very large field of mathematics; as far I know, the idea of noncommutativity is always present, but precise notion of ``to quantize something'' may vary from one subject to another}, \emph{deforming} a manifold is simply putting a one-parameter family of new noncommutative products on the set of functions on this manifold. We impose that it reduces to the usual commutative product when the parameter goes to zero. In order to speak of \emph{quantization}, we ask the first order term in the expansion with respect to the parameter to somehow ``contain'' the symplectic structure given on the original manifold.

Questions that arise in this context are: is it possible to study causality in a noncommutative framework ? does it apply to real physics ?

The main result of the present work is not to directly address these large questions, but to build a concrete example in which one can work. Namely, we consider the anti de Sitter space --- that is the simplest non trivial solution of Einstein's equations with constant negative curvature --- that we endow with a black hole structure defined from the action of a subgroup of the isometry group. Then we select the physical part of the space --- the one which is causally connected to infinity --- and we perform a deformation of that part.

The work is divided into three main parts. In a first  time (chapter \ref{ChapBHinAdS}) we define a ``BTZ'' black hole in anti de Sitter space in any dimension. That will be done by means of group theoretical and symmetric spaces considerations. A physical ``good domain'' is identified as an open orbit of a subgroup of the isometry group of anti de Sitter. 

Then (chapter \ref{ChDefoBH}) we show that the open orbit is in fact isomorphic to a group (we introduce the notion of \emph{globally group type} manifold) for which a quantization exists. The quantization of the black hole is performed and its Dirac operator is computed.

The chapter \ref{ChapDefo} is given in a pedagogical purpose: it exposes generalities about deformation quantization and careful examples with $\SL(2,\eR)$ and split extensions of Heisenberg algebras. Explicit decompositions are given for every algebra that will be used in the thesis in chapter \ref{ChapThoComsGroupes}. It serves to make the whole text more self contained and to fix notations. Basics of quantization by group action are given in appendix \ref{SecDefAction}. 

One more chapter is inserted (chapter \ref{ChapNoteDev}). It contains two small results which have no true interest by themselves but which raise questions and call for further development. We discuss a product on the half-plane (or, equivalently, on the Iwasawa subgroup of $\SL(2,\eR)$) due to A. Unterberger. We show that the \emph{quantization by group action} machinery can be applied to this product in order to deform the dual of the Lie algebra of that Iwasawa subgroup. Although this result seems promising, we show by two examples that the product is not universal in the sense that even the product of compactly supported functions cannot be defined on $AdS_2$ by the quantization induced by Unterberger's product. 

 Then we show that the Iwasawa subgroup of $\SO(2,n)$ (i.e. the group which defines the singularity) is a symplectic split extension of the Iwasawa subgroup of $SU(1,1)$ by the Iwasawa subgroup of $SU(1,n)$. A quantization of the two latter groups being known, a quantization of $SO(2,n)$ is in principle possible using an extension lemma (section \ref{SecExtLem}). Properties of this product and the resulting quantization of $AdS_l$ were not investigated because we found a more economical way to quantize $AdS_4$.

