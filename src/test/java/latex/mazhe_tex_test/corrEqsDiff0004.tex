% This is part of Exercices et corrigés de CdI-1
% Copyright (c) 2011,2014
%   Laurent Claessens
% See the file fdl-1.3.txt for copying conditions.

\begin{corrige}{EqsDiff0004}

\begin{enumerate}

\item
$y'-2ty=t$ est une équation linéaire non homogène, donc nous résolvons d'abord l'équation homogène $y'_H=2ty_H$, qui amène
\begin{equation}
	\ln(y_H)=t^2+C,
\end{equation}
c'est à dire
\begin{equation}
	y_H(t)=C e^{t^2}.
\end{equation}
La méthode de \href{http://fr.wikipedia.org/wiki/Méthode_de_variation_des_constantes}{variation des constantes} nous dit de chercher la solution du système non homogène sous la forme $y(t)=C(t) e^{t^2}$. En remplaçant dans l'équation, nous trouvons $C'(t)=t e^{-t^2}$, c'est à dire
\begin{equation}
	C(t)=-\frac{ 1 }{2} e^{-t^2}+K,
\end{equation}
et donc
\begin{equation}
	y(t)=(-\frac{1}{ 2 } e^{-t^2}+K) e^{t^2}=-\frac{1}{ 2 }+K e^{t^2}.
\end{equation}
Le problème de Cauchy dit de résoudre
\begin{equation}
	y(1)=-\frac{ 1 }{2}+Ke= e^{-1/2},
\end{equation}
donc de prendre $K= e^{-3/2}+\frac{1}{ 2e }$.


\item
Nous commençons par résoudre l'équation homogène $y'_H+y_H\tan(t)=0$. Cela donne
\begin{equation}
	\ln(y_H)=-\ln\left( \frac{1}{ \cos(t) } \right)+C.
\end{equation}
Ici, nous effectuons un changement de nom pour la constante : $C\to \ln(C)$, et nous utilisons la formule de somme des logarithmes. Nous obtenons $\ln(y_H)=\ln(C\cos(y))$, et donc
\begin{equation}
	y_H(t)=C\cos(t)
\end{equation}
comme solution générale de l'équation homogène. En ce qui concerne la solution de l'équation non homogène, nous posons
\begin{equation}
	\begin{aligned}[]
		y(t)&=K(t)\cos(t)\\
		y'(t)&=K'\cos(t)-K\sin(t),
	\end{aligned}
\end{equation}
que nous remettons dans l'équation de départ :
\begin{equation}
	K'\cos(t)-K\sin(t)+\frac{ K\cos(t)\sin(t) }{ \cos(t) }=\sin(2t).
\end{equation}
Les deux termes en $K$ non dérivé se simplifient et il reste
\begin{equation}
	K'(t)=\frac{ \sin(2t) }{ \cos(t) }
\end{equation}
qui s'intègre très facilement une fois que l'on a pensé à utiliser la formule $\sin(2t)=2\sin(t)\cos(t)$ :
\begin{equation}
	K(t)=-2\cos(t)+C,
\end{equation}
et la solution du problème non homogène est
\begin{equation}
	y(t)=K(t)\cos(t)=\big( C-2\cos(t) \big)\cos(t).
\end{equation}
La résolution du problème de Cauchy est $y(0)=-2+C=6$, donc $C=8$.

\item
$y'+y\cotg(x)=5 e^{\cos(x)}$. L'équation homogène associée est $y'_H+y_H\cotg(x)=0$, dont la solution est donnée par
\begin{equation}
	\ln(y_H)=-\ln\big( C\sin(x) \big),
\end{equation}
et donc par
\begin{equation}
	y_H(t)=\frac{ C }{ \sin(t) }
\end{equation}
où nous avons renommé $C\to C^{-1}$. La méthode de variations des constantes demande de substituer ceci dans l'équation :
\begin{equation}
	\begin{aligned}[]
		y(t)&=\frac{ K(t) }{ \sin(t) }\\
		y'(t)&=\frac{ K' }{ \sin(t) }-K\frac{ \cos(t) }{ \sin^2(t) }.
	\end{aligned}
\end{equation}
Nous trouvons
\begin{equation}
	\frac{ K' }{ \sin(t) }-\underbrace{K\frac{ \cos(t) }{ \sin^2(t) }+\frac{ K }{ \sin(t) }\cotg(t)}_{=0}=5 e^{\cos(t)}
\end{equation}
où, comme toujours, les termes en $K$ se simplifient, et il reste $K'(t)=5\sin(t) e^{\cos(t)}$, d'où $K(t)$ se trouve. La solution finale est alors
\begin{equation}
	y(t)=\frac{ -5 e^{\cos(t)+C} }{ \sin(t) }.
\end{equation}

La condition de Cauchy demande $y(\frac{ \pi }{ 2 })=\frac{ -5+C }{ 1 }=-4$, et donc $C=-1$. La solution est donc
\begin{equation}
	y(t)=\frac{ -5 e^{\cos(t)}-1 }{ \sin(t) }.
\end{equation}

\item
\item
$x^3y'+(2-3x^2)y=x^3$. Nous allons commencer par diviser l'équation par $x^3$. Nous devrons discuter à la fin ce qu'il se passe en $x=0$. L'équation homogène est $y_H'/y_H=(3x^2-2)/x^3$, et la solution est
\begin{equation}
	y_H(t)=Kx^3 e^{-x^2}.
\end{equation}
Nous appliquons la méthode de variation des constantes, et nous mettons
\begin{equation}
	\begin{aligned}[]
		y(x)&=K(x)x^3 e^{1/x^2}\\
		y'(x)&=K'x^3 e^{1/x^2}+K(3x^2-2) e^{1/x^2}
	\end{aligned}
\end{equation}
dans l'équation de départ. Après simplification nous trouvons
\begin{equation}
	\begin{aligned}[]
		K'(x)&=x^{-3} e^{-1/x^2}\\
		K(x)&=\frac{ 1 }{2} e^{-1/x^2}+C
	\end{aligned}
\end{equation}
La solution à l'équation non homogène est donc
\begin{equation}
	y(x)=\frac{ x^3 }{2}+x^3 e^{1/x^3}.
\end{equation}
Cette solution n'est pas définie en $x=0$, et, pire, $\lim_{x\to 0} y(x)\neq 0$. Donc la seule solution possible qui passe par $x=0$ est la solution identiquement nulle.
	
\item
L'équation homogène est $y'_H/y_H=2\cotg(2x)$, dont la solution est
\begin{equation}
	y_H(x)=K\sin(2x).
\end{equation}
\end{enumerate}

\end{corrige}
