% This is part of the Exercices et corrigés de mathématique générale.
% Copyright (C) 2009,2012
%   Laurent Claessens
% See the file fdl-1.3.txt for copying conditions.
\begin{corrige}{TP40001}

    %TODO : refaire le dessin
%La fonction est dessinée sur la figure \ref{LabelFigExpsqret}                                                                                 
%\newcommand{\CaptionFigExpsqret}{La fonction de l'exercice \ref{exoTP40001}}                                                                 
%\input{Fig_Expsqret.pstricks} 
D'abord, la fonction ne commence à exister qu'à partir de $x=-1/2$, donc il faut calculer le volume entre $-1/2$ et $0$. En utilisant la formule $V=\pi\int_a^bf(x)^2dx$, nous devons calculer
\begin{equation}
	V=\int_{-1/2}^0 e^{2x}(2x+1)dx.
\end{equation}
Cela se fait par partie. Nous posons
\begin{equation}
	\begin{aligned}[]
		u&=2x+1	&dv&= e^{2x}\\
		du&=2	&v=&\frac{1}{ 2 } e^{2x}.
	\end{aligned}
\end{equation}
Donc l'intégrale à calculer est
\begin{equation}
	I=\int e^{2x}(2x+1)=\frac{1}{ 2 } e^{2x}(2x+1)-\int  e^{2x}=x e^{2x},
\end{equation}
et le volume est
\begin{equation}
	V=\pi[x e^{2x}]_{-1/2}^0=\frac{ \pi }{ 2e }.
\end{equation}


\end{corrige}
