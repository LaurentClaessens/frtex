% This is part of Exercices de mathématique pour SVT
% Copyright (C) 2010
%   Laurent Claessens et Carlotta Donadello
% See the file fdl-1.3.txt for copying conditions.

\begin{corrige}{TD3-0004}

	Lorsque $n=0$, la formule est vraie parce que $u_0x=bn+x$ si $n=0$.

	Supposons maintenant que la formule soit vraie pour un certain $k$, c'est à dire que $u_k=kb+x$. Dans ce cas, le terme suivant de la suite vaut
	\begin{equation}
		u_{k+1}=u_k+b=bk+x+b=(k+1)b+x,
	\end{equation}
	ce qui est bien la formule demandée pour le terme $u_{k+1}$.

	Par récurrence, la formule est vraie pour tous les termes de la suite.

\end{corrige}
